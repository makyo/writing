Did you know that I used to read tarot cards? I still have the
embarrassingly large collection of decks, books, and other accessories
that go along with the practice. ~I pull them out every now and then to
remember the person that I used to be. ~I used to be intensely focused
on the subconscious and all of the ways in which it wound itself through
our waking lives. I used to daydream about spending the requisite hours
necessary for a 78 card spread using every card in the standard deck,
even if I only did it once, ~At one point, I even vowed to do one
reading for myself a day for 78 days in order to write a book about the
experience (an idea that crops up with just about every interest I pick
up, I should note).

I've talked about change before, and I have even
\href{http://adjectivespecies.com/2012/03/21/makyos-kaddish/}{laid bare}
some of the changes I have gone through personally. ~Even though my
fascination with tarot has waned, I still retain the general interest in
the ways in which the subconscious works in our lives, and I can still
appreciate the deep symbolism that goes along with it. I would be lying,
in fact, if I were to say that there wasn't some subconscious link tying
me to the furry fandom. And, having had a few conversations on that
point, I think that the same holds true for a lot of us here.

What, then, might be some of the subconscious reasons for wanting to
join in a fandom of like minded individuals, spending hours online or
sometimes hundreds (or thousands) of dollars just for the chance to
interact in person at a convention? What would pull someone to a
loose-knit group of individuals with the general theme-in-common of
totally digging animals anthropomorphized to some extent, or perhaps
humans similarly zoomorphized? ~It is, of course, one of those questions
that has a different answer (or perhaps several) depending on who you
ask, but I think that they are likely to fall into several loose
categories.

First, though, I think it's beneficial restrict ourselves a little in
order to focus better on the task at hand. ~There are quite a few
reasons that someone might wish to join in a subculture, be around close
friends, construct a chosen family for themselves; let's set those
aside, however, and focus on the reasons that someone might be willing
to construct for themselves an avatar through which they may interact
with others. ~While it's certainly not a universal in our fandom, I do
think that character creation and interaction are still quite common,
and figure large in the ways in which we communicate, even if it is only
to purchase art.

One of the reasons that immediately springs to mind for me is escapism.
This requires a little bit of explanation on how I really got into the
fandom in the first place, though, and I hope you'll forgive the brief
digression. Around about the second half of 2000, a lot of things
happened at once within my mom's side of the family (my parents having
originally gotten divorced when I was quite young), and the tension
between my mom and then step-dad grew daily, eventually to the
dissolution of that marriage as well. The first of the divorces happened
when I was too young to remember, but the second occurred near the
beginning of high school. ~I had just come out as, at the
time,~homosexual, as well, and the combined stress led to a strong
desire to escape to some sort of place where these issues didn't loom
quite so large.

I think that this was a common theme among several of my friends within
the fandom at the time, as well. Although, by virtue of being able to
even escape onto the Internet to pretend to be animal people, we led
rather privileged lives, we all had stresses of a sort, or own realities
looming over us, providing the desire to escape into a fantasy world.
I've mentioned before the startling banality of a lot of this fantasy in
which we took part, with folks hanging out in parks or bars, being
students or programmers, and I think that reflects a bit of that
escapism: getting away from a hectic life to take part in what your
segment of society views as normal. To fantasize about normalcy, even
with that element of magical realism inherent in being a fox-kid, shows
the need to get away from life as it stands.

Of course, despite the normalcy~striven~for by the crowd that I hung out
with, there are certainly more fantastic elements to the fandom. ~With
character creation and world building, just about anything is possible,
anywhere from simple non-binary gender roles to plant-cats and
digital-huskies, from vast changes in financial class and social status
to vast changes in size, even whole constructed realms with a fleshed
out backstory, rules, or laws of physics. Beyond simply being happily
normal, fantasy can satisfy out sense of grandeur.

Many individuals have a need to better themselves in any number of ways:
to become thinner, to get rich, to win friends or defeat enemies.
Fantasy provides one outlet for this. ~Through the process of character
creation, one can construct an avatar that fills this need for
grandiosity. By becoming, however temporarily or shallowly, a fantastic
entity, we can satisfy some of our mythic desires.

This need to better ourselves needn't be on either extreme, of course.
~For many, simply the feeling of fulfillment involved in creating the
person that we really wish to be is enough. ~For myself, I think this is
of prime importance. Now that I've grown up, left college behind, and
moved away from family (thank goodness), furry has taken on less of the
escapist overtones and simply become the place where people can strive
to be what they wish to be. ~Much of the psychological reasons for this
were covered in JM's previous post on
how~\href{http://adjectivespecies.com/2012/09/17/our-fursonas-are-happier-than-we-are/}{our
fursonas are happier than we are}.*

This was hammered home recently when I received word of a good friend's
passing. Although he and I hadn't had much of a chance to interact in
the past few years since he joined the army, prior to that he was
someone that I looked up to and trusted in a sort of chosen-family,
big-brother way. ~In fact, there are several people like that still in
my life, those friends, usually older than myself, whom I sort of
adopted as people to follow back when I was in that escapist mode.
Thinking about this after my friend's death originally made me feel a
little guilty, a little creepy; it honestly made me feel like kind of a
sad person for having been raised by a bunch of older gay guys
pretending to be animal people in an Internet gay bar.

In the end, though, we all grow, change, and mature over time, and I
think that I've come out alright. Rather than focusing on living a
normal life online, I'm lucky enough to be living something like that in
person in my own way, and my interactions with others through an avatar
have reflected that. The death affected me deeply due to it being a
rather blatant signal of that change, and now I know that I'm using the
fandom more to help fulfill my needs to become what I want to be: not a
fox person, \emph{per se}, just my ideal self. ~This goal of fulfillment
is something that I see in a lot of people within the fandom, too.
Beyond simply playing an anthropomorphic animal, they are playing what
they wish to become.

These are just a few of the important factors of draw to the fandom, of
course. ~Part of the whole reason of {[}a{]}{[}s{]}'s existence in the
first place is to try to explore those factors. ~I had originally
thought that it might have lasted for a few articles and then devolved
into a current-events site, or maybe into just reviews of all the
wonderful creations out there. The cool part about our subculture,
however, is that we truly do become part of it, and for all sorts of
reasons. These reasons, these draws, these subconscious aspects of our
participation all shape the way we interact with each other through our
chosen and created avatars, and help shape those avatars in turn on a
very fundamental level. I encourage everyone to consider the
subconscious aspects of why they are involved with the fandom: that sort
of introspection is always quite valuable.

\begin{center}\rule{0.5\linewidth}{\linethickness}\end{center}

* I should note that, as I was travelling for work, I wrote this on the
plane before getting a chance to read JM's delightful article. ~I
apologize if this seems a bit repetitive!
