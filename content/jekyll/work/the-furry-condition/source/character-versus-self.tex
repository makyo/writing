When I first got into furry, I was probably fourteen or fifteen. ~I know
that it was the fall semester of my freshman year of high school, and
that I started getting into it in my downtime in my first computer class
at school (well, during class, too), as well as at home. ~I wound up
finding Yerf and FluffMUCK back in their prime, and played around with
IRC on YiffNet, as well. ~I found the whole thing from a website I was
on called Puberty101~- which now sounds like a pedophile's paradise; the
name was later changed to GovTeen - a forum site for (supposed) kids to
ask questions of other (supposed) kids about things like sex and
sexuality, emotions, and all that jazz. ~Just so happened that I
stumbled over a few posts regarding this thing called furry, one of
which had this abstruse collection of letters, numbers and punctuation
at the end, which was described as a `fur code'.

I had already been all about the good old furry favorites like Disney's
Robin Hood, The Rescuers, Mrs.~Frisby and the Rats of NIMH, the Redwall
books, and so on. ~Finding the fur code and what it meant at that time
in my life led to a perfect, terrible storm of destruction for any hopes
of normalcy I had planned for my life. ~I latched right onto it and,
after spending three dumb days as a dragon, settled on a red fox with
two tails as my character and dubbed myself Ranna. ~This was the subtle
point that would take me the better part of ten years to disentangle:
character creation.

I sometimes wonder if people involved in LARP communities, those in the
SCA, or even pencil and paper RPG players get quite as involved in their
characters as furries do. ~I honestly don't know, as most of my
knowledge is gained from an outside, media-tainted perspective, but I
suspect that it might be a little different for furries for a couple of
reasons. ~First and foremost is that our characters are intended to be a
representation of ourselves. ~The thing that drew me in about FurCodes
was the `T' segment:~If you had the chance, would you want to become a
real furry. ~This wasn't just something fun we did or some historical
accuracy we~strive~for - people actually really, truly desired to become
their characters. ~I'm sure there are folk in the SCA or in LARP groups
that really do desire to be in the role they're playing, but that leads
us to the second point.

Furries don't necessarily role play outside themselves. ~Someone who
gets so into renaissance festivals that they wind up working there and
living the characters on weekends is casting themselves into a totally
different time, where the modern conveniences of life are gone and
everything is fundamentally different. ~Furries - and, though I'm
speaking from experience here, I know it doesn't apply to everyone - are
perfectly content to act out their day to day, mundane, boring-ass lives
as anthropomorphic canines (statistically speaking). ~As I was growing
up through high school, I hung out with a crowd made up of furry gamers,
programmers, and computer nerds; not just the players, but the
characters as well. ~As I grew and moved to college, I decamped from
FluffMUCK and moved over to FurryMUCK to spend most of my time in The
Purple Nurple, an online, text only gay bar where predominately gay
furry yuppies aired their college and post-college woes. ~We weren't
just pretending to be cat- and dog-people, and we weren't just chatting
about work, we were cat- and dog-people chatting about work.

Of course, I wasn't totally secluded in my world of young professional
furry gay men, I hung out elsewhere online and experienced everything
from multi-session, all-hours of the night role playing (usually dirty)
to entire relationships enacted strictly in-character. ~However, while
there were always `OOC', or out-of-character moments, everyone was
joined together in the fact that they \emph{were}~their character. ~Even
when I was in college, the music department, a decidedly close-knit
group, contained several people who were just in it because they
happened to be good at playing, say, the oboe, and could give a shit
less about music, being an instrumentalist, or even making money off
their skill. ~In my experience, people like that in furry are rare:
there's the occasional person who has no real attachment to the fandom
other than they simply happen to be good at some aspect of it, but they
seem to be far from the norm.

All of this adds up to something that I feel is fairly unique to the
fandom. ~It is a strange line that divides character from self, in a
fur. ~The line is semipermeable as some would gladly view themselves as
their character as a sort of whole-body dysphoria, but there's still the
separation between that aspect of personality and the person as a whole.
~Our characters are intangible, non-spatiotemporal; they aren't
something that can be touched or felt, and are closer to an idea than
anything real. ~However, they form an integral part of our concept of
self, whether or not we would actually like to be our anthropomorphic
fox character in real life. ~They inform our view of the world around
us, as well, and not just in some vaguely foxish or wolfish way.

There is no denying that a good portion of the community revolves around
art - visual and otherwise. ~As with any group of people, though, skill
in one particular field is not evenly distributed, and while there are
definitely a lot of amazing artists within the fandom, they are still a
minority. ~We rely on the skills of a relatively small sub-set of our
community to provide us with the more tangible representations of our
characters, and here is where this blurred line between character and
self can cause issues. ~However, the way in which furries interact with
creators in the community differs greatly from the way in which a
professional artist would interact with a client in a few very important
ways. ~A client may commission an artist for a piece of artwork to
appreciate or for others to appreciate - that is, something to hang in
their house or something to hang in public. ~With ~music, you can branch
out and say that a client may commission the artist for a piece of music
to perform. ~In all of these cases, though, nothing works quite like it
does in the fandom: with furry commission, you're not simply
commissioning a piece of art to hang around the house and show others,
you're commissioning a representation of your \emph{self}.

Several seemingly unique issues in the way that artists and clients (or
`commissioners', as they're called, leading one artist to create a
``feral Commissioner Gordon'') stem from this strange difference. ~Some
of the onus of creation is moved from the artist to the client in that
much of the picture is designed by the client instead, because, after
all, it is the client's character and the artist's talent. ~This seems
to work closer to standard work-for-hire relationships, except that it
has strange inflections on licensing: FA notably specifies that uploads
fall under a policy of `by you/for you', where a user may upload a
picture that they created or that was created for them. ~Rather than
falling under a standard work-for-hire relationship where it is the
artist's talent and the client's art, there exists a continuing tension
between the two parties, the artist maintaining near full rights over
their creation while the client's rights remain in shady limbo - they
maintain rights over the intellectual property of their character, and
have some vague sense of ownership over the picture they've received,
with a shadowy idea of where they're allowed to show it.

As a personal example, I was commissioned for a three-movement work for
French horn and string base to be performed on my senior recital. ~As I
had been used to the standard furry way of doing things, I insisted that
the instrumentalists specify rather more than less of the work, a fact
that led to much strife and pain in getting the piece actually
performed. ~I was unable to live up to their expectations (they wanted
me to write like Hindemith, and I'm not Hindemith), they were
unmotivated to rehearse a piece that they felt they had a hand in
creating, and my composition professor was baffled by the whole
scenario. ~My senior recital turned out to be one of the most
disappointing experiences in my life, largely in part due to the fact
that I had failed to properly execute the commission that was expected
of me.

From the other side, an
\href{http://www.furaffinity.net/user/pseudomanitou}{artist} on FA
recently wrote a journal about possibly offering prints of works that
were commissioned from him, mentioning that since it was work-for-hire,
he would split profit with the client who had commissioned the piece in
the first place. ~The
\href{http://www.furaffinity.net/journal/2840110/}{result}~was rather
out of proportion with the original post and helped to illuminate
several of the differences between the professional art world and the
art world contained within the furry fandom. ~``My talent, not my art
{[}is for sale{]}. A commissioner buys my talent to make their art,''
the artist writes, leading to a slew of comments ranging from decently
positive to stunned and angry. ~This standard practice is in direct
opposition to the way the furry art world works - limited rights to the
artist's art is for sale, rather than simple access to their talent.

No small amount of drama has originated from this scheme. ~While the
artist above relinquishes their rights to the piece they've created to
the client as part of standard business practice, this is not the usual
within the fandom, and a client doing something such as uploading their
art to be seen by a wider audience on other furry art sites such as
fchan, e621, or pawsru.org can certainly lead to plenty of strife.
~There is the occasional artist who will upload their art to these sites
on their own, but the fandom has largely set them up as their villains,
several of the sites or members of the sites buying readily into that
label and stirring things up on their own. ~This concern over use of art
is doubly strange for a community so focused on appropriating heavily
licensed characters such as those from Sonic the Hedgehog or anything
from Disney for themselves.

The concepts of character and self are rooted deep in the furry
community. ~Making a negative comment about someone's fursuit or images
of their character can lead to trouble, as the words can be seen as a
slight against that person. ~After all, the fursuit or image is a
representation of the character's owner - even if you agree that a thing
is ugly, a careless phrase can cause offense if that thing is dear to
you. ~The result is something akin to an offshoot of the Dunning-Kruger
effect - unskilled people holding illusory superiority while skilled
people hold illusory inferiority - in that the one who receives a
representation of their character is likely to hold it to some illusory
ideal higher than just any similar piece. ~Meaning in art is a tough
subject, and it's only made more complicated within the fandom when it
comes to character art.

The two intertwined entities of character and self comprise a large part
of furry. ~The fandom as it is is hard enough to pin down to any one
definition, and I think that's due in large part to the myriad ways in
which one interacts with one's character or characters. ~For some, their
character is inextricably a part of themselves, closer to an anima or
animus in the Jungian sense. ~For others, myself included, a character
may carry smaller aspects of personality, and not, as a result, be as
all-encompassing. ~Speaking for myself, I have three or four of what I
would consider characters that I often interact with, and each acts
differently, each more focused on a different aspect of my personality.
~This didn't use to be the case, though, as I previously had a single
character that was more all-encompassing and close to my self.

Along with the shift in character interaction came a shift in friend
circles, and it left me wondering how much this internal interaction
define how we build up and maintain our lives within furry. ~I asked
around on \href{http://twitter.com/adjspecies}{Twitter}~and got a few
answers: the way in which we relate to our characters does seem to have
some relation to the types of people we find ourselves friends with.
~Whether that's cause, effect, or some sort of subconscious correlation,
I can't say.~~All of this pondering around the psychological aspects of
pretending to be an animal person with a lot of other people pretending
to be animal people may just be another symptom of being a
firmly-entrenched member of the very same fandom. ~A commenter on a
previous entry used the word `avatar' instead of character, and I feel
that this was an appropriate choice of words, moreso than character. ~A
character is an entity not necessarily connected to some person in
reality, but an avatar has connotations of incarnation and appearance of
something outside the world in which it interacts. ~\emph{This}~is the
idea behind our characters: they aren't just some sort of disjoint idea
that relates back to us, even if we create more than one. ~They are
aspects of us, and as such, are integral to us. ~No wonder we can get so
touchy in regards to our interactions with them.
