If we accept the fact that the furry subculture, the fandom as a
cohesive group of somewhat like-minded individuals, has only existed for
about thirty years, then we have available to us a growing and expanding
membership at the beginning of what I hope to be a long thread of human
society. We're still in that bright, almost expansionist era of our
creation where we are doing out level best to create more than we can
consume. We bring in new members not only through the shared interest in
anthropomorphics, but also through both the vibrancy of our existence
and the social currency of our creative output. Furry, such as it is, is
on the rise.

We are still young though, there's no getting around that.

Thirty years, in the grand scheme of things isn't really all that long
of a time. The United States has lasted eight times that long,
Christianity approaching 70 times, and, according to some, the universe
almost 200 times that long, and that number is considered very, very
small by many others. Our vibrancy and social currency is strong, but we
are not the only group on the rise out there. In western culture, the
anime fan base is taking a similar track, as have countless other
subcultures and fandoms before it. Our output is copious and so, in
turn, is our social currency, but they are not out of proportion.

Our fandom is young, and given the median age of about twenty years old,
we are a fandom made up of many, many young people. Really, then, it's
no surprise that a single death among our ranks affects so many of us so
greatly.

As I mentioned last week this article was one that has been in the works
for a bit, and was intended to go live last week. I, like JM, like to
get the article done a day or so ahead of time in order to make sure
everything is set to go off without a hitch. Unfortunately, while I had
this article halfway done, I heard the distressing news of the loss of
two furries via several posts on FA. I waffled for a few days about
whether to continue on with the publication of this post in tribute or
to hold off out of respect, and, at the last minute, wound up coming to
this compromise of a weeks delay for a respectful entry.

Death and the larger concept of mortality have been our fixation for
almost all of recorded history. It's arguable, really, that death and
mortality have been the fixation of life for its entire existence here
on earth. It's something of a milestone in life when we start to realize
that we're mortal, that we will end and that at that point, something
fundamental about our existence will change, whether it's entering into
heaven or simply the same unknown we return to that we were a part of
before birth. For me, it was about the time I turned eight or nine and,
leaning against my mother's front while watching TV, I heard her
heartbeat and it hit me, in a very logical fashion, that at some point
that heartbeat would stop and my mom would be no more. I suppose it
happens to everyone now and then, but from an individual's perspective,
the idea that life will eventually come to a stop is something that
focuses the mind and all but forces introspection.

Death is always a tricky subject, but especially so in a societal
context. ~Death has become an industry in Western culture; not just
dealing with the remains of our loved ones respectfully, but also the
industry of delaying death and the industry dedicated to bereavement.
~Whether or not the concept of the end of one's life is cause for
introspection, it's something that society has grown up to deal with.
~There are arguments to be made for the fact that death - or at least
protection from early death - is at the center of society and
governance. ~The sharp contrast between life and death is often at the
center of much of religion and art as well, both social concepts. ~It
makes sense, then, that a subset of society (and of religion and art, if
you look at furry that way) would also have its collective mind so
focused by loss.

We have at least two benefits within furry, however. ~First of all,
we're still relatively small. ~The Tucholsky quip that ``The death of
one man: that is a catastrophe. One hundred thousand deaths: that is a
statistic!'' would be difficult to hold true in our subculture of one or
two hundreds of thousands (an arguable point, I'm sure). ~For us, one
death is a tragedy, but given our small size, any number of deaths would
likely be as much a tragedy. ~Much of the basis for this quote has to do
with Dunbar's number, the suggested limit of stable relationships one
individual can maintain; with a community of our size and a rough
estimate of perhaps 150 for Dunbar's number, that means that, no matter
what, in the event of a~catastrophe, the chances of one being directly
affected, either through personal involvement or a personal
relationship, are much, much higher.

The second, and perhaps more important benefit is that furry is based
around a willful membership. ~\emph{We}~identify as furries, whether or
not the interest in anthropomorphics is innate, whether or not we feel a
connection with animals. ~It is a choice, much more than skin color or
biological sex could ever be. ~Our membership in the subculture comes
primarily with the benefits of social currency and standing within the
smaller group, and in a limited setting with such a friendly group, it's
hardly surprising to see loosely connected people paying their respects
to the dead and the bereaved. ~On the FA profile page of any deceased or
grieving member of the fandom, one is likely to see that nearly every
shout or comment on a journal is another fur offering their sympathies.

The interesting side of this is that many, if not most of those leaving
their shouts and comments do not actually know either the bereaved or
the deceased. ~ They have found out about it through their own social
networks. ~In our socially oriented fandom with a relatively small mean
degree of separation between individuals, news about anything travels
fast. ~If one sees a friend grieving over a loss, and makes one mention
of it, chances are good that someone not even involved will feel moved
and may even leave their own note.

Nothing is ever quite so simple, of course, and there are a few
downsides and negative aspects to our relationship with death.
~Primarily, just because we know or know of someone does not necessarily
mean that we like them. ~Many simply keep their peace in such
situations, but some have noticed that individuals will occasional
create puppet-accounts on social sites in order to post a negative
comment or two, or even use their own account to rail against the
deceased or their loved ones. ~I feel that much of this is likely due to
the anonymity provided by interactions on the Internet, but I could be
wrong. ~Perhaps there is an additional aspect to our social nature or
our tightly-knit web of relationships that makes it easier for one to
express their views, both positive and negative, but that said, I hear
far, far less about this happening in person than online.

An additional factor to take into account is that the fandom
\emph{is}~growing, and at quite a clip. ~There seems to be hundreds of
new furries each day. ~Dragoneer, the owner of FurAffinity, recently
mentioned that, in 2011, there were anywhere between 300-500 new
accounts created per day for a total of 145,787 new accounts in that
year alone, most of which were estimated to be unique, non-group
accounts. ~Along with the growth of the fandom comes a greater chance of
losing one's individuality in life and not being noticed quite as much
in death. ~However, even if the number of random strangers comforting us
in our grief declines or the number of shouts from those who didn't know
the dead starts to decrease, our membership still gets us a caring
family and many ready friends.

In the end, however, death within the fandom is still something that
strikes us strongly. Perhaps it's due to our small size, or our
tightly-woven net of interpersonal relationships, or even due to the
online nature of much of our interaction, but no matter what, it's
comforting to know that there are those out there who, whether or not
they knew us, would feel our loss. So let this article stand in memoriam
of FirePyro and Athus, Waarhorse and Randomonlooker, Ponybird and Loki,
and all the others who have entered into our lives through furry and
then gone. I'm opening up
\href{http://forums.adjectivespecies.com/viewtopic.php?f=4\&t=21}{a
topic} in the forums for additional names to be added to the list; if
the fandom has lost someone in particular that you know of, feel free to
add them to the list of those to be remembered.
