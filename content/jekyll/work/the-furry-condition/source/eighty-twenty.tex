One of the interesting things about running a blog is that you get to
write about what's important to you. ~And one of the interesting things
about running a blog with more than one contributor (hi guys!) is that
rather than focusing on the whole field, you're more able to spread the
labor around and focus on specific things within the field that are very
important to you. ~Given that I've already written a more broad-picture
article on gender and am now about to delve into another 2000 word essay
on the same, it's safe to say that I think the whole thing's terribly
interesting, and that furry itself is probably one of the more
interesting subcultures in which to examine gender, sex, and sexuality.

As I did in the
\href{http://adjectivespecies.com/2011/11/16/boys-girls-and-the-in-betweens/}{previous
article}, I feel the need to provide the following information and
disclaimers about myself. ~Firstly, I am a biological male, I do not
identify as male-gendered, and in terms of sexuality, while I'd call
myself pansexual, I am engaged to another man. ~Since that's what I've
got to work with, that's the viewpoint I'll be writing from, even though
I'll try to draw as much as I can from others. In addition, the title is
in reference to results provided by Klisoura's Furry Survey, which will
be mentioned within the article itself. ~Some of the thoughts in this
article come from the responses to the {[}adjective{]}{[}species{]}
\href{http://survey.adjectivespecies.com/sexuality-and-gender}{survey}
on gender identity and sexual orientation in the fandom. Finally, I know
that my articles are wordy, perhaps more so than they need to be, but
given that this topic is especially important to me, I do hope that
you'll forgive a slightly longer read.

Now that we've satisfied that nagging part of me that needs to make
disclaimers\ldots{}

Part of what got me interested in this whole topic to begin with is the
way I spent most of my time in the fandom for the first five or six
years of my time here. ~Without going into more detail than has already
been covered, I spent a lot of time hanging out with mostly gay guys
online, primarily on MUCKs and IRC. ~It was what I'd call a comfortable
existence. ~My daily routine online consisted of connecting and
immediately heading to the gay-bar-analog, whether it was an IRC channel
or a room on FurryMUCK, to spend some time chatting it up, or maybe even
looking for some hot, hot text-only action.

And it was pretty fulfilling, too! ~I met some wonderful people I still
love to spend time around (hi guys!), had my fair share of relationships
that occupied heart and mind almost completely while they lasted, and
just generally lived out my little hedonist life as a red, then an
arctic fox. ~I explored some things that I would never do in person, and
some things that aren't even possible offline, but in all, it was a
young gay man's paradise; sex without consequences, a large dating pool,
and a surrounding subculture that was almost fanatically accepting.

There were a few little things, however, that I hardly noticed at first,
but started to bug me more and more as time went on.

I've noticed a trope in western gay culture, such as it is, that
discovering you're gay goes through five main stages. ~Put glibly:

\begin{enumerate}
\def\labelenumi{\arabic{enumi}.}
\tightlist
\item
  Age 5-12: ``ew, girls are icky!''
\item
  Age 13-14: ``I'm supposed to like girls now\ldots{}''
\item
  Age 15-25: ``ew, girls are icky!''
\item
  Age 26-32: Maturity
\item
  Age 32+: A mystery. ~Some say The Gay ends, some say that this is
  about 102 in gay years, and some say that a few mythical couples live
  on\ldots{}
\end{enumerate}

Alright, so that was put \emph{very} glibly. ~Even so, I bring this up
in continuation from last week's article, Participation Mystique,
wherein I mention some of the participation mystique that gay men have
with western, or at least American gay culture.

There is a certain~rebelliousness that we (and I say ``we'' freely; I
identified as gay for quite a while) buy into. ~It starts with the
rebelliousness that many teenagers go through without further prompting,
continues on through liberation to college or working life where we know
everything, and peters out around the time we land a job or career we
aim to keep for a while. It's a rebelliousness against the
heterocentrism that is inevitable in a world that, to requote and
oft-quoted statistic, is 90\% heterosexual. ~The bias is justified,
sure, but we're up-and-coming young adults and there's no reason we
shouldn't assert not only our existence, but our membership to the gay
culture, our participation mystique.

It's been successful to some extent, as well. ~The whole ``we're here,
we're queer, get used to it'' scene has done much to push the culture
and its members into the conscious mind of America, and change is indeed
happening at both a state and national level. It's the return to the
``ew, girls are icky!'' stage that I find intriguing, though. ~A focus
on marriage rights, matronly pop-stars, and men having sex with men is
not the only thing that the gay culture brings with it. ~Of note to us
is a sort of misogyny that is based within this rebelliousness, a
rejection of the female body as being unappealing which seems to go
hand-in-hand with the trope of straight men liking lesbian porn due to
the lack of male bodies in the picture. ~While it's a subtle sort of
misogyny that is based around the bearer of the bias' own state more so
than the bearer of the brunt of the bias (that is, this particular bias
is based in the fact that gay men do not like women, rather than the
fact that women are perceived as fundamentally inferior in some way), it
is still just that.

It is what it is, though. ~My high school history teacher said several
times that, in order for a segment of society to gain what they perceive
to be equal status, they have to push a little too hard, go a little too
far, in order to let things swing back toward the middle.

It is what it is, I should say, except in the case where you have a
population that is effectively 80\% male and 20\% female, rather than
the standard fifty-fifty split. Here in furry, we have a predominately
young male culture, anonymity provided by the Internet, a sexually
liberated atmosphere, and a group that is decidedly accepting of most
anything. ~In short, we have a perfect storm for something that smells
good to gay men. ~While there are countless roots into the fandom, I
don't doubt that several are through the exploration of homosexuality
online. ~I don't doubt it, because that's how I got here: a combination
of some people posting in a forum for gay teens and
some\ldots{}uh\ldots{}stories on a certain \emph{nifty} website.
~Needless to say, given all that, it's no surprise that there is the
concentration that there is of young gay and bisexual men within the
fandom.

I know that this was a long, round-about way to get here, but I feel
that it really is very important to understanding some of the misogyny
within the fandom. ~The misogyny that I'm speaking of, in particular, is
the reaction to sex within an adult image or story. ~We really are a
tolerant crowd, and there's room for everyone within this fandom. ~That
the subject matter drop-down when submitting a piece of art to
FurAffinity includes such things as ``paw (tame)'', ``pregnancy
(adult)'', and ``abstract'' (while somehow managing to leave off
``crafts'') is telling of just how open a community we really are.

We're all welcome here, and yet still there is this strange misogyny
that expresses itself almost as heterophobia in the reaction to art.
~What would an image depicting a straight couple having sex be on FA
without the ``this would be better if they were both male'', ``ew, grody
vagina :('', and ``you're cute, so I guess I'll just cover up the other
side of the screen'' comments? ~It's become pervasive on FA, respondents
to the
\href{http://survey.adjectivespecies.com/sexuality-and-gender/}{survey}
have mentioned it, and I've started noticing it within day-to-day
interactions with those around me, as well.

This is, of course, only one example of the sex and gender bias within
the fandom, of course. ~Along with our unique brand of heterophobia,
there do seem to be some unique gender roles that we've appropriated for
each other here in furry. As with most gender roles, they focus on
dichotomies and binary states. Men are x, women are y, and never the
twain shall meet. I tried to pull together three good examples, but
there are, of course, plenty more than that.

\begin{itemize}
\tightlist
\item
  \textbf{Female as creator, male as consumer} This, as with all of the
  examples I have here, is based in part off a gender role that is
  common in fields such as crafts or amateur art. That is, the female is
  seen as the one who takes the time to create, the one who would do
  such a thing as a profession, while the male is seen as the consumer,
  the one who would buy the created object. Though there are certainly a
  good number of male creators and female consumers within the furry
  fandom, it does seem that there is something of an expectation for the
  female furries to be the artists and fursuit makers, those who are
  creating, while the males are the ones browsing along the aisles of
  the dealers den, looking to purchase.
\end{itemize}

With this, as with most gender roles, there is little danger in bucking
the trend, but the pressure to go along with it remains. One will not be
castigated because one is a female consumer or male creator, but there
is still an expectation that things will work a certain way, and perhaps
a bit of disappointment when they don't. It is interesting to see the
differences in sex between those who are roaming the aisles and those
who are working the tables at a convention dealers den, however,
especially given the reported demographics of the fandom as a whole. *
\textbf{Female as nonsexual, male as hypersexual} A friend on twitter
recently mentioned that one of his favorite things about a certain adult
website was that it provided some insight into the feminine state of
mind when it came to sex. ``Society makes that hard to see,'' he said.
``Since for girls, sex is some big secret for the most part, when guys
are concerned.'' This is a codified gender role that goes way, way back;
centuries, even. That a female would ever enjoy sex was something that
was simply beyond the ken of many, and to this day, that remains a
concern within society.

Conversely, that a man might not be all about sex violates the code of
machismo that, if nothing else, is codified in western media, if not
society as a whole. There is a growing population of those - male and
female alike - within the fandom, as noted by a respondent to the
sexuality and gender survey, who identify as either asexual or
non-gendered. What my friend was bemoaning was the double standard and
that surrounds sexuality between the genders. This is perpetuated, to
some extent, within the fandom by the western culture that surrounds so
much of it. While a female bucking this trend is not likely to be called
a nymphomaniac, nor is a male likely to be called a eunuch, that it's
strange and new for us to see the opposite sides is telling of how
gender works within the fandom and our society as a whole. *
\textbf{Female as offline, male as online} One of the interesting
experiences surrounding gender that I had in college had to do with the
gender differences between the majors offered by the university. I went
to a school that very much bought into a lot of old-school ideas, from
the way it treated the arts to the ways it expected students to act.
Students and parents bought into this, as well - there was another, more
liberal school in the state, and our goal seemed to be ``don't be
them.'' So, not only were female engineers and scientists more rare, but
they more readily bought into certain roles such as ``nerd'' that males
didn't necessarily need to buy into. You could be a ``jock'' male
computer science major, it seemed, but you couldn't be anything but a
``nerd'' female computer science major.

This is a wide-spread issue that is being focused on by many better
minds than my own, but it's effects are also seen within the fandom.
Along with the creator role mentioned above, it seems like the females
of the fandom are not expected to be as willing to partake in MUCKs,
IRC, or even forums to some extent as much as males are. Combined with
the previous point about sexuality, and it is unsurprising just how much
of the population of Tapestries, a sexual and BDSM MUCK, is male.

I know that I've likely gone on for far too long, and probably lost
readers along the way, but I feel that this is an important topic for
the fandom to consider. We are an open-minded bunch, all told, but there
are a few sticking points where we have our troubles, and one of the
biggest problem spots has to do with gender. Even if it's not
necessarily the cause for huge amounts of drama, it always seems to be
riding beneath the surface of our interactions, making itself known here
and there in all our myriad means of communication.

Rather than end this overly long article on a simple concluding
statement, thought, I want to take the more proactive approach by
putting out a call for submissions. I've written this ``Eighty-Twenty''
article from the standpoint of a mostly-male furry in a mostly-male
fandom. What I think we really need, though is the ``Twenty-Eighty''
article written from a female standpoint about how the fandom works from
that point of view. I know that a few of you (hi guys!) have already
approached me about the possibility of writing such an article; well,
let this be your call to action - I don't think I'm alone in wanting to
hear more sides of the issue!

\begin{center}\rule{0.5\linewidth}{\linethickness}\end{center}

If you're interested in writing the companion piece to this, you may get
in touch with us via \href{mailto:submit@adjectivespecies.com}{email},
or send us a note on twitter to @adjspecies; we'd certainly welcome a
guest post to help fulfill this need in the community.
