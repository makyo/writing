Like many of those who who identify as members of the furry fandom, I
joined at a relatively young age. ~I was reminded of this, recently,
when a friend from years ago came out to visit, this last weekend. ~When
he and I were talking most frequently, that was eleven or twelve years
ago, which would've made me (gulp) fourteen or fifteen. ~I've been
dwelling on that point for the last few days, as I worked up the outline
of the rest of this article, and things finally fell into place when I
consider who I was and where I was in life at that time. ~I was young,
for sure, and just getting into the whole furry thing, watching artists
on Yerf and VCL (and Side7 and Elfwood, oh man\ldots{}) create these
awesome drawings, most of which seemed to be spur of the moment things,
or works of art created for the sake of creating art. ~Some, however,
were commissions, and that was something I just could not fathom.

An artist - someone I didn't even know - would draw whatever I told
them. \emph{For money!}

It boggled the mind, to be sure. ~I found the concept amazing, and spent
all of thirty seconds researching the idea before noticing the price of
a commission: \$50. ~At the time, I made that much in two months of
allowance. ~Once I could drive, my allowance went up, but then I was
expected to pay for my own gas as I drove back and forth a few times a
week between my mom's and dad's houses in the decidedly fuel
\emph{in}efficient junker I had been lent. ~It wouldn't be until I was a
few years into college that I paid an artist for a commission of my
character.

Money plays a not insignificant part in our fandom. ~While art was, for
a while, the thing that everyone tried, there was still a growing, core
group of artists that provided much of the output and garnered much of
the attention by offering a steady stream of commissions and filling our
VCL feeds, at first, and then our FA watch lists. ~For those who are
unable to draw their characters to their own satisfaction, all it took
was a few bucks, or a few hundred, dropped on a commission, a short, or
long, wait, \emph{et voila}, your character in a visible format to share
with the world. ~The financial transactions became more pronounced as
fursuiting began to gain in popularity, as the core group manufacturing
some of the most visible fursuits was even smaller, and the price point
higher. ~Finally, conventions offer their own unique financial burden
for those involved, whether it's simply the cost of attending one's
local convention or the price of airfare halfway around the world to
attend a con in another country.

However, there seems to be some additional doxa surrounding money within
the furry fandom. ~The ``poor fur versus the rich fur'', for example, is
a trope that plays itself out regularly in the comments on images and
journals on FurAffinity, particularly on the post of an artist offering
commissions. ~It usually begins with an ``I would, but I can't afford
it'' comment, and can often spiral into an argument from there. ~Much
has been written on this in the past, as this seems to stem from the
idea of the poor envying the rich and the lifestyle that they represent,
but in this case, the leisure either perceived or imagined, takes the
format of numerous commissions, a fursuit, and regular attendance at
popular conventions.

This ties into another example of the layers of meaning around money
within the fandom: being judged on the amount and status of one's
material possessions, usually in the form of commissions. ~A good
example would be the non-artist who commissions countless pieces and
reposts them all to FA, garnering followers and social status by
spending money. ~That is, of course, a cynical way of looking at it, and
perhaps a more kind-hearted explanation is that the individual is very
much into the visual representation of their character, and has the
money to spend to make that happen. ~Either way, the fact that the idea
of a member of the fandom gaining social standing by purchasing drawings
of themselves, as it were, points to the fact that this is something we
take into account on some level when interacting with those around us.
~After all, if someone has plentiful drawings of their character in a
myriad of styles, it's certainly easier to picture interacting with them
in some sort of furry world during RP, to name only one perceived
benefit.

The idea goes beyond just the consumers, however, and extends even to
the creators. ~We all know the overextended artist, ever taking more
commissions without finishing the previous batch, their work-load piling
up as they offer reassurances with one hand and sketch-stream
commissions with the other. ~Or there is also the under-priced artist,
who has decided on \$5 as a good price for a sketch, \$10 for color,
\$15 for shaded despite the obvious quality of their work and the time
spent on it. ~There are countless additional tropes involving the artist
and the role they play with the audience and their patrons, however, and
many surround the idea of money within the fandom.

``So what, you ridiculously wordy fox?'' I hear you saying. ~``What's
the big deal? ~We're a subculture dominated by westerners, and those
western types tend to be capitalists; is it really so surprising that
money would play a large factor in our fandom?''

Well, no, it's not surprising in and of itself. ~Within a western
capitalist society, money is exchanged for goods or services in order to
represent a fair trade for work performed. ~To extend that into our own
social group is only second nature: we offer money in return for the
work of rendering our characters visually, for a costume that we can put
on in order to act the part, or for the chance to go visit hundreds (or
thousands - hey folks at AC!) of like-minded individuals in one spot for
a wonderful weekend or two a year.

What~is interesting, however, is the complex interaction between
cash-money and social currency, which features prominently in our
interactions. ~I'm not kidding when I say ``complex'', either. ~Social
currency and financial currency are two topics that are, on the surface,
linked: by creating something worth buying, you are, in effect, making
something which has improved your social standing. ~Capitalist societies
don't necessarily work this way, of course, and so the relationship
between the two exists in a sort of tension revolving around worth: ``is
this worth something?'', ``am I worth what I'm paid?'', ``what worth
would I gain by having more images of a fox-man I claim is me?''.
~Rather, it's likely more instructive to examine the ways in which money
aids and hinders social currency within the fandom.

The number one way in which having more money would aid one's social
standing is by being a party to the act of creation. ~The root concept
of a commission is that of two parties, the artist and the patron,
working together to create an item worth something by each contributing
something of worth. ~For the artist, this is their talent, skill, and
time; and for the patron, it is their ideas and character or characters
- the subject matter. ~Money changes hands, here, and social currency is
boosted. ~The purpose of the money is to offer something in exchange for
the patron's boost in social currency; the artist can create their own
by producing works that are not commissions, such as their own personal
art or art to sell in one form or another.

Perhaps a more simple example, however, is the convention. ~For a
convention, the attendee is willing to exchange money for social
interaction. ~Social interaction of any kind works into one's social
standing, and increasing the outlets and venues for that interaction
helps to diversify one's standing. ~It always helps to prove that one is
not simply some sort of program on the Internet, nor a meat popsicle
incapable of interacting with others.

Where does money hinder social currency, then? ~Well, one of the primary
ways in which the two oppose each other is the increased divisiveness
that is inherently part of a financial class-structure. ~The whole
rich/poor distinction can be taken on an individual basis and split
further into richer-than-me/poorer-than-me and does play a factor in our
lives no matter how much we intend to keep it at bay. ~Being able to
interact effectively across perceived financial boundaries is part of
learning to live within a hierarchy, after all. Within the fandom, this
shows its face in myriad ways: the artist who takes on several
inexpensive commissions to make rent, the fan who overspends in order to
be able to attend a convention, or even the aforementioned comments on
commission posts about not having enough money and the wrangles that
ensue.

Beyond that,however, financial and social currency do not map exactly
onto each other. ~That is, a monetary expenditure is not correlated one
hundred percent with a social currency gain. ~At times, it can seem to
be the opposite - when one first gets a commission from an artist of
some renown, the number of page-views skyrockets, new faves, new
watches, and new comments all seem to come in a flood. ~However,
comparing that with the faves, comments, and views of some other
commissioners, even of the renowned artist's post of the same image, can
be a little disheartening. ~It's in our nature to compare, as was
mentioned, and noting that our own meager following seem to be the only
ones appreciating our post as compared to that of the artist shows what
appears to be a disparity in gain: we gained our social status through
our financial contribution, and it's up to us to ensure that the gain
was worth the money we spent.

This division of worth is a complex and difficult one to understand, of
course, and I know that I am oversimplifying greatly here by leaving out
aspects such as personal and aesthetic worth gained from things such as
commissioned art and fursuits, not to mention the intensely personal
gain experienced from seeing a loved one at a convention felt by many.
~However, it was enough to broach the subject: money is one of those
strangely simple ideas that has grown strangely complex ancillary
meanings over time, and the concept is not made any simpler by pitting
it against the nuances of social standing and currency that are so
important within our subculture.

There is still room to explore, of course. ~Without spoiling too much of
what I have planned, I would like to explore both the concept of
business and its interaction with our subculture - whether it's a furry
business or a non-furry business targeting furries - as well as more
from the creator's side of the trade, and what all it means to take
money in order to produce a representation of someone else's character.
~An exchange, whether of trust and social standing or of simple monetary
funds, is a complicated thing, and we are continually carving out our
own niche, making our own markets, and coming across our own problems in
that arena.
