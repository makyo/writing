\chapter*{Acknowledgements}

As with any written work, there are many to thank who helped along the way. Thanks to Rob Baird for the boundless optimism and flattering words, as well as the inspiration from his Cannon Shoals stories; to Zeta for all her heart; and to JC and Robin for their help on reading and editing. Thanks, also, to Thurston Howl Publications, Sofawolf Press, and Red Ferret Press for working with me on three of these stories.

And thanks, of course, to the polycule. Just as each story here explores a different theme meaningful to me, each of you fills a different role in my life and make me all the more complete.

\chapter*{About Sawtooth}

\vspace{0.4cm}

Sawtooth, Idaho is a city slightly larger than Twin Falls, with perhaps 50,000 residents. As a university town, much of the younger population is only living there temporarily, but it leads to the core of the town being fairly well-off. The University of Idaho, Sawtooth itself is the largest employer in the city, followed by the tech industry, and then manufacturing.

This is a world populated by anthropomorphic animals. As such, there are certain rules that apply to the universe to maintain consistency.

There is no restriction on interspecies relationships, though it generally falls under the "queer" umbrella in-universe, perhaps as a holdover from when the ability to bear children was more important. There is no hybridization beyond what might be expected with animals in our world. That is, the offspring of a dog and a coyote would be a coydog, but a cat and a mink would not be able to reproduce.

Queer relationships and identities beyond species, while still thought of as minority identities, are more widely accepted due in part to this, certainly moreso than in America at time of writing.

``Domesticated'' species (housecats, dogs, etc.) were, historically, a sign of noble birth. However, it has largely become discredited as anything beyond selective breeding from centuries past.

Sawtooth natives are Idaho native species. There are no ``fictional'' species (dragons, griffons, etc). While species plays a slight role in size, it is not extreme. For example, weasels would be shorter than wolves, but only by, say, a foot or so.

There are biological and anatomical considerations to be made: scent plays a larger role in the sensory landscape; anatomical differences in tails, genitalia, paws, etc. do exist; there are relatively few reptiles due to harsher winters; and so on.

There is no historical artificial origin story for multiple anthropomorphic species beyond parallel evolution. Species historically grouped together in clans, population exploded during the industrial revolution, and the concept of clans has largely been abandoned. Due to the evolutionary advantages of being able to digest high levels of protein and fat, only carnivores and omnivores have managed to evolve to this point.

The world of \emph{Sawtooth} is open, and I freely invite anyone to write stories of their own set in the town of Sawtooth or the universe as a whole, with no limits. Because of the license, I ask only that any use of the \emph{characters} in the stories be credited back to the original source.

Beyond that, the world is largely as it stands, including much of the same historical, literary, and geopolitical events. Should you be inclined to use this setting, feel free!

\chapter*{}

\begin{description}
    \item[\emph{The Fool}]
    --- Copyright \copyright\ 2017 Madison Scott-Clary\\
    \emph{The Fool} originally appeared as \emph{The First Step} in \emph{Arcana: A Tarot Anthology} edited by Madison Scott-Clary and published by Thurston Howl Publications.
    \item[\emph{Disappearance}]
    --- Copyright \copyright\ 2017 Madison Scott-Clary\\
    \emph{Disappearance} appeared in \emph{Hot Dish Vol. 3} published by Sofawolf Press.
    \item[\emph{Fisher}]
    --- Copyright \copyright\ 2019 Madison Scott-Clary
    \item[\emph{Centerpiece}]
    --- Copyright \copyright\ 2016 Madison Scott-Clary\\
    \emph{Centerpiece} appeared in \emph{Knotted --- A BDSM Anthology Vol. 2} published by Red Ferret Press
    \item[\emph{You're Gone}]
    --- Copyright \copyright\ 2018 Madison Scott-Clary
    \item[\emph{Overclassification}]
    --- Copyright \copyright\ 2017 Madison Scott-Clary
    \item[\emph{Acts of intent}]
    --- Copyright \copyright\ 2017 Madison Scott-Clary
    \item[\emph{Every Angel is Terrifying}]
    --- Copyright \copyright\ 2019 Madison Scott-Clary
    \item[\emph{What Remains of Yourself}]
    --- Copyright \copyright\ 2017 Madison Scott-Clary
    \item[\emph{A Theory of Attachment}]
    --- Copyright \copyright\ 2017 Madison Scott-Clary
\end{description}
\newpage
\thispagestyle{empty}
\null
\vfill

\noindent Madison Scott-Clary is a transgender writer, editor, and software engineer. She focuses on furry fiction and non-fiction, using that as a framework for exploring across genres. She has edited and written for [adjective][species] since 2011, and edited \emph{Arcana: A Tarot Anthology} for Thurston Howl Publications in 2017. She is the editor-in-chief of Hybrid Ink, LLC, a small publisher focused on thoughtful fiction, exploratory poetry, and creative non-fiction. She lives in the Pacific Northwest with her cat and two dogs, as well as her husband, who is also a dog.

\begin{center}
    www.makyo.ink

    www.hybrid.ink
\end{center}

\vfill
