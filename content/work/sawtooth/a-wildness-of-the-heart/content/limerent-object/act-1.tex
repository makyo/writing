\chapter{Limerent Object}

\section{}

I wrap emotions in the cool embrace of jargon to soften sharp edges and take the sting out of ones I feel too keenly. It's why I got into this field. It's why I studied what I did. Of course I care for my patients, and of course I love what I do, but my reason for being here, for being a psychologist, is a simple insatiable need to explain away my emotions.

I've talked about it with my therapist at length\footnote{We all have them, therapist-therapists. I would never trust a therapist who does not.}. We talk about my need to hide behind words as a way of reducing my vulnerability. They become armor, when taken in this sense.

There's a tension, then, between these two explanations: to put it the way I did at the beginning is to allow words to be a useful tool to define the edges of my emotions and perhaps make them easier to digest and understand in the process.

To hear Jeremy's suggestion, though, my words are a means by which I might reduce my responsibility to actually feel the emotions I try to define.

Thus me, sitting here on my lunch break, writing journal entries on the steno pad I use in sessions.

Despite the utility I know there to be within the act of journaling, something which I've recommended to countless patients of mine, it's never quite something that I've picked up for myself. I always felt like maybe I was supposed to do something \emph{more} than just write about what I had done during the day, so I'd go off onto long philosophical tangents like this, and then I'd start to feel guilty for not writing about what I'd done during the day. No matter what, it felt like I was doing something wrong, like I was incorrectly doing the thing I knew how to describe to those who looked to me for instruction.

When I'd brought this fact up to Jeremy, he laughed and called me a ``fucking nerd'' and then talked me through what we thought my goals should be:

\begin{itemize}
\tightlist
\item
  I should write about the feelings that I have during the day until I've finished a complete thought about them, and then stop.
\item
  If an event comes up, I should feel free to record it, but not feel obligated to.
\end{itemize}

It sounds almost simplistic, but I get what he's doing. He's giving me permission to write about feelings instead of actions, to not feel bad about, as he put it, ``leaning on {[}my{]} upbringing and the way {[}I{]} think in complete sentences and five paragraph essays''. Of course, along with that, he wants me to actually feel things and process them rather than just wrap them up with a label and set them on a shelf. Use the things that are my strong suits to bolster the things that I'm weak at.

So, what am I weak at?

I think I'm weak at processing my emotions in a way that feels like growth. I wrap them up in words and I try to talk my way through processing them, but it's a performative sort of vulnerability that doesn't lead to any growth. As a result, I wind up in one of two situations:

\begin{itemize}
\tightlist
\item
  I feel the same things over and over again with no change in how I process them; or
\item
  I feel new things that I've never felt before, and rather than try to understand them, I shove them off to the side.
\end{itemize}

The latter is something that I've been making some progress at recently, as I learn to improve my emotional literacy, but the former is a habit that I need to break.

To go along with this task, then, what are my strengths?

I think that Jeremy pinned them down quite neatly. I think that my upbringing, strict as it was, instilled in me a sense of duty - first to my parents, then to my school, then to God and my time in seminary, and now to my patients. Also, he is not wrong in joking about thinking in essays; being so bound up in language is a net positive in that it allows me to take what patients tell me and turn it into something actionable for them, just as it previously allowed me to take the scripture that I read and turn it around through hermeneutics and truly understand it in a way that simply living as a believer would not.

Alright, so how can I build up the weaker portion of myself with my strengths?

I think that the best way to put the goal is to use my language skills to journal about emotions so that I have a record. If I could study scripture and study psychology, then surely I can study my own notes and from there, learn and grow to handle my emotions in a more deliberate and constructive fashion, right? Basically, write what I feel and how I react so that next time I can react better.

All of this, however many hundreds of words, all because I told Jeremy that I think I have a crush on a girl and didn't know what to do about it.

Ah well, I suppose that this has already been therapeutic, in its own way. I have a task I can set for myself, and, knowing me, all I need to do is let my sense of duty loose on it and we'll see if it bears any fruit in the weeks and months to come.

\section{}

I have noticed over the years that we tend to place benches in the strangest of places. I noticed this at Saint John's, those years ago back in Minnesota. The placement of benches ought to be deliberate. There ought to be some sort of goal in putting them where we do. A bench placed in a part with a careful view across the grass, through the trees, down the street would be ideal. You could look at the kits playing in the grass, the trees moving in the breeze, down the bustling street. Instead, we place them facing buildings along sidewalks.

Or, here at work, we place them facing a parking lot. I know, of course, that this bench is here because it is intended to be a place to wait for someone to come pick you up in our car-ridden town. I \emph{know} this, and yet this bench feels so fantastically pointless. There is one in front of short-term parking which feels far more apt a place for such a thing, but no, perhaps that was not enough: this one is along the side of the building, facing that overflow portion of the lot that on some days sees no use at all.

There is this occasional fad among certain groups on the Internet, I've been told, of seeking out so-called liminal spaces. I think that the term is ill-fitting. Liminality has a very specific meaning. I do not think that many of the places described as ``liminal'' that show up on social media and forums on the 'net are liminal so much as abandoned and vaguely spooky. They are not a place between, they are not a place one transits, not a border. They are simply poorly lit or forgotten.

The important thing about liminality, though, is not that a place be forgotten and certainly that it not be in any way scary, but that it should slip and slide beneath your interest. Liminality requires some form of passing through, It needs to be a border that you cross or a place that you enter for the sole purpose of exiting. Abandoned shopping malls are not literal. A barn, canted awkwardly to the side with age, standing alone in a field is not liminal.

A parking lot is liminal. An airport is liminal. A drive-thru is liminal. These are the spaces that exist only to be traversed. They are the spaces where, should you get stuck in them, you will be struck by the unnerving quality of the experience. They are not places that you visit. They are places that, should you visit, you will feel unwelcome because they resist the idea of doing so. They push back at you, in some intangible way, and say: ``You are not meant to be here.''

I am stalling.

It's perhaps a little strange that I seem to get the most out of journaling during my lunch breaks. To me, it feels as though I ought to be doing something so personal and introspective back at home, rather than sitting out on that awkwardly-placed bench in front of the office in that liminal parking lot, but there is something about the discomfort of that place combined with me already being in the therapeutic mindset that makes this the ideal situation.

I am stalling, though, because I know that it is easier for me to get caught up in words than to actually do the work at hand. Perhaps I am in the mind of liminality because this idea of liking someone, of wanting to pursue a relationship, is so new to me.

I have long since acknowledged that, despite my ability to listen actively and to guide patients through therapy, I am insufferable. I do not mean to denigrate myself in this. It is a fact and I am comfortable with my role in life. I am autistic and comfortable with all that comes with that (indeed, it works to my advantage in my professional life as I work primarily with other autistic individuals). I have few friends outside of a professional context. I do not enjoy drinking. I am devoutly Catholic. I suspect, for some whom I met at university, even at the private school before that, that I am out of place for being so `low' a species in such lofty places as those, for such are the places for the cats and dogs of the world, not a coyote who has, in their mind, pried himself up from the blue-collar professions of his ancestors or some imagined poverty.

Along with all of this, however, has come with a necessary distance from romance and relationships. This is another thing that I am comfortable with. The celibacy that was in my future as a priest was not a thing that I was in any way uncomfortable with, and when I moved on from that life I saw no reason to change that. I do not enjoy the word `single', because that implies something `less than' in today's society. I am happy alone.

I have, at various points in life, picked up a romantic twinge, and when I do, I cherish it. I will sit with that feeling and enjoy it, and then I will put it up on some shelf within me to be a part of my life, and yet in some way apart from it.

It is not unlike praying in that sense: God is always a part of my life, and yet is apart from it. I do not subscribe to many of the modern evangelical takes on religion, wherein God is within you and Jesus in your heart, but something far more conservative and old-fashioned. God is beside me, perhaps. Above me. He is with me, but not within me.

Another way to look at this is perhaps that these feelings are embers, or the smoldering of paper that has not yet caught fire into a relationship. You can see the faint tint of red crawling along the fibers of the paper, and yes, I suppose that you could blow on it and coax it into something more, but better, for me, to watch it slowly consume the paper, enjoy the beauty of the ember and the delicacy of the papery ash it leaves behind, and then, once it has gone out, acknowledge that it has left me a new person.

That, however, is not what Kay has done. She has flared into my life as a bright spark. It is not the slow crawl of smolder along paper but the bright flash of magnesium caught fire. Unstoppable. Undousable. Inevitable.

This --- this and the fact that that we both have what sound like single letters for names, Kay and Dee --- is why I brought her up to Jeremy, this brightly burning light in my life that has suddenly claimed me. This feeling is new. It's novel. I have had what I had assumed `crushes' were before, but to be smitten is a very new feeling for me, one that I do not quite know how to approach.

Kay and I met during the last year of her undergrad and the first year of my graduate studies at UI Sawtooth. She had taken a job in the campus library to help pay her way through school, working in the interlibrary loan office, a service that I was starting to use more in earnest.

That's three years gone now, though, and that these feelings were not in place soon after we met clouded my judgement when I started to pick up so intense a set of emotions. When one feels a yearning that saps one's strength, one expects that this is to be fairytale-level pining. Love at first sight. Smitten by looks. Utterly taken with the ways in which one speaks.

But no, when I first met Kay, I had made a mental note that she was a conventionally attractive coyote, no-nonsense and to the point, a fastidious dresser, and almost frighteningly competent. I read in her some of the same facets of autism that I see within myself, and I suspect much of her quiet efficiency stemmed from the fact that she, like me, often found herself feeling insufferable. It has taken me training and practice to soften my voice, to understand expressions, postures, and the vocal tics that make up people. I feel myself to be an empathetic person, a fact which drove me first to ministry and then to psychology, but to actually connect with those around me on an individual basis took effort.

I freely admit that the ILL office was not necessarily the type of place where one focuses on exemplary customer service, but still, this did not seem to be something that Kay was interested in in the slightest. She was there to do her job, do it quickly, and do it well. After a few visits picking up and returning books\footnote{More than I needed, perhaps. I had access to ILL as long as I was a student, and I took fantastic advantage of it.}, I decided that I would try to befriend her and find out how much we had in common.

Was this some early expression of my feelings toward her? I do not know. I do not remember feeling in any way romantic toward her at the time, yet for me to deliberately seek friendship from someone was not a thing that I might otherwise have done. I do remember thinking at the time that had I asked her to talk over a coffee, that would have carried such connotations, so instead, the next time I had an order of books to pick up, I simply asked her major.

For some reason, I remember that she had been in the middle of typing something when I had asked, claws clicking on the keys, and that she had stopped and blinked rapidly at the screen, and I imagined thoughts crunching out of gear within her head.

``Music,'' she had said. ``Music composition, actually. Why do you ask?''

I shrugged. ``I don't know. I just always seem to wind up talking with you here, so I was wondering. You don't seem like one of the salaried employees.''

Her smile was wry as she replied, ``I'm not, no.''

I don't remember if we talked about anything else that day, and there were not any stand-out conversations over the next however many times I saw her in the office, though we soon started talking every time I came by and the few times I saw her in passing both in the library and on campus. At some point, we simply\ldots became friends. I do not know whether we would have done so without me having acted with the intent to do so. Perhaps we would have. I do not remember thinking about intent-of-friendship much after that first conversation, so perhaps all it took was that opening question.

We slid effortlessly into a routine of weekly lunches. I went to a few concerts with her, though she knew far more about the music being played than I and I often felt in over my head as we listened to the instrumentalists on stage. I was surprised to find on the first concert that she wore earplugs throughout. I did not find the music to be too loud, some string quartet, perhaps, but she explained to me that it kept her from getting overwhelmed.

At the end of her time at UI Sawtooth, I had the chance to attend her senior recital, where several other students from the various departments performed a few short compositions of hers. The music was cerebral and, to my ears, dissonant, even dark, but it was as fastidious as her in a way that I cannot explain. I applauded heartily and after the show we hugged and she invited me out to drinks with her family, who all proved quite friendly and much like her. Thinking back, I suspect that must have made quite the sight: four coyotes sitting around a table at a fairly nice restaurant, speaking in essays to expound on whatever thesis has come into their heads.

Spending time with other autistic folks was not a strange occurrence to me, as I had known a few in university and had of course met several in my training, but for some reason, that night was the first time I could say that I felt comfortable in that portion of my identity. I felt at home with others, and, strange as it seems to say, rather like a member of their family.

My lunch break is nearing its end, out here in the liminal lot, so I should probably hold off from writing any more, but I should note before I do that it \emph{is} interesting that much of what I describe here in retrospect bespeaks an early attraction that I had not at the time attributed to budding romance or anything so grand.

Perhaps it was, in the end.

\section{}

I had a sudden memory today, during my final session of the day.

I was seeing a young wolf who was struggling with getting settled in his degree at university and feeling guilty for not taking to it right off the bat, and I had a sudden recollection about leaving Saint John's, all those years ago. It was strange to feel that sudden demanding of attention come over me, that sudden rush that required me to think about the past \emph{right now}, with no recourse to avoid it. It led to an unintended silence between me and my patient.

``So, I don't know,'' he said after the silence grew uncomfortable. ``What do you think?''

I don't remember quite how I came to the decision to share, but I suppose I must have weighed my options and figured it might be worth it to bring the memory into the conversation. Perhaps I thought it would be a piton of shared experience on which to hang some form of step forward for him.

I leaned back in the chair and adopted what I imagine was a thoughtful expression. ``You know, back before I decided to go into social work, I went to school in order to become a priest. Catholic priests generally get their masters in divinity, which includes a ton of theology --- something I really loved --- and psychology, but also the practical aspects of ministry. I was great at the first two, but the third, well\ldots{}'' I trailed off and gestured at myself.

The wolf cocked his head.

``It's difficult to be balance being the leader of a congregation with being an awkward mess in social situations.''

He laughed. ``Okay, yeah, I can see that.''

``So, they were nice about it, and we weighed our options,'' I continued. ``One option was for me to switch from an MDiv degree to getting a degree in theology, which would be all that intellectual research and navel-gazing that I loved without the pastoral care I was simply not cut out for.''

``But you didn't,'' he hazarded, gesturing at my degree on the wall. \emph{Master of Social Work, Univerity of Idaho, Sawtooth.}

``I didn't, no. Another option that we discussed was me transferring out of the program and into something else. I opted for that.''

``How, though? Why? Like\ldots why would you choose that over the other option or toughing it out in your divinity stuff?''

I shrugged. ``It was complicated.''

``You're telling me,'' he said, rolling his eyes.

``I know you don't share my faith, but those are the terms I think of that conversation happening in, so you'll have to forgive me for using them.''

He nodded for me to continue.

``There is a process called discernment in these situations, wherein you do your best to discern God's plan for you in life. Do you go on to get married? Do you go into monastic life? Ministerial life? Hermitage?''

``And you chose married life?''

My mind flashed to Kay and I held back a wince. I laughed to hide my discomfort. ``I chose helping others, which is why I'm here, but God's plan for me was to do so on an emotional level rather than a spiritual one, one on one rather than in a congregation. It's easier for me to be genuine one on one with someone than it is for me to relate to a group. It's easier for me to have a conversation and work out problems together than it is for me to teach and preach. My advisor agreed, and said, \emph{God needs saints more than he needs priests,} which stuck with me.''

He nodded, and I could sense a hint of impatience in the gesture. ``I don't have God to lean on, though. I'm not sure I believe in any overarching plan for me to reach a goal or whatever.''

I shook my head. ``No, I understand, that's just the language I was using at the time, but the act of discernment is what I'm getting at. It's not a one-and-done deal. At least, not for most of us normal, imperfect folks. It's an ongoing conversation we have with ourselves about what's important to us and how we get where we want to be.''

At that, he relaxed. ``Oh, yeah, I get that. Like, I went into school thinking it'd be great to do chemistry and get into, I dunno, materials science or something, and I'm struggling with that. I guess a better way to think of it would be something like, uh\ldots I guess I'm having this conversation--''

``Or you need to,'' I interjected. ``The discomfort may be a sign that the conversation isn't over yet.''

``Yeah, I need to continue to work on this process of nailing down what it is I want to do.''

I nodded. ``And one thing that falls out of that is that you're going to learn more from yourself and maybe things change. There's no harm in them changing, you're just getting new data from yourself about it.''

He brightened up. ``Yeah. Yeah! I like that.''

I like him. I like all of my patients, of course, but he's a good kid, and far smarter than I was at his age.

I was just so sure of myself, back then. I was positive that I wanted to get my BA, learn Greek, then my MDiv, then head back here to Sawtooth and start right away in the ministry. My parents were also incredibly pleased with this decision, if decision it was. It felt like a decision at the time, but now I'm not so sure.

Did I decide to do something that felt so self-evident? Was it just the path of least resistance? I remember when I began to struggle, when I decided leave the program, that conversation with God. I remember admitting it to myself, the confession the next morning, the meeting with Father Borenson, my advisor.

But was \emph{that} a decision? Was I giving responsibility to God for an action that I myself took?

These feelings of doubt have been cropping up more and more, recently. I do not doubt in God, but I am beginning to question my relationship with Him. Saying ``God knows what is best'' is an awfully handy way to absolve oneself from the responsibility for one's actions.

I know it's right for me to not be in ministry. I wouldn't make a good priest. I wouldn't be happy, and thus my congregation wouldn't be happy.

But I don't know if my path here, to this point in my life, has what's required to be called a decision. I wound up in secular life, but I wasn't thinking what that would entail. All I was picturing is that I would not be Father Kimana.

Now, here almost in my thirties, all of the decisions seem so much bigger, even if their impacts are smaller. That's not to say that pursuing Kay would be a small thing. It has the potential to be huge. It just doesn't have the change-your-life-in-an-instant quality that leaving Saint John's did. It would be a process. Admitting feelings, dating, marriage, children\ldots all decisions in and of themselves, all with the potential for failure, incomplete success, or mismatches in expectations.

I should go home and eat. I love my patients --- nerds, to the last --- and they always get me thinking, but lately, all this rumination\ldots{}

I should go home and eat.

\section{}

It is a Saturday today and I have no clients, so I am attempting to write at home rather than on a bench somewhere or slouched in my office chair, and am actually using my computer for it this time rather than scribbling on a steno pad. I have to admit that I feel very strange writing like this. It feels almost like a violation of a habit, despite having only been at this for a few days.

I have put some further thought into what I wrote about over the last two days, about the fact that there may have been some hints at romance or a crush or what-have-you prior to the time when Kay moved away.

I do not think that, at the time, I was thinking in terms of romance, and I also don't think that it was on Kay's mind either. Her parents may have been of the mind that we might have been going out with each other, but I do not know.

However, I am also not sure that my conscious self was entirely in line with my subconscious at the time. I speak now in retrospect, of course, and at the moment I know very well that they often float closer and further away from each other in terms of agreement, so I do wonder whether or not my subconscious was heading deeper into a desire for more than friendship.

This means that there are two possible scenarios to consider:

\begin{itemize}
\tightlist
\item
  My subconscious mind was starting to, as a client put it the other day, catch feelings, and thus the situation I find myself in now has a longer history than expected
\item
  The history behind this current set of emotions has some later starting point and the way in which Kay and I became friends has no bearing on the present other than as an interesting story.
\end{itemize}

If the former is the case, then I think it is worth some introspection as to what about our in-person interactions might have drawn me to her romantically. As I mentioned, she was frightfully smart. She was kind. She was not unattractive, either, and as a coyote, certainly someone who ought to have been in the market for me.\footnote{I know that many of the more liberal bent are increasingly okay with interspecies relationships, but, liberal as I try to be, my upbringing and my time within the church seem to have set me on the straight and narrow path, here.}

If the latter is the case, however, then I have to wonder why it is that such feelings did not form until distance became an issue, for less than a month after that dinner with Kay and her parents, she moved away from UI Sawtooth to prepare for her masters at UI Boise and our communication moved almost entirely to email and PostFast messages. I know that we tried to call once or twice, but neither of us is particularly keen on phones.

When I speak with my patients struggling with anxiety disorders, one of the exercises that I have them perform after a panic attack is to walk back to when the panic attack started and write down what they were doing and how they were feeling. Once they have done that a few times, they can look for similarities in the reports, and then they can start walking back further from the starting point of the attack in order to discover potential triggers. Knowing those, they can begin working on coping and avoidance mechanisms.

I know that I am trying to justify to myself my work on this journal so far, but I think that this retrospection is part of what I am doing with the project. I am not sure that I want to cope or avoid these feelings that I'm having, necessarily, but I do want to at least better understand when they began, and by understanding the past, better understand the present.

So, in that spirit, I think the first time I noticed this crush on Kay was perhaps six months ago. I remember having spent an evening talking with her about music, about which she has been slowly teaching me, both of us sending each other videos to watch and counting down from three so that we could hit play at the same time and talk about what we were both hearing as it happened.

After the conversation, I had gone to bed thinking that there was something about that particular interaction that felt oddly intimate to me, and when I lay in bed, instead of falling asleep quickly as usual, I spent a while thinking back to her senior recital and that hug that we shared after. In particular, I was thinking about the combination of the feeling of her cheekfur, soft and dry against my own, and her scent.

The room had had more than enough scent mitigation in place, and I know that the sort of non-scent of scent-block had a tendency to cling to fur a while after having been in a room where it had been layered on thick.

However, while the audience had been sitting still and watching the concert, Kay had been up on the stage for much of the performance, conducting, playing the piano, and speaking about the music she had written and I suspect that that combined with any nerves she may have felt prior to and during the performance must have had her a bit worked up, for she smelled more strongly than I'm sure I did.

I remember laying in bed, breathing shallowly as I tried to recall that scent in its most intricate details. My thoughts became fractal in my weariness and I found myself refining and refining my memories. Did she smell of exertion? Did she smell of cleanliness? Did she smell fresh? She smelled of all three, so what were the percentages of each within her scent as a whole.

I remember feeling a pang in my chest as I realized that I wanted to experience that again. That scent, the feeling of her cheek against mine. I wanted it desperately. I craved that moment, drawn out and extended.

I am no stranger to sexual fantasies. I have had them plenty in my life, and am not ashamed to admit that. Celibacy does not preclude one from having desires, and as long as they do not become covetous, God does not proscribe them. But the thing that sticks with me about this night of fantasizing is that there was nothing sexual about it. I did not fantasize about Kay and I some day having sex, of all the things we might do along those lines. Instead, I fantasized about hugging her, breathing deep, then leaning back and, for some reason, brushing my thumb over her cheek.

I don't know why, but that night, that act picked up a talismanic significance, as though were I to perform the ritual --- the hug, the breath, the brush through fur --- in precisely the correct way, I might somehow feel a light more intense than the sun wash through me, feel a rush of fulfillment, feel a sense of rightness and completion.

Finally, I remember praying. I remember speaking to God and holding in tension my words to Him and these feelings that I was having. I remember asking Him what this meant. What, O Lord, does it mean to desire fulfillment from another person? I do not want to possess them. I do not want to lay with them. I am not even sure that I love them. I just want to be happy with them, want them to be happy, and yet in such a specific way. What does it mean?

The little voice through which God speaks was silent. I was not surprised --- the domain of God's works are not the petty interpersonal relationships between individuals but rather whether or not their lives are lived in grace, and whether or not they strive to bring grace to the world around them.

I was not surprised, but I was, admittedly, disappointed. I try not to be disappointed in the ways of the Lord, of course. It's not His job to solve my problems, and to expect him to do so is silly.

I perhaps just wanted some guidance.

\section{}

If I assume that there is some subconscious root of my rising feelings toward Kay, and if I am to continue working backwards from a known starting point, then let me step back from that first recognition of those feelings.

We have countless hours of conversations over PostFast and email. We have fallen into the habit of at least saying hi to each other once a day. Sometimes, that is all the conversation that we have and the rest of our days will be eaten up by work and study, by reading and hobbies; and sometimes we will spend entire evenings talking, listening to music or watching comfort videos in the background while we engage on a more constant level.

Before we both wound up on PF, though, we had been emailing back and forth. We still do, on occasion, for when thoughts require something less immediate, something more structured than instant messaging\footnote{I have sometimes considered why this might be the case, and I have two main thoughts on the issue. The first is that email allows for threaded conversations. One can respond to a particular email, perhaps even after the conversation has continued from beyond that point. This also allows one to reply inline, even, interjecting thoughts between points one's interlocutor has made. The second is that as a self-advertised ``mobile first'' application, PF limits the width of the text per message to what might fit on a phone screen, even when using their desktop application, and something about reading a very narrow, very long block of text feels like a misuse of the medium.}. Sending each other essays and bulleted lists and long quotations that we have found interesting.

I had planned to dig back through those conversations for my Saturday afternoon task, hunting for hints of yearning among however many thousands of words we've shared. But, as happens, I got caught up in the business of the day. I wrote that entry earlier full on planning this, and then I remembered I had to vacuum the last remnants of winter coat from the floor. Having vacuumed, I figured I might as well use that momentum to clean the kitchen, and while there, I remembered that I needed to cook for the week.

Not all plans were made to be followed to the T, though.

Instead of sitting down at the computer and digging and rereading and reliving --- or attempting to --- I set myself to mindless tasks through which I could live in memory, instead. I thought back instead of read back, and I did my best to put words to my feelings at the time.

Kay and I's first lunch together was an accidental affair. During that final semester, I was spending my afternoons sitting in on sessions and, towards the end, holding supervised sessions of my own. I learned early on that a lack of calories in my system would lead to irritability and an increased difficulty in masking for the sake of my patients, so I began leaving my final seminar and heading straight for the student union for lunch before my first sessions began.

The food there was not great. You grow up on a farm in the northwest and you get used to a certain type of food. Sure, there are plenty of steaks and burgers at home, but you also have a healthy selection of homegrown produce and homemade canned goods. There is little enough profit in the industry for family farms, so my parents saved money where they could by growing what they were able to.

The student union, though, had four restaurants. A burger joint, a bagel shop, a soup-and-salad place, and a Mexican restaurant, all of them chains. The soup-and-salad place was my go-to, most days: they were the most likely to have an interesting selection on a day-to-day basis, they were the most likely to have vegetables other than shredded lettuce, and they were the least likely to leave me with an upset stomach later on in the afternoon, even if they were also the most expensive.

I smile to think back on the sheer number of combo meals I ate there. Half salad --- usually Caesar --- cup of soup, and square of focaccia, all arranged neatly on a tray. Few of the soups were memorable, of course, but almost none of them were bad. I was willing to accept ``consistently okay'' food.

I was waiting in line, lost in thought, watching the woman on the other side of the counter scoop lettuce and croutons into a bowl where it would be tossed with dressing, when Kay sidled up behind me and said, ``Hey, Dee.''

I will admit that the context shift of seeing her outside of the library initially caught me off guard. Always, I had been standing before a counter waiting on one of the employees to fetch my books off the shelf. Always, there had been a barrier between us, a requisite space that kept us apart.

Now, though, she was right behind me, standing closer than any counter would have permitted in the past. I hesitate to say that I didn't recognize her out of this context, for her voice was still the same and I could easily put voice to name in my head, but it took a few seconds for it to sink in that, hey, this was Kay. We had talked. We knew each other.

I had known that she was shorter than I, but I hadn't realized just how much. I could see over the top of her head between her ears. I also hadn't noticed her scent before, at least not to this extent. The library was full of the scents of others, despite the open spaces and constant air circulation, so it was far more difficult to pick out an individual's scent over any other's. Now, it was far more distinct, closer, more present.

It was not unpleasant, of course. She smelled of coyote and femininity and slowly fading scent block. There was no scent of stress, either, something I hadn't noticed had always been present in the library until its absence here.\footnote{I freely acknowledge that not all have the attraction to libraries that I do, and that the stress-scent I had been experiencing there could just as easily have been something more universally ambient than related to Kay herself.}

I realized I'd been staring and snapped to attention. ``Kay, hey, sorry. Long morning. Lunch break for you, too?''

She nodded. ``Yep. Theory classes first thing in the morning, then work, then techniques in the afternoon. I steal lunch when I can.''

``I've not seen you come through here before,'' I said, handing over my card to the cashier. ``Don't think I've ever seen you outside the library, come to think of it.''

She shrugged. ``Forgot my lunch.''

I waited for her to pay and pick up her own tray of food. I remember, for some reason, that she had ordered a full salad with strips of chicken on top.

I also remember that there was no discussion of us sitting at the same table and eating together. This was unusual for me. I struggle to eat around others without feeling hypervigilant over how I must appear to others. Too many frowns for chewing too loud, too many admonitions to slow down. That I would just walk over to a table with someone and share a meal with them without thinking was a strangeness that struck me only after the fact.

We talked a little, though I've largely forgotten about what. I remember asking what techniques classes were, and I remember she asked me what I did for work, but the rest must have been small talk that slipped from our minds.

All I remember is the not-unpleasant sensation of seeing something out of place. Kay belonged in the library. That was the context in which she fit most easily. That she might exist outside, might have a life, might actually be a real person, with real hopes, real dreams, the very real need to eat added depth to her, and while, on thinking back, I'm sure there was no early hint of a crush, there was no small amount of pride in the small success of seemingly made a friend after setting my mind to the matter.

\section{}

I took Sunday off to focus on church, but I have two things of note today:

The first is that I typed up and sent the previous entries that I have written to Jeremy. I will include his full response to me here:

\begin{quote}
Dee

Good to hear from you, man! I applaud the work that you've done so far here. I know that it can be really hard to buckle down and get to the actual work of parsing your feelings, but this is really great stuff. I like that you are using the journal entries to get out some of your current feelings that don't just surround this crush, though I also like that you call yourself out on stalling. You have talked before about struggling with emotional literacy, and I have to say, I think you're doing a stellar job of improving on your skills. Keep up the good work and try to employ more of that vocabulary where possible.

One thing I do want to mention however, and don't take this as a knock about what you've got down already, is that I think a great next step would be for you to tackle what it is that you're feeling \emph{now}. You've told a really coherent tale of how you got here, and now it's important that you focus on what you're feeling at the moment. Pry at some of those threads and follow them to see where they go. Here are some questions to get you started:

\begin{itemize}
\tightlist
\item
  You mention your feelings on God not providing you the guidance that you wish you had. I here you, and I know that can be frustrating. Perhaps one thing you could look into is your own response to your feelings on Kay within the context of your spirituality. Do you your beliefs influence your thoughts on her? Do you feel that being a spiritual person has an effect on your relationship to her?
\item
  When last we spoke, you mentioned that you weren't sure that these feelings were ``real''. What do ``real feelings'' mean to you? What quality keeps these feelings from being ``real''?
\item
  From the outside, you seem stuck. You don't seem to want to push for something more between you and Kay, and you certainly don't want to pull back from her. The next step in this project should be to find actionable paths forward. Why don't you start by simply enumerating options. What could moving forward look like? What might stepping back look like?
\end{itemize}

Seriously buddy, this is really great stuff. Not usually what I see in journals, but you've always been a heck of a writer.

Remember to breathe!

Jeremy

This electronic mail message and all attachments may contain confidential information belonging to the sender or the intended recipient. This information is intended ONLY for the use of the individual or entity named above. If you are not the intended recipient, you are hereby notified that any disclosure, copying, distributing (electronic or otherwise), forwarding or taking any action in reliance on the contents of the information is strictly prohibited. If you have received this electronic transmission in error, please immediately notify the sender by telephone, facsimile or email and delete the information from your computer.
\end{quote}

Jeremy brings up a lot of very good points and I will admit that I am both pleased that he has recognized the work that I have done already (and that he apparently enjoyed my writing) and also a little frustrated that I still have so far to go. However, I recognize that the latter sensation there is a fallacy and the result of me not being mindful and in the moment. The mind is ever drawn to conclusions and finales, when in reality, this is a process, not an end goal to be achieved. He is very careful in his writing there, in that he specifies that I should ``see where they go'' and ``what could moving forward look like''. This is very process-oriented language, and something that I would do well to engage with, myself.

I can also tell that he is gently nudging me away from being hung up on the past. I know that that hug with Kay has stuck with me, and that I have done a very good job on latching onto moments when we have gotten particularly close or that she has shown me a level or quality of attention that has felt particularly fulfilling. It is important to have good memories, but it is also important to not stagnate.

The second item of note is that I had a dream about Kay last night.

Dreams are such nothing things. At best, they represent a means by which the unconscious mind adapts to stressors in order to build up defense mechanisms, and at worse they represent random firings of neurons in the sleeping brain --- neurons that perhaps fired rather a lot during the day.

Dreams are such nothing things, and yet to them we pin so much meaning.

I dreamed that Kay and I were back at her senior recital, except that she was sitting next to me in the audience instead of up on stage, and we were watching her works being performed together. We were silent, rapt. The whole audience was rapt. The works were of breathtaking beauty\footnote{Not that they weren't very good at the time, of course, though they were certainly beyond my ability as an active listener, and beauty often seemed not to be the goal. She tried to teach me about them, once, but we are not built the same.} and when they were finished, the applause was so uproarious that she was not able to make it back up to the stage to take her bow. I tried to help her, but she got separated from me and was drawn off.

She did not seem displeased by this, however when I called after her, I, as in so many other dreams, dreams I'm sure we \emph{all} have, had no voice. I was barely able to manage a whisper, and my muscles grew so weak and my limbs so heavy that I fell over and that's when I woke up.

Powerlessness, separation, falling, these are all common features in dreams, and yet I am pretty firmly in the school of dream interpretation being largely bunk. Sleep is a protective action for the body, and dreaming is just the same for the mind. It is unguided, and serves to provide a break from taxing both our physical and our mental forms.

But we are so hard-wired to read deeper meanings into the mindless mutterings of countless neurons. ``What does it mean that she was sitting next to me? Does it mean anything in particular that we were separated from each other? Why did I become so weak without her presence?'' I am Nebuchadnezzar seeking my Daniel, not the other way around. There is no one to interpret my Mene, Tekel, and Peres but myself.

Some part of me craves answers to these questions and so many more, but there are no answers to be had because they are non-questions. They are questions one might ask the sky supposing only that that is where God resides.

The writing on the wall. Hah! Dreaming of someone that you have a crush on means absolutely nothing, and yet it certainly feels like it must mean \emph{something.} It has left me spinning with so much to think about and a lot to feel whether I want to or not.

I did not dream again last night.
