\subsection{August 18, 2008}
\subsubsection*{The Background}
I feel oddly blank.  Distressingly so.

\subsubsection*{The Drawing}
Exploring the void with the Universal Waite\cite{tarotRWS}.

\begin{enumerate}
  \item Four of Pentacles.
  \item Death
  \item Seven of Pentacles
  \item Eight of Pentacles
  \item Judgement
  \item Two of Pentacles
  \item The High Priestess
  \item The Star
  \item Nine of Pentacles
  \item Eight of Swords
\end{enumerate}
\emph{[No Image]}
%\includegraphics{file}

\subsection*{The Reading}
I'm unhappy about feeling as blank as I am.  I drew cards aimlessly and
interpretted them as they came, letting the meanings come rather than
resorting to a book or even caring about what I consciously knew about
the cards.  No reversals or anything were used.  Just one card at a
time.

\subsubsection*{The First Card}
The Four of Pentacles strikes me as a smug card.  I know that it should
be seen as smug on the outside and hurt on the inside, but sometimes,
it's so hard to get past the little twist of the man's eyebrows, that
little bit of contentment.  Or the way he holds the Pentacle on his lap,
deliberately casual, with his chin resting on his hand and the other arm
in his lap, as if he wants you to notice it.  A man hiding behind his
money.

\subsubsection*{The Second Card}
Death, more than any other card, reminds me of the passage of time.
Nothing can stop that horse.  The king couldn't stop it, nor will the
cleric, the woman, nor the child.  The sun still shines, the river still
flows with the boat on it.  Death is inevitability.

\subsubsection*{The Third Card}
The Seven of Pentacles is a card of hard-earned rewards, but now I'm
noticing the slightly unhappy cast to the farmer's face, as if it wasn't
quite as much as he was expecting this year, or if the rewards aren't
quite up to his high standards.  It is working towards perfection but
never quite getting there - doing great things but driving yourself
always harder.

\subsubsection*{The Fourth Card}
In the Eight of Pentacles, I see a person doing what they love, having
retreated from society or societal norms to do so.  It is caring more
about your art than those who might appreciate it, and doing so because
it is your job, as ordained by the person you are, rather than any
authority.  Some people work at what they do because that is what they
must do.

\subsubsection*{The Fifth Card}
Judgement shows all those nameless, faceless people who follow an idea.
They are sheep, and happier for it.  It is hard to see this card as bad,
though, as one should never underestimate the power of people in groups
following an idea.  Still, it makes me feel like an outsider.

\subsubsection*{The Sixth Card}
The Two of Pentacles seems almost to be a hallucination at this point.
It is the view of work being done by an outsider who doesn't know what
is going on.  It is a verb, describing the action of an act so esoteric
it seems to be magic.  Even the ships seem perplexed.

\subsubsection*{The Seventh Card}
The High Priestess is guarding something, but you'll never get to know
what.  Her face is so composed, so blank and she's so still that not
even her expression or body language can give away the landscape behind
her, however breathtaking it might be.  She still holds the law in her
hand, so I'm afraid to ask sometimes.

\subsubsection*{The Eighth Card}
The Star works behind the scenes doing what must be done, even when
things fall apart.  It is the inner workings of the emotional mind that
keeps me going even when I'm crashing.  She is the inner alchemist that
churns and toils to keep the juices flowing, as it were, to keep the
bellows moving and the fire stoked in my mind and heart.

\subsubsection*{The Ninth Card}
The Nine of Pentacles is distraction.  Someone did the work and it's
there to appreciate, but only in a vague sense of the word, strolling
through the gardens and taking it in, while really you're thinking about
who did what with whom last week, or something to appreciate in a
picture in the office.  And the bird who may recognize any sort of
beauty has a hood on, kept from seeing what's really around for the
whims of another.

\subsubsection*{The Tenth Card}
The Eight of Swords is martyrdom, being bound to the stake of sorts.  It
is willingly giving yourself up to those hostile forces around you.  Or,
and I hate to think about it, it is being too weak and caving in to
those whims.  Doing what you're told because you haven't the strength to
do what you must.

\subsubsection*{Analysis - after the fact}
I had to wait until the next day to finish analyzing the cards ---
usually, I outline everything and take a day or two to add in the
details --- as soon as I drew the last card, the blankness of emotion,
of intellect, and of Self drove me to bed, leaving me to be enveloped
somewhere more comfortable than in front of my computer.  As I'm feeling
much different today, I shall try to finish this chapter.  My aim is to
recall last night as best as I can and provide an interpretation based
on what I wrote, and then go back through the cards and provide a new
interpretation using more traditional definitions.

The ten cards I drew and interpretted briefly last night outline a path.
Briefly, smug and content in my comfortable new major with my knowledge
and ease of composing, shown in the first card, changes little in the
second card.  No one cares at work, my music still isn't getting
performed, and I feel no closer to graduating.  The third card is partly
my realization of this and partly that part of me that wishes to excel,
needing to work harder; and the fourth card shows me about where I am
now: feeling the need to retreat a little in order to do what I like
doing, since it sometimes feels as if I've chosen a major that no one
sees the use of, least of all my parents.

The next several cards seem much more speculative.  Begining with the
Fifth card, the music world of today, particularly in education and
popular circles is shown as sheep following a leader, some composer (the
sixth card) working their magic and becoming successful by writing the
music that sells with good marketing rather than good musicianship.  The
seventh and eighth cards are likely what I would do when confronted with
such an individual.  While I work my own `magic' with my music, I seem
more to be incomprehensible, but, running my own publishing business, I
hold the law in my hands.  The eight card shows, in particular, my
realiziation of this, of trying to keep alive a portion of my art that
no longer serves the function of art, without an audience.  

The ninth card shows more of the world around me, of music and art in
general falling by the wayside, as it has in the past few years, to
background noise.  Music that one hears in restaurants and on the radio
just because it's there, not for the purposes of enjoyment or
enlightenment.  The tenth card is my big ``I told you so'' moment to the
world.  I left the education system to get away from the way this
country seems to embrace that background-noise ideal.  I tried to change
that with my publishing company and all I did was cause trouble.  Take
that.  Or else the second meaning comes into play and I wind up hurting
for money, writing that very music that I despise so much.

Reading strictly from the rather stupid subconscious seemed to be easier when pulling
definitions, but stringing together the meanings into a coherent
analysis was certainly much more difficult.  When I apply the conscious
intellect inherent in analysis of the cards, though, the meaning becomes
much more evident and easier to string together into a time-line without
having to stretch so much.  This proves my earlier thoughts, that tarot
requires all levels of consciousness in order to be used successfully.
As such, many of the meanings I pulled from inside are still accurate in
a way: the overall premise of the storyline changes, but the problem
remains the same, as that of the weakening world of art.

Whereas I mentioned I was smug and content in my musicianship, a more
traditional standpoint shows less about the musicianship than about the
musician.  I'm creating structure in this world through my art, but also
using it to turn inwards and as a protective shell for myself, in the
first card.  The second card traditionally means changed, and I chose
the meaning of inevitability.   The two go hand in hand, and perhaps
this is a signal of change to come, what with my growing business.
Perhaps my structures should be more productive than they are now.  The
original meaning of the third card is indeed a much happier one than
I've chosen, but along the same lines of completion.  I think that mine
still applies, however, the constant need to excel is not a bad need to
have --- the work is not yet done, and this is shown in the next card.
The work is never done.

Here the reading diverges much more from what I saw in the subconscious
drawings.  The fifth card shows a call to change from within (or,
rather, a call to recognize what change has already happened), and the
sixth card represents the balancing act that is life in general.  Rather
than the outside perspective I saw before, these two cards indicate a
rebalancing of my life.  While composition will still be very important
to me, I think I need to realize that I've found my calling in life more
in the publishing of great new works of music.  

The High Priestess represents passivity, possibly in excess.  I do still
believe that music and art are not currently being herded to a very
constructive destination by both artists and policy makers in this
country, but I think that now is a good time to work against that with
my own music and my company to help get the music of others out there in
the open.  If I hesitate too long, my work will be ineffectual against
this slow drift into mediocrity.  The Star as the eigth card, however,
does caution a pause to build my resources, to grow into this position
without rushing headlong into it.  The Star and The Tower often suggest
each other, and I see The Tower looming on either side of this card:
either from rushing the process or waiting too long, I'm liable to
destroy myself.

The final two cards show offer a bit of advice about the way things are
running.  First of all, the Nine of Pentacles as the ninth card suggests
the recognition of rewards.  This has multiple meanings in context --- I
must recongize the greatness of the composers who ask to publish through
my company, I must recognize the need for my company, and I must
recognize the importance of my customers, for they are literally my
financial reward.  However, I must be careful of oppression, as shown by
the final card.  I should avoid blindfolding myself and oppressing
myself by painting myself into a corner, as it were, and I must also
avoid adding that blindfold to others, both composers and customers.
Those other styles of music do have a place in our world, and I've been
good so far about sending artists interested in that to other venues
more willing to publish their works, but I will need to keep up with
that so that I don't discourage them.  I will also need to keep an open
mind about what I will be able to publish, lest I deprive the world of
that music, or the composers of their recognition.

These two readings were particularly helpful to me.  I learned the place
of each of the levels of consciousness in the Tarot workspace, as well
as my limitations in reading from the spot.  I hope that my skills as a
reader will improve through this experiment, and I also hope that what I
learn from my readings will remain applicable in my business throughout
my life, as this reading seems to have proved to me just how important
this idea is.  As a pertinent final note, my card for the day was the
Seven of Swords, cautioning leaping into things without a plan.  This
applies to both the business as discussed above, but also my method of
reading.  I did enjoy the looseness without a spread, but I should plan
to work harder on reading on all levels.

