This is just a quick follow-up with some further information about the
\href{http://adjectivespecies.com/2013/08/07/species-selection-and-character-creation/}{Species
Selection and Character Creation}~article posted last week. ~I normally
post on Wednesdays and I had an article that could have been scheduled
today, but with that article likely needing more space than this one and
the desire not to distract from it with a simple addendum, I figured I'd
swap the two days around and give tomorrow's~\emph{real} article its
time as the featured post!

Last Wednesday, even as the article was going live, I was packing up my
laptop for an afternoon at a coffee shop
(\href{http://www.alleycatcoffeehouse.com/}{The Alley Cat}, where the
phone is always answered with a personable ``\emph{meow!}'')~where I
would spend a few hours talking with the inimitable Klisoura about
furries and data. ~Among other topics (some of which will show up here
on {[}a{]}{[}s{]} quite soon), we poked around some of the species data
a little further, and found some more interesting facts. ~That, combined
with some input from others both on Twitter and FurAffinity, and some
volunteers in private communication, got me thinking that more
information is always better than less, and so here we go!

\subsubsection{Common Terms}\label{common-terms}

Over the process of exploring the data with Klisoura, we removed several
common words such as the name of the species, articles such as `a' or
`an', and so on. ~However, we left in many additional terms that showed
variation between species as they do help show the differences in the
ways in which people thought of their characters. ~A few of these words,
such as `love'/`loved' or `personality' show up on every chart, of
course, but at different rates, showing a stronger sense of, say,
personality alignment with one species, but with a greater sense of,
say, loving with another species.

However, this tends to hide some of the differences in responses that
show species perception rather than character perception due to their
relative prevalence. ~By removing these common words as well, we find
that the words associated with the stereotypes or perception of a
particular species are emphasized even further, and those differences
made plain. ~Check it out below!

\subsubsection{Additional Surveys, Visualization, and
Exploration}\label{additional-surveys-visualization-and-exploration}

The amount of data amassed is quite large. ~Current data sets include
the Furry Survey from 2009 until present (though we will not be
providing information from current data until the 2013 survey itself is
finished), the 2012 {[}adjective{]}{[}species{]} Census and Survey, and
all of the {[}a{]}{[}s{]} small polls and surveys, not to mention
aggregated data from other sources such as the IARP and other surveys,
and scrape-able data-sources which we have used
\href{http://vis.adjectivespecies.com/furrysurvey/extras/}{in the past}.

As I am fond of saying in the Exploring the Fandom Through Data panel*,
exploration is a cycle of sorts: from collection of data through
understanding, giving back, dialog, and back to data collection. ~This
is a big portion of that cycle. ~When we pull together data from the
various sources, that's a big part of the understanding stop of that
cycle, just as presenting visualizations such as the word-clouds is a
big part of the giving back portion. ~By presenting this data in a form
that shows some of the story behind it, we can start a dialog between
those who produce the results and those who consume them, which leads
right back to the beginning: collection. ~This, of course, is a fancy
way of saying, we invite comments and questions by posting these results
freely. ~More than that, we love the feedback, because that's what helps
drive us to ask new questions, explore new topics, and try to understand
more of our subculture.

We got several responses to the last post, and I think it would be good
to expose some of this process to all so that we can see what goes on in
this whole cycle.

\begin{itemize}
\tightlist
\item
  \textbf{I'd like to see X species/Why didn't you do X?} - We have data
  for several species, plus several write-in answers for additional
  species that were not available through the check-boxes. ~However, as
  the number of respondents nears one for each given species, two things
  can happen to the data: it can either get skewed wildly in
  inappropriate directions, or it can near the normal distribution of
  words within any given text. ~For example, if we were to take this
  here paragraph, we'd see a fairly normal distribution of words, with a
  slightly higher weight on `species', but nothing out of the norm.
  ~However, if you were to respond to your choice of species of ``fox''
  with ``fox fox fox fOX FOX FOX OH MAN I LOVE FOXES'', then, as you can
  see, the distribution is wildly skewed toward `fox'. ~This was the
  reason for us restricting data to the more popular species responses
  out there: we are more likely to see trends that might, in some way,
  represent those who respond with a given answer.\\
\item
  \textbf{This totally jives with why I chose X/I can't understand why
  people would answer in such a way!} - First of all, these are only
  general trends that express the reasons for choosing a species to
  represent oneself. ~The are hardly guides, and they often fall along
  social perceptions of the species in wider culture, outside of furry
  (thinking of wolves in a pack, speedy cheetahs, or cunning foxes is
  hardly out of the norm for western society). ~Secondly, did you take
  the \href{http://furrypoll.com}{Furry Survey}? If something seems
  missing, it could be your response!\\
\item
  \textbf{What about fandom perceptions that make species more
  appealing?}~- I mentioned in last week's article that there were what
  I termed ``self-reinforcing stereotypes'' associated with many
  species. ~For instance, Altivo mentions those who would choose fox,
  husky, or horse due solely for their perceived sexual role within the
  fandom. ~This is most assuredly worth an article of its own, but in
  brief, that is a difficult thing to measure both in the data as
  explored and also in the responses to the questions asked at the
  Species Selection and Character Creation panel. ~Needless to say, we
  haven't forgotten about fandom-specific stereotypes as a factor in
  selection, simply that the point of the article was to explore
  selection as a more general topic.\\
\item
  \textbf{Have you tried correlating against X?/What further things can
  be done with the data?} - This sentiment is perhaps best expressed by
  FA user \href{http://www.furaffinity.net/user/nexrad/}{NEXRAD} in
  their
  \href{http://www.furaffinity.net/view/11303171/\#cid:72968774}{comments}~on
  the Jackals/Coyotes post on FA. ~There is a lot -~\emph{a lot} - of
  data in all of the responses to the Furry Survey. ~In fact there are a
  stupefying number of data points in any~\emph{one} year of the survey!
  ~We can look for trends, such as we have done with the species, or
  model relationships based on correlations or clustering as was
  suggested. ~All of these are possible, but they take time and we are,
  for the most part, lay-critters doing the best we can outside our day
  jobs, and checking our work before sending it out into the world.
  ~(Additionally, {[}a{]}{[}s{]} has some restrictions that prevented
  the topic from being explored further in last week's article: we try
  to keep our articles at about 2,000 words or under to help with
  readability and comprehension, and so the best place for such work is
  in future articles, posts, and visualizations!)
\end{itemize}

Finally, we'd like to reiterate the sentiment that has been in place
with the Furry Survey for several years now. ~We do our best to present
a fairly solid breakdown of the information provided in the surveys, but
we welcome requests for larger data sets from other researchers in the
future. ~These aren't available for direct download currently, and will
take some time to anonymize and prepare, but they do exist, and the same
holds true as with ``more information'': more \emph{eyes} on that
information is always better!

\begin{center}\rule{0.5\linewidth}{\linethickness}\end{center}

* Which, if everything works out okay, I should be able to provide as an
updated recording soon. ~We have video and audio recordings from RMFC
this year, and if their quality is good enough, we'll pull them together
and put them up on Vimeo as we did
\href{https://vimeo.com/adjspecies}{last year}.
