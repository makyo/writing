One of my classmates in college was pursuing what I believe was a double
major in engineering and music composition. He was a pretty great guy,
at his most helpful when it came to the discussions on sound and
acoustics. He was also a huge nerd, but so were we all: we were the
first class to help get the composition department at the university up
and running, so we were the ones actually pushing to get the degree
program started - my nerdiness took the form of running the composition
lab.

For his junior recital, one of the two we were required to give
consisting entirely of pieces we composed, he performed an extended
three-movement piece for solo French Horn titled ``Journey To Arelle''.
It's one of those titles you have to say out loud to get the joke. The
song was a tone poem about what mental processes a character left to
idle on Word of Warcraft must go through when their player went off to
``deal with RL''.

The idea of RL - ``Real Life'' - in opposition to things furry is, I
think, an interesting and telling one. There's a lot to be said for
immersion when it comes to gaming, for sure, but many furries apply it
to much more than just an experience that can be had sitting at a
console. We're hardly the only ones, of course, but it helps in
understanding just how the fandom works to know that it occurs in a
context that is not always ``real life.''

Role-play in and of itself is usually set as an opposite to real life.
The idea of something in opposition to structured activities such as
role-play is not a new one; this is easily seen in the previous example,
of course. One is spending the time and effort to pretend to be this
character within the set bounds of the game, computer or otherwise, in
which that character ultimately resides. There is a literal role to play
of some other living (or perhaps undead) being, here, and to attend to
daily tasks that may be wildly out of character if not outright out of
period is certainly returning to ``real life''. There just isn't the
connection tying the two lives together, there.

The difference between a strict role-playing type scenario and furry,
however, is that furry has no rules, no objectives, and no canon. This
isn't to say that it can't, of course, as plenty of folk I know within
the fandom play furry-themed RPGs such as Ironclaw or Usagi Yojimbo, or
even appropriate not-strictly-furry games to their own uses, creating
new species to be used in, say, Star Wars themed pencil-and-paper
role-playing games.

Furry lacks a central story, though: there's no canon to guide us other
than the shared interest that ties us together. In our case, though we
often play the roles of our created or chosen characters in various
ways, from interacting with them in text-only chat rooms and MU*s to
commissioning artwork or dressing up in giant animal bags at
conventions, we don't have rules or story to separate out a perfectly
livable daily life as an animal person from a perfectly livable daily
life as someone pretending to be an animal person.

I think this shows that furry is something beyond just role-play: it's a
whole separate context, a separate life lived in opposition to what a
lot of people still think of as ``normal''. We incorporate role-play as
a tool rather than as some sole form of interaction. We live our lives
out as furries here and there, but for the large part, much of our
interaction within the fandom remains a form of escapism. Beyond that,
however, furry as a subculture is still seen by many both inside and
outside the fandom as an interest that's bizarre at best, downright
abnormal at worst.

This isn't an opinion held by just those outside, as I've said. The fact
that we maintain such a strict separation of concerns when it comes to
our shared affinity for anthropomorphized animals and day-to-day
interaction with those who don't share our interest shows our own
willingness to accept what we consider a normal life alongside the lives
we lead within our chosen subculture. It's willful and, as JM and I both
point out, hardly negative and not without utility. A sense of normalcy
pays off just as much as all that we gain by virtue of this
transgressive subculture.

This isn't the type of thing that furry is alone in creating. There are
other hobbies and lifestyles - especially the latter - which readily fit
into a separate context from everyday life. These are the types of
things where one might find oneself being reminded, ``don't cross the
streams''. The further something is from being regarded as a part of the
main-stream (you'll forgive the mixed metaphor, here), the more likely
it is to be seen as constructive when one prevents it from overlapping
with day-to-day life. Philately, while definitely a bookish and
stereotypically nerdy sort of hobby, is something one might freely talk
about with friends and coworkers outside the stamp-collecting
subculture. One's collection of firearms or bedroom proclivities rarely
mix well in so-called polite company without also being some sort of
transgression.

This holds especially true for lifestyles. In recent years, even in this
last year, being lesbian, gay, or bisexual has hardly entailed the same
amount of hiding a core part of oneself at work and with friends,
separating out a portion of life from what's considered normal by
society at large. This wasn't always the case, though, and it's humbling
to look back, as someone who grew up fitting more or less solidly into
one of those categories, and see how differently the world works today
in terms of ``crossing the streams''.

The interesting thing to consider with this analogy is the level of
choice involved in furry as compared to sexual orientation. I used the
term ``lifestyles'' intentionally above, though it's fallen out of favor
when referring to one's orientation, because of the fact that there
exists a significant portion of the furry world that lives furry,
identifies as furry, and feels that they don't necessarily have a choice
about doing so, much in the same way that many live gay, identify as
gay, and feel they don't have a choice in the matter. One can look at a
hobby from the outside and see it as something that someone chooses to
do and generally be correct about that, but not always. For some, those
often called lifestylers, it truly can be seen as something more akin to
an orientation or identity than a simple hobby, and thus be harder to
separate from every day contexts.

JM and I have both discussed the usefulness in both accepting and
rejecting a separate context for furry in our lives, depending on the
scenario, and I think this acceptance of our subculture as a
slightly-less-than-real life when stood up next to what so many of us
refer to as ``RL'' is worth taking a step back and looking at. It's
hardly a big thing, or an exciting thing, or a new thing, but it does
show the ways in which we differentiate furry from other things in our
lives, and even define the boundaries of what each of us considers to be
the furry fandom.
