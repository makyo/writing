Furry as a subculture may not be ``mainstream'', but neither is it
small. The fandom has grown by leaps and bounds over the last few
decades with expanding easy access to the Internet, the proliferation of
furmeets and conventions, and even just plain old word of mouth.
Estimates put the current size of furry at somewhere between
20,000-50,000. This is, of course, a very rough guess based on responses
to The Furry Survey and other polls out there, but even at this size,
we're talking about a good-sized town (Fort Collins, Colorado, where I
live, has about 70,000 people living in it, and about 25,000-30,000 of
them attend or are otherwise affiliated with Colorado State University,
so maybe we can guess at the size of a popular American university),
with one very important distinction. A city in America has a council and
a mayor, and belongs to a congressional district and a county, which fit
within a state, which fits within the country, which is part of several
overlapping groups of nations, all of which are (currently) stuck on one
world. It's as if much of our culture here comprises a series of nested
centralized forms of leadership and government. Even the university
analogy works similarly.

Furry~as a subculture, however, is almost completely decentralized. Many
of us meet up and talk on the Internet, where we share our art and
ideas, but many of us do not. Many of us meet up in person at furmeets,
conventions, or even unrelated events such as parades, but again, many
of us do not. The whole concept that ``many of us do, but many of us
don't'' is consistent across all of furry and can be applied to creating
art, role-playing, fursuiting, or most any activity that takes place
within the fandom. Given this decentralized and diverse fandom which
nonetheless holds itself together, how does the~concept of leadership
fit in?

The word ``leadership'' has a formal ring to it, but can be used to
describe any form of guided social interaction, however informal or
unintentional. In fact, one of the primary ways in which leadership is
shown within the fandom is that of small groups leading through their
own interactions. This way of leading by example is often a good source
for the spread of memes, ideas that pass from person to person. It's
almost a type of group-think at times, as after all, we're already
trained to think along similar lines, given that we're all generally
interested in this one larger trope of animal-people. There are those
with the social currency or visibility that can wind up leading these
trends within the fandom in their own way, however unintentionally.
Trends such as the rise in popularity of streaming artwork or
Your-Character-Here commissions, or trends in the music we listen to, or
even the ways in which fursuiters act (there was, for quite a while, a
swishy sort of ``fursuit walk'' that would cause the suit's tail to wag
which seems to have diminished in popularity over the last year or two).

Another similar form of leadership within our community is that of
incidental leadership, and this is primarily shown through the
intentional promuglation of ideas, which can take place through content
production or actual leadership within events such as furmeets or cons.
This can occasionally be bound up in the idea of popularity (a muddied
word if ever there was one), but that certainly isn't always the case.
This is, I think most visible within the area of visual arts, where
artists will influence styles that will persist and grow based on their
popularity, such as the paintings of Blotch or the fur detail in
Ruaidri's art. However, this extends far beyond that, and fursuits are
another place in which this is visible. A certain manufacturer's
fursuits may be readily picked out of a crowd, such as those made by One
Fur All Studios, or certain expressions may become more and more
popular, such as the ``Pixar Look'' or the sunken ``3-D eyes'' style.
All of these things point to the subtle leadership that goes along with
content creation, especially in a culture such as ours where it's not
only common but almost expected for such content to be published for the
widest possible audience on sites such as Weasyl, SoFurry, and
FurAffinity. Even {[}adjective{]}{[}species{]} could be said to fall
into this category, as it is our intent to publish our works in an
easily accessible way for the widest audience, even though we have no
intentional designs on leadership.

Finally, there are some instances where there are quite formally defined
leadership roles, whether it be the committees running conventions, or
site administrators and volunteers such as those that run FA or Wikifur.
These are the instances in which the leadership aspect gets closest to
actual governance, in that the board running a convention does so by
having each member fill a specific role, heading their own team of
volunteers, in order to accomplish a certain goal. The administration of
a content-hosting website faces similar challenges, often solving them
in similar ways: by delegating certain tasks to people in specific
leadership roles in order to accomplish a goal, such as content
moderation. These are pretty common and well established practices as
well, with few systems working in different ways - Reddit is a good
example of a content-hosting website that eschews leadership (for the
most part) in favor of quality-voting; Discourse, a forum, works
similarly, by letting users with more points do more in the way of
moderation. However, these examples of con boards and site admins are
very specific to their purpose and rarely escape beyond their bounds and
into the wider world; though to be sure, some leaders within these roles
also carry additional social status due to their roles within their
domains (viz, Samuel Conway or Dragoneer).

Is this bad? Having a decentralized subculture with a fluid sense of
leadership? I don't think so. It's certainly not just a furry thing, as
there are countless examples (just as there are countless
counter-examples) of groups of people such as ours being decentralized
with a fluid leadership. However, I think that it is central to our
identity as far as it can be, in such a decentralized group. How, then,
does it benefit us? That is, how does this affect our forward motion
with regards to change? That is a complicated sort of question to answer
(given how many words it took to ask!), but I think one worth looking
into. How is it that, given our lack of a sense of centralized
leadership, or even a cohesive\ldots{}well, anything, that we have
perceptible shifts in artistic styles, convention habits, or even the
shared interests or our new membership.

When it comes to art, we benefit from the lack of canon, the lack of a
need to utilize any particular set of characters, clothing, style, or
even content to any of our visual art. In a way, that seems to give us a
little too much freedom, in that ``overwhelming choice'' sort of way.
We're nothing if not inventive, though, and I think that there has been
a large increase in the amount of artists and the quality of the art
produced over time despite the fact that we have no guiding canon to
work within. Much the same goes for fursuits, and this is helped out
even further by the fact that many of the techniques and standards are
being created out of whole cloth by the makers within the fandom. Not
just the makers, either, as fursuit performance has changed in its own
right over time. Of course, writing benefits from this as well, given
the additional challenge of creating well-written furry works that are
truly pertinent and not just incidental - that is, not just a story
where the characters happen to be furries; something which has been
accomplished in increasingly wonderful variety over the years.

It's not just content creating that has changed, though, but our styles
of personal interaction, both online and off. The ideas of characters
have shifted in prominence due to the shift in online interaction from
that of the more purely art-based worlds of Yerf! and VCL, to the mix of
art- and social-based worlds of FA, SoFurry, and Weasyl, to the mostly
social networks of Twitter and the like. These are, of course,
generalizations, and certainly applicable outside of furry as the
Internet matures, but given how much of the fandom does take place
online as well as how many of us fit into the ``early adopters''
category, it's certainly affected us as well. The same could be said for
offline interaction, as the common and socially acceptable behaviors at
conventions (two things which don't always overlap). What is generally
recognized as a proper con-going attitude has changed with the
increased~prevalence~of conventions around the world and on just about
any given weekend.

There is a constant stream of new members to the subculture as more and
more people find furry through the Internet, through friends, or just
invent it independently on their own. For those who find it through
others, however, they are influenced immediately by their first
impression, gained from their acquaintance with someone experiencing the
whole of furry at a certain point in time. As these new folks join the
fandom, they also help steer it by adding weight to whatever drew them
to the fandom in the first place, and I think that this accounts for
some of the ways in which our culture grows. If you were to find the
fandom through, say, an artist, and thought of furry primarily as a
group of individuals who put prime importance on art, then that might be
your defining furry aspect. This is how it was for myself, and it took
me nearly ten years to really even understand the whole fursuiting thing
and why it was even a big deal. This sort of bias helps to reinforce and
further some of the aspects of our subculture. Sure, ``new talent'' is
joining the fandom, but so to is someone interested in a certain aspect
of it, adding their own weight and input to that area. We don't move
forward in the same direction all at once.

In reality, this is a large part of what furry is all about for a lot of
its constituents: the fact that the fandom is decentralized allows one
to make their own way, but we are not without social direction, given
our guiding interest in anthropomorphics and animals. It runs counter to
enough of what we face in day to day life that it's refreshing in a way,
for a good number of us, to be a part of something that doesn't quite
follow the same hierarchical strictures of so much of the rest of
society. It's a place where anyone can be in a leadership role without
necessarily needing to be a leader. Talking with others, producing
content, or even acting in a governing position of something such as a
con or website are all things that we can do here that, in their own
way, wind up giving back to our subculture and helping make it what it
is.
