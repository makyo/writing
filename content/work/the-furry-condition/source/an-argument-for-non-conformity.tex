Okay, so the title is a bit grandiose.

I want to address some of the ideas that JM's
\href{http://adjectivespecies.com/2013/06/17/an-argument-for-conformity/}{previous
article} brought up for me. It's a magnificent read about the ways in
which the mainstream can benefit those who participate, touching on
privilege, presentation, and what we do in private. JM and I seem to
come to a firm agreement that his articles are the more immediately
applicable, whereas I'm busy navel-gazing; furry does not occur in a
vacuum, though, so perhaps I ought to talk some more about the wider
social implications of furry.

As with anything that can be simply negated by adding `non-' or `ab-'
(you know, like
\href{http://en.wikipedia.org/wiki/Young_Frankenstein}{Abby-Normal}),
there are two sides to the coin, and more often than not, the
interaction between the two is hardly a simple binary, often involving
friction, and sometimes quite a lot at that.

As many readers can attest, there has been a wave of ``be yourself''
propaganda pushed on children and young adults in America over the last
thirty or so years, appeals to the sense of non-conformity that each of
us carries within us to some extent. Much of this, of course, was awful,
saccharine filler that served no purpose other than to make someone
money, and blanket non-conformity is hardly something I'd advise someone
to undertake. However, just as in the rest of the world, furry has
something to benefit from careful application of non-conformity.

Non-conformity and subculture have mixed for a long, long time. Anyone
who has been part of the goth scene, or the punk scene before it, or the
rock scene before that, or the jazz scene before that, knows this. These
are, of course, examples that take the idea of non-conformity and spread
it throughout the very interest that brings them together, turning it
into something of a fandom itself. Even beyond the idea of fandom,
though, non-conformity and its close cousin, transgression (an act that
goes beyond generally accepted boundaries), have served groups within
society as long as there has been society; one need only look to the
history of early Christianity to see that. Non-conformity and
transgression are hardly artifacts of modern western society.

There are, in fact, a lot of things about furry that can be seen as
transgressive, both within and outside of the fandom. Some minor
transgressions, acts that take place outside accepted boundaries, are
seen as core ore close to our subculture in many instances:
street-fursuiting, a propensity for collecting stuffed animals, or even
hanging tasteful furry art in the home or office
(\href{http://characters.openfurry.org/image/43}{these}
\href{http://characters.openfurry.org/image/92}{two} pieces grace our
walls right in the entryway, along with a ton of pictures of our dogs)
are just a few ways in which we can step into furry space in a non-furry
context, even if only a little bit. Minor transgressions, to be sure,
but it's easy to see the roots of transgressive behavior within our
fandom. What could be more non-conforming than not conforming to the
generally accepted species, after all?

This is, I believe, part of the reason for the relatively accepting
nature of furry as well. A group which is, in a way, transgressive at
its core is often a safe space for those with a stake in otherwise
transgressive behavior. This is more than just ``falling in with a bad
crowd'' - after all, we're not \emph{that} bad, are we? Rather, this
goes along with the idea of finding a safe space for oneself. A safe
space is, in some ways, a space in which one can engage in either
transgressive behavior or discuss, think about, or otherwise wax
metatextual without fear or repercussion, or at least in the hopes that
that's the case. This is the purpose of the
\href{http://en.wikipedia.org/wiki/Safe-space}{safe-space} signs in
schools, which serve this purpose in a subtler way, after all: in a
place where acknowledging LGBT issues positively might be seen as a
transgression, or at least a form of non-conformity, these signs show
that the educator is attempting to create a place free of that
association.

When it comes down to it, the ideas of non-conformity and transgression
serve an important role to minority identities. As
\href{http://theorts.tumblr.com/post/53262160482/invisibility-illegibility-thoughts-on-why}{this}
article bluntly puts it:

\begin{quote}
Queerness is not just about whom or how you fuck. It is also about not
being part of that mainstream culture, about being decidedly against
that mainstream culture. It is about disruption. It is about putting
things at risk.
\end{quote}

Of course, both that quote and my own words are incautious: minority
identity, and in this example, queerness, are generalizations used to
described trends in identity shared within a social group. I know there
are several individuals who would disagree. I have my own hesitancy,
here. There is an uncomfortable stage for some in the reclamation of a
word where it still carries some of its old connotation before the new
one has gained general acceptance. ``Queer'' is in that space for me,
because it still has its connotation of ``weirdness'', it still denotes
transgression. I'll hasten to add, though, that this is an ongoing
process, within myself even as it is within society at large. The word
``straight'', after all, has been largely accepted to simply imply
heterosexuality, despite its prior connotations of ``going straight'',
where homosexuality was seen as crooked or deviant (which has been
notably lamp-shaded by the movie
\href{http://en.wikipedia.org/wiki/Bent_(film)}{Bent}).

However, I think that the word ``queer'', and others like it, are
important in the sense that this sort of non-conformity is vital to
identity. When it comes to arguing identity (that is, discussing the
point with the goal of changing minds, not necessarily having shouting
arguments - though sometimes that too), it is advantageous for the
argument to be cast in one's own terms. When the argument from a
minority is cast in the terms of the majority, the minority often only
receives relatively small concessions, rather than recognition.
Transgressive language and non-conformity help to recast the argument so
that there is a greater likelihood of one's point being made
forcefully.*

While conformity is generally the province of the majority,
non-conformity is hardly detrimental to it. The culture of the majority
is a static behemoth, whose only purpose is to remain precisely where it
is, as it is. This is all well and good for those within the culture who
benefit from that stasis, but this isn't the case for everyone, and
often isn't even the case for the actual by-the-numbers majority of
individuals wrapped up in society. Minority culture and identity,
subversive and transgressive, have the job of pushing the majority
culture forward in such a way as to improve life for more and more of
those in society, attempting to break that stasis to benefit those
involved with their culture and identity. A lot of social progress that
humanity can claim comes from this tension and friction; the majority
promises safety, the minority promises progress. Both have a purpose.

So, let's tie this back to furry and the idea of conformity.

When it comes to JM's article, I really must stress that I
whole-heartedly agree with it. There is a lot to be gained in terms of
safety by conforming to the majority. One furthers one's standing within
that culture by not, say, wearing a collar to one's interview. This
helps in terms of personal progress: a better job, perhaps a greater
amount of respect from those around you, and yes, even the possibility
of using that progress towards one's goals within the fandom (EF2015
sounds like a good idea - JD's been talking about it for a while now).

Non-conformity is nothing to feel bad about, however. Neither is
conformity! Both have their purpose in our lives, and every single one
of us expresses both in some way or another at different times and in
different aspects of our social interaction for our own reasons. Even
furry. Transgressive acts such as street-fursuiting, publicly visible
gatherings such as conventions, and even talking about furry from a
critical theory standpoint on a publicly visible website have helped to
legitimize furry as an identity, a membership, a subculture. Conformity,
on the other hand, helps many the individual members of furry to keep
things moving forward by benefiting from what the majority has to offer
to those who go along with it.

\begin{center}\rule{0.5\linewidth}{\linethickness}\end{center}

* Note that this is a very reductive view on critical- and queer-theory,
topics very much worthy of their own post(s). I have to get to the point
somehow, though! If this sort of thing is interesting to you, I highly
suggest prowling around more: there's a ton out there.
