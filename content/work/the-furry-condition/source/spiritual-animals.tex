Spirituality is one of those slippery words that can be ridiculously
hard to pin down. ~ I've found that you can usually tell when one of
those is coming up by looking at the length of it's Wikipedia article,
as odd as that sounds. ~If the article can basically get right to the
point and then spends the rest of the time exploring fine details such
as history, examples, and important figures, then the topic is not
likely very complex to define. ~If it wanders down a long path, peppered
with links, is topped with a sidebar and tailed by a category
box\ldots{}well, needless to say that
\href{http://en.wikipedia.org/wiki/Spirituality}{Spirituality}'s
Wikipedia article is a prime example of a ``difficult topic''.

It really seems to come down to the fact that spirituality means
different things to different people, has to do with the search for
meaning in things that we don't understand and don't seem to be
explainable by science, and is self-referential: numinous things are
spiritual, spirituality has to do with numinous things. ~While my gut
instinct tells me that the concept of a spiritual fur has been on the
decline in recent years, I still see and hear mention of it quite
frequently, in some form or another. ~Us spiritual animals have rich
histories to draw on, adopt, and appropriate, not to mention the ones we
create for ourselves, and we seem to have done so with a will.

``Spiritual'' can be used to describe many things, and means many
different things to different people, of course. ~To some it's a way or
means of exploring issues or answering questions to which they do not
have an answer, and to others it's more of an adjective attached to
things that are inexplicable, and yet to others it's a state of being
they maintain throughout their lives. ~In general, though, it all seems
to have to do with meaning. ~I've gone
\href{http://adjectivespecies.com/2012/04/11/meaning-within-a-subculture-part-1/}{on}
(and
\href{http://adjectivespecies.com/2012/04/12/meaning-within-a-subculture-part-2/}{on}
and
\href{http://adjectivespecies.com/2012/04/13/meaning-within-a-subculture-part-3/}{on})
about the importance of meaning as it pertains to furries, and, as part
of my preparation for writing this article, I went back through my
notes. ~As I did so, it became clear that this fixation on meaning
involved with spirituality is thoroughly tangled up with furry. ~After
all, what would be more obvious as we investigate the meaning of
creating an avatar of ourselves as some other species than to consider
the spiritual side?

I must add the caveat that spirituality is by no means a universal with
those that identify or are interested in furries and anthropomorphics.
~In fact, atheists and agnostics seem to outnumber those who identify
with a particular spiritual path such as Christianity or paganism. ~This
is, of course, referring only to
\href{http://vis.adjectivespecies.com/furrysurvey/religion.html}{responses}
on a survey to a question utilizing the word `religion' rather than
spirituality, and in this respect, my gut feeling is that it's fairly
accurate. ~However, I do get the feeling that many who may have
responded with `atheist' or `agnostic' might still feel, in some way,
spiritual. ~I, for example, fall within the agnostic slice of that pie
(or, well, doughnut), yet still can't deny experiences that I could only
call spiritual, at least at the time. ~This, along with similar
sentiments held by several friends, is harder to quantify, yet still a
valid point to make: our reactions to the unknown and our explorations
of meaning go beyond simply the actions taken to explore those things,
touching also our emotional and intellectual outlook on life.

Of all of the spiritual influences within the furry fandom, two seem to
be far, far more prevalent than any others: the loose-weaved
generalization of ``Native American'' spirituality and some aspects of
Japanese spirituality. ~The former may well have been a product of the
'80s and '90s, when many of those who responded to Klisoura's survey
were born, and which featured, among other things, a spike or resurgence
of interest in Native American spirituality. ~The reason that I
mentioned this as a loose-weaved generalization and then put ``Native
American'' in quotes is that it is difficult to pin down ``Native
American'' spirituality to just one tribe; rather, it seems to be a
collection of influences from several North and South American tribes
(some notable ones being the Chippewa, from whence came much of the
writings on Totemism; many tribes more focused on shamanism as it's
traditionally described through central America, with a focus on Power
or Spirit Animals; and down into the South American continent, which
provides art and architecture from the Inca and the like). ~Many furries
who incorporate elements of these spiritual origins into their own lives
seem to do so because of the draw provided by the very concept of
Totemism: the fact that one might have a power animal, that one might
share aspects of that animal's personality or physical attributes, and
that one might draw personal or spiritual power from such a totem
provides a clear draw for those interested in anthropomorpics.

On the other side of the world, however, the Japanese have amassed a
large amount of folklore surrounding many different animals. ~The tales
that surround foxes and the native raccoon dogs (\emph{N. procyonoides})
in the most pertinent forms of \emph{kitsune}~and \emph{tanuki}~are
those that are most familiar to the western-dominated furry subculture.
~These two in particular, but other supernatural beings (\emph{yōkai})
related to animals such as the Beckoning Cat (\emph{maneki neko}) have
crept into western culture through various media outlets, and
specifically into the popular furry fandom through the crossover links
with the anime fandom. ~With their connotations of shapeshifting, of
being in a relationship, and of animals interacting with the world
around them in supernatural ways, it's unsurprising that the fandom
would draw much from these.

These, of course, are only two examples of the way spirituality and
folklore have influenced the furry fandom and woven ties deep into our
subculture, influencing everything from the ways we feel about our
connection to animals to something as simple (well, ``simple'') as
character creation. ~Many of the most popular species out there are
related in some way to a species that is important to at least one
culture in a spiritual way. ~Wolves have their legends in both North
America and Europe, horses have their adherents in~Scandinavia and
throughout Eurasia, foxes and coyotes have their trickster backgrounds
(not to mention jackals and many other such canids), and even kangaroos
have their own legends to go with them, not to mention the spirituality
that goes along with big cats all over the world.

\It seems that part of what draws us to the idea of anthropomorphism is
the meaning attached to an animal. ~Whether that means that an
individual is influenced in their character by the spiritual
associations or that their spiritual associations are influenced by
their subconscious choice of character likely varies by the individual,
but the important aspect seems to be that it adds intensity to the
choice. ~When one person elects to create a character of a fox, they may
do so because that species offers the intensity of meaning, that certain
``it just fits'' \emph{je ne sais quoi}~that helps to complete the
process of character creation. ~It's a powerful sensation, one supposes,
and just as often leads to a proliferation of characters in order to fit
all those intense moments in life, or one character locked down forever
that provides the best fit in all scenarios.

This is evident beyond just the spiritual associations that are attached
to certain species, though ``spiritual'' being such a difficult word to
pin down, that's a broad statement in itself. ~Many individuals may find
that intensity of meaning provided by the social connotations of species
that are not necessarily considered spiritual, in the traditional sense
of the word (though I should note that the Wikipedia page for
``tradition'' is nearly as complex as that for ``spirituality''). ~Dogs,
for instance, carry significance in the society beyond the legendary,
though many contemporary works have started to include some of that in
their status. Specifically, dogs seem to be drifting toward some
apotheosis of animal companionship, as evidenced by works such as
\emph{Shiloh}, \emph{Old Yeller}, \emph{Lassie}, \emph{Where the Red
Fern Grows}, and
\href{http://www.abebooks.com/books/famous-dog-novels-lassie-marley/dogs-fiction.shtml}{countless
others}. ~Dogs are only one example, however; house cats, race and work
horses, and many others all have built up their own social significance
that adds to the meaning of the character one creates.

The thing that got me thinking about this in the first place was a
hashtag that floats around twitter once a week: \#TMITuesday. ~It's
really no secret that people change throughout their lives around
adolescence, and I am no different. ~I have, on one of my bookshelves,
books that range from the Bible to the Quran, the Celestine Prophecy to
books on tarot cards (not to mention a modest collection of decks). ~I
was very, very much into the concept of spirituality, specifically the
introspective aspects of it (as if that wasn't obvious), and amassed
quite a collection~of materials related to that interest. ~My choice of
characters, then, was not mere consequence. ~As I was first getting into
the fandom, I began as a red fox, taking from the species many spiritual
aspects both learned and imagined. ~I created my character based around
the intensity of meaning surrounding a supposed slyness, a dash of
mystery, and a generous helping of playfulness that I gleaned from
outside sources and my own thoughts.

As time went on, that shifted toward arctic fox after sifting through
vague correlations in much the same way that I learned to read tarot
cards; I felt snarky, arctic foxes looked snarky stealing bits of food
from polar bears,~thus a correlation was demonstrated. ~Another example
was the way in which I changed with the seasons. ~What might be called
Seasonal Affective Disorder in others, I deemed in a hazy way a
correlation between the way the arctic fox's coat goes from a fluffy
white to a scraggly salt-and-pepper. ~Even as my interest in
spirituality waned over time, I still felt the need for that intensity.

Other species choices were much more, well, specious. ~I created a
wolverine character meant mostly to get different reactions in places I
frequented on MUCKs, and the whole otter thing was due mostly to wanting
to get a fursuit, but finding out that white fur can be hard to make
look how you want.~This intensity of meaning became evident in the
different ways I felt interacting as each of the characters in turn.
~There was something distinctly lacking from my interactions as a
wolverine and an otter, and making them ``mine'', as it were, took a
force of effort, rather than being a consequence of my selection, having
some sort of spiritual or social meaning behind their creation. ~I
failed with Happenstance, the wolverine character, and I succeeded only
through force of will (and money well spent on a fursuit) with Macchi,
the otter character.

In many bookstores, there is a certain area, usually just a shelf or
section of a shelf, sometimes an entire room dedicated to the act of the
practical, personal application of spiritual ideas. ~Many focus on
meditation practices, prayer, research, manipulation of certain objects,
or even diets and other practical matters. ~Others provide descriptions
and hint at exercises intended to guide one down their own exploratory
spiritual path rather than provide clear directions of one sort or
another. ~I prowled my way through this section often through at least
one period in my life. ~Many members of our subculture, and countless
more outside the contiguous fandom, whether they identify more with
therians, weres, some other subculture, or none at all, have found a way
to integrate many aspects of what is called spirituality readily into
their lives, however. ~We seem to have done well by ourselves in that
respect, making something as important as identification of a personal
spirit animal, totemic guide, or other spiritual-animal connection a
part of our day-to-day lives.

I know that this is a large topic, and I know that I have not done it
justice, due to my incomplete knowledge. ~I know, for example, that I
was unable to provide adequate words to the Totemism topic that is so
dear to many of my friends, and I deliberately skirted the topic of more
conventional, more organized religions on the grounds that I have very
little experience with such things, and don't know too much of how furry
interacts with the social aspect of spirituality as structured in
religion beyond a few conversations I've had with a very kind
{[}a{]}{[}s{]} reader. ~I
\href{http://adjectivespecies.com/2012/01/29/from-the-survey/}{know}
that many of you feel a spiritual connection with furry, and I invite
you to leave comments with your own stories, thoughts, and words on the
subject here, or, if such things are too personal and you still wish to
share, to email me at
\href{mailto:makyo@adjectivespecies.com}{\nolinkurl{makyo@adjectivespecies.com}}.
