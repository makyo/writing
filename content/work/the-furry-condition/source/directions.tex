And now comes the time when I have to write a very important, rather
personal article. Again.

I can see you all bracing yourselves already. ``Oh no, here goes
Makyo\ldots{}''

I actually feel kind of bad for starting the article out like that, to
be honest. Recently, my boss (and biggest ally at work) resigned by
calling us all into a meeting and announcing, ``So now is the time when
I announce you that I'm quitting'' and we all sat there in stunned
silence. Was he kidding? It was certainly in his style.

It's tough to have your most important personal ally leave through
something that could just as easily be taken as a joke. I'm sure I
wasn't the only one expecting him to start out that way and then launch
into some sort of deadpan ``ha ha just kidding here are the plans for
machine view we have to get it done by 14.04.''

I'm hardly leaving. I've invested quite a bit of time, no small amount
of money, and basically just as much of my furry life as I can in this
project - it means a lot to me. I think it's a fantastic outlet for
myself, and a pretty good resource for the community, providing both the
resources needed to investigate furry on many levels, as well as a place
for \href{http://adjectivespecies.com/contributing/}{any others to do
so}. This has allowed {[}a{]}{[}s{]} to become precisely what it has.

As I've
\href{http://adjectivespecies.com/2012/03/21/makyos-kaddish/}{mentioned
before}, I had few plans for {[}a{]}{[}s{]} other than to maybe just be
an outlet for me to kvetch about furry. I thought it would be a simple
little place for me to write down my thoughts about what bugged me, or
at least fascinated me, about this subculture. There's no small amount
of that, after all - no one version of any community is going to fit any
person 100\% given that a subculture is made up of \emph{at least} as
many versions as there are members.

Once I started poking around more, thinking more, and especially giving
the site its own personality rather than just some wordpress blog with a
bad crop of one of my own commissions at the top, things started to head
in their own way.

I was pleased. Often, in fiction writing as well as music composition,
the point at which a piece really starts to take off for the creator is
the point at which it starts to take on a life of its own. Not that it
violates the outline or plan or anything, just that characters start to
fill out the story and play their roles, the melodies start to take
their own inevitable paths toward resolutions (or not, as the case may
be).

One thing that I've noticed, however, is that the part of us that is
actively engaged in the process of creation, however subconscious it may
be, can surprise us. More than that, creation does not take place in a
vacuum. Rather, the process of creation is the product of not only
ourselves, but also the various levels of society in which we reside.
The creation that is {[}a{]}{[}s{]} is not just some isolated production
of Makyo's view of furry. That view of furry changes as I grow, and as
furry changes.

I grow and change because there are people in my life that come and go,
influences that wax and wane. When I started writing, I was dealing with
some of the troubles I was having with gender in my life by role-playing
as a mix-gendered character, hiding it from those around me because of
feared backlash. As I slowly opened up to myself and others, so to did
the circles in which I counted myself a part: I started talking more
with people with similar experiences and, though I'm a little sad to
admit it, lost touch with some of the absolutely delightful folk I met
online while in the beginnings of my exploration.

So of course {[}a{]}{[}s{]} is going to change. I'm going to change over
the process of two and a half years, there's no reason that the things
that I produce are, for some reason, \emph{not} going to change!

Keeping that in mind, take into account the fact that, over these last
few years, the site itself has grown to include a separate project, Love
- Sex - Fur, focused on the curious intersection of furry, romance, and
sexuality. It has grown to adopt the Furry Survey from one of our
initial contributors, Klisoura. It has grown to include not just myself
and Klisoura, but also JM, Kyell, Zik, Rabbit, Jakebe, and many guests
besides. All of these additions, most especially the additions of
written contributions help to shape the ways in which {[}a{]}{[}s{]}
grows. It's self-reinforcing, too, a type of feedback: the site attracts
a certain type of mind that writes in a certain way, which attracts more
of the same, building in strength and attracting more of the same until
it starts to refine itself.

\begin{center}\rule{0.5\linewidth}{\linethickness}\end{center}

These last few months have been pretty tough for me. The issue of mental
health crops up with some frequency here on
{[}adjective{]}{[}species{]}, but I think it's one worth covering, from
the point of view of introspection, and particularly due to furry as a
factor, as well as a utility in the recovery process.

I know it may perhaps seem routine at this point, but it wasn't
intentional that I take each winter off from {[}a{]}{[}s{]} and several
other projects as some sort of sabbatical. Rather, my genetic propensity
toward seasonal depression, and mental illness in general, tends to lead
to inadvertent medical leaves primarily in the fall and winter.

I first started to take the concept of mental health seriously in my
later years of college when I started having trouble making it to work
on time because I was terrified of the people I'd encounter on the way.
I was afraid that there was some aspect of myself that was attracting
attention. I was worried that walking in a certain way, or perhaps too
close to others, would lead to accusations of sexual harassment. I was
worried to the point where I started having to plan my arrival on campus
between classes, or if I happened to show up during a rush, to move from
bathroom to bathroom on the way to my lab in the library, from the
animal sciences building, to the plant sciences building, to the general
education building, to the library, and to my chair, moving from island
of stillness to island of solitude.

Much of this subsided with the adoption of a regular schedule in a much
smaller and more restricted environment when I took my first job. I'd
drive for an hour (53 miles) by myself to the office, hang out with the
same four or five people out of a group of twenty or so for eight to
twelve hours, then drive for another hour to my home, where I lived with
the same two or three others.

At first, it was actively refreshing! The drive was a nice way to unwind
after work, especially on those days when I was working ten or twelve
hours: not only could I just Not Work for a while, but I was alone and
didn't have to interact. I discovered audiobooks and plowed through
several over and over again, taking comfort in the lack of surprise in a
repeated storyline.

Eventually, the 60-80 hour weeks started to wear on me, as did the job
itself, and I noticed that it was easier for me to hide in my cube, or
simply head home and spend the entire weekend indoors, or perhaps in the
backyard engaging in some solitary hobby or another (I'm a big fan of
home brewing - there's few surprises and basically zero conversation
involved in mashing grain by oneself).

I've written about it before, but as it piled up, I found myself seeking
medical help, even to the point of medication in the form of
anxiolytics. This eventually culminated in a mental health emergency
that lead to forced time off work, and eventually leaving the job for
one that would be a better fit for my well-being.

\begin{center}\rule{0.5\linewidth}{\linethickness}\end{center}

{[}a{]}{[}s{]} had drifted from where I felt that the site would've gone
had it just been me, and I felt torn.

I had this idea that {[}a{]}{[}s{]} would be some grand outlet for
exploring the furry fandom in such a way as to be all inclusive and yet
still introspective, a place for data and doxa, an accessible place
without being condescending.

Pure dreams, of course. Once the project started to take off, my role in
it was limited to my own production, and once others came on board, the
project took off in a purely unpredictable direction. It was
exhilarating and fantastic to watch, seeing the ways in which others
also thought about the fandom joined in and started taking the
conversation all the further.

The benefit of this, and particularly of the voice of JM, is that the
site grew rapidly without becoming unstable. The growth is visible not
only in the plain-old-numbers sense of our viewers per day has
increased, but also in the spread of our articles. Some were picked up
by FurCast, some were discussed on Flayrah, one was even nominated for
an Ursa Major Award, along with the site as a whole.

The downside, however, was that as the direction of the site shifted and
steered in other directions, I felt as though my own contributions were
less and less relevant. This is embarrassing for me to admit, of course,
but worth admitting all the same. I felt that what I had to offer when
the site started out was not worth offering to what the site has become,
based solely on the numbers game I was playing in my head.

To me, there felt like two obvious choices available: I could try and
steer the site back toward what I wanted at the risk of its
accessibility by a wider audience as well as every other author that
participated, or I could step down from my own contributor status and
let the site go in the direction it seemed destined to go.

Binaries are false (at least, most of them), however, and destiny only
means so much when there are multiple parties with their own free wills
(or semblance thereof, but that's out of scope for this article) acting
within a single setting. And besides, I run the tech side, and the
organization's in my name. So the true reality is that there's some
middle road, some third path to be taken to ensure that the site
fulfills its potential without slowly narrowing in scope until the
audience of everything outside that scope becomes irrelevant and bored.

\begin{center}\rule{0.5\linewidth}{\linethickness}\end{center}

In 2012, as I recovered, I took the winter off. However, during the
summer of 2013, I was blindsided by the fact that, despite a lifetime of
some semblance of consistency, the symptoms of panic took a hard left
into psychosis-ville and I found myself stuck in London, head jerking
violently to the left every few seconds, hearing voices telling me to
throw myself to my death from the office balcony, and believing with an
earnestness never before experienced that the entirety of the world knew
every terrible secret of mine.

There's a lot to be said for the ways in which the mind works to adapt
to situations that force a reevaluation of the way of life. There comes
a point, however, when your mind throws up its hands (or paws, or wings,
or what-have-you) and basically just refuses to come out of its room,
subsisting on pita chips and hummus and brandy until the world changes
its mind, straightens up, and flies right.

Of course, the world had hardly changed at all. I mean, sure, work trips
are stressful, and I came home at a bad time with the Colorado floods,
but there wasn't anything happening in my life to warrant the magnitude
of my own personal breakdown. And that's what it was: a breakdown. The
term isn't a medical one, but is generally accepted a temporary, drastic
mental shift featuring large and out of place anxiety, depression, or
other psychological symptoms, and for me this took the form of
debilitating panic attacks, auditory aberrations, the return of a
nervous tic, and various other symptoms.

I stopped writing. I very nearly stopped working, and was subsequently
reprimanded. I haven't contributed to any side projects (my own or
others such as Weasyl) since. I maxed out my credit card on treatment. I
spent all of my energy on improving, with the gracious help of my
partners and friends.

\begin{center}\rule{0.5\linewidth}{\linethickness}\end{center}

I left {[}a{]}{[}s{]} in the hands of JM, as I did before, and couldn't
have asked for a more capable administrator while I sorted things out.
In that time, guest articles were published, correspondences managed
(with some notable exceptions, for which I'd like to offer public
apology in addition to my private ones).

Had this been just a Makyo Production, I would've lost everything and
not had much of a reason to continue after the shame of having to stop,
no matter the legitimacy of the initial reason. It's hard to invest so
much of oneself into something, fail so completely on a personal level,
and then just start right back up.

Thankfully, this isn't a Makyo Production. Never will be. I've got a
job, a husbandog, a catfriend, two pups and a kitty, a house-husky, and
a whole host of friends and acquaintances I can lean on for support.
I've got obligations outside of {[}a{]}{[}s{]}, just as JM does; just as
Jakebe and Rabbit and Zik and Kyell and Klisoura do.

The root of the problem, then, comes down to one of identity. I had
built up in my mind this picture of myself as an avatar of a website.
Nothing so grandiose as to be delusional, and quite a bit more paranoid
than that implies. Every interaction with the organization, whether or
not directed at me, whether or not positive or negative, became a
personal indictment, the entirety of the subculture that had been my
home and to whom I had looked up to for more than a decade now looking
down on me and deeming me unworthy of even contempt. It was an extension
of the paranoia infecting my personal life.

That's not even remotely close to the reality, though. In retrospect,
it's one of those instances that's relatively clear in hindsight, even
if completely obfuscated in the heat of the moment. I think it's worth a
brief discussion on the way {[}a{]}{[}s{]} actually works.

{[}a{]}{[}s{]} is a platform. It's a collection of individuals who are
willing to put in somewhat regular effort in investigating the furry
fandom, plus a welcoming base of support for any individuals who would
like to make a point about the fandom through a guest post. As far as
published opinion goes, there \emph{is no} such thing as an
{[}a{]}{[}s{]} viewpoint. There is no direction for the site except
outwards. All the site can be is embodied in the contributors and
creators, which is exactly as it should be: the precise same thing
applies to the furry fandom as a whole. There is no guiding light. There
is no canon. There are no leaders. There are no directions in which our
community should or ought to head, because the directions in which our
community should head consist of simply ``outward''.

The Furry Poll is not, never has been, and never will be a scientific
study of the furry fandom. That is the purview of scientists, and one
ought to look to the IARP for such information. If one wants to think of
the Poll, it's best to think of it as a market survey: a simple view of
the market as viewed through the eyes of willing participants. The goal
is not to make broad and sweeping statements of absolute truth about the
furry subculture, but to view through our communities eyes the
demographic and psychological makeup of the community. It's a snapshot
of how a good portion of the community views itself.

Love - Sex - Fur is not an endorsement of sexuality within the context
of furry in any sense, but an exploration of the intersection of two
very important parts of life for so many in our community. There is no
denying that sex and sexuality play an important part in the human
experience (evidenced by so many laws, regulations, social mores, and
taboos surrounding it), and when it comes down to it, we interact with
each other in the context of human society. And yet the furry subculture
plays an enormous role for many of us, so it's no wonder that there is
some intersection. It's as much worth acknowledging and exploring as
species selection and character creation.

Our goal, in moving forward, should be to emphasize this, and we have
already taken steps to that end. The goal is for {[}a{]}{[}s{]} to be
less some news outlet ``voice of the community'' - we aren't, and it's
questionable whether such a thing is possible - than to be a platform
for introspection and exploration. We are less a voice than we are a
medium. Seriously, I think the project compares more favorably to
publishing platforms such as \href{http://medium.com}{Medium} than
anything else. As Rabbit
\href{http://adjectivespecies.com/2014/01/24/when-youve-said-too-much/}{put
it}, it's a platform for exploring the ``why'' and ``how'', rather than
the ``what'', ``where'', ``who'', and ``when'', something that can only
be done by all of us. If there \emph{is} any steering to be done of the
project, I think it is to ensure that this is more visibly the case.

\begin{center}\rule{0.5\linewidth}{\linethickness}\end{center}

The fandom is enormous, and {[}a{]}{[}s{]} is a view of this enormous
thing as it evolves over time. There are at least as many different
views of the fandom as there are members of it, if not more. This is a
feature, I think, rather than a defect: after all, if we were all so
thoroughly aligned that there were fewer views of our subculture, then
the complexity of our community would be replaced with loudly agreeing
with each other.

I thought in my moments of panic that maybe I ought to step down, or
perhaps force {[}a{]}{[}s{]} in one direction. In reality, both were the
products of my own anxiety, so thoroughly a part of who I am that my
mother, in an email, apologized to me for her role in my genetics (an
apology borne out of anxiety if ever there was one, with absolutely no
recourse to reality).

I try to tie my own writing back to furry whenever possible, and I think
I've done an okay job of including it throughout this screed, but I
think it's worth calling out one last time that furry is one of those
anchors that keeps me tied down to a grounded and full life. This
community provides me with a support structure that I feel was honestly
lacking from my life before, and I'm not only super pleased that I have
it, but super pleased that it's available to anyone and everyone, no
matter how they need it.
