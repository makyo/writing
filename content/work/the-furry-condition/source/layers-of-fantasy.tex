I found out recently that there's a name for the concept behind the
movie \emph{Inception: mise en abîme}. ~It's a French phrase which means
``placed into abyss'', and refers not only to the sort of
dream-within-a-dream concept so heavily pounded upon in
\emph{Inception}, but also the concept of any thing within itself, such
as a representation of the painting within a painting, or the feeling of
standing between two mirrors and seeing the infinite representation of
self receding into the distance. ~It also has to do with different
layers of representation and meaning in art, and, even though I've
mentioned before that it's surprising how mundane much of our
interaction is, that's what strikes me about the prevalence of fantasy
and science fiction within the furry fandom's artistic output. ~It is a
sort of stacking of different layers of fantasy, with our focus on
anthropomorphic animals being layered atop science fiction or fantasy
elements.

I suspect that a lot of why this weird dichotomy of mundane and
fantastic trends within the fandom is so striking to me is due to the
different avenues into the fandom that we've taken. ~Speaking for
myself, I found the fandom along a decidedly mundane path - Yerf.com.
~Even though I'd read all the Redwall books at the time, had watched
Disney's Robin Hood over and over, and spent much of my time in
elementary school role-playing scenes from The Phantom of the Opera with
a friend wherein everyone was a cat, none of those actually struck the
furry chord, as it were. ~It was finding PacRat's art on Yerf.com,
images of furries in more mundane settings, that got me into things. ~I
liked the fact that species became more an aspect of self rather than
some fantastical attribute about some fictional character.

That is, of course, not the only route into the community: several
people I know have talked about their entry into the fandom being based
around some of those things that I already mentioned, such as Redwall.
~In fact, a good majority of my friends found their way into furry
through the more fantasy-oriented routes, and that struck me as
interesting, as here we were, already pretending to be animal people.
~It was intriguing to think of layering fantasy atop fantasy like that.
~It's difficult, of course, to draw a hard and fast line between these
two routes, as there are several people who are content living in a
mostly mundane world set perhaps a few years forward or backward in
time, or even a mundane life in the far future or distant past, yet I do
feel that there is a difference in mindset between the more and less
fantasy oriented furs.

I suppose that the difference between these two views of the fandom
isn't so much that we're applying our culture to a fantasy setting
versus a mundane setting, so much as how we view our focus on our
characters. ~If one views one's character as some sort of fantastic
being, some concept of self with additional elements which extend beyond
the norm, it's easier to place oneself in a fantastic setting. ~From the
other point of view, if one views one's character as one's self, simply
expressed differently, or as something one possesses rather than one is,
then it might feel more comfortable to exist in a setting closer to the
one inhabited by the player - that is, a more mundane set of
circumstances. ~The difference there being that there is a bit of a
divide, no matter how vague, between two sides of looking at one's
character - as fantasy, or as mere re-representation of self.

This sort of thinking struck me as interesting back when I was first
getting into the fandom, on one of my first sojourns onto a MUCK. ~When
you describe your player using Triggur's seemingly omnipresent
editplayer command, you are given the opportunity to set a bit, or
attribute, on your character to say whether or not you can fly. ~I had
personally thought this rather strange: I was just a teenage fox guy,
living in a teenage fox guy world, where I had surrounded myself with
several other teenage fox, cat, or what-have-you friends living in the
same world. ~What use did I have for flying? ~I set the bit in order to
more thoroughly explore the MUCK that I had wound up on (Zorin's
FluffMUCK), as it was needed to do things such as go up, instead of just
north, south, east, or west. ~Every now and then, I would play around
with it, flying up above the park, the main location on the server,
where I could joke around with friends or get away from the inane
chatter below, but I never really thought of it as flying, per se.

It wasn't until I started to explore further on other MUCKs such as SPR
and FurryMUCK where role-play was taken more seriously than it was on my
original hang-out of choice, that I found out that it really did matter
to people less mundane than I whether or not the flying bit was set.
~Although in the long run, I wound up simply finding another, older
crowd of more mundane fox, cat, and wolf people to hang out with, it
always stuck with me that here I was, a fox guy that could fly for, in
my case, no real reason. ~I never flew (I rarely do much more than hang
out in one room, to be honest), and even to this day, never really
consider it flying. ~However, having seen and, once or twice, taken part
in more serious role-play in a more fantastic setting than what amounts
to a glorified chat room most days, I can say that this is likely due to
me just not being a very fantasy-oriented person, and perhaps there's a
personality trait that helps determine whether or not one feels more
comfortable interacting in a fantastic or mundane setting.

The downside to all of this, of course, is that it becomes difficult to
maintain without potentially losing some aspect of the fantasy. ~A furry
story set in a fantasy setting runs the risk of being a fantasy story
wherein all the characters are animal people for no~discernible~reason,
or perhaps a furry story in which fantastical things keep happening with
little explanation. ~Perhaps that's the sign of a really good furry
role-player or writer, though, being able to maintain a level of
coherence within all the separate layers of fantasy. ~The requirements
for a furry fantasy to be pulled off well require miscibility: the risk
is great of having a fantasy that happens to be furry or vice versa, and
so it seems to be important that furry be either a strong part of the
fantasy or at least part of the plot in order for everything to work out
well.

Another downside to these different routes into the fandom is the
segregation that is built into that fact. ~That western society views
role-playing of most types as a geeky pursuit and geeks as a
frowned-upon minority, it's no surprise that the same outlook can carry
over into furry pretty easily, given how much of the fandom is based in
western society. ~Perhaps that's a big claim for me to make, but having
seen the way that the issue of ``RP'' can polarize furries, I'm not sure
of what other explanation there might be. ~There are those who totally
buy into their character, and especially into the fantastic aspects of
them, and there are those who are in the fandom for some other reason,
perhaps more of an affinity than an identity. ~The two groups
occasionally have their clashes, with arguments being based around the
one group ``powergaming'' the other, or the other group being too
serious or roleplaying in comments. ~As yet, at least, the clashes seem
to mostly involve the two groups poking fun at each other.

Furry is a fantasy, there's no way around it - at the very least, it is
a hobby that revolves around what could basically be explained as
fantastic creatures with human attributes (or vice versa, of course),
and on the other end of the spectrum, it can be seen as a set of people
with identities that more closely match that of some other species
besides their own, those who are perfectly willing to buy into the
fantasy. ~Adding additional fantasy on top doesn't always work out quite
as expected, but seems to be the natural course of events in that it's
so easy to extend furry beyond its roots and into such realms. ~Some
just like their animal people to be pretty normal, though, and that's
okay, too. ~It's long since gotten to the point where the fandom is big
enough to hold all of us.
