\documentclass{memoir}

\title{Seasons}
\author{Madison Scott-Clary}

\usepackage[backend=biber,style=authoryear-ibid]{biblatex}
\bibliography{seasons}

\begin{document}

The power of the cyclical nature of the year is of an importance that draws the heart onward\footnote{To be more exact, due to the (generally) linear nature of time, years spiral up. Days, of course spiral forward.}, and that which moves the heart is fair game for poetry. The demarcations for this cycle are the two solstices, with secondary markers at the equinoxes. One finds oneself at the longest night of the year and knows that, from there onwards, it is downhill into summer.\footnote{I am not sold on this metaphor; uphill bears both positive and negative connotations, and it is difficult to say which to apply when. Ask a poet.} One finds oneself at the longest day of the year and before oneself lies cooler times.

The concept of seasons and seasonality is well known within poetry. Exploring that is beyond the scope of this paper.\footnote{Or perhaps my abilities as an author.} To rely on synecdoche is the best one can manage with a topic so large. To that end, it is worth exploring the poetry of Dwale in such a context.

Dwale (1979--2021\footnote{I learned about Dwale's passing the day after it did so, on July 2nd. A portion of this paper was written toward the end of June. I was unable to share that with it, and the fact weighs on me.}; it/its) was a poet living in the Southern United States. As a member of the furry fandom, it presented itself as a 'cabbolf' --- a cat/rabbit/wolf hybrid --- often dressing in a Russian kosovorotka or Middle Eastern shalwar kameez. \parencite{dwale}

Its work is described as focusing on ``altered states of consciousness\ldots{}poverty, addiction, subjectivity, and the transience of existence'' \parencite{dwale}, though to reduce its body of work to any or all of those provides an inexact picture of its writing. This will be touched on in a future section on translation, but needless to say, this paper will focus on its work through the lense of seasonal progression.

\newpage

\section*{Spring}

\begin{verse}
The seasonal storms have poured upon the grassy flat, \\
The leafless stalks abound like thirsty mouths. \\
Puddles form and soon are swarmed with little fish, \\
And all the arid life has fled despair.

And here, wrapped in rain, lies the oldest soul, \\
The changes wrack his bones with painful cold. \\
His skin is like the sky at night, as many scars \\
Have marked his hide as there are glinting stars.

At once he feels his lungs become bereft of breath, \\
His daughter nudges him, to no effect. \\
She walks away rememb'ring days they stalked the plains, \\
Within her womb there grows a golden bloom.

\parencite[26]{leaves}
\end{verse}

Spring is commonly associated with newness. New growth, new life, new warmth under a new sun. 

\begin{verse}
Stirring suddenly from long hibernation \\
I knew myself once more a poet \\
Guarded by timeless principalities \\
Against the worm of death, this hillside haunting; \\
And presently dared open both my eyes

\parencite[165]{graves_poems}
\end{verse}

\begin{verse}
\emph{Mi no ue no} \\
\emph{tsuyu to mo shirade} \\
\emph{hodashikeri}

Heedless that the dews \\
mark the passing of our day --- \\
we bind ourselves to others

\parencite[11]{issa}
\end{verse}

\printbibliography

\end{document}
