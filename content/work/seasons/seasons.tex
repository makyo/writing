\documentclass[12pt,onesided]{memoir}

\usepackage{fontspec}
\setmainfont{Gentium Book Basic}

\usepackage[
  letterpaper,
  includeheadfoot,
  margin=1in
]{geometry}

\title{Seasons}
\author{Madison Scott-Clary}

\usepackage[backend=biber,style=authoryear-ibid]{biblatex}
\bibliography{seasons}

\linespread{1.5}

\begin{document}

\maketitle

A year spirals up.

A day, a week, a month, they all spiral, for any one Sunday is like the previous and the next shall be much the same, but the you who experiences the differing Sundays is different. It is a spiral, proceeding steadfastly onward. A day is a spiral, with each morning much the same as the one before and the one after. A month, following the cycle of the moon

But a year, in particular, spirals up. It carries embedded within it a certain combination of pattern, count, and duration that delineates our lives better than any other cyclical unit of time. Yes, a day is divided into night and day, and those liminal dusks and dawns, but there are \emph{so many of them}. There are so many days in a life, and there are so many in a year that to see the spiral within them does not come as easily.

Our years are delineated by the seasons, though, and the count of them is so few, and the duration long enough that we can run up against that first scent of snow late in the autumn and immediately be kicked down one level of the spiral in our memories. What were we doing the last time we smelled that non-scent?\footnote{Scientists have described the `scent of snow' as the air being too cold for the olfactory system to register scents.} What about the time before?

The power of the cyclical nature of the year is of an importance that draws the heart onward, and that which moves the heart is fair game for poetry. The demarcations for this cycle are the two solstices and two the equinoxes. One finds oneself at the longest night of the year and knows that, from there onwards, it is downhill into summer.\footnote{I am not sold on this metaphor; both uphill and downhill bear positive and negative connotations, and it is difficult to say which to apply when. Ask a poet.} One finds oneself at the longest day of the year and before oneself lies cooler times.

Dwale (1979--2021; it/its) was a poet living in the Southern United States. It was moderator of and, for a term, president of the Furry Writers' Guild, and was known for facilitating the `coffeehouse chats', hour-long lectures surrounding various writing topics that took place twice a week. Its work is described as focusing on ``altered states of consciousness...poverty, addiction, subjectivity, and the transience of existence'' \parencite{dwale}, though to reduce its body of work to any or all of those provides an inexact picture of its writing. This will be touched on in a future section on translation, but needless to say, this paper will focus on its work through the lens of seasonal progression. 

The concept of seasons and seasonality is well known within poetry. Exploring that is beyond the scope of this paper.\footnote{Or perhaps my abilities as a writer} To rely on synecdoche is the best one can manage with a topic so large. To that end, it is worth exploring the poetry of Dwale in such a context.

\section*{Spring}

Spring is commonly associated with newness and beginnings. New growth, new life, new warmth under a new sun. A season of green things: of buds greening bare trees, of grass poking through late snows, or perhaps the greenery of gardening as one buys flats of flowers or sows vegetable seeds in the expectation of a harvest later on.

Spring is also associated with growth in a broader sense. It's the time when plants race toward the heavens, or leaves burst out from reanimated branches seemingly overnight. It's the time when you can almost feel your hair growing, or perhaps your dreams swelling in some sympathetic expansion of their own.

And, importantly, spring is the season of expectations. The year may start on the first of January, a convenient fiction provided to us by the need to start it \emph{somewhere}, but the expectations for the rest of the year lay dormant in the mind until spring. January first is the time to make the resolutions and the rest of winter is the time to try them out, whether tentatively or with great passion, but the setting of expectations for the year doesn't come until the trauma of the year before has settled into uneasy memory --- or, to use an outdated metaphor, expectations are not set until one stops writing the previous year on the date line of one's checks.

Although it often engaged with expectations in its work, Dwale tackles the subject of Spring in the context of beginnings and growth infrequently, seeming to prefer Autumn. One small example of this comes from a short \emph{renga} that took place on Twitter:

\begin{verse}
Blackbird headed south\\
Down to the hawks and kudzu\\
Six months 'til winter

\parencite{dwale_haiku}
\end{verse}

While we are verging into the territory of summer here, as ``six months 'til winter'' implies, we do get a sense of those expectations settling into place, a feeling of ``ah, so the year is going to be like \emph{this}''. We also get that sense of growth and greenness with the mention of kudzu, a plant known for its rampant growth, quickly covering all it can in green.

Blackbirds, while often showing up in the context of winter, do occasionally make their presence known in writings that take place during other seasons. Stevens, for example, has

\begin{verse}
XII \\
The river is moving. \\
The blackbird must be flying.

\parencite{blackbird}
\end{verse}

\noindent wherein the thought of a river moving again being of note implies a thaw after a long winter, a world in which this could not possibly be the case without the blackbird also flying. There is a movement thawed, here.

Some of the reason for this paucity of spring-themed poetry is doubtless selection bias: a chapbook titled \emph{Face Down in the Leaves}, with its cover of frost-rimed leaf-litter, is unlikely to contain any paeans to new growth.

Instead, we are presented with works that focus on the fact that spring is also the time for harrowing. It's the time for tearing up that which was old, the earth that was compacted by time and snow, in order to make room for that growth which is going to come soon, whether we like it or not (the topic of unwanted growth is a topic for later in the year\footnote{Or perhaps later in life, when cancer may rear its ugly head. It is proving quite difficult to write about even seasons of new growth and beginnings without death-thoughts creeping in.})

This untitled work will stand as our example:

\begin{verse}
The seasonal storms have poured upon the grassy flat, \\
The leafless stalks abound like thirsty mouths. \\
Puddles form and soon are swarmed with little fish, \\
And all the arid life has fled despair.

And here, wrapped in rain, lies the oldest soul, \\
The changes wrack his bones with painful cold. \\
His skin is like the sky at night, as many scars \\
Have marked his hide as there are glinting stars.

At once he feels his lungs become bereft of breath,\footnote{When its friends learned of its passing, many of us decided to memorialize it with poetry of our own \parencite{memorial}. While I lack the feel, my attempt also incorporated the loss of breath: \begin{verse}Beneath that evening's breeze the sickly sweet \\ \vin and brazen scent of countless flow'rs \\ awoke inside of you a darkened sleep \\ \vin Of dreams dug deeper than the soil. \\ Oh, we are waking minds who missed that scent! \\ \vin What hope have we who wait in life, \\ who sit and pray and watch for your next breath? \\ \vin Our hope can only reach for ends --- \\ To wit, to see you wake and meet a mind \\ \vin Too keen to weed a garden clean --- \\ For we exhaled when you breathed in that breeze \\ \vin and flowers wreathe your sleeping form.\end{verse} Perhaps it is the idea of the cessation of the cyclical nature of breath that brings with it thoughts of death.} \\
His daughter nudges him, to no effect. \\
She walks away rememb'ring days they stalked the plains, \\
Within her womb there grows a golden bloom.

\parencite[26]{leaves}
\end{verse}

This poem\footnote{The choosing of these four poems to focus on was originally intended to be for a music project. These were to be the texts for four art songs in a collection also named ``Seasons''. Every now and then, I get it into my head that maybe I can go back to writing music instead of words, and am quickly disabused of the notion when I sit down to do so. The Madison who wrote music has long since passed.} in three stanzas is largely in an even meter (sometimes iambic, sometimes trochaic), though we are presented with two instances in the first lines of the first two stanzas where that pattern is broken (``The seasonal storms'': ˘ -- ˘ ˘ and ``And here, wrapped in rain'': ˘ -- -- ˘ --). When this is taken with the middle verse's assonance and rhymes, we pick up a sense of a stumble mid-gallop. Although the procession of time may be linear, the procession of the seasons may be interrupted by little stalls, little snowy loops back into winter as spring presses on towards summer.

These variations in prosody combined with the third verse being ``played straight'', such as it were, add up to a sense of growth, of rushing forward when Winter (we assume the oldest soul to be) breathes his last. Here, we might picture that final snow, Spring nudging winter, and realizing that all she has left are her memories of him and her child, Summer, still unborn within her.

This, after all, would be her new beginning. She is no longer bound to winter as she might have been before; there are to be no more of those loops back into snow, she's on her own now, pacing into the grassy flat with its puddles of fish.

Issa says,

\begin{verse}
\emph{Mi no ue no} \\
\emph{tsuyu to mo shirade} \\
\emph{hodashikeri}

Heedless that the dews \\
mark the passing of our day --- \\
we bind ourselves to others

\parencite[11]{issa}
\end{verse}

Spring is nothing without Winter. Even when it has its own snows, Spring is what it is specifically because it isn't Winter. There's that vernal equinox and then suddenly the days are longer than the nights, the world begins anew, and all that is in it does so as well. As with us: we are nothing without those around us, and we are us specifically because of those in our lives. There is our meeting and then suddenly that which makes us \emph{us} is fuller than before, and we carry within us the golden bloom of who we are to become.

We are the seasons that comprise our lives. We are beholden to the passing of our days as they are, yes, but we are also unable to truly, truly begin something anew. We are are also comprised of that which came before, and are bound to those around us.\footnote{After all, I was bound to Dwale; that's why this essay exists. That's why what little poetry I have exists. I could appreciate the music within poetry, but it wasn't until I met Dwale, became bound to it in friendship, that was able to understand poetry better on its own terms.}

Also throughout Dwale's seasonal work is the concept of vegetation. In spring, we have the grass, those leafless stalks that open up with the rain.

Here, this new grass is anthropomorphized: as new grass grows, it unfurls from the curl that it was before, forming almost a funnel which, in this instance, becomes a thirsty mouth. They live lives as full as Spring and Winter do in our poem, and one might picture their journey from thirsty mouths to rattle-dry stalks, dusty and tan, as summer fades.

Issa says,

\begin{verse}
\emph{Ukigusa ya} \\
\emph{ukiyo no kaze no} \\
\emph{iu mama ni}

Floating weeds, as blow \\
the winds of the floating world --- \\
drifting and drifting

\parencite[18]{issa}
\end{verse}

There is some world that is not ours superimposed on the one we live in. This floating world is that which shows plants as the thirsty mouths that they are, shows the floating weeds as integral parts of the world, rather than some thing to be removed.\footnote{Something about the numinous inspires reading the animate into the inanimate (if plants could be called such) and no one that I have talked to who dwells on their sense of the numinous can either explain or deny this. Wands of living wood! The true cross! The tree of life! Secret lives of secret cells keep hope alive that one day I might speak with you again. All four seasonal poems dwell on this.} Perhaps this is the one that plants experience most clearly, where Spring may nudge Winter and, finding him dead, walk out into new grass and memories.

It is this world that poetry most clearly provides a glimpse into. It contains those symbols which pass fleetingly through our lives, drifting on by as the seasons progress.

``Here is the difference betwixt the poet and the mystic,'' Emerson cautions. ``That the last nails a symbol to one sense, which was a true sense for a moment, but soon becomes old and false.\footnote{I will admit that I veer towards mysticism, here. ``Mysticism consists in the mistake of an accidental and individual symbol for a universal one,'' Emerson goes on to say, and I will not deny my propensity toward doing so, but such is the problem with an essay. Can you really blame me for wanting to pin down the love of lost friends lest it squirm away into nothing, into some dusty old box high up on a shelf labeled simply `regrets'?} For all symbols are fluxional.'' \parencite[33]{emerson} We have in Dwale's work a glimpse of the symbol of the hard death of Winter, of Spring with Summer in her belly.

%%%%%%%%%%%%%%%%%%%%%%%%%%%%%%%%%%%%%%%%%%%%%%%%%%%%%%%%%%%%%%%%%%%%%%%%%%%%%%%

\section*{Summer}

As the year continues on its upward spiral, we come to one of those strange apogees of the longest day. Strange because yes, of course it bears meaning as the longest day, and yet the start of Summer never seems to fall directly on that day, does it? There is doubtless some good reason that, at least here, that is the first day of summer rather than midsummer.

And yet even that isn't always accurate, is it? Some years, summer doesn't feel like it has truly hit until well into July, when the temperatures climb and the rain becomes a distant memory.\footnote{And perhaps your well dries out when you head out of town for you husband's surgery, so your dog-sitters to have to figure out water, leaving you to fret and pace around the hotel room, and maybe that's the time you decide, ``You know what? Work is so terrible that I think I'll apply for grad school.'' But you have to provide a sample of analytic writing to do so, so you pick one of your friend's poems to analyze, and two weeks later --- when you've come home to no water and a dog whose health is steadily declining though you don't know it yet --- your friend is dead.} You're left feeling miserable for weeks on end, wishing for even a drizzle to quench your thirst, or even a bit of cloud cover at night, enough to maybe knock the temperature down into the low seventies so you can finally, \emph{finally} get some sleep and yet the days spiral forwards through heat-haze.

\begin{verse}
Summer, season of hot insomnia, \\
That much never seems to change at all. \\
Laying awake in the red desert night, \\
I shape forest from shade and wait for fall.

Ten years now gone,\footnote{It's 2022 as I write this, which means that, come September, it will have been ten years since Margaras died.\par His was the first death that really hit me. The first one I was really able to comprehend. Koray came into the bar, asked if this was the place he would have frequented, passed on the news, and then left.\par It was crushing. It destroyed me. I am still not entirely sure why, since we were friends, yes, but we were hardly so close as to warrant the reaction that I had, and yet I did.\footnotemark\par And yet I did and now, a decade later, I only think of him on the anniversary or when I come across the notifications I have from him and from Koray. Maybe that's why there's that worry about the box labeled `regrets'. I have my regrets for Margaras, and the amount by which those are outweighed by the good memories is too small for my liking.}\footnotetext{Not unlike Dwale, I suppose. Perhaps a good chunk of this --- of both of their deaths --- is due to just how little I interacted with them through anything other than text. I met Dwale once in person, and never met Margaras. I listened to Margaras's music and listened to audio versions of Dwale's stories, but other than that, they were relegated to words on a screen.} and who thought I would miss \\
Cricket songs, cicadas and katydids? \\
Then I'd gladly have grabbed a big hammer, \\
Smashed them flat as Pinocchio's conscience.

Testing palisades of clocks and yardsticks, \\
No advent waits for the restive dreamer. \\
I bandage my tattered, bitten left hand \\
And shed the smoke rings on my cloven finger.

\parencite[8]{leaves}
\end{verse}

The poem follows a similar structure to that chosen for Spring: three stanzas of four lines each, often falling back into a stressed-unstressed (or vice versa) meter, though far more free. We have a few more near rhymes, (`at all' and `for fall', and, to a lesser extent, `dreamer' and `finger'), plus a few pleasing instances of alliteration (`\emph{cri}-\emph{cket}\ldots{}ci-\emph{ca}-\textbf{das}\ldots{}\emph{ka}-ty-\textbf{dids}').

Also as before, there is a volta in the third verse. Whereas with Spring, we switched point of view from Winter to Spring, here, we switch away from from the concrete world and into something more abstract. Where we start with hot deserts, forest shade, katydids and hammers, now we are confronted with unknown tools of measurement, dreams, and smoke rings. We have that which defines itself in the external world and that which we define internally, and with those two poles, we are left to extrapolate what is between them.

Issa says,

\begin{verse}
\emph{Natsuyama ya} \\
\emph{Hitori kigen no} \\
\emph{Ominaeshi}

On the hill of summer \\
Stands the slender maiden flower \\
In a solitary humor

\parencite[65]{issa}
\end{verse}

The slender maiden flower is the slender maiden flower. We have no say in its existence except that we might pick it, trample it, or leave it be. It is itself, in all its glory --- or at least all its solitary humor. The flower defines itself and though we may take action on it, may think it beautiful or ugly or lonely or austere, that doesn't matter to the flower.\footnote{For a while, I was quite caught on the idea that others have agency of their own. Of course they do, I mean, I just found it marvelous that this was the case. There was no way that they could not, right? They live and love and feel just as much as I do, so I can't say that this same applies to people; they define themselves, sure, but they can actively change how I create meaning from their existence.\footnotemark}\footnotetext{Of course, having written this, I feel bad for the flower. Perhaps it desperately wants to be seen as austere instead of lonely, as beautiful instead of ugly. Ask a botanist.}

``Summer, season of hot insomnia / That much never seems to change at all'' speaks well to this. Summer is Summer. It is the season of hot insomnia and it doesn't care how tired we are. It's not that it is inimical to us so much as existing within its own external nature. It exists in that floating world that is separate from us. It does not know us, it knows only itself. It's hyperreal, perhaps, only casting its shadow into our reality.\footnote{It was briefly hinted by its partner that Dwale's death may have been an overdose --- or at least a drug-induced accident --- due to a heroic dose of DXM. I don't know if this is true, and the tweet where she suggested this has since been deleted. Hell, I don't even know if I was meant to see that information. Still, it sticks in the mind. On mentioning my own explorations into mushrooms as a way to set aside, for a few weeks, the obstinate burden of depression, Dwale readily agreed that this would probably be good for me. It seemed knowledgeable, as though it knew that smear of other reality (or hyperreality) well enough.\par If this story of its death \emph{is} the case, then that makes me wonder just what those last hours, last minutes, last seconds might have contained. Was it filled with pleasure? Did it, as a Muslim, feel closer to Allah? Did it know what it had done, panic, and try to retreat? Terror hems these thoughts in.}

Issa says,

\begin{verse}
\emph{Mi no ue no} \\
\emph{kane tomo shirade} \\
\emph{yusuzumi} 

Heedless that the tolling bell \\
Marks our own closing day --- \\
We take this evening's cool

\parencite[39]{issa}
\end{verse}

This is the inverse, the other pole of our spectrum. Whether or not the bell tolls for us and our day, whether or not the evening's cool is of that floating world, we still can define ourselves and our actions in the face of it. We are the ones who can take that cool as some small respite from the hot insomnia that the Summer might otherwise offer. We can define ourselves in that context, and by that, we can define the world around us.

In this sense, the cool evening and the end of our day --- indeed, the season of hot insomnia that never changes --- is something over which we can layer an artificial definition. The semiosis in play allows us to turn Summer into a sign that we can interpret. Our artificial definitions apply to us, even if the heat of the day doesn't give a damn about us. ``Testing palisades of clocks and yardsticks, / No advent waits for the restive dreamer'' because we restive dreamers are only able to measure by our artificial definitions.

But that cannot be all. There has to be more than the external and natural, that which defines itself, and the internal and artificial, that which is defined by us. We smash the insects flat with a hammer, correct? We build air-conditioned bedrooms to be able to get our sleep, correct? What is in the middle is agency. It is the permission we give ourselves to form these definitions in cooperation with the world around us. We can cry out at the sight of blackbirds bursting from the trees, because that is a thing that we have the power to do, ourselves:

\begin{verse}
X \\
At the sight of blackbirds \\
Flying in a green light, \\
Even the bawds of euphony \\
Would cry out sharply.

\parencite{blackbird}
\end{verse}

It is the act of taking meaning from each other, as well, for each of us has our own agency: we can interact with each other and influence each other's definitions of ourselves.\footnote{Viz. me meeting Dwale in the writers' guild and deciding --- actively deciding --- that I would be its friend. It wasn't lacking, and neither was I, but something about someone who might choose `it/its' as pronouns, someone who could engage with poetry in a way that had always eluded me. Doubt nips at my heels, though. Is ``deciding to be someone's friend'' a normal thing to do? Was that weird? Did it resent me for-- but I shouldn't be thinking like this.}

As that golden bloom of Summer defines itself as all things must, and we have to take it at its word. We can kvetch about the insomnia of Summer, that which makes us sweat through the sheets so that the thought of touching someone else makes one feel clammy and disgusting\footnote{Just me? No? Maybe just me.} all we want, but that doesn't mean anything to Summer. It just also doesn't stop us from layering our own definitions atop that.

\begin{comment}

%<!-- RELATED -->

%%%%%%%%%%%%%%%%%%%%%%%%%%%%%%%%%%%%%%%%%%%%%%%%%%%%%%%%%%%%%%%%%%%%%%%%%%%%%%%

\section*{Autumn}

\begin{verse}
\emph{Face down in the leaves}

We crawl through moist humus like millipedes, \\
Feasting on dirt and dead, crumbling leaves \\
While striped skies cycle through violet hues, \\
While time's kisses take the shape of a bruise. \\
Endeavors wear the warmer years away, \\
Reduced at last to heaven's dormant clay. \\
Alive, I lick brambles until my tongue \\
Tears, despairing ever being so young.

I think of you. I don't smile when I do.

A moment more and then the day is gone, \\
In evening grey, we mourn the vanished dawn, \\
And so on, maybe waiting for someone \\
To come drag us back to where we belong. \\
In dreams we interred, with your pure throat bare, \\
I know your breath, your jasmine-scented air. \\
Alive, a god to mites and mud-daubers. \\
The harvestmen scuttle and bob onwards.

\parencite[9]{leaves}
\end{verse}

%<!-- RELATED -->

%"To Autumn" verse 1 by Keats

\begin{verse}
Season of mists and mellow fruitfulness, \\
\vin Close bosom-friend of the maturing sun; \\
Conspiring with him how to load and bless \\
\vin With fruit the vines that round the thatch-eves run; \\
To bend with apples the moss'd cottage-trees, \\
\vin And fill all fruit with ripeness to the core; \\
\vin \vin To swell the gourd, and plump the hazel shells \\
\vin With a sweet kernel; to set budding more, \\
And still more, later flowers for the bees, \\
Until they think warm days will never cease, \\
\vin \vin For Summer has o'er-brimm'd their clammy cells.

\textbf{TODO CITE}
\end{verse}

\begin{verse}
\emph{Intercession in Late October}

How hard the year dies: no frost yet \\
On drifts of yellow sand Midas reclines \\
Fearless of moaning reed or sullen wave \\
Firm and fragrant still the brambleberries \\
On ivy-bloom butterflies wag

Spare him a little longer, Crone \\
For his clean hands and love-submissive heart

\parencite[23]{graves_intercession}
\end{verse}

Issa says,

\begin{verse}
\emph{Akatombo} \\
\emph{kare mo yubo ga} \\
\emph{suki ja yara}

Red dragon-fly ---
He's the one that likes the evening,
Or so it seems.

\parencite[65]{issa}
\end{verse}

Issa says,

\begin{verse}
\emph{Akikaze yo} \\
\emph{hotoke ni chikaki} \\
\emph{toshi no hodo}

O winds of autumn!
Nearer we draw to the Buddha
As the years advance

\parencite[11]{issa}
\end{verse}

\begin{verse}
III \\
The blackbird whirled in the autumn winds. \\
It was a small part of the pantomime.

\parencite{blackbird}
\end{verse}

%%%%%%%%%%%%%%%%%%%%%%%%%%%%%%%%%%%%%%%%%%%%%%%%%%%%%%%%%%%%%%%%%%%%%%%%%%%%%%%

\section*{Winter}

``Now Winter comes slowly, Pale, Meager, and Old,'' Winter sings in \emph{The Fairy Queen}.

(Later) ``Prays the Sun to Restore him, and Sings as before.''

\begin{verse}
\emph{Dirt Garden}

My garden of foxtails and milk-thistle, \\
Alive and wild, more so than tended rows \\
In growth, has died. I killed them a little, \\
The crab-grass clumps, Datura and nettle. \\
``Time and time, I commit these small murders, \\
To whose benefit?'' I ask why and wonder, \\
The scent of sap on scuffed and bloody hands. \\
If I indwelt some luring scrap of land \\
Far from here, secluded, my own to call, \\
I would welcome these same weeds, one and all, \\
To plant their roots in my warm, earthen roof, \\
Just they and I, with no need of reproof, \\
And thank the thorns for making a hale fence, \\
The compost for being my winter blanket.

\parencite[5]{leaves}
\end{verse}

%<!-- RELATED -->

Issa says,

\begin{verse}
\emph{Kore ga maa} \\
\emph{tsui no sumika ka} \\
\emph{yuki goshaku}

Is this it, then, \\
My last resting place --- \\
Five feet of snow!

\parencite[37]{issa}
\end{verse}

Issa says,

\begin{verse}
\emph{Arigata ya} \\
\emph{fusama no yuki mo} \\
\emph{Jodo yori}

A blessing indeed --- \\
This snow on the bed-quilt, \\
This, too, is from the pure land

\parencite[46]{issa}
\end{verse}

\begin{verse}
\emph{Lament for Pasiphaë}

Dying sun, shine warm a little longer! \\
My eye, dazzled with tears, shall dazzle yours \\
Conjuring you to shine and not to move \\
You, sun, and I all afternoon have laboured \\
Beneath a dewless and oppressive cloud --- \\
A fleece now gilded with our commen grief \\
That this must be a night without a moon \\
Dying sun, shine warm a little longer!

Faithless she was not: she was very woman \\
Smiling with dire impartiality \\
Sovereign, with heart unmatched, adored of men \\
Until Spring's cuckoo with bedraggled plumes \\
Tempted her pity and her truth betrayed \\
Then she who shone for all resigned her being \\
And this must be a night without a moon \\
Dying sun, shine warm a little longer!

\parencite[206]{graves_poems}
\end{verse}

\begin{verse}
She, then, like snow in a dark night \\
Fell secretly. And the world waked \\
With dazzling of the drowsy eye \\
So that some muttered `Too much light,' \\
And drew the curtains close \\
Like snow, warmer than fingers feared \\
And to soil friendly; \\
Holding the histories of the night \\
In yet unmelted tracks

\parencite[143]{graves_poems}
\end{verse}

\begin{verse}
She tells her love while half asleep \\
In the dark hours \\
With half-words whispered low:

As Earth stirs in her winter sleep \\
And puts out grass and flowers \\
Despite the snow \\
Despite the falling snow

\parencite[173]{graves_poems}
\end{verse}

\begin{verse}
I. \\
The snow is falling, \\
sleeping, \\
whispering, \\
dreaming of water.

II. \\
Gold, silver, iron, stone; \\
pure and gentle, silently melting, \\
the sun sings softly through the quiet ice.

III. \\
A single snowflake awakens, \\
shimmers, \\
glows, \\
watches the world with weary eyes, \\
darkens, \\
settles, \\
and disappears.

\parencite{esch}
\end{verse}

\begin{verse}
I \\
Among twenty snowy mountains, \\
The only moving thing \\
Was the eye of the blackbird.

VI \\
Icicles filled the long window \\
With barbaric glass. \\
The shadow of the blackbird \\
Crossed it, to and fro. \\
The mood \\
Traced in the shadow \\
An indecipherable cause.

XIII \\
It was evening all afternoon. \\
It was snowing \\
And it was going to snow. \\
The blackbird sat \\
In the cedar-limbs.

\parencite{blackbird}
\end{verse}

\section*{Cycles}

``As such, every reading of every poem, regardless of language, is an act of translation: translation into the reader's intellectual and emotional life. As no individual reader remains the same, each reading becomes a different --- not merely another --- reading. The same poem cannot be read twice [...] the poem continues in a state of restless change.'' \parencite[46]{weinberger_paz_2016}

\end{comment}

\printbibliography

\end{document}

% vim: spell
