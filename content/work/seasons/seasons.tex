\documentclass{memoir}

\usepackage{fontspec}
\setmainfont{Gentium Book Basic}

\usepackage[
  letterpaper,
  includeheadfoot,
  margin=1in
]{geometry}

\title{Seasons}
\author{Madison Scott-Clary}

\usepackage[backend=biber,style=authoryear-ibid]{biblatex}
\bibliography{seasons}

\linespread{2}

\begin{document}

\maketitle

A year spirals up.

A day, a week, a month, they all spiral, for any one Sunday is like the previous and the next shall be much the same, but the you who experiences the differing Sundays is different. It is a spiral, proceeding steadfastly onward. A day is a spiral, with each morning much the same as the one before and the one after. A month, following the cycle of the moon

But a year, in particular, spirals up. It carries embedded within it a certain combination of pattern, count, and duration that delineates our lives better than any other cyclical unit of time. Yes, a day is divided into night and day, and those liminal dusks and dawns, but there are \emph{so many of them}. There are so many days in a life, and there are so many in a year that to see the spiral within them does not come as easily.

Our years are delineated by the seasons, though, and the count of them is so few, and the duration long enough that we can run up against that first scent of snow late in the autumn and immediately be kicked down one level of the spiral in our memories. What were we doing the last time we smelled that non-scent? What about the time before?

The power of the cyclical nature of the year is of an importance that draws the heart onward, and that which moves the heart is fair game for poetry. The demarcations for this cycle are the two solstices, with secondary markers at the equinoxes. One finds oneself at the longest night of the year and knows that, from there onwards, it is downhill into summer.\footnote{I am not sold on this metaphor; uphill bears both positive and negative connotations, and it is difficult to say which to apply when. Ask a poet.} One finds oneself at the longest day of the year and before oneself lies cooler times.

Dwale (1979--2021; it/its) was a poet living in the Southern United States. Its work is described as focusing on ``altered states of consciousness...poverty, addiction, subjectivity, and the transience of existence'' \parencite{dwale}, though to reduce its body of work to any or all of those provides an inexact picture of its writing. This will be touched on in a future section on translation, but needless to say, this paper will focus on its work through the lens of seasonal progression. 

The concept of seasons and seasonality is well known within poetry. Exploring that is beyond the scope of this paper.\footnote{Or perhaps my abilities as a writer} To rely on synecdoche is the best one can manage with a topic so large. To that end, it is worth exploring the poetry of Dwale in such a context.

\section*{Spring}

Spring is commonly associated with newness and beginnings. New growth, new life, new warmth under a new sun. On of green things: of buds greening bare trees, of grass poking through late snows, or perhaps the greenery of gardening as one buys flats of flowers or sows vegetable seeds in the expectation of a harvest later on.

Spring is also associated with growth. It's the time when plants race toward the heavens, or leaves burst out from reanimated branches seemingly overnight. It's the time when you can almost feel your hair growing, or perhaps your dreams swelling in some sympathetic expansion of their own

And, importantly, spring is the season of expectations. The year may start on the first of January, a convenient fiction provided to us by the need to start it \emph{somewhere}, but the expectations for the rest of the year lay dormant in the mind until spring. January first is the time to make the resolutions and the rest of winter is the time to try them out, whether tentatively or with great passion, but the setting of expectations for the year doesn't come until the trauma of the year before has settled into uneasy memory ---  or, to use an outdated metaphor, expectations are not set until one stops writing the previous year on the date line of one's checks.

Although it often engaged with expectations in its work, Dwale tackles the subject of spring in the context of beginnings and growth infrequently. One small example of this comes from a short \emph{renga} that took place on Twitter:

\begin{verse}
Blackbird headed south\\
Down to the hawks and kudzu\\
Six months 'til winter

\parencite{dwale_haiku}
\end{verse}

While we are verging into the territory of summer, here, we do get a sense of those expectations settling into place, a feeling of ``ah, so the year is going to be like \emph{this}''. We also get that sense of growth and greenness with the mention of kudzu, a plant known for its rampant growth, quickly covering all it can in green.

Some of the reason for this paucity of spring-themed poetry is doubtless selection bias: a chapbook titled \emph{Face Down in the Leaves}, with its cover of frost-rimed leaf-litter, is unlikely to contain any paeans to new growth.

Instead, we are presented with works that focus on the fact that spring is also the time for harrowing. It's the time for tearing up that which was old, the earth that was compacted by time and snow, in order to make room for that growth which is going to come soon, whether we like it or not (the topic of unwanted growth is a topic for later in the year\footnote{Or perhaps later in life, when cancer may rear its ugly head. It is proving quite difficult to write about even seasons of new growth and beginnings without death-thoughts creeping in.}).

This untitled work will stand as our example:

\begin{verse}
The seasonal storms have poured upon the grassy flat, \\
The leafless stalks abound like thirsty mouths. \\
Puddles form and soon are swarmed with little fish, \\
And all the arid life has fled despair.

And here, wrapped in rain, lies the oldest soul, \\
The changes wrack his bones with painful cold. \\
His skin is like the sky at night, as many scars \\
Have marked his hide as there are glinting stars.

At once he feels his lungs become bereft of breath, \\
His daughter nudges him, to no effect. \\
She walks away rememb'ring days they stalked the plains, \\
Within her womb there grows a golden bloom.

\parencite[26]{leaves}
\end{verse}

This poem\footnote{The choosing of these four poems to focus on was originally intended to be for a music project. Every now and then, I get it into my head that maybe I can go back to writing music instead of words, and am quickly disabused of the notion when I sit down to do so. These were to be the texts for four art songs in a collection also named "Seasons".} in three stanzas is largely in an even meter, though we are presented with two instances in the first lines of the first two stanzas where that pattern is broken (``The seasonal storms'': ˘ -- ˘ ˘ and ``And here, wrapped in rain'': ˘ -- -- ˘ --).

These variations in prosody combined with the third verse being ``played straight'', such as it were, add up to a sense of growth, of rushing forward when Winter (we assume the oldest soul to be) breathes his last. Here, we might picture that final snow, Spring nudging winter, and realizing that all she has left are her memories of him and her child, Summer, still unborn within her.

\printbibliography

\end{document}
