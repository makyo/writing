\documentclass[12pt]{memoir}

% Todo:
% * Maybe something about Quaker funeral
% * Dwale asking for spooky music

\usepackage{fontspec}
%\setmainfont{Gentium Book Basic}
\setmainfont{Coelacanth}
\setmonofont{Ubuntu Mono}
%\newfontfamily\IPAfont{Liberation Serif}
\newfontfamily\IPAfont{Coelacanth}
\newfontfamily\Warnfont{Noto Serif CJK JP}
\newcommand\Warn{\Warnfont ⚠\normalfont}

\frenchspacing

\usepackage{multicol}

\usepackage[backend=biber,style=authoryear-ibid]{biblatex}
\bibliography{seasons}

\usepackage[hidelinks]{hyperref}

\usepackage[letterpaper]{geometry}
%\usepackage[
%  paperwidth=5.5in,
%  paperheight=8.5in,
%  layoutwidth=5.5in,
%  layoutheight=8.5in,
%  vmargin=0.5in,
%  outer=0.5in,
%  inner=0.75in,
%  includeheadfoot,
%  twoside,
%  showcrop
%]{geometry}

\title{Seasons}
\author{Madison Scott-Clary}

\begin{document}
\OnehalfSpacing

\maketitle

\begin{center}
  \footnotesize \emph{Content note:} this document contains frank discussions of death and grief, including descriptions of the euthanasia of a pet (marked with \Warn~).\normalsize
\end{center}

\vskip1cm

\begin{verse}
\centering
  \emph{What means death or grief} \\
  \emph{In the face of endless time?} \\
  \emph{Slow-turning seasons.}
\end{verse}

A year spirals up.

A day, a week, a month, they all spiral, for any one Sunday is like the previous and the next shall be much the same, but the you who experiences the differing Sundays is different. It is a spiral, proceeding steadfastly onward. A day is a spiral, with each morning much the same as the one before and the one after. A month, following the cycle of the moon

But a year, in particular, spirals up. It carries embedded within it a certain combination of pattern, count, and duration that delineates our lives better than any other cyclical unit of time. Yes, a day is divided into night and day, and those liminal dusks and dawns, but there are \emph{so many of them}. There are so many days in a life, and there are so many in a year that to see the spiral within them does not come as easily.

Our years are delineated by the seasons, though, and the count of them is so few, and the duration long enough that we can run up against that first scent of snow\footnote{Scientists have described the `scent of snow' as the air being too cold for the olfactory system to register scents.} late in the autumn and immediately be kicked down one level of the spiral in our memories. What were we doing the last time we smelled that non-scent? What about the time before?\footnote{Seasons being a handy way to count the years, I am, at time of writing, more than `the time before' years gone from when I last smelled this back in Colorado, taking a walk to clear my head after yet another argument in the Writers' Guild chat, leaving it to the mods. There is perhaps something to be said about the inevitability of a spiral.}

Or perhaps one thinks across the spiral. One, stuck in Winter, thinks back to Summer --- ah, such warmth! --- and tries to remember what it was one was doing then. ``Only silhouettes show / in the billowing snow,'' Dwale writes \parencite[19]{leaves}. ``Remembering months, now / gone when new blooms would grow.''

The power of the cyclical nature of the year is of an importance that draws the heart onward, and that which moves the heart is fair game for poetry. The demarcations for this cycle are the two solstices and two the equinoxes. One finds oneself at the longest night of the year and knows that, from there onwards, it is downhill into summer.\footnote{I am not sold on this metaphor; both uphill and downhill bear positive and negative connotations, and it is difficult to say which to apply when. Ask a poet.} One finds oneself at the longest day of the year and before oneself lies cooler times.

Dwale (1979--2021; it/its) was a poet living in the Southern United States. It was moderator for and, for a term, president of the Furry Writers' Guild, and was known for facilitating the `coffeehouse chats', hour-long lectures surrounding various writing topics that took place twice a week. Its work is described as focusing on ``altered states of consciousness...poverty, addiction, subjectivity, and the transience of existence'' \parencite{dwale}, though to reduce its body of work to any or all of those provides an inexact picture of its writing. This will be touched on in a future section on translation, but needless to say, this paper will focus on its work through the lens of seasonal progression. 

The concept of seasons and seasonality is well trod within poetry. Exploring that is beyond the scope of this paper.
\begin{comment}
\footnote{Or perhaps my abilities as a writer}
\end{comment}
To rely on synecdoche is the best one can manage with a topic so large. To that end, it is worth exploring the poetry of Dwale in such a context.

\clearpage

\section*{Spring}

A season of newness and beginnings: new growth, new life, new warmth under a new sun.

A season of green things: buds greening bare trees, grass poking through late snows, the greenery of gardening as one buys flats of flowers or sows vegetable seeds in the expectation of a harvest later on.

A season of expansive growth, when plants race toward the heavens, or leaves burst out from reanimated branches seemingly overnight. It's the time when you can almost feel your hair growing, or perhaps your dreams swelling in some sympathetic expansion of their own.

And, importantly, a season of expectations. The year may start on the first of January, a convenient fiction provided to us by the need to start it \emph{somewhere}, but the expectations for the rest of the year lay dormant in the mind until spring. January first is the time to make the resolutions and the rest of winter is the time to try them out, whether tentatively or with great passion, but the setting of expectations for the year doesn't come until the trauma of the year before has settled into uneasy memory --- or, to use an outdated metaphor, expectations are not set until one stops writing the previous year on the date line of one's checks.

Although it often engaged with expectations in its work, Dwale tackles the subject of Spring in the context of beginnings and growth infrequently, seeming to prefer Autumn.\footnote{Dwale tweeted,\footnotemark~``Pondering on why my poetry features dead leaves so prominently, I think it was because the area in which I grew up was heavily forested. Dead leaves were everywhere, so they became like a classical element: earth, wind, water, fire, and so, so many dead leaves.'' \parencite{dwale_leaves_tweet}, which provides some insight into this.}\footnotetext{A month before it died, no less.} One small example of this comes from a short \emph{renga} it took part in on Twitter:

\begin{verse}
Blackbird headed south\\
Down to the hawks and kudzu\\
Six months 'til winter

\parencite{dwale_haiku}
\end{verse}

While we are verging into the territory of summer here, as ``six months 'til winter'' implies, we do get a sense of those expectations settling into place, a feeling of ``ah, so the year is going to be like \emph{this}''. We also get that sense of growth and greenness with the mention of kudzu, a plant known for its rampant growth, quickly covering all it can in green.

Blackbirds, while often showing up in the context of winter --- there is something about the contrast, the beat of wings against the stillness of snow-dulled landscapes --- do occasionally make their presence known in writings that take place during other seasons. Stevens, for example, has

\begin{verse}
XII \\
The river is moving. \\
The blackbird must be flying.

\parencite{blackbird}
\end{verse}
wherein the thought of a river moving again being of note implies a thaw after a long winter, a world in which this could not possibly be the case without the blackbird also flying. There is a movement thawed, here.

Some of the reason for this paucity of spring-themed poetry is doubtless selection bias: a chapbook titled \emph{Face Down in the Leaves}, with its cover of frost-rimed leaf-litter, is unlikely to contain any paeans to new growth.

Instead, we are presented with works that focus on the fact that spring is also the time for harrowing. It's the time for tearing up that which was old, the earth that was compacted by time and snow, in order to make room for that growth which is going to come soon, whether we like it or not (the topic of unwanted growth is a topic for later in the year\footnote{Or perhaps later in life, when cancer may rear its ugly head. It is proving quite difficult to write about even seasons of new growth and beginnings without death-thoughts creeping in.})

This untitled work will stand as our example:

\begin{verse}
The seasonal storms have poured upon the grassy flat, \\
The leafless stalks abound like thirsty mouths. \\
Puddles form and soon are swarmed with little fish, \\
And all the arid life has fled despair.

And here, wrapped in rain, lies the oldest soul, \\
The changes wrack his bones with painful cold. \\
His skin is like the sky at night, as many scars \\
Have marked his hide as there are glinting stars.

At once he feels his lungs become bereft of breath,\footnote{When its friends learned of its passing, many of us decided to memorialize it with poetry of our own \parencite{memorial}. While I lack the feel, my attempt also incorporated the loss of breath: \begin{verse}
Beneath that evening's breeze the sickly sweet \\
\vin and brazen scent of countless flow'rs \\
awoke inside of you a darkened sleep \\
\vin Of dreams dug deeper than the soil. \\
Oh, we are waking minds who missed that scent! \\
\vin What hope have we who wait in life, \\
who sit and pray and watch for your next breath? \\
\vin Our hope can only reach for ends --- \\
To wit, to see you wake and meet a mind \\
\vin Too keen to weed a garden clean --- \\
For we exhaled when you breathed in that breeze \\
\vin and flowers wreathe your sleeping form.\end{verse}\par
Perhaps it is the idea of the cessation of the cyclical nature of breath that brings with it thoughts of death.} \\
His daughter nudges him, to no effect. \\
She walks away rememb'ring days they stalked the plains, \\
Within her womb there grows a golden bloom.\footnote{A dandelion, perhaps, those yellow suns dotting perfect fields a perennial memory of summer. As May put it:\begin{quote}
``What I love is their scent.'' She held it up for em to sniff. ``They smell like muffins. How can anything that smells like muffins be bad?''\par
\parencite[162]{toledot}
\end{quote}}

\parencite[26]{leaves}
\end{verse}

This poem\footnote{The choosing of these four poems to focus on was originally intended to be for a music project. These were to be the texts for four art songs in a collection also named ``Seasons''. Every now and then, I get it into my head that maybe I can go back to writing music instead of words, and am quickly disabused of the notion when I sit down to do so. The Madison who wrote music has long since passed.} in three stanzas is largely in an even meter (sometimes iambic, sometimes trochaic), though we are presented with two instances in the first lines of the first two stanzas where that pattern is broken (``The seasonal storms'': ˘ -- ˘ ˘ and ``And here, wrapped in rain'': ˘ -- -- ˘ --). When this is taken with the middle verse's rhymes and other examples of assonance (`become'--`bereft'--`breath' stands out), we pick up a sense of a stumble mid-gallop. Although the procession of time may be linear, the procession of the seasons may be interrupted by little stalls, little snowy loops back into winter as spring presses on towards summer.

These variations in prosody combined with the third verse being ``played straight'', such as it were, add up to a sense of growth, of rushing forward when Winter (we assume the oldest soul to be) breathes his last. Spring nudges him, and realizing that all she has left are her memories of him and her child, Summer, still unborn within her, walks those plains with only his memory.

This, after all, would be her new beginning. She is no longer bound to winter as she might have been before; there are to be no more of those loops back into snow, she's on her own now, pacing into the grassy flat with its puddles of fish.

Issa says,

\begin{verse}
\begin{multicols}{2}
\emph{Mi no ue no} \\
\emph{tsuyu to mo shirade} \\
\emph{hodashikeri}

\columnbreak

Heedless that the dews \\
mark the passing of our day --- \\
we bind ourselves to others
\end{multicols}
\vspace{-1em}
\parencite[11]{issa}
\end{verse}

Spring is nothing without Winter. Even when it has its own snows, Spring is what it is specifically because it isn't Winter. There's that vernal equinox and then suddenly the days are longer than the nights, the world begins anew, and all that is in it does so as well. As with us: we are nothing without those around us, and we are us specifically because of those in our lives. There is our meeting and then suddenly that which makes us \emph{us} is fuller than before, and we carry within us the golden bloom of who we are to become.\footnote{Or, to continue to use Dandelions as an example, the seeds we are to leave behind to grow in others, borne on warm breezes.}

We are the seasons that comprise our lives. We are beholden to the passing of our days as they are, yes, but we are also unable to truly, truly begin something anew. We are are also comprised of that which came before, and are bound to those around us.\footnote{After all, I was bound to Dwale; that's why this essay exists. That's why what little poetry I have exists. I could appreciate the music within poetry, but it wasn't until I met Dwale, became bound to it in friendship, that was able to understand poetry better on its own terms.}

Also throughout Dwale's seasonal work is the concept of vegetation. In spring, we have the grass, those leafless stalks that open up with the rain.

Here, this new grass is anthropomorphized: as new grass grows, it unfurls from the curl that it was before, forming almost a funnel which, in this instance, becomes a thirsty mouth. They live lives as full as Spring and Winter do in our poem, and one might picture their journey from thirsty mouths to rattle-dry stalks, dusty and tan, as summer fades.

Issa says,

\begin{verse}
\begin{multicols}{2}
\emph{Ukigusa ya} \\
\emph{ukiyo no kaze no} \\
\emph{iu mama ni}

\columnbreak

Floating weeds, as blow \\
the winds of the floating world --- \\
drifting and drifting
\end{multicols}
\vspace{-1em}
\parencite[18]{issa}
\end{verse}

There is some world that is not ours superimposed on the one we live in. This floating world is that which shows plants as the thirsty mouths that they are, shows the floating weeds as integral parts of the world, rather than some thing to be removed.\footnote{Something about the numinous inspires reading the animate into the inanimate (if plants could be called such) and no one that I have talked to who dwells on their sense of the numinous can either explain or deny this. Wands of living wood! The true cross! The tree of life! Secret lives of secret cells keep hope alive that one day I might speak with you again. All four seasonal poems dwell on this.} Perhaps this is the one that plants experience most clearly, where Spring may nudge Winter and, finding him dead, walk out into new grass and memories.

It is this world that poetry most clearly provides a glimpse into. It contains those symbols which pass fleetingly through our lives, drifting on by as the seasons progress.

``Here is the difference betwixt the poet and the mystic,'' Emerson cautions. ``That the last nails a symbol to one sense, which was a true sense for a moment, but soon becomes old and false.\footnote{I will admit that I veer towards mysticism, here. ``Mysticism consists in the mistake of an accidental and individual symbol for a universal one,'' Emerson goes on to say, and I will not deny my propensity toward doing so, but such is the problem with an essay. Can you really blame me for wanting to pin down the love of lost friends lest it squirm away into nothing, into some dusty old box high up on a shelf labeled simply `regrets'?} For all symbols are fluxional.'' \parencite[33]{emerson} We have in Dwale's work a glimpse of the symbol of the hard death of Winter, of Spring with Summer in her belly.

%%%%%%%%%%%%%%%%%%%%%%%%%%%%%%%%%%%%%%%%%%%%%%%%%%%%%%%%%%%%%%%%%%%%%%%%%%%%%%%

\clearpage

\section*{Summer}

As the year continues on its upward spiral, we come to one of those strange apogees of the longest day. Strange because yes, of course it bears meaning as the longest day, and yet the start of Summer never seems to fall directly on that day, does it? There is doubtless some good reason that, at least here, that is the first day of summer rather than midsummer.

And yet even that isn't always accurate, is it? Some years, summer doesn't feel like it has truly hit until well into July, when the temperatures climb and the rain becomes a distant memory.\footnote{And perhaps your well dries out when you head out of town for you husband's surgery, so your dog-sitters to have to figure out water, leaving you to fret and pace around the hotel room, and maybe that's the time you decide, ``You know what? Work is so terrible that I think I'll apply for grad school.'' But you have to provide a sample of analytic writing to do so, so you pick one of your friend's poems to analyze, and two weeks later --- when you've come home to no water and a dog whose health is steadily declining though you don't know it yet --- your friend is dead.} You're left feeling miserable for weeks on end, wishing for even a drizzle to quench your thirst, or even a bit of cloud cover at night, enough to maybe knock the temperature down into the low seventies so you can finally, \emph{finally} get some sleep and yet the days spiral forwards through heat-haze.

\begin{verse}
Summer, season of hot insomnia, \\
That much never seems to change at all. \\
Laying awake in the red desert night, \\
I shape forest from shade and wait for fall.

Ten years now gone,\footnote{It's 2022 as I write this, which means that, come September, it will have been ten years since Margaras died.\par His was the first death that really hit me. The first one I was really able to comprehend. Koray came into the bar, asked if this was the place he would have frequented, passed on the news, and then left.\par It was crushing. It destroyed me. I am still not entirely sure why, since we were friends, yes, but we were hardly so close as to warrant the reaction that I had, and yet I did.\footnotemark\par And yet I did and now, a decade later, I only think of him on the anniversary or when I come across the notifications I have from him and from Koray. Maybe that's why there's that worry about the box labeled `regrets'. I have my regrets for Margaras, and the amount by which those are outweighed by the good memories is too small for my liking.}\footnotetext{Not unlike Dwale, I suppose. Perhaps a good chunk of this --- of both of their deaths --- is due to just how little I interacted with them through anything other than text. I met Dwale once in person, and never met Margaras. I listened to Margaras's music and listened to audio versions of Dwale's stories, but other than that, they were relegated to words on a screen.} and who thought I would miss \\
Cricket songs, cicadas and katydids? \\
Then I'd gladly have grabbed a big hammer, \\
Smashed them flat as Pinocchio's conscience.

Testing palisades of clocks and yardsticks, \\
No advent waits for the restive dreamer. \\
I bandage my tattered, bitten left hand \\
And shed the smoke rings on my cloven finger.

\parencite[8]{leaves}
\end{verse}

The poem follows a similar structure to that chosen for Spring: three stanzas of four lines each, often falling back into a stressed-unstressed (or vice versa) meter, though far more free. We have a few more near rhymes, (`at all' and `for fall', and, to a lesser extent, `dreamer' and `finger'), plus a few pleasing instances of alliteration (`\emph{cri}-\emph{cket}\ldots{}ci-\emph{ca}-\textbf{das}\ldots{}\emph{ka}-ty-\textbf{dids}').

Also as before, there is a volta in the third verse. Whereas with Spring, we switched point of view from Winter to Spring, here, we switch away from from the concrete world and into something more abstract. Where we start with hot deserts, forest shade, katydids and hammers, now we are confronted with unknown tools of measurement, dreams, and smoke rings. We have that which defines itself in the external world and that which we define internally, and with those two poles, we are left to extrapolate what is between them.

Issa says,

\begin{verse}
\begin{multicols}{2}
\emph{Natsuyama ya} \\
\emph{Hitori kigen no} \\
\emph{Ominaeshi}

\columnbreak

On the hill of summer \\
Stands the slender maiden flower \\
In a solitary humor
\end{multicols}
\vspace{-1em}
\parencite[65]{issa}
\end{verse}

The slender maiden flower is the slender maiden flower. We have no say in its existence except that we might pick it, trample it, or leave it be. It is itself, in all its glory --- or at least all its solitary humor. The flower defines itself and though we may take action on it, may think it beautiful or ugly or lonely or austere, that doesn't matter to the flower.\footnote{For a while, I was quite caught on the idea that others have agency of their own. Of course they do, I mean, I just found it marvelous that this was the case. There was no way that they could not, right? They live and love and feel just as much as I do, so I can't say that this same applies to people; they define themselves, sure, but they can actively change how I create meaning from their existence.\footnotemark}\footnotetext{Of course, having written this, I feel bad for the flower. Perhaps it desperately wants to be seen as austere instead of lonely, as beautiful instead of ugly. Ask a botanist.}

``Summer, season of hot insomnia / That much never seems to change at all'' speaks well to this. Summer is Summer. It is the season of hot insomnia and it doesn't care how tired we are. It's not that it is inimical to us so much as existing within its own external nature. It exists in that floating world that is separate from us. It does not know us, it knows only itself. It's hyperreal, perhaps, only casting its shadow into our reality.\begin{comment}\footnote{It was briefly hinted by its partner that Dwale's death may have been an overdose --- or at least a drug-induced accident --- due to a heroic dose of DXM. I don't know if this is true, and the tweet where she suggested this has since been deleted. Hell, I don't even know if I was meant to see that information. Still, it sticks in the mind. On mentioning my own explorations into mushrooms as a way to set aside, for a few weeks, the obstinate burden of depression, Dwale readily agreed that this would probably be good for me. It seemed knowledgeable, as though it knew that smear of other reality (or hyperreality) well enough.\par If this story of its death \emph{is} the case, then that makes me wonder just what those last hours, last minutes, last seconds might have contained. Was it filled with pleasure? Did it, as a Muslim, feel closer to Allah? Did it know what it had done, panic, and try to retreat? Terror hems these thoughts in.}\end{comment}

``Sleep, or don't.'' Summer yawns, lingers beneath the eaves and between still branches, bothers not with such as us.

Issa says,

\begin{verse}
\begin{multicols}{2}
\emph{Mi no ue no} \\
\emph{kane tomo shirade} \\
\emph{yusuzumi} 

\columnbreak

Heedless that the tolling bell \\
Marks our own closing day --- \\
We take this evening's cool
\end{multicols}
\vspace{-1em}
\parencite[39]{issa}
\end{verse}

This is the inverse, the other pole of our spectrum. Whether or not the bell tolls for us and our day, whether or not the evening's cool is of that floating world, we still can define ourselves and our actions in the face of it. We are the ones who can take that cool as some small respite from the hot insomnia that the Summer might otherwise offer. We can define ourselves in that context, and by that, we can define the world around us.

In this sense, the cool evening and the end of our day --- indeed, the season of hot insomnia that never changes --- is something over which we can layer an artificial definition. The semiosis in play allows us to turn Summer into a sign that we can interpret. Our artificial definitions apply to us, even if the heat of the day doesn't give a damn about us. ``Testing palisades of clocks and yardsticks, / No advent waits for the restive dreamer'' because we restive dreamers are only able to measure by our artificial definitions.

But that cannot be all. There has to be more than the external and natural, that which defines itself, and the internal and artificial, that which is defined by us. We smash the insects flat with a hammer, correct? We build air-conditioned bedrooms to be able to get our sleep, correct? What is in the middle is agency. It is the permission we give ourselves to form these definitions in cooperation with the world around us. We can cry out at the sight of blackbirds bursting from the trees, because that is a thing that we have the power to do, ourselves:

\begin{verse}
X \\
At the sight of blackbirds \\
Flying in a green light, \\
Even the bawds of euphony \\
Would cry out sharply.

\parencite{blackbird}
\end{verse}

It is the act of taking meaning from each other, as well, for each of us has our own agency: we can interact with each other and influence each other's definitions of ourselves.\footnote{Viz. me meeting Dwale in the writers' guild and deciding --- actively deciding --- that I would like to be its friend. It wasn't lacking, and neither was I, but something about someone who might choose `it/its' as pronouns, someone who could engage with poetry in a way that had always eluded me. Doubt nips at my heels, though. Is ``deciding to be someone's friend'' a normal thing to do? Was that weird? Did it resent me for-- but I shouldn't be thinking like this.}

As that golden bloom of Summer\footnote{Of dandelions:\begin{quote}``Of course. They are a weed, yes. Or often thought of as one. The leaves make a good salad, though, and I was told that you could dry, roast, and grind the roots to make a coffee substitute.''\par
\parencite[161]{toledot}\end{quote}\par
They are death in Summer, I've always felt. I was always supposed to kill them, and they were always the sign of a dead lawn. Still, I read all about them on realizing how good they smelled and grew my little obsession. I passed it on to the characters in my books, and let them feel out that connection to death so that I could do so from a distance.} defines itself as all things must, and we have to take it at its word. We can kvetch about the insomnia of Summer, that which makes us sweat through the sheets so that the thought of touching someone else makes one feel clammy and disgusting\footnote{Just me? No? Maybe just me.} all we want, but that doesn't mean anything to Summer. It just also doesn't stop us from layering our own definitions atop that.

%%%%%%%%%%%%%%%%%%%%%%%%%%%%%%%%%%%%%%%%%%%%%%%%%%%%%%%%%%%%%%%%%%%%%%%%%%%%%%%

\clearpage

\section*{Autumn}

Autumn bears a strange dichotomy of plenty and impending naught. In Autumn, we harvest. We think of squash and gourds. We think of wheat, rye, corn, those fields all tan and gray. Those rattle-dry stalks we met in spring are born here.

The grain is in the silo. The gourds and potatoes are in the cellar. The fruit has been canned, the hay mown and baled, and we have never seen so much food, it seems.

And yet now is the time we consider empty stomachs. There is a particular Autumnal anxiety\footnote{Or perhaps a fear. Halloween lies there, doesn't it? There is a terror to your work, something existential, but you were also a fan of horror. Your story was going to be the one that started that other fiction podcast we were planning on, where bummers were welcome to complete the dichotomy\footnotemark~with The Voice of Dog where there were none.\par
I don't know why I associate you so heavily with both terror and horror. You were a delight to be around, and your work is not \emph{all} terror or horror. I wouldn't call your personality dark, or at least no darker than fallen leaves-- but I am getting ahead of myself.}
\footnotetext{``I had read the sign,'' I wrote for one of my only attempts at horror/terror \parencite{plu}. ``And had immediately fallen down into the space defined by that dichotomy, the gap between had-to-be and could-not-be. Dichotomy? Dialectic? There was no telling anymore, no matter how many times I'd tried to paste one word or the other onto the two phrases. Were `dichotomy' and `dialectic' a dichotomy or dialectic?''\par
Clearly, I'm still shaky on the difference, despite those seven weeks in DBT (the D stands for `dialectical', after all), but at least I recognize it; I can just dwell in that space between two truths. Best I can do when I'm about to write however many hundreds of words on dialectics/dichotomies.} that lays bare future hunger and says, ``See? It doesn't matter how much you have stored away. This is Winter.''

It's easy to lean on one or the other. Keats, for example, is impressively himself about the whole season:

\begin{verse}
\emph{To Autumn}

Season of mists and mellow fruitfulness, \\
\vin Close bosom-friend of the maturing sun; \\
Conspiring with him how to load and bless \\
\vin With fruit the vines that round the thatch-eves run; \\
To bend with apples the moss'd cottage-trees, \\
\vin And fill all fruit with ripeness to the core; \\
\vin \vin To swell the gourd, and plump the hazel shells \\
\vin With a sweet kernel; to set budding more, \\
And still more, later flowers for the bees, \\
Until they think warm days will never cease, \\
\vin \vin For Summer has o'er-brimm'd their clammy cells.\footnote{I know that this line has little to do with cells in the biological sense, but how poetic a description of cancer!\footnotemark~Cells living in eternal summer, growing and growing, over-brimming in unchecked autolysis.}
\footnotetext{\Warn~They said it was just a lipoma, and then they stopped looking. Even though we told them she'd had a lipoma removed from atop her head back when we adopted her, back when she was a puppy, they stopped looking. They stopped looking! They said she was too fat, said as they peered over their imagined glasses at us, as though it were our fault that she was no longer so svelte, and then they sent us home. They sent us home! They said it was a benign lump and that German Shepherds just get those sometimes, that she was just too fat because they can be such couch potatoes, and then they stopped talking to us because they were too busy, too busy, too busy. A year later, she had slowed down to the point where she refused to go outside. She began spending all day, all night in the bathroom. That last day, her gums turned white and her belly was visibly swollen. That last night, she died\footnotemark~in my arms.}
\footnotetext{\Warn~I know that I'm trying to square what I have of Dwale with its death, but when Falcon died in my arms less than six months later, then I really, \emph{truly} knew what death looked like, and now I have to square that with Dwale's passing as well. Did it, too, cry? Did it, too, try to hide? When it breathed its last, did it slump over to the side and stay warm far longer than one might expect? There was no one there to chide us and send us home that I can blame; there's no cancer, if that ephemeral mention is to be believed, that lurked beneath the surface. It was and then it wasn't, and the only referent I have is a dog who died too young. I'm ashamed that I can't help but make the comparison.}

\parencite[249]{keats}
\end{verse}

While Stevens is much more austere about the whole season:

\begin{verse}
III \\
The blackbird whirled in the autumn winds. \\
It was a small part of the pantomime.

\parencite{blackbird}
\end{verse}

In Keats's work, we see the lush language that we expect out of a romantic poetry. Even in a free meter, there is a sharp focus on technique that one expects from Keats in particular, with well-balanced assonance of both nasals (/{\IPAfont m}/, /{\IPAfont n}/) and sibilants (/{\IPAfont s}/, /{\IPAfont z}/, /{\IPAfont ʃ}/) leading to a sense of fullness, or perhaps the final warm breeze of the year.

The winds in Stevens's verse are not warm, though. With the aforementioned austerity, we are given one of the first cold winds of the year, and we see that the trees have lost their leaves already, miming against the sky as they are.

While I hesitate to say that Dwale walks a middle path here, its work does feature elements of both plenty and paucity. By establishing these two poles, we can then begin to triangulate where the poet believes Autumn lies.\footnote{This, after all, is what I'm trying to do, I think. I can't ask it where Autumn lies. I can't ask it if it feels the same way about the onrushing cold that I do, about saying farewell to the heat of Summer. I can't ask it if its moods are still defined by the school year, as mine are, these many years gone, with stress peaking around what used to be the end of term and depression creeping in around that first week of school. I can't ask it many things. I can't ask it anything.}

\begin{verse}
\emph{Face down in the leaves}

We crawl through moist humus like millipedes, \\
Feasting on dirt and dead, crumbling leaves \\
While striped skies cycle through violet hues, \\
While time's kisses take the shape of a bruise. \\
Endeavors wear the warmer years away, \\
Reduced at last to heaven's dormant clay. \\
Alive, I lick brambles until my tongue \\
Tears, despairing ever being so young.

I think of you.\footnote{By its absence, I feel its presence, and yet I continue to try and gaslight myself into believing that it never existed. Is it gone? It must be. Was it ever there, though? Was it a real person? Was it someone so grounding that I felt childish before it? Was it someone I had the chance to meet back in 2015, where I stared longingly at its kosovorotka in gold-trimmed black, wishing I was brave enough to wear something like that? We'll never know, I suppose. One more thing I'll never be able to ask you.} I don't smile when I do.\footnote{Maybe I will, some day. I'd sure like to think so.}

A moment more and then the day is gone, \\
In evening grey, we mourn the vanished dawn, \\
And so on, maybe waiting for someone \\
To come drag us back to where we belong.\footnote{After all, ``Would God that I had died for thee'' (2 Samuel 18:33, KJV) is a sentiment at least 2,400 years old.} \\
In dreams we interred, with your pure throat bare, \\
I know your breath, your jasmine-scented air. \\
Alive, a god to mites and mud-daubers. \\
The harvestmen scuttle and bob onwards.

\parencite[9]{leaves}
\end{verse}

For Autumn, we are greeted by the vision of plenty and naught in the form of fallen leaves. The bare trees speak to a lack, and so the leaves on the ground bear testament to this. And yet the leaves themselves are someone's plenty, are they not? The millipedes, the mites and mud-daubers, the harvestmen all have a place to live, have food for the season, even if we have already collected ours. Everything is always food for something.\footnote{Even if that something is time.} The leaves are food for the insects, and they leave behind the humus, which will be a slow food for things too small to see.

And we, perhaps, are food for that ground.\footnote{Were you buried, Dwale? I realize that I don't actually know. When Idun passed on news of your passing, she also asked what observances should be made for a Muslim who has passed. I know that expressing one's wishes for when one dies are not always something does with one's partner --- hell, I don't know that any of my partners and I have talked about it, though it \emph{is} in my will --- but it does make me wonder: were those customs upheld?\footnotemark~I realized, also, that I don't know how much of your identity was known by your family. I have to interpret your life only to the extent that I can interpret your poetry: I haven't the ear, I have only the words, and you are not around to ask.}
\footnotetext{Every time I take the long way home from the store because traffic sucks or highway 2 is too much, I think about stopping by the mosque that I pass and asking about this. It's always also couched in that selfish desire to also ask after a framework for dealing with grief.\par
When I was talking about lack of framework in the context of this essay, a friend sent me a link to a tweet wherein the poster states ``An american \emph{(sic)} is told a thousand different ways that experiencing grief is abnormal, improper, and something to be done in private on your own time.'' \parencite{grief1} This is stated in contrast to the Jewish practice of sitting shiva and the following sheloshim which provides a structured procedure for engaging with grief. Another user replied that this might just be a white, middle-class American thing: ``White Anglo Saxon Protestant based communities may lack rituals for mourning. I don't know that world. But everyone from Black Americans to Latinx to AAPI to ethnic white  communities (Polish, Italian, Ukrainian etc) have ways to mourn that aren't exactly hidden.'' \parencite{grief2}.\par
So here am I, bathed in white cultural protestantism and puritan work ethics, having nothing to hang my grief on but a desire for resolution, for even a hint at a framework. Five years after Margaras's death, when I was still trying to process what life without him would actually be like, I wrote:\begin{verse}
  \textit{Yit'gadal v'yit'kadash sh'mei raba}\\
  Would that I had the faith\\
  To pray daily.\\
  Eleven months to let you go,\\
  And an amen to end the sorrow.\par
  \parencite{uvaip}
\end{verse}
I still wish for that. I wished it then when I was trying to figure out why I was less of a person even five years on, and I wish it now that I have to mourn both Dwale and Falcon at the same time. I have nothing to lean on but confusion and words.} This idea that we, too, might be a feast of plenty to someone is not a new one --- `food for worms' is an idiom for a reason. It isn't for the world at large, and it isn't for poets. Even Dwale tackles this in the poem that will be used for Winter.\footnote{The me who is writing this from top to bottom is dreading this. I applied to grad school with the poem I plan on using, and have already bathed myself in it once, and to do so again feels exhausting before the fact.}

And yet there is another layer of lacking here: we lack the absent interlocutor. \emph{We} have buried \emph{our} dreams, here, those dreams where \emph{I} know the scent of \emph{you}. This, as before, features a turn from the external and impersonal to the internal and personal. Toward the end of the first verse, after language surrounding the world around us, we get not only an action that we take (and how delightful, that homonym in `tears'), but the feeling of despairing that comes with it.

Autumn is, it seems, a dialectic: two things can be true at the same time. Plenty and paucity. Alive and dead. Impersonal and personal. There is an eternety between each of those sets of truths, as though Autumn, more so than the rest of the seasons, holds on the longest. ``How hard the year dies: no frost yet,'' Graves writes in \emph{Intercession in Late October}. \parencite[23]{graves_intercession} ``Spare him a little longer, Crone / For his clean hands and love-submissive heart.''\footnote{Who knows how much of my skittishness around winter is a me thing or an us thing. Spare me a little longer.} 

Issa says,

\begin{verse}
\begin{multicols}{2}
\emph{Akatombo} \\
\emph{kare mo yubo ga} \\
\emph{suki ja yara}

\columnbreak

Red dragon-fly --- \\
He's the one that likes the evening, \\
Or so it seems.
\end{multicols}
\vspace{-1em}
\parencite[65]{issa}
\end{verse}

Despite being the in-between of Summer and Winter, something that seems as though it ought to be a smooth transition between hot and cold as Spring tried to be, Autumn steadfastly refuses to be anything other than its own entity. We are unsure\footnote{After all, I think our well was out into Autumn, or maybe it had \emph{just} recovered. We were borrowing water from the neighbors for the dogs --- Falcon, who was dying, and Zephyr, who probably knew. I had burnt out so hard at work I had to take a leave of absence, had to spend sixteen hours a week in therapy, and on going back to work realized I still hated everything. I'm unsure even now whether life would have been easier without that grief. There is now dialectic between you being alive, of course, but there is this dialectic within me being unsure of whether or or not I've processed your death.\footnotemark~Sometimes I have, and sometimes I have to stop writing this essay for five days because looking at it makes me cry.}\footnotetext{\Warn~Ditto with Falcon. Sometimes I'm able to make it an entire day not thinking about her, and then I'll be laid low by an evening of flashbacks, the way she slumped to the side, just how long her body stayed warm\ldots} of whether or not we like Autumn;\footnote{Despite what Autumn bitches would have you believe.\footnotemark}\footnotetext{It's me. I'm bitches.} surely some seem to, but this duality makes it elusive. Rather than shy away from it and decide to let it sit or cleave to it and enjoy every minute, we always have a little bit of that space between ourselves and the season, a little bit of that eternity.

We think of it. We don't smile when we do.

Issa says,

\begin{verse}
\begin{multicols}{2}
\emph{Akikaze yo} \\
\emph{hotoke ni chikaki} \\
\emph{toshi no hodo}

\columnbreak

O winds of autumn! \\
Nearer we draw to the Buddha \\
As the years advance
\end{multicols}
\vspace{-1em}
\parencite[11]{issa}
\end{verse}

We are helpless before the onward spiral of the year.

%%%%%%%%%%%%%%%%%%%%%%%%%%%%%%%%%%%%%%%%%%%%%%%%%%%%%%%%%%%%%%%%%%%%%%%%%%%%%%%

\clearpage

\section*{Winter}

``Now Winter comes slowly, Pale, Meager, and Old,''\footnote{I wish you had died in Winter, Dwale. I wish you'd lived to comfort me through Falcon's death. Hell, I wish you'd lived to comfort me through your \emph{own}. I wish you'd lived to the winter of your life, not to a mere 42 years. The very beginning of Autumn, for you! You had your plenty and your paucity. I wish you'd made it to `Pale, Meager, and Old'.\par Of course I wish you hadn't died at all, but I wish you'd died in Winter if you had to.} Winter sings in \emph{The Fairy Queen} \parencite{purcell}.

Winter creeps. It eases into place. Even if there is a sudden, blustering storm to begin the season, that is but the first noise you hear. It creeps and crawls in because it cannot but creep and crawl. It's old. It's tired. ``Dying sun, shine warm a little longer!'' \parencite[206]{graves_poems} we may beg,\footnote{And perhaps do. I am not --- most of us are not --- immune to that simple desire that we have a little more time together. Another year, another month, another day. Even another hour together,\footnotemark~enough time for me to tell you that I think of you often, that you mean a lot to me, that I hope you understand it.\par 
The last time we talked one-on-one, you was a month before you died, June 5\textsuperscript{th}. You pinged me in the Guild chat asking me to DM you because you couldn't find my chat in your client: ``Could you DM me right quick?''. I sent you a surprised-looking sticker and you said, ``Uh, just wanted to say that although I don't know what I ever did to deserve having your support, I do see and appreciate it.''\par 
``You just strike me as an earnest and well-spoken. You do quite a bit to buoy others up, and that ought to be returned in kind,'' I said, but honestly, how the hell was I supposed to respond to that? That I think we both wind up in that spot where impostor syndrome becomes more dire? That any praise, any validation becomes almost too hot to touch?\par 
It's a haunting sort of message to be left with. That I should praise your work (for I think it was in response to my review of \emph{Face Down in the Leaves}) and inspire that terrifying ordeal of being seen has me confused and upset, but I don't know that there's much to be done about it. I wanted to type so much more than I did, but this wasn't the place. The conversation wasn't open after that. It had closed up, the point having been made --- inadequately by either of us, I suspect --- and then I never got to say anything about it again.}\footnotetext{\Warn~I never really got that with Falcon, either. She was sick that morning, and then we drove to the emergency vet, and she was inside for perhaps ten minutes before the vet came out and said she had perhaps six hours to live, and it was time to make a choice. Six hours of pain and a death in agony or one hour of pain, dampened by drugs to lessen the shock, and then a blissful sleep. A slump to the side, a last breath, eyes open, mouth open, warm there on the floor.\par I never got that with her. I never got it from you, only a tweet the day after you died from your partner, but it was somehow less real, less immediate, because of course it was. ``I do not know which to prefer,'' Stevens writes \parencite{blackbird}. ``The beauty of inflections / or the beauty of innuendoes, / the blackbird whistling / or just after.'' } but all it wants to do is lie down and blanket the land.

Much of the imagery in poetry around Winter picks up on this, and we commonly see instances of blankets, of beds, of rest.

Issa says,

\begin{verse}
\begin{multicols}{2}
\emph{Arigata ya} \\
\emph{fusama no yuki mo} \\
\emph{Jodo yori}

\columnbreak

A blessing indeed --- \\
This snow on the bed-quilt, \\
This, too, is from the pure land
\end{multicols}
\vspace{-1em}
\parencite[46]{issa}
\end{verse}

Perhaps it is because we so often experience Winter through the lens of contrast. We experience Winter through the warmth of fire. We experience Winter because it is \emph{out there} and we are \emph{in here} (or, failing that, we are experiencing Winter directly because we are \emph{out there} and would very much rather be \emph{in here}). We think of snow on the ground, we think of blankets because, yes, of course it looks that way, but also because we are primed to think of winter in terms of the contrasts to how cold such a blanket must be.

Or perhaps we think of Winter this way, of snow as a blanket, of sleepy silences, because the world really does seem to be asleep. It goes beyond mere hibernation; the whole world --- the Earth, the sky, the rivers and lakes --- all seem to be asleep. ``I wonder if the snow loves the trees and fields,'' Lewis Carroll writes \parencite{carroll}. ``That it kisses them so gently? And then it covers them up snug, you know, with a white quilt, and perhaps it says, ``Go to sleep, darlings, till the summer comes again.''''

Even the snow itself seems destined for sleep, drifting down lazily in fat clumps or being blown nearly sideways, helpless, only to be piled up in immobile drifts.\footnote{I go back and forth on what death must feel like. There are times when I think it must be like this slow fade to sleep. It has to have that hypnogogic quality. You could try and stay awake, but gosh, it would be nice to get some rest.\begin{verse}
Her eye turns inward, \\
vision dims and movement stills \\
as winter claims her.\par
\parencite{pale_she}
\end{verse}\par And sometimes, I think it must be like a wave, rolling up to subsume you, and it's so, so much bigger than you are that there is nothing you could possibly do to stop it:\begin{verse}
A flash of coppery sweetness, \\
A clearing of the sinuses, \\
A burst of unnamed colors, \\
A rush of creativity, of wonder, \\
Velvety softness, a low hum, \\
And then the wave recedes. \par
\parencite{rush}
\end{verse}\par And I'm sure it has much to do with the way in which one dies. Perhaps there is terror. Perhaps there is relief.\footnotemark}\footnotetext{\Warn~I'm sure that Falcon felt a bit of both. She was in so much pain, and yet she was stuck at the vet. She was delirious with pain meds. She was surrounded by the worst smells, people she didn't know. She was in a cage until we got back.\par I have no idea what you must have felt.} It's destined for sleep because what else would a blanket do?

\begin{verse}
I. \\
The snow is falling, \\
sleeping, \\
whispering, \\
dreaming of water.

{[\ldots]}

III. \\
A single snowflake awakens, \\
shimmers, \\
glows, \\
watches the world with weary eyes, \\
darkens, \\
settles, \\
and disappears.

\parencite{esch}
\end{verse}

Wallace ties in this sleepiness\footnote{And, in sleep, there is that not-knowing, not-caring. There is something of death in sleep, and for one who years for such, that bears much allure.\begin{verse}
Pale she sleeps, sleeps still. \\
Waking her may have listened. \\
Endless winter calms. \par
{[\ldots]} \par
If spring never comes, \\
pale she supposes, that's fine. \\
Winter is for dreams. \par
She'll dream, or she won't. \\
She'll carry on or she won't. \\
Cold has claimed heartwood. \par
\parencite{pale_she}
\end{verse}\par But I shouldn't be talking this way.} with contrast\footnote{Contrast, then, with wakefulness. Contrast with life. ``Find all the happy pictures of Falcon that you can,'' my therapist suggested. ``Tell yourself stories about them.''\par It works some of the time.} --- as we shall do before long --- by contrasting the stillness of a world asleep with our faithful blackbird:

\begin{verse}
I \\
Among twenty snowy mountains, \\
The only moving thing \\
Was the eye of the blackbird.

\parencite{blackbird}
\end{verse}

Similarly, Graves has,

\begin{verse}
She, then, like snow in a dark night \\
Fell secretly. And the world waked \\
With dazzling of the drowsy eye \\
So that some muttered `Too much light,' \\
And drew the curtains close \\
Like snow, warmer than fingers feared \\
And to soil friendly; \\
Holding the histories of the night \\
In yet unmelted tracks

\parencite[143]{graves_poems}
\end{verse}

``As Earth stirs in her winter sleep,'' he writes elsewhere \parencite[173]{graves_poems}. ``And puts out grass and flowers / Despite the snow / Despite the falling snow.'' Winter has crept in and tucked the world away to sleep for a while, and though we might stretch and peek out and, seeing the sun, think to ourselves, ``I really must get up,'' we are helpless to actually do so. Make attempts, sure, but there is no waking from Winter on any terms other than Winter's.

It plays into the timelessness of that serenity.\footnote{Wishful thinking.} It is so quiet!\footnote{Yet more of the same.} The contrast is so high! Morning light is the same as noon light is the same as afternoon light. How could time pass? Winter will not permit it.

\begin{verse}
XIII \\
It was evening all afternoon. \\
It was snowing \\
And it was going to snow. \\
The blackbird sat \\
In the cedar-limbs.

\parencite{blackbird}
\end{verse}

And, of course, perhaps we think of Winter this way because that very danger that keeps us inside. Winter, death-season, can have snow as a funeral shroud as easily as a blanket. We are not \emph{in here} simply because it is cold \emph{out there}, but because that very cold brings death with it.\footnote{Guidelines for reporting on suicide state that you should not report on the method with which someone killed themself. This, apparently, does not apply to the living. Terry Gross asked Allie Brosh during an episode of ``Fresh Air'' how she imagined committing suicide and, rather than keeping that close to her heart, the author explained that she had planned on freezing herself to death through a mechanism I won't describe.\par And, as the guidelines say, that has stuck with me. I think about it every time it gets cold.\footnotemark~I thought about it that night after Falcon passed. I thought about doing just as Brosh said, and finding a way to experience that very sort of blanket, that very shroud.}\footnotetext{Which, I realize, is the opposite of death-thoughts elsewhere in the year. Perhaps Autumn is the season for thinking of fire, Spring the season for leaping, and so on.}

Issa says,

\begin{verse}
\begin{multicols}{2}
\emph{Kore ga maa} \\
\emph{tsui no sumika ka} \\
\emph{yuki goshaku}

\columnbreak

Is this it, then, \\
My last resting place --- \\
Five feet of snow!
\end{multicols}
\vspace{-1em}
\parencite[37]{issa}
\end{verse}

The discursive nature of this section might itself be related to the blunted vision of the world after snow.\footnote{Or maybe just because I'm riddled with memories. They pock my surface, keep me from moving smoothly through analysis. Could I write about the season of Winter in some more cohesive manner had not the ground been covered with that shitty slush the day that Falcon passed?} How can we define the world around us when we can barely make out its edges? We cannot define Winter because it's so blurry around the edges. Obscured. Defined by contrast, but only the contrast of blackbirds or bare tree limbs, rather than one hill from the next, one house from the next. We can only pin it down by walking those paths, one by one, heading up to the top of the hill and looking down from there, walking up the drive to get a better look at the numbers tacked to the side of the house. There are so many perhapses and maybes\footnote{Better, I think, than if-onlys and if-I-had-justs. There's that wishful subjunctive, as always. Would God that I had died for thee. I don't really feel that for you, Dwale, and I don't know whether to feel sorry or grateful for that. I've felt it for others, as is perhaps obvious, and it is one of the worst feelings I've had in my life. I'm sorry that there is something about our friendship that precludes that, I guess, but I'm more grateful that I don't have that feeling associated with our memories.} to be had.

\begin{verse}

VI \\
Icicles filled the long window \\
With barbaric glass. \\
The shadow of the blackbird \\
Crossed it, to and fro. \\
The mood \\
Traced in the shadow \\
An indecipherable cause.

\parencite{blackbird}
\end{verse}

`Indecipherable' indeed.

And so, with an eye warmth and cold, to contrasts, to blankets and sleep, to softness and inexactitude, to death, we come to our final poem:

\begin{verse}
\emph{Dirt Garden}

My garden of foxtails and milk-thistle, \\
Alive and wild, more so than tended rows \\
In growth, has died. I killed them a little, \\
The crab-grass clumps, Datura and nettle. \\
``Time and time, I commit these small murders, \\
To whose benefit?'' I ask why and wonder, \\
The scent of sap on scuffed and bloody hands. \\
If I indwelt some luring scrap of land \\
Far from here, secluded, my own to call, \\
I would welcome these same weeds, one and all, \\
To plant their roots in my warm, earthen roof, \\
Just they and I, with no need of reproof, \\
And thank the thorns for making a hale fence, \\
The compost for being my winter blanket.

\parencite[5]{leaves}
\end{verse}

This fourteen-line poem is one of half-rhymes and mixed meter. We have `all' and `call', as well `roof' and `reproof' (which, depending on your accent, may not be a complete rhyme; many have roof as /{\IPAfont ʊ}/ or even /{\IPAfont ɵ}/)\footnote{My accent\footnotemark has roof as /{\IPAfont ɯ}/ vs reproof as /{\IPAfont u}/.}\footnotetext{I do not know your accent. I do not know where you came from. I do not remember your address. I do not remember your voice. I met you twice in 2015, back at the final Rainfurrest, and all I remember was your hat, your hair, your kosovorotka. I remember Mando better, and saw him only a little bit more. I remember JM introducing you as the one who wrote ``the best story in the fandom, I hear'', but that's about it. ``Behesht'', was it? The story about reaching paradise? The story of a post-apocalyptic wasteland, of the slow death of life, of the drive to press on towards something better that can only be stopped by death?\par 
I looked the story up when thinking about this, and came across the lines:\begin{quote}``Peace, my brother,'' he said. ``Come with us, and leave these wretched places behind. Where we are going is far better.''\par
When I inquired as to where that might be, he smiled and said a single word: ``Behesht.'' Their destination was nothing less than Heaven itself, the hidden garden which is the reward of believers.\par\parencite{behesht}\end{quote}\par 
That was back in 2015, though, so perhaps not. The dates don't add up. That was seven years before you died. It's one of those things where you couldn't have known. You couldn't \emph{possibly} have known, and yet I suppose you bore within yourself the seeds of your death from birth, just as we all do.\footnotemark}\footnotetext{Or, at least. I know I do. I know that I'm stuck with those death-thoughts, the ones that won't leave, will only curl up into a little purring ball in the corner of my mind, unwilling to let me out of its sight.\par 
Only, sometimes it feels it must traipse across my lap as cats\footnotemark do, bunting its head against my arms, needle-sharp claws digging into my thighs, demanding that it receive the attention its due. ``Think of me,'' it says. ``Think of me and dream of me. Pet me and stroke me. Let me know that you love me, in your own fearful way.''}\footnotetext{Death, a constant, refuses to leave. I start writing this essay on death, and then the vet calls: your cat's asthma isn't asthma, it's metastatic lung cancer. Just keep her comfortable.\par
You who read this, or perhaps future Madison, I don't write this that you feel sorry for me, but only for a little validation. Dwale dies. Falcon dies. Turtle is dying. Someone other than me must know.}, but beyond that, we get only hints of assonance: `hands'/`land'.\footnote{My accent has hands as closer to /\IPAfont{ɛ}/ vs land as /\IPAfont{æ}/.}

We come around once more to the cyclical nature of time, as subsequent re-reads of the poem cycle us through multiple meanings.

Once more, the first half of the poem focuses on concrete imagery (``My garden of foxtails and milk-thistle'', ``The scent of sap on scuffed and bloody hands'') and actions (``I killed them a little, / The crab-grass clumps, Datura and nettle'', ``I ask and wonder'') which, when contrasted against the turn toward the more hypothetical and contemplative second half, offers on second reading a sense of immediacy.

On one's first read, one is confronted with the unwelcome nature of the real and the welcome nature of the hypothetical: these are weeds that must, according to some external source, be pulled, and yet in some perfect world, one might welcome them in. In both of these cases, the tension lies in the volta halfway through, where one imagines that the poet stands up from toil, a pile of vegetation at its feet, wipes the sweat from its brow, and asks for the hundredth time, ``Time and time, I commit these small murders, / To whose benefit?''\footnote{Time and again, these small deaths! I've read my Job. I've listened to my Bernstein. I know my right as God's creation to call them to account. And yet, \emph{If I summoned him and he answered me, I do not believe that he would listen to my voice} (Job 9:16, NRSV).\par
``Lord God of Hosts, I call You to account! / You let this happen, Lord of Hosts!'' the narrator cries in Bernstein's ``Kaddish'' \parencite{kaddish}, and though that symphony caught me up in whorls of meaning, years ago when I first heard it, I don't think I understood this urge until these last few years. I lost Cullen, I lost Morgan and Tirix and Brone, I lost Dwale, I lost Falcon, and now I'm losing Turtle.\par
``Tin God, your bargain is tin! / It crumples in my hand,'' the narrator continues, and the words tear at me now. I've read my Job and I've listened to my Bernstein and I know, now, what it means to call God to account. I know what it means to weep, to pull at my clothes, at my hair. I know what it means to have food turn to ash in my mouth.}

From the second read on, however, as the reader re-evaluates the work, we know that the `garden' in the first line is more than just a wistful statement, but a more active contrast from the external source. More than letting them grow wild, would the poet perhaps plant them intentionally? A thistle provides a beautiful purple blossom, and Datura's white trumpets its own poisonous beauty; why not? Arctic foxes, by virtue of their diet, wind up planting gardens above their dens, scanty cold-weather flowers peeking through after winter.\footnote{A small obsession:\begin{verse}Arctic fox's den \\
adorned with flowers and snow \\
garden in winter\par
\parencite{arkie}
\end{verse}\par
It sticks with me, apparently. In my own writing, I've dug deep into the beauty of dandelions. Puffballs, sure, but also:\begin{quote}
``Me, though, I like the flowers. They are too complicated for their own good in this stage, are they not? Sure, they close up and then become the puffballs that spread them further and further, but here, they are almost platters of yellow.''\par
\parencite[162]{toledot}\end{quote}\par
Ioan, a few paragraphs above this, even talks of thistles.\par
Weeds are those whose goal is to cling desperately to life \emph{even} in death. Weeds don't wish for death, they accept it as inevitable more easily than us poor fools. The one speaking in that quote, after all, \emph{is} named `May Then My Name Die With Me'.}

Even reading the poem top to bottom on repeat, one picks up subsequent layers one after another. Is the poet wishing for solitude? There is this rejection of external requests for someone's imagined benefit and talk of hedging (perhaps literally) oneself in ``with no need for reproof''. Is the poet musing on death\footnote{\Warn\Warn\Warn~She was doing so bad that morning. Her belly was bloated and her gums were white and she was so lethargic. We didn't know it, but she was in shock at the time. So, we called the emergency vet and they said, ``Oh, yeah, that sounds serious. Bring her in.''\par
``I'm up in Sultan. It'll be a good half hour before I make it down,'' I said.\par
``Well, hurry.''\par
I took her outside and she awkwardly peed at the plum tree, so I waited while Zephyr and JD watched from the back door, and then, suspecting that she was maybe in too much pain to jump into the back of the car, I got a towel around her midsection and helped to lift her into it, but I don't think I did a very good job. She didn't yelp out in pain or anything, but she looked so stunned once she'd made it into the back seat. She stared up at me with eyes that showed fear, showed pain, showed some existential terror that I didn't know yet, because I think she knew. Something about the shock that she was in left enough knowledge in her, that I think she knew she was gone.\par
I buckled her in by her harness, and drove. I talked to her all the way down the hill, down to Snohomish. I told her it'd be okay, that I was hurrying, that I loved her. I told her that she'd be alright, that it was probably just bloat, and that they could do surgery to fix it, that she'd be in pain for a few weeks, but she'd survive. She was only nine, after all, right? So many more years of chasing Zephyr around. So many more years of herding the cat.\footnotemark\par
I sat with her in the back seat of the car, on hold with the emergency vet, while she rested her head on my lap. I was so cramped back there, I'm so tall, and my knees were mashed against the back of the driver's seat. She rested her head on the seat, nose hanging down into the messy footwell, and I pet her for nearly an hour as the hold music continued to play, interrupted by announcements to bring your pet in for vaccinations, please do not hassle the staff, we offer natural treatments for horses. I listened to the ``You are caller number \emph{n}'' messages, I waited while \emph{n} ticked down 7 6 5 4 3 3 3 3 3 2 2 2 1 1 and then I got to speak to the operator, letting them know that I was in the lot, and they said that I should've just pulled into one of the appointment spaces and brought her inside stupid stupid stupid and they wheeled a gurney out to meet me because they weren't sure if she could walk but she could so they put a leash on her and brought her inside and I had to wait in the car, watching as she made it in the lobby and got so scared that she urinated on the floor and lay down because she didn't want to be taken back to the X-ray machine without me and I think that's because she knew. She knew! She \emph{knew}. She was dying and she knew. She knew \newcommand{\sheKnew}{
she knew
she knew
she knew
she knew
she knew
she knew
she knew
\phantom{s}he knew
she knew
she knew
she knew
she knew
s\phantom{h}e\phantom{ }knew
sh\phantom{e} knew
s\phantom{h}\phantom{e} knew
s\phantom{h}\phantom{e} kne\phantom{w}
she knew
she knew
sh\phantom{e} knew
\phantom{s}he\phantom{ }knew
s\phantom{h}e kn\phantom{e}\phantom{w}
s\phantom{h}e knew
she kn\phantom{e}w
she\phantom{ }kne\phantom{w}
sh\phantom{e} \phantom{k}new
she \phantom{k}\phantom{n}e\phantom{w}
sh\phantom{e} knew
she \phantom{k}ne\phantom{w}
\phantom{s}\phantom{h}e\phantom{ }knew
she \phantom{k}new
sh\phantom{e} knew
she\phantom{ }\phantom{k}new
sh\phantom{e} \phantom{k}\phantom{n}ew
\phantom{s}\phantom{h}e\phantom{ }knew
s\phantom{h}\phantom{e}\phantom{ }knew
she \phantom{k}\phantom{n}e\phantom{w}
she\phantom{ }\phantom{k}\phantom{n}e\phantom{w}
she k\phantom{n}e\phantom{w}
\phantom{s}he\phantom{ }kn\phantom{e}\phantom{w}
she kne\phantom{w}
\phantom{s}h\phantom{e}\phantom{ }\phantom{k}ne\phantom{w}
\phantom{s}\phantom{h}e\phantom{ }k\phantom{n}e\phantom{w}
she\phantom{ }knew
sh\phantom{e} knew
sh\phantom{e} \phantom{k}n\phantom{e}\phantom{w}
sh\phantom{e} \phantom{k}n\phantom{e}w
\phantom{s}he\phantom{ }k\phantom{n}\phantom{e}\phantom{w}
s\phantom{h}\phantom{e}\phantom{ }\phantom{k}\phantom{n}ew
sh\phantom{e} knew
\phantom{s}\phantom{h}e\phantom{ }knew
\phantom{s}he k\phantom{n}\phantom{e}\phantom{w}
sh\phantom{e} \phantom{k}\phantom{n}\phantom{e}\phantom{w}
\phantom{s}h\phantom{e}\phantom{ }\phantom{k}\phantom{n}e\phantom{w}
sh\phantom{e}\phantom{ }k\phantom{n}\phantom{e}w
s\phantom{h}e\phantom{ }\phantom{k}n\phantom{e}\phantom{w}
\phantom{s}he \phantom{k}\phantom{n}e\phantom{w}
\phantom{s}\phantom{h}\phantom{e} k\phantom{n}\phantom{e}\phantom{w}
s\phantom{h}e kn\phantom{e}\phantom{w}
\phantom{s}h\phantom{e} k\phantom{n}e\phantom{w}
sh\phantom{e} \phantom{k}\phantom{n}e\phantom{w}
\phantom{s}\phantom{h}e\phantom{ }k\phantom{n}\phantom{e}\phantom{w}
\phantom{s}h\phantom{e}\phantom{ }\phantom{k}n\phantom{e}w
\phantom{s}\phantom{h}\phantom{e}\phantom{ }k\phantom{n}\phantom{e}w
s\phantom{h}e\phantom{ }\phantom{k}\phantom{n}e\phantom{w}
\phantom{s}he\phantom{ }\phantom{k}n\phantom{e}\phantom{w}
\phantom{s}\phantom{h}\phantom{e} \phantom{k}\phantom{n}\phantom{e}\phantom{w}
\phantom{s}\phantom{h}\phantom{e}\phantom{ }\phantom{k}\phantom{n}\phantom{e}w
\phantom{s}\phantom{h}\phantom{e} \phantom{k}n\phantom{e}\phantom{w}
\phantom{s}h\phantom{e}\phantom{ }k\phantom{n}\phantom{e}\phantom{w}
\phantom{s}\phantom{h}e \phantom{k}new
sh\phantom{e}\phantom{ }k\phantom{n}\phantom{e}w
\phantom{s}\phantom{h}\phantom{e} \phantom{k}\phantom{n}\phantom{e}\phantom{w}
s\phantom{h}\phantom{e} \phantom{k}n\phantom{e}\phantom{w}
s\phantom{h}e\phantom{ }\phantom{k}ne\phantom{w}
\phantom{s}\phantom{h}\phantom{e}\phantom{ }knew
she\phantom{ }\phantom{k}n\phantom{e}\phantom{w}
\phantom{s}he\phantom{ }\phantom{k}\phantom{n}\phantom{e}\phantom{w}
s\phantom{h}e kne\phantom{w}
\phantom{s}h\phantom{e}\phantom{ }\phantom{k}\phantom{n}\phantom{e}\phantom{w}
s\phantom{h}e \phantom{k}\phantom{n}\phantom{e}\phantom{w}
\phantom{s}\phantom{h}\phantom{e}\phantom{ }kn\phantom{e}\phantom{w}
\phantom{s}\phantom{h}\phantom{e}\phantom{ }\phantom{k}\phantom{n}ew
\phantom{s}\phantom{h}e \phantom{k}n\phantom{e}\phantom{w}
s\phantom{h}\phantom{e} \phantom{k}\phantom{n}\phantom{e}\phantom{w}
\phantom{s}\phantom{h}e \phantom{k}\phantom{n}\phantom{e}\phantom{w}
\phantom{s}he\phantom{ }\phantom{k}\phantom{n}\phantom{e}\phantom{w}
\phantom{s}h\phantom{e} k\phantom{n}\phantom{e}\phantom{w}
\phantom{s}\phantom{h}\phantom{e}\phantom{ }\phantom{k}\phantom{n}\phantom{e}\phantom{w}
\phantom{s}\phantom{h}\phantom{e}\phantom{ }\phantom{k}n\phantom{e}\phantom{w}
\phantom{s}\phantom{h}\phantom{e}\phantom{ }\phantom{k}n\phantom{e}\phantom{w}
\phantom{s}\phantom{h}\phantom{e}\phantom{ }\phantom{k}\phantom{n}\phantom{e}\phantom{w}
\phantom{s}\phantom{h}\phantom{e}\phantom{ }\phantom{k}\phantom{n}\phantom{e}\phantom{w}
\phantom{s}\phantom{h}\phantom{e}\phantom{ }\phantom{k}\phantom{n}\phantom{e}\phantom{w}
\phantom{s}\phantom{h}\phantom{e}\phantom{ }k\phantom{n}\phantom{e}\phantom{w}
\phantom{s}\phantom{h}\phantom{e}\phantom{ }\phantom{k}\phantom{n}\phantom{e}\phantom{w}
\phantom{s}h\phantom{e}\phantom{ }\phantom{k}\phantom{n}\phantom{e}\phantom{w}
\phantom{s}\phantom{h}\phantom{e}\phantom{ }\phantom{k}\phantom{n}\phantom{e}\phantom{w}
\phantom{s}\phantom{h}\phantom{e}\phantom{ }\phantom{k}\phantom{n}\phantom{e}\phantom{w}
\phantom{s}\phantom{h}\phantom{e}\phantom{ }\phantom{k}\phantom{n}\phantom{e}\phantom{w}
\phantom{s}\phantom{h}\phantom{e}\phantom{ }\phantom{k}\phantom{n}\phantom{e}\phantom{w}
}
\par 
She knew.}\footnotetext{
And now, with Turtle, we finally have a chance for something else. We have weeks, maybe months with her. We don't have eight hours of trauma that I'm sure I'll never be able to forget, that time will only blunt the impact but never the memory.\par
We know. We can journal her breathing, her energy, her mood. We can make these last weeks or months great for her. We can give her a little blob of sour cream every time she gets one of her steroid pills, a treat to go with a little bit of bitterness.\par
We know, so we can get her wet food to eat as well as dry.\par
We know, so we can invite Ash and Merry, from whom we adopted her, over to see her one last time, laugh at how Merry calls her `the dirigible'.\par
We know, and it's so, so much easier that way.\footnotemark\par
We know.}\footnotetext{
\Warn~Had we known\footnotemark with Falcon, how much more time would have made a difference? Would it still have been traumatic if we'd had a few days with her rather than a few hours? Would I have been able to bring her home with some hefty painkillers to live a little longer by our sides, or in our bathroom? Surely we could have made her life a good one, those last few days; I know I'm still glad that her last meal that morning was some of that wet food that she loved, even though it doubtless sat inside her bleeding belly, undigested. JD and I still would have laid on the floor with her and watched her die, but would JD have sobbed, ``Come back, come back''? Would I have needed --- twice! --- to step out of the room to `deal with the paperwork' just so I wouldn't be around her still warm body?}\footnotetext{
And what about with you, Dwale?\par
With Morgan, we knew, yes? We had \emph{years} to prepare for it, because she lived so much longer, so much better than we had thought. Cancer claimed her in the end, but we knew, and she worked for FC, and I saw her that last time with her hair only freshly grown back from her recent chemo, and she laughed with me. That was enough time for me to compose a mental goodbye, even if only for myself.\par
We didn't know with Margaras, and that hit me for days and days and days. We didn't know for Cullen, and that hit me for weeks and months.\par
And we didn't know for you, and now I'm doing my level best to process that through words --- who knows how successfully, because some part of me is trying to convince the rest of me that this isn't actually doing the work, that bathing in the grief in an attempt to define its center isn't moving on --- and hoping against hope that that fantastical place where grief no longer claws at my insides is closer rather than farther away.} when confronted with vegicide? An ``earthen roof'' has plain enough meaning.

And yet, even Winter must die, yes? That, after all, is right where we started. Winter, dead, and Spring with Summer unborn in her belly. The poem is bound in cycles. Our reading is bound in cycles. The year is bound in cycles and as we spiral up through the seasons, they leave us changed. We are not who we were last Winter, and yet it is Winter still. Every Winter is different, and in that they are the same.

How hard the year dies, and yet there is Spring. She has walked the grassy flat with him for the last time and, the golden bloom of Summer\footnote{Perhaps even a dandelion.} in her womb, has naught to do but nudge him to no effect.

%%%%%%%%%%%%%%%%%%%%%%%%%%%%%%%%%%%%%%%%%%%%%%%%%%%%%%%%%%%%%%%%%%%%%%%%%%%%%%%

\clearpage

\section*{Spiral}

To return to Spring, to make it through that cycle of growth, of insomnia and harvest and frost, is to stand at a precipice. It is to stand right up against the edge of that spiral, lean over carefully, peer down into the depths from however many storeys up, and wonder. It is to confront memory in the form of heights. It is to regard the spiraling days, weeks, and months to either side of you, give them the acknowledgement they deserve, and then return to peering down into the depths.

To return to Spring is to hit that vernal equinox, look down, and feel the steam of memory, the heat of the last year, washing up over your face. What was that album again? ``Memories Come Rushing Up to Meet Me Now''?

We remember last Spring. We remember Autumn, because there it is across from us. We remember Winter and Summer through some haze --- Winter is still too fresh, Summer so long ago --- by peering through the haze of the days around us.

It is nearly a cliché to try and find the spirals in everything. A snail's shell! A nautilus! Find the Golden Spiral in every hidden cloister.

A cliché is a metaphor, though, and a metaphor is a framework upon which much can be built. We can look at the cliché that is the spiral and say to ourselves, ``Hmm, what else fits this pattern?''\footnote{Our relationships, perhaps. My relationship with Dwale spiraled, after all. One long spiral from when we met, however many years ago, and then that spiral made its full cycle when it died, and now I'm on to the next loop, able to look down on what time we had together.}

Fit, then, a poem --- a particular poem, yes, but also the idea of a poem --- to this framework. A poem is a spiral. Poetry is a spiral. Writing poetry. Reading poetry. Burying oneself in words too rich to taste --- it is all a spiral.

Elliot Weinberger, in his survey of translations centuries of translations of one of Wang Wei's poems, does an admirable job of this. Throughout the years, he views the ways in which translations move: the views of the poem, the views of the time the poem was written, the views of the place in which it was written. Orientalism stains so many of them, especially those so early on. Even those most contemporary run into certain levels of inexactitude that miss the ways in which the languages translate. Is it `shine' or `reflect'? Simply `sound', or something more complex such as an `echo'?

``{[E]}very reading of every poem, regardless of language, is an act of translation: translation into the reader's intellectual and emotional life. As no individual reader remains the same, each reading becomes a different --- not merely another --- reading,'' as he so succinctly puts it \parencite[46]{wangwei}. ``The same poem cannot be read twice {[\ldots]} the poem continues in a state of restless change.'' It is all very Heraclitus.

By virtue of the reader's ever-shifting state of mind, they constantly re-translate otherwise static text, even from minute to minute, and build up a library of meaning from a single work. Reading a poem is as much a form of self-definition as it is of entertainment.

The spiral of the year, of the month, of the day applies as well to the poem, the stanza, the line. We've spiraled our way from spring to spring. We've spiraled our way upwards, using on its seasonal poetry as synecdoche for seasonality in poetry as a whole, through each of those seasons, and so the only fitting end is to use one last poem of Dwale's as a synecdoche for poem as spiral. We can read a microcosm of the spiraling year into a single poem. Start at the beginning, and when you get to the end, start over because you're already a different person.\footnote{And that's not so bad, is it?}

And so, one more time before setting aside the topic to steep for another year, let us address one of Dwale's poems:

\begin{verse}
\emph{Poem for a Deceased Lover}

Seven days\footnote{The instances of seven and eleven in this work may call back to \emph{shiva}. As a Muslim, though, the periods would have been different for Dwale.} had passed when I heard you died, \\
A message in the warm morning hours. Dawn \\
Rose, and no one said how I should go on, \\
Or wade this mire without my only guide.

Flown to space by what callous earth destroyed, \\
I chase the long-flying radio waves. \\
Far away from grief and a potter's grave, \\
I sift to find again your breathing voice.

Teacher, my every thought was yours to thresh,\footnote{A counterpoint to Spring, perhaps; thresh in Autumn, harrow in Spring.} \\
So now what sure course would you recommend? \\
Your kind words turned to shrapnel in the end, \\
Pieces of you left here in my heart's flesh.

Lover, did you mean to leave this deep wound? \\
I would sell my world to kiss you farewell. \\
Eleven years facing perpetual Hell, \\
And all I can say is, ``Too soon, too soon.''

\parencite[14]{leaves}
\end{verse}

If we are to tackle this as Weinberger does (and as we have touched on before), a good place to begin would be the prosody and sonority. We are again confronted with lines that follow a unique meter of iambs/troches interrupted towards the end with a spondee: ``Seven days had passed when \emph{I heard} you died'' works out as three troches, a spondee, and then an iamb (we could call it a `third epitrite', apparently, or we could be realistic), and ``A message in the warm morning hours.\footnote{Again, accents may complicate this, as `hours' may be one or two syllables.} Dawn'' as three iambs, a troche, and a spondee. 

This type of analysis may at times act as a desiccant, drying out an otherwise lush poem, but it does serve its purpose in giving us a glimpse at just why a poem makes us feel the way we do. When taken with the familiar half- and almost-rhymes (this time in \emph{ABBA} format), we are once more faced with a stumbling feeling that, in this case, perhaps speaks to trying to make one's way through the day with tear-clouded eyes.

Upon returning to the top and reading the poem through, one is struck by a sense of distance contrasted with the particular intimacy that comes with a wound. `Seven days', `flown to space', `long-flying radio waves' all speak to the impossible gulf between life and death, while `pieces of you left here in my heart's flesh', 'this deep wound', and `kiss you farewell' describe a closeness that crosses boundaries, a breach of an integument.

The grief is shown in the freshness and immediacy of the words. `Wade this mire' feels impossible in so low a place. ``I sift to find again your breathing voice'' shows the urgency that follows loss, the hasty need to find what is no longer there.

And yet, even within the span of the poem, we see that urgency lessen. We hear uncontrollable, gasping sobs calm down into mere crying. We are not yet at sniffling, at the dull pressure in our head that follows actually crying, but we are at least able to speak, by the end, our sorrow. ``Too soon, too soon,'' we say, and it is no soft platitude,\footnote{Platitudes are for others. They are for those trying to convince each other that they are saddened by this change. They are performative.
\begin{verse}
{[\ldots]} \\
``Good man, good man,'' they mutter, \\
doing all they can to convince each other \\
through well-rehearsed performances, \\
that this must be the case. \\
The silently bereaved already sit graveside.\par
\parencite{penguins}
\end{verse}\par
But grief, true bereavement, is not performative, it is reflexive. Add in the fact that I'm helpless before my compulsive explanation and beholden to my graphomania, and this was my grief over Dwale. I could not sit, silent, by the graveside. I could not sit \emph{shiva}. I could not bury myself in a community that is willing to support me, but what I could do is use the framework of words to pull meaning from that which feels too big to make sense. I \emph{do} have tools, even if it may not feel like it when grief burns particularly bright.} but our meager attempt to put into words what we are feeling when what we are feeling is still too hot.

Despite mentions of Hell\footnote{And I sure hope that the torment of plagues and politics doesn't last eleven more years, much less for perpetuity.}, it is comforting to see here that grief has transmuted into sadness. We have climbed that year-long spiral eleven times,\footnote{And while this may have been longer than Falcon lived, longer than she made our lives a joy, we got to make her entire life a good one.} we have had our period of lamentation, the soul has been purified, and we can see what it is to live life without them.\footnote{And it will live on at least as long as I do, will it not? I would that it had not died at all, but as it had to, at least I have the ability to think about it, love it from across that infinite gulf in my own, awkward way. I have the privilege of being able to memorialize it. I have my threnody, and through that, its works are set for those to see who might not otherwise.} Sure, we will always hunt their breathing voice, their kind words remain with us, we will never kiss them farewell, but it is now comprehensible. We can intellectualize their loss. We can pull it into words and set it before us. We can read our grief from top to bottom and then start once more at the top. We know it well, our sadness, and each time we take our trip\footnote{This is not a new idea, of course. In my choral conducting courses, we talked about taking `the seven trips through the score' in order to tease it apart so that we could put it back together with our students. Again, though, that Madison has passed.} through the text, we can feel its impact soften. It does not leave us, but it becomes  a part of us.

And now, when we spiral around once more to the top of the poem, we can look down over that perilous edge and see what we were. We can see the way we bury our face in a pillow we hug to our chest so that the gasping, choking sound of our sobs is muffled --- from whom? Perhaps even this version of us, here in the future --- however many levels down. We can look down to the level just below us and see how we're starting to come to terms with that loss. It was not a smooth transition, this integration of loss into ourselves, but now that we've once more reached the first line, we are no longer ``I, who grieves'',\footnote{Or perhaps ``I, who writes paeans to grief in the footnotes of an essay and worries that this is not doing the actual Work''. Just me? No? Maybe just me.} but perhaps ``I, who has grieved''. We can think about how our love is borne out of the solar system on those radio waves (for what else is WiFi?) and, even if we do not smile, we do not cry.

We can look up, too. We can look up and see all of the other times we \emph{will} read the poem and imagine who we might be. Might we be someone who can read through this poem and only \emph{remember} the us who was so torn by grief that they couldn't breathe for sobbing? A hazy memory, one where we remember that us as some different person.

And so we read the poem again and see something new --- aha! Is ``I sift to find again your breathing voice'' an anaphora? --- and it all becomes a little softer, a little more abstract. We read and read. We come back to our poem years later and it inspires nostalgia in us. Nostalgia! Simpler times for simpler versions of ourselves. A little younger, a little dumber, but no less capable of feeling.

Issa says,

\begin{verse}
\begin{multicols}{2}
\emph{Ro no hata wa} \\
\emph{Yobe no warai ga} \\
\emph{Itomagoi}

\columnbreak

Around the hearth --- \\
The smile that bids us welcome \\
Is also a farewell!
\end{multicols}
\vspace{-1em}
\parencite[101]{issa}
\end{verse}

A year passes and we look down through the haze of time, down along that spiral.

We read the poem again, re-translate it for ourselves, and spiral through the lines and verses.

A year spirals up, and so, too, does a poem.

% (Probably some gentle self-deprecation around writing a paean to grief in an attempt to get out of doing the actual work required to process it)

% (Positive outlooks: even if she only made part of our lives good, we made Falcon's whole life good, she had that last delicious meal; Dwale is honored and will live on in its writing and our memories, how lucky am I that I get the chance to be a part of that?)


\printbibliography


\end{document}

% vim: spell
