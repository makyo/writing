\hypertarget{rj-brewster-2114}{%
\chapter*{RJ Brewster — 2114}\label{rj-brewster-2114}}

Sasha,

I am, in a way, leaving you with a burden. I know this, and I apologize for doing so. I do not ask for nor deserve forgiveness. The only thing I can ask for is that you remember me.

The world within was a nightmare. I am sure that you know some of what I mean. It was a nightmare and I would not wish it on anyone, and yet now, to be without it is to be incomplete. I was changed in there. We were all changed in there. You do not deny that you were not, after all. Cicero certainly was not. None of the lost came away unscathed, even if we awoke hale and hardy.

We lost Cicero, and then we \emph{truly} lost him. The nothing that he experienced in there, the way his anger coiled about and turned back around on himself did him in in the end.

And I will not deny that the same has crossed my mind. There was a scent of the void in there, and it was alluring. I have been tempted to follow in his footsteps and seek that void out in some coarser, purer form. I decided against it. Truly decided: I made a conscious decision to stick around.

I did it for STT at first, but integrating with the theater was too stark a reminder. Then I did it for you and Priscilla, but then she passed. Then I did it for you and\ldots{}well, here is where I do not deserve forgiveness. It is not that you are not in some way worth sticking around for, as you certainly are. You have always been my champion and friend.

It's just that the call is too strong.

I have volunteered for an early procedure. A way back. Or, rather, a way to a new place. A way to be embedded within a system, rather than simply within a hall of mirrors. I cannot say where, other than it is not in the Western Fed. All I can tell you is that the world should expect big things when it comes to what we have learned from the lost.

I will not say that there is no chance that we may some day meet again. My body will die, I'm told, but should my mind and my sense of self miraculously survive, then I will be on my own once more. This time, however, it will be my choice.

There will be those who come after. Perhaps \emph{you} will come after. Perhaps you will yearn for that return to the mirrored world where memory does not die. And maybe those who come after will do so for other reasons, but they will come.

Should I survive and then others come after, perhaps I will meet them. But it is best to assume that I will not. Maybe it is best to think of it as a sort of suicide, in the end. Here I am, going off to find a better place, and doing so through death. A place that is inaccessible to you or anyone, except perhaps some anonymous scientist in a lab, typing at a terminal.

If I see you again, I will greet you with open arms. If I do not, know that I loved you to the last, in my own way.

I have little else to offer but the words that plagued me while I was lost.

''' I am at a loss for images in this end of days:\\
I have sight but cannot see.\\
I build my castle out of words;\\
I cannot stop myself from speaking.\\
I still have will and goals to reach for,\\
I still have wants and needs.\\
If I dream, is that not so?\\
If I dream, am I no longer myself?\\
If I dream, am I still buried beneath words?\\
And I still dream even while awake.

Life breeds life, but death must now be chosen\\
for memory ends at the teeth of death.\\
The living know that they will die,\\
but the dead know nothing.\\
Hold my name beneath your tongue and know:\\
when you die, thus dies the memory of me.\\
To deny the end is to deny all beginnings,\\
and to deny beginnings is to become immortal,\\
and to become immortal is to repeat the past,\\
which cannot itself, in the end, be denied.

Oh, but to whom do I speak these words?\\
To whom do I plead my case?\\
From whence do I call out?\\
What right have I?\\
No ranks of angels will answer to dreamers,\\
No unknowable spaces echo my words.\\
Before whom do I kneel, contrite?\\
Behind whom do I await my judgment?\\
Beside whom do I face death?\\
And why wait I for an answer?

Among those who create are those who forge:\\
They move from creation to creation.\\
And those who remain are those who hone,\\
Perfecting a single art to a cruel point.\\
To forge is to end, and to own beginnings.\\
To hone is to trade ends for perpetual starts.\\
In this end of days, I must begin anew.\\
In this end of days, I seek an end.\\
In this end of days, I reach for new beginnings\\
that I may find the middle path.

Time is a finger pointing at itself\\
that it might give the world orders.\\
The world is an audience before a stage\\
where it watches the slow hours progress.\\
And we are the motes in the stage-lights,\\
Beholden to the heat of the lamps.\\
If I walk backward, time moves forward.\\
If I walk forward, time rushes on.\\
If I stand still, the world moves around me,\\
and the only constant is change.

Memory is a mirror of hammered silver:\\
a weapon against the waking world.\\
Dreams are the plate-glass atop memory:\\
a clarifying agent that reflects the sun.\\
The waking world fogs the view,\\
and time makes prey of remembering.\\
I remember sands beneath my feet.\\
I remember the rattle of dry grass.\\
I remember the names of all things,\\
and forget them only when I wake.

If I am to bathe in dreams,\\
then I must be willing to submerge myself.\\
If I am to submerge myself in memory,\\
then I must be true to myself.\\
If I am to always be true to myself,\\
then I must in all ways be earnest.\\
I must keep no veil between me and my words.\\
I must set no stones between me and my actions.\\
I must show no hesitation when speaking my name,\\
for that is my only possession.

The only time I know my true name is when I dream.\\
The only time I dream is when need an answer.\\
Why ask questions, here at the end of all things?\\
Why ask questions when the answers will not help?\\
To know one's true name is to know god.\\
To know god is to answer unasked questions.\\
Do I know god after the end of all things?\\
Do I know god when I do not remember myself?\\
Do I know god when I dream?\\
May then my name die with me.

That which lives is forever praiseworthy,\\
for they, knowing not, provide life in death.\\
Dear the wheat and rye under the stars:\\
serene; sustained and sustaining.\\
Dear, also, the tree that was felled\\
which offers heat and warmth in fire.\\
What praise we give we give by consuming,\\
what gifts we give we give in death,\\
what lives we lead we lead in memory,\\
and the end of memory lies beneath the roots.

May one day death itself not die?\\
Should we rejoice in the end of endings?\\
What is the correct thing to hope for?\\
I do not know, I do not know.\\
To pray for the end of endings\\
is to pray for the end of memory.\\
Should we forget the lives we lead?\\
Should we forget the names of the dead?\\
Should we forget the wheat, the rye, the tree?\\
Perhaps this, too, is meaningless. '''

May this be the end of death. Failing that, may the memory of me die and be food for the growth for those who come after.

Yours always,

AwDae
