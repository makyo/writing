\begin{quote}
\emph{``All artists search. I search for stories, in this post-self age. What happens when you can no longer call yourself an individual, when you have split your sense of self among several instances? How do you react? Do you withdraw into yourself, become a hermit? Do you expand until you lose all sense of identity? Do you fragment? Do you go about it deliberately, or do you let nature and chance take their course?''}
\end{quote}

\noindent The Post-Self universe is an open setting for exploring the ramifications of being able to create copies of oneself, of what it means to undergo individuation, of what it means to let memories build up and up and up within oneself. What began as a simple shootpost on Twitter turned into a collaborative storytelling project, then an ARG which told the story of Dear, Also, The Tree That Was Felled and Ioan Bălan. Thanks(?) to the COVID-19 pandemic in 2020, this became the basis of that storyline in \emph{Qoheleth,} book I in the Post-Self cycle.

As an open setting, all of this information is free to use for your own purposes under a Creative-Commons 4.0 Attribution license. The stories wouldn't be what they are without the contributions of others.
