\section*{The universe}

\subsection{Immersive tech}

Beginning in the late 2100s, immersive computing technology began to become commonplace. The mechanism by which one enters the 'net is a set of implants taking the form of metallic contacts on the middle carpals of the fingers, near-field pads beneath the skin of the forehead, interferites --- microscopic neural blockers that prevent one from acting out in reality what happens when delved in --- and an implant along the spine starting at the fifth cervical vertebra and running down to the bottom of the thoracic vertebrae. The exocortex contains much of the technology that actually controls the experience of interacting with the sim.

The net is comprised of simulated areas, or sims, where one can interact with objects and other people. Online, one is perceived through an avatar, or av, which can be whatever shape one chooses. These can be made, customized, purchased, and sold.

It's like VR, only actually good.

A new take on sims are fully immersive sims, wherein one becomes something more abstract than an avatar, such as an entire room, where moving means controlling lights or sound, and sensations can be those of microphones or any other sensor one might like.

\subsection{Earth}

Sometimes referred to as `phys-side', Earth continues to tick along.

\subsubsection{Early 2100s}

At this point, the governments of earth are divided into two large political units comprised of smaller countries. The two largest players are the Western Federation (WF) and the Sino-Russian Bloc (S-R Bloc), but others include the North-East African Coalition (NEAC), and Southeast Asia/Pacifica (SEAPAC). Many countries still remain independent, with Israel being a notable example.

The previous century is described as troublesome, and there's a marked decline in population, with global population hovering at around 7 billion. The climate has suffered greatly, but things are still habitable.

\subsubsection{Around 2170}

While the climate has continued to suffer somewhat, income inequality has continued to increase and, under the guise of helping poorer families out, several governments have started to incentivize uploading, though in reality it comes across as thinly-veiled eugenics. {This is largely due to influence sys-side by members of the Ode and Jonas clades, notably due to the work of Do I Know God After The End Waking}

\subsubsection{Early 2300s}

Earth is described as a `shithole'. Global warming has proceeded to the pace where much of the population below a certain latitude lives below-ground, though many have simply moved towards the poles. Air quality is\ldots not great, and many spend as much time as possible on the 'net in sims, with children getting implants at around 5 years old, though the minimum upload age remains 18.

\subsection{The System}

Created in the early 2100s, the System (a vague name to keep the original project secret, though one which stuck around) allows for uploaded consciousnesses to live functionally immortal lives.

\subsubsection{Systime}

The System measures time with systime. This takes the format of \emph{years since 2124}+\emph{day of the year} \emph{24-hour time}. For instance, Secession took place on 1+21 19:00 first contact from the Artemisians occured at 222+148 3:06.

The date of midnight on January 1, 2124 was chosen as the opening of the reputation markets, as such a time scheme was needed for marking transactions. The use of systime is not universal among the inhabitants of the System, as getting the current time (an experience akin to remembering what time it is) provides both systime and standard phys-side dates, but those who work most often with history and sim design rely on it heavily for both mapping events and seasons of the year, should the sims in question require seasons.

\subsubsection{Uploading}

Uploading is a one-way, destructive process. The body dies while the consciousness continues within the System. There is a small chance of failure (around 1\% as of 2130, \textless0.5\% as of 2140, \textless0.25\% as of 2150, \textless0.001\% as of 2200).

Consciousnesses are uploaded to the system at the L5 point via the Ansible, a networked series of upload centers with a direct radio connection to the System itself. By the 2300s, this is largely automated and consists of signing a form and hitting a button.

Once uploaded, individuals are greeted by volunteers (later automated) to orient them to the concepts of creating clothing, simple objects, moving between sims, sensorium messages, and forking. Early uploads tend to live communally in larger sims, and many remain there, while the rest tend to flock towards smaller communities of like-minded individuals.

\subsubsection{Forking}

Introduced almost by accident, the concept of forking allows one to create a new \emph{instance} of oneself. This copy is completely identical, but as soon as they're created and their experiences begin to differ, that instance starts to undergo the process of \emph{individuation}. They form their own memories, and their experience of the world is colored by those memories.

An instance may \emph{quit}. When they do so, their memories are provided to their \emph{down-tree} instance to remember or not in a process called \emph{merging}. A merge may be wholesale (sometimes described as \emph{blithe}) or \emph{cherrypicked}, wherein the down-tree instance is able to choose some of the memories but not others in a labor-intensive process.

The greater the individuation between and up- and down-tree instance, the greater the chance for \emph{conflicts}. These occur when memories don't line up --- that is, the experiences may be of the same event, but the conclusions drawn from the event may be different. As time goes on, individuation will affect the entire personality of an individual, as personality is built in part atop memories. Cocladists who have diverged by decades or centuries may find such merges incredibly difficult.

Forking incurs a reputation cost. This is tied to available capacity on the System, and as capacity grows, the cost of forking decreases, to the point where, in the 2300s, it's negligible. This cost is incurred after five minutes of forking or as soon as that instance forks, whichever comes first. The new instance begins with reputation equal to the cost of forking, though transferring reputation within a clade is possible. Several other things such as information production and exchange, sim creation, and some experiences can lead to reputation exchange.

The \emph{root instance} of an individual will find it very difficult to quit as, to quote May Then My Name Die With Me of the Ode clade, ``the System is not built for death''. This applies to their \emph{up-tree} instances as well; it is easier to quit the shorter one has been around or if a newer up-tree instance exists (for instance, if Jace Doe\#Tracker forks into Jace Doe\#1234abc, \#Tracker may quit easily right away, though it will get steadily more difficult as \#1234abc individuates; similarly, if \#1234abc forks into Jace Doe\#5678def and \#5678def individuates long enough, \#1234abc will find it difficult to quit).

\subsubsection{Clades and dissolution strategies}

Groups of instances forked from a single individual are known as \emph{clades}. Although these are all highly unique, the oh-so-human need to bucketize the world into useful categories has led to three general strategies:

\begin{description}
\tightlist
\item[Taskers]
Taskers fork infrequently and only ever for short-lived tasks, choosing to remain primarily a clade of one. \emph{Example:} Tycho Brahe (from \emph{Nevi'im}) is a tasker who forks so rarely he has a lot of trouble even managing it. Merging back down to his \#Core proves difficult.
\item[Trackers]
Relying more heavily on forks to accomplish tasks, trackers may keep instances around for months or years, and sometimes more than one at a time. However, these instances tend to retain a strong sense of identity with their root instance and will almost always merge back down. \emph{Example:} Ioan Bălan, as a tracker, forks quite often for eir work, but those forks tend to be associated with projects and, on completion, will merge back down into eir \#Tracker instance (with a few notable exceptions: Codrin Bălan individuated enough to become eir own person, and Sorina Bălan forced her own individuation to leave memories behind as best she could).
\item[Dispersionistas]
Dispersionistas don't give a fuck. They fork at need and those forks may quit, may retain some sense of their identity, or may individuate and become their own individuals down the line. \emph{Example:} Michelle Hadje founded the Ode clade, which nominally has 100 members, but they're not super strict about it and many have long-lived instances they don't really talk about.
\end{description}

Clades can form quasi-familial units or not even really talk to each other; it's really up to the individual. There's a mild taboo against relationships between \emph{cocladists}, though the greater they have differentiated, the less that seems to be an issue.

\subsubsection{Sims}

Locations in the System are known as sims, an artifact from the pre-System 'net days. Sims may be public or private. Public sims are usually open to anyone and can be accessed by querying the perisystem architecture for their \emph{tags} (e.g: Josephine's\#aaca9bb9).

Private sims are generally owned by a single individual, clade, or family. These sims generally have much more restrictive \emph{ACLs} (from `access control lists', but now generally used to refer to fine-grained permissions) which can limit who may enter, whether or not the location is visible to others, who in the sim may create new objects, modify boundaries, and so on. The owners have full ACLs, including the ability to grant others owner status and rescind their own (though every sim must have at least one owner).

\subsubsection{Reputation market}

Although by the 2200s the System mostly exists as a post-scarcity society (or non-society, as it is not at all unified), a market was put into place early on when capacity was at a premium. This market worked on reputation (marked Ŕ) which was gained via recognition. Appreciation of someone or the works they produce increases their reputation, which can then be spent on various things such as forking (which only costs a nominal amount by 2250), creating sims, seeking information from individuals, and so on.

With technological advancements increasing System capacity exponentially, the reputation market shifted in purpose early in the 2200s to be a place for sharing information between individuals, with one gaining reputation by way of producing content and spending it by requesting content from others.

\subsubsection{Perisystem architecture}

The perisystem architecture is the conceptual foam of computer-stuff in which individuals reside and items such as sims, food, very nice fountain pens, and very fine paper exist. However, it also contains large amounts of information in the form of books, the reputation market, and various information feeds.

Some maintenance of the perisystem architecture is required, usually by engineers both sys-side and phys-side. In the instance of the two launch vehicles, for instance, PA engineers managed the DMZ {later called Convergence}

\subsubsection{Other notes}

\begin{description}
\tightlist
\item[Children]
Not a thing, sorry.
\item[Pets]
While there is no uploading of pets, many common animals can be created.
\item[Communication between sys-side and phys-side]
Communication between the two levels of existence was limited to text-only until A/V communication was unveiled in 2350 based on information gained from the Artemisians.
\end{description}

Any other questions? Feel free to \href{https://makyo.is}{ask}!
