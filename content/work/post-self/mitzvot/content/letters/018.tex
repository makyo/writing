\hypertarget{ioan-bux103lan-the-bux103lan-clade}{%
\subsection{Ioan Bălan — The Bălan clade}\label{ioan-bux103lan-the-bux103lan-clade}}

\begin{quote}
systime 229 (2353) (transmission delays)
\end{quote}

Hi all.

I hope you've been doing well of late.

It's been heartening watching everyone reconnect over the last year, if I'm honest. I know I say it just about every time I write, but I've been worried. You've all mentioned in the past feeling like I'm someone grounding that you can talk to, and\ldots well, I hope this isn't weird of me to say, but I've been feeling protective of you all of late. It's not quite the realm of parenthood or anything like that, but it does kind of feel like I'm watching over the clade, in a way. I don't know if it's a root instance thing, a shared past thing, or a me-as-I-am-now thing.

It's probably the last.

I think it's high time to admit aloud that all of these memories of Rareș are starting to pile up for me, and this protectiveness stems from those memories of him after mom and dad's death. I've been struggling to keep my mind off him, honestly. There have been a few abortive attempts at pulling the thoughts together into a book or screenplay or something, just as a way to process my feelings.

The thing is, if I want to be successful at something like that, I'll have to actually sit down and research the past. That's where I've been failing. I know it's something I'd need to do if I'm to do any project like that justice, and probably something I need to do if I'm to find a way to come to terms with the past, but there's some emotional block. Lately, every time I get close to engaging with the topic head on, I have a panic attack. Honest to goodness, full blown, hyperventilating-and-feeling-like-I'm-dying panic attack.

It's something I've been working on a lot with Sarah. I certainly don't like the feeling, but neither do May or, when she's around, Sasha like seeing that happen.

I know you know more about this than I do, Codrin, but please let me work on this myself.

Anyway, that's only part of why I'm writing. The way that this topic has affected me has led to a series of conversations between May and I around the interplay of immortality and relationships. I know I won't do the topic justice, so she's written up some of her thoughts, which I'm including here.

\begin{quote}
One unintended consequence of immortality is not just that memories of relationships pile up, but the \emph{way} in which they pile up. We do not simply remember lost loves with fondness, but also with caution.

It seems counter-intuitive, does it not? We might expect that our everlasting lives might add in some more cavalier attitude toward the relationships that we form. This has not borne out over the centuries. We do not find ourselves trying ever new things in the ways in which we form relationships; perhaps some do, but neither of our clades do. We keep our lives as a whole interesting, but we constantly refine our relationships.

The Ode speaks of honing and forging, and so many of those who have uploaded and sought out entanglement have found themselves honing rather than forging. It is a search for the more perfect love. We speak constantly of ``learning from our mistakes'' and ``doing better by them/ourselves''.

This is no bad thing! We do this out of a desire to be better people in the ways that we are closest to others. These just happen to be the ways most likely to hurt others, too. We shy away from trying new things with our relationships because that puts our view of ourselves as good people at risk.

And so we look back on the relationships that we have formed, kept, lost, or let slip away into so many years, and we remember the good times cautiously. We hunt for the things that went wrong, we see all of the places where we fucked up and we tear them apart as one might a hole in a piece of clothing: thread by thread. We idly pull a thread, inspect it, and hunt for the weak point that led to the hole forming in the first place. We think back on arguments and hunt for where we could have kept it from blossoming into a fight. We think back on missed expectations and wonder what we might have said. We think back on crossed boundaries and hunt for a sign pointing to the boundary that we simply overlooked.

It is a fool's errand and we are dumber than a bag of rocks for doing that, and yet we keep on doing so. It is so incredibly difficult to stop, is it not?

And yet, as the Ode goes on to say, ``To forge is to end, and to own beginnings. To hone is to trade ends for perpetual perfection.'' That perfection, it says, is ``Perfecting singular arts to a cruel point.''

The Ode is just a poem, it is no holy text — what was it Emerson said? The poet nails a symbol to a sense that was true for a moment but soon becomes false, while the mystic mistakes the singular for the universal?\footnote{This one took some digging. It's from his essay ``The Poet'': ``Here is the difference betwixt the poet and the mystic, that the last nails a symbol to one sense, which was a true sense for a moment, but soon becomes old and false. For all symbols are fluxional {[}\ldots{]} Mysticism consists in the mistake of an accidental and individual symbol for a universal one.'' Where do they even find this stuff?} — but every poem is open to interpretation and analysis. The author of the Ode was not wrong. We shy away from those ends that hurt and any beginnings that might follow in favor of our dreams of perpetual perfection.

This applies just as readily to familial relationships as it does to romantic ones. Ioan and I do this in our relationship just as much as ey does this when ey remembers Rareș and every single Odist does when thinking about the poet. Our immortality gifts us the ability to do this to an uncomfortably endless degree.

I will quote Sasha's gentle warning here with her permission: ``The danger in ceaseless memorialization is how close it lies to idolatry. To elevate the dead to such a status is to ceaselessly perfect the imperfectable.''

Ends happen. There may yet come a day when Ioan and I decide to go our separate ways. We know that it will hurt, and it is easy to focus only on that and hone and hone and hone. That is not all we can do, though; we can also hope that it will be with love, that we will go our own ways and own what beginnings may yet be in front of us.
\end{quote}

She's right. Of course she is, I mean. Not only does she have more experience than literally any of us in this matter, but for more than two centuries, it has been a daily focus of hers. I have to catch myself from endlessly focusing on things I could have done better. Could I have stayed in touch with him? Should I have encouraged him to upload? Worrying about these things is the fool's errand she describes.

These are just things that have been on my mind, by the way, I don't mean this as any sort of admonishment with how any of you are tackling the issues that have taken up the greater part of our worries the last few years. We're just doing the best we can with what we have, and what we have isn't always the healthiest when it comes to coping mechanisms.

Anyway, beyond that, things are going well. \emph{I\&R}'s release last month seems to have gone over well enough. I imagine that's due in no small part to the preparation that Jonas and the rest of the eighth stanza have put into ensuring it lands as they'd wish. It has yet again come off as ``just slightly too fantastical to be real, but sure makes a good story'', much as \emph{Perils} and the \emph{History} did. Ah well. I'm still proud of it, and I'm not unhappy with where we've wound up.

Aurel's off with Sasha now, and has been for a few months. For a while there, her periods of solitude were coming pretty often, and ey was popping in and out of existence with some frequency, but she seems to be settling down into a more predictable pattern. It's my hope that ey'll eventually be able to spend a year or so at a time with her, if not longer.

She's been doing well, too. I think she's really starting to come into her own as Sasha. Always in threes, but still always Sasha. She's been getting a bit grumpy about the whole spotted skunk thing, though, and I think that, before long, she'll see if she can find a way to go back to striped skunk. She keeps complaining about the shorter tail. Aurel's been teasing her by calling it cute, eliciting the usual threats of biting.

She's just about wrapped up her work on the companion volume to the \emph{History}, which she's tentatively calling simply \emph{Ode}. I've had a chance to read it and\ldots well, I'll let her share it when she's ready. It will take a lot of work for it to have the effect she plans, and the consequences will be far-reaching for the Ode clade. She says she won't publish it for another decade or so for reasons which will become clear when you have the chance to read it. In the interim, she's mentioned a few other writing projects she'd like to tackle, all of which sound good.

Debarre's back with E.W., which is good to see, and given the fact that we're now plopped right in the middle of a forest sim, they've come over to visit and camp a few times. Or, well, Debarre will come stay with us for most of the day while E.W. and Sasha go off and explore, and then they'll meet back up around dinner when Sasha returns to Aurel. Debarre's loosened up some, but I don't think he'll ever be totally comfortable with Sasha, which she seems to have accepted.

It's getting on bed time and May's whining at me most pitifully, so I'm going to go ahead and get this sent off before I ramble any more.

We all send our love to you and yours, and hope the universe is treating you well.

Ioan Bălan
