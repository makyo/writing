\hypertarget{layers}{%
\section*{Layers}\label{layers}}
\addcontentsline{toc}{section}{Layers}

\begin{quote}
There is a holy mind that is above fear.
\end{quote}

I was reading through a book recently --- \emph{Discipline} by Dash Shaw --- and came across this phrase, and I've been completely unable to dislodge it from my mind.

I'll be in the kitchen making a mess or at work trying to research some fantastically boring topic, and then it'll just kind of pop up in my mind, and I won't be able to do anything about it. \emph{There is a holy mind that is above fear,} I'll think. \emph{There is a holy mind that is above fear, there is a holy mind that is above fear, there is a holy mind\ldots{}}

I can't quite pin down what it is about this phrase that leads to it getting stuck so easily in my thoughts. It's pretty quippy. It would fit on a bumper sticker, and I would very much like to meet the person who would put such on their car. It's also quite pleasant to say, all those small words that just kind of tumble out of your mouth. The `that is' stuck in there, to our modern ears, adds in a little sense of formality.

But I think the reason that it gets stuck in my head is that it says so very little. There is this holy mind, of course, and it is above fear. There's nothing convoluted about the statement. Thing \emph{A}, a holy mind, exists relative to thing \emph{B}, fear.

Then you get stuck on trying to figure out what a holy mind is, though. It's something that can't include fear, right? That's part of the logic of the statement: it's above fear, not made from it, in whole or in part.

And what is meant by `fear'? Hearing someone talking about fearing God is not uncommon; why is that separate from the holy mind?

Both of these are unknown things with unknown qualities held in very real relation to each other, and that combination of the concrete (or as concrete as a metaphor can get) relation abstract concept holds the mind in thrall.

So much is left unsaid that the statement can come off as both self-evident to the point of not saying anything and abstruse to the point of not meaning anything, yet it rings true enough for me that it will get stuck in my head, and I'll start diving down through these layers of meaning to try and pull it apart, to dig in the spaces between the letters of `holy' and `fear' for any extra meaning left behind by a long-dead author.

Layer after layer after layer, digging their way through levels of applicability and practicality, from that surface level interaction of what is holy and what is fear all the way down to that ineffable sense of the numinous that we can never explain and nonetheless still have within us.
