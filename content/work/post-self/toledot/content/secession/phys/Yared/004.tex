\hypertarget{yared-zerezghi-2124}{%
\chapter{Yared Zerezghi—2124}\label{yared-zerezghi-2124}}

\emph{Mention how the System almost feels like its own nation, mention L\textsubscript{5} but only in passing,} the note read. \emph{Expect agreement from a new faction. Act pleasantly surprised.}

As he had found himself doing increasingly often, Yared stepped out of his apartment to walk the town and draft his new post in his head. They used to flow so easily, when each one did not feel like some school assignment.

He walked out past the coffee shop, waving to the woman behind the counter, and shaking his head to an offer of coffee. He was already wired enough.

He kept on walking, instead, out and down the street past apartments, the store where he bought his food, apartments, the restaurant that he ate at once every other week, and yet more apartments. Out and out until he ran into that patch of scrub that somehow never got developed, then right and into where the scrub turned into scattered bushes, and then trees. There had been a fence, once, but all that remained were the posts.

He'd never bothered walking up here until he'd accepted the unnerving assignment to convince everyone to secede. Explicitly, to convince the DDR and various governments to allow it, but implicitly, he felt, to convince those he talked to on the System, as well. Convince True Name and Jonas to suggest it from the other side.

It had been unnerving at first, at least.

Why would he, a nobody who dumped all his free time into the 'net, into the DDR, be expected to make any change? He knew that, once a referendum was picked up by more than a couple of the various legislatures, it was hopeless to expect the DDR had any real impact. It became the joke that he was sure so many thought it was.

He had picked up the topic of the System's individual rights as his next pet topic, for even though he had felt little interest in the System or its labyrinthine technologies at the time, when the previous bill he had hyper-fixated on had failed on the floor, and after a night of far too much tej, he needed to set his mind on \emph{something.}

He didn't know why he did this, why he felt the need to dive into politics. He was a no one in Addis Ababa, a city which paled in importance in the NEAC, a governing body that paled in comparison to the others in the world.

He had a data analysis job he could do from home reasonably well, and he didn't slack off while at work (though he did leave DDR alerts on in his field of view). He made enough of a living to stay in his apartment in an alright part of town. He was comfortable. He had no plans to upload.

Or hadn't previously. The more he learned, the more enticing it seemed.

It certainly seemed like an easier life than this, accepting messages from shadowy government agencies to try and influence what was supposed to be a direct means of being represented in the legislatures of the world. It was one thing to try to do so from one's own perspective, but to accept such influence, even if he was only paid in coffee and cake\ldots{}

It had surprised him that he had even picked up the task at first. Secession seemed like such a strange thing to ask for. What did the NEAC---or any government, really---gain by having the System secede? What was the System doing that threatened them so much? There was the brain-drain that some feared, but this seemed to rely on some more basic instinct or need to have that which is different separated from that which was familiar.

He didn't know why he had picked up the task, but it was working, even on him. \emph{Especially} on him. The idea of secession from a government's point of view was one that fit neatly into his worldview without him needing to change anything, and that was strange in and of itself.

The System probably should secede. At that point, uploading became a simple matter of emigration, one to a country that was guaranteed to grant you residency. Not only that, but, though the cost might be high and the move permanent, it offered a ready-made haven for refugees, whether from the increasingly hot climate or the countless little spats along disputed borders. Uploading was an option for those who had nowhere else to go, and one that offered them more freedom than any other country on earth.

And this new idea that had started showing up, first in his conversations with True Name and Jonas, and then on the DDR in general, of tacking the System onto one of the launches for the L\textsubscript{5} station construction. The timing---True Name and Jonas, then the DDR---made him wonder if the Council of Eight had its fingers in other pies, too.

He wasn't sure how to feel about this. What an opportunity that had presented itself! All those arguments about the resources the System used would be all but put to rest. The station would house it, the station's solar power source would power it, and the Station Hotel's revenue would fund it. It would be another part of the tourists' experience. There were already plans for a new transmission system that would be easy enough to build for uploads to make it from Earth to the System without having to fly to the station first.

It was all starting to feel like such a good idea, and some part of him felt embarrassed that Councilor Demma's bald-faced political machinations were working just as well on him as they promised to on the masses that filled the DDR forums.

He realized he'd been so lost in thought that the wooded grove had already spat him out the other side, back into heat and back into traffic.

``Well, shit,'' he mumbled, and began the long trek back to his apartment, polishing the draft of his post in his head.

\begin{quote}
I won't lie, I'm pleased to see this discussion take a turn to the positive. There are some great minds thinking and talking here. Here on the DDR forums, out on the 'net, and now out in the subcommittees that will feed into the legislatures of the world.

What heartens me more than that, however, is to see some names that I had previously seen arguing \emph{against} independent rights now campaigning \emph{for} them (or, at the very least, neutral in tone). This is how the DDR is meant to work: it's a forum for us, the rank and file of the nations of the world, to be able to participate in the legislative process that will bind us in more ways than of old. No more relying solely on representatives. No more collecting signatures for yet another petition that will fall on deaf ears. No more letter writing campaigns that doubtless fed countless shredders and trash folders.

To those arguing for independent rights, keep working hard, as there is still much to be done, but to those who are arguing against this referendum, I would like to address a few of those points that seem to keep cropping up:

\emph{The System has no meaningful way for us to control its goings on, and thus could be a good place for disaffected citizens to coordinate with phys-side agents on acts of terrorism.}

This is one of those arguments that is difficult to refute because, on the surface, it is indeed a potential reason that one might upload.

That said, enough thought about how international terrorism works is enough to put this to bed as yet more FUD. First of all, it is the responsibility of each country to monitor their own citizens to within the limits of their national policies (and, let us not kid ourselves, well beyond). If a disaffected citizen is willing to engage in a terrorist act on their home soil, then it is the responsibility for the government to deal with that individual.

I will grant that this leaves the upload to contend with. There is no easy way to detect whether or not the System has punished them, and there's certainly no way for them to be extradited, should they be discovered.

Do not doubt your respective governments' abilities to track these actions, however. It is something of an open secret that they are always a decade ahead of us mere mortals when it comes to encryption, and thus cracking of those encryption methods used ten years prior. They'll be able to track communications from the System easily enough, just as they track any other form of text-based communication.

(And to my NEAC government handler who reads all of my posts, finger hovering above the big, red `arrest' button: hello! I hope that you are well.)

\emph{Without clear news sources coming out of the System, there is no way for us to tell that the Council of Eight is effective at governing those sys-side.}

Disregarding the Council of Eight's mandate to ``guide but not govern'', I'm curious, now! What would a ``clear news source'' would look like?

When one thinks about news sources here, one thinks of a stream of information about concrete events: what hurricane hit which part of North America; what stock jumped to what price; what the cricket scores are. These are all \emph{things.} They all have to do with \emph{stuff} or \emph{places} or \emph{money.}

Think of one thing that has made news recently that does not have to do with any of those things. I will preempt many of your examples:

\vspace{-0.25em}

\begin{itemize}
\tightlist
\item
  Legislation---that is, new laws to govern stuff, places, or money.
\item
  Scientific advances---that is, new ways to work with stuff, places, or money (and before you suggest theoretical sciences, consider that those are future ways to work with stuff. Psychological breakthroughs? Better ways to keep us happy so that we can produce and consume more stuff).
\item
  International relations---that is, which group people in which places have which stuff that which other group of people want.
\item
  Technological breakthroughs---stuff.
\item
  Exploration---places.
\item
  Travel, entertainment, comedy---commodified experiences.
\end{itemize}

\vspace{-0.25em}

Here are some things that you might find in this theoretical news source that also appears in ours:

\vspace{-0.25em}

\begin{itemize}
\tightlist
\item
  Opinions
\item
  Interpersonal relations
\item
  Religion
\end{itemize}

When one is unbound by the constraints of stuff, places, or money, one finds that there is little news that is worth treating as news.

Doubtless they have news out there. I don't mean to imply otherwise. Of what worth would it be to us to know of a cult surrounding, say, some upload who has found a neat thing to do with forking? Of what use is the knowledge of what is the new, hottest sim? Which of us really, truly cares about their petty squabbles?

I would say that I do, but lets be honest, I can't even begin to understand those, but I can certainly respect their rights to have them.

Now, tell me what effective governance looks like in such a system. Resources are controlled through the reputation market. As far as I can tell, there is no murder, there are no wars, fights can be over in a blink if one of the parties just leaves, and the worst offense someone can commit is stalking, and even then, one can be bounced from a sim.

We come yet again to the idea of speciation. We are fundamentally different. Or, to use a metaphor from the first point, this is an entire \emph{society}, human or otherwise, that is fundamentally different, as one might see with the vast gulf between customs in different areas of the world.

\emph{The L\textsubscript{5} station has no obligation to host the System.}

Correct, and yet they volunteered. This is a non-argument for a non-problem.

They are an international cooperative effort with business interests involved. The System is neither of those, true, but it is also not \emph{not} those, either. A nation to cooperate? It is not a nation, but I believe I've argued the point that, given fundamental differences, it might as well be. A business? It is not a business, but it does have employees and businesses associated with it, and it produces some delightful results in terms of the new ideas that constantly flow through the communications channels.

Friends, I struggle to see the merit of many of these arguments, and of the ones that do hold water, there are sensible compromises available. These people are \emph{people,} and it has long been established that people deserve rights. They are a \emph{culture,} and it has long been established that cultures deserve protection.

Vote for the granting of rights. Vote yes on \emph{referendum 10b30188}

Yared Zerezghi (NEAC)
\end{quote}
