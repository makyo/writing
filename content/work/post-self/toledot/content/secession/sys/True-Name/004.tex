\hypertarget{true-name-2124}{%
\chapter{True Name—2124}\label{true-name-2124}}

It had initially taken some getting used to, meeting with one's up- or cross-tree instances. Michelle, in her role in helping tie the cost of forking to the reputation markets, had certainly done it a number of times before, but, as the cost of a new fork was only applied five minutes after it had been created, all of her forks to date had been short-lived in order to conserve her reputation for some imagined future date.

The date had come and gone, now, so True Name---and likely all of the other Odists---had had to learn how to interact with the other copies of Michelle Hadje/Sasha that had sprung so quickly into being and immediately began to diverge.

The fact that those who matched Michelle and those who matched Sasha were evenly distributed had helped at first. There had been some oddness in talking to a Michelle-alike, given the countless memories of the constant shifting between the two forms, but that had had a different flavor to it than talking to another Sasha-alike. Seeing a form and a face that so clearly mirrored her own was not exactly unnerving so much as uncanny.

As the days and weeks went by, however, the forks diverged further and further, and different cares painted different faces, different habits were formed and dropped, and it became less like talking to an alternate version of oneself and more like talking to a twin, a sibling.

So it was when The Only Time I Know My True Name Is When I Dream met with That Which Lives Is Forever Praiseworthy.

Her initial impression is that the other skunk had shifted her wardrobe to look more professional, choosing a loose-fitting pantsuit in muted blue that had been in style before Michelle had uploaded. This also included a pair of pince nez glasses perched atop her muzzle which, when True Name inquired, Praiseworthy explained were non-prescription, and ``something I am just trying for the moment. They are quite annoying, but still fetching.''

Beyond that, however, Praiseworthy had decided to divest herself of many of the personality traits that had made Sasha Sasha. Gone were those aspects of childishness that Michelle had long held onto, and gone was the exhaustion that had lingered for years after getting lost.

\emph{I have changed, too, at that,} True Name thought. \emph{I have become the politician, working with Jonas. Praiseworthy has become something else.}

The two skunks shook paws, and then Praiseworthy drew True Name into a hug. It was surprising. Something about it felt both natural and performative, as though this was just a thing that one did when one had a role to play.

``True Name,'' Praiseworthy said. Her smile was warm and earnest, and she spoke with willing paws, palms up. ``It is nice to see you again.''

She laughed. ``I suppose so. You have changed quite a bit in so short a time.''

The other skunk bowed, laughing. ``As have you, my dear! And that is why you have come here, is it not?''

``I guess it is, yes. The more I work with Jonas, and the more I talk with the Council and phys-side---the more politicking that I do---the more I feel the ways in which my attitude and expressions are lacking.''

Praiseworthy nodded. ``Yes, you do still have some of the stiffness about you, and there are some sharp edges that\pagebreak\ could do with softening.''

``Softening?''

``Yes. It is mostly a matter of appearance and affect, though. You should not blunt your wit or intellect, just your tone and features.''

True Name frowned. ``I am not sure what you mean by blunting or softening, though.''

Praiseworthy took her gently by the elbow and started walking through the grass. They had decided to meet on a portion of Michelle's dandelion-ridden sim, far away from their root instance, but in a place that was still familiar to both.

``Take your walk, for instance. Even now, as we are just out for a stroll, you walk with purpose. Your shoulders move too much. Remember, if you keep them pointed straight ahead and shift the rolling motion to your hips, it will lead to others seeing more feminine aspects in you.''

She tried to keep her shoulders still as they walked, immediately feeling a slight strain in her hips.

Praiseworthy laughed. ``You do not need to keep them level to the ground, just perpendicular to the direction you are walking in. But here, no need to practice too hard. Fork, holding in your mind a pelvis just a hair wider than your own, but keeping your hips the same width. It will mean slimming down a little.''

``I can do that?''

``Of course. Zeke dreamed some algorithmic magic behind the scenes. You can fork yourself into most anything that can be consensually held in the mind.''

True Name nodded warily, holding this new image of herself in her mind.

``Perfect,'' Praiseworthy said, moving to take this new fork by the elbow and nodding to the original instance of the skunk. ``Now you quit. No need to incur a charge. Michelle, no need to accept further memories from us for the day.''

The skunks tilted their heads in unison.

``Michelle will be getting a pile of memories, if she wants, as I will have you fork a few more times yet. I have been letting her know when she can ignore further merges, as I have done this quite often.''

The first True Name nodded, then disappeared.

True Name felt down her flanks, taking a few more steps and finding it far easier to walk casually and still keep her shoulders pointed forward. She nodded approvingly. ``Excellent. What other suggestions do you have?''

``For your role, you will need to carefully balance cute, attractive, and competent. If you go too far towards cute, then it will be difficult for you to be taken seriously. The same if you go too far attractive because you will be just a pretty face. If you go too far competent, you will be seen as dour and unpleasant.''

Praiseworthy stopped her and turned her gently to look at her face.

``Now, first, your eyes will need to be just a hair larger, your ears slightly rounder, your cheeks fuller, and you will need fewer but longer whiskers. Can you hold those in your mind?''

She closed her eyes, picturing what she knew of herself in her mind, and forked.

``Goodness.''

She opened her eyes again to look at the fork, immediately laughing and shaking her head.

``Am I cute?'' the new skunk asked.

``Adorable, but that is not quite the direction we want to go. You look closer to a teddy bear.''

She rolled her eyes, then quit.

``Let us try one at a time. You will need to work fairly quickly to avoid the hit in reputation. Fork once, and then that fork will continue to look as you do now, while you work progressively on each of those steps.'' When True Name did so, Praiseworthy nodded. ``First, rounder ears.''

The new fork perked up when her down-tree instance forked and quit, the new instance having slightly rounder ears. She nodded, smiling.

``Excellent. Now the whiskers. Great. Cheeks? And\ldots eyes. Fantastic.'' Praiseworthy smiled after all the forking had been completed, then nodded to the first of the new instances, who quit.

The option for a rush of memories was provided to True Name, who, on a whim, accepted it, now remembering what it had looked like from the outside as her face had grown\ldots well, cuter. It had worked well.

The two skunks worked through a short laundry list of changes. True Name grew an inch or so taller, her shoulders became the slightest bit flatter without getting broader, her back straighter.

One last time, she forked to get a good look at herself to compare with what she remembered from before the process.

She was, indeed, cuter, but this was tempered by a more conventionally attractive body type, staying shy of being both adorable and overtly attractive. This somehow combined into a look that was more professional. It made her look, she realized, like a public figure.

``Oh, this is delightful.''

Praiseworthy beamed. ``I am glad that you enjoy.''

They worked next on how to better her affect. Smile more earnestly, laugh more easily, transition from those expressions to stern or confident or pitying. There were a few more forks as they worked on ways to soften True Name's voice, pitching it just a little lower, rounding some of the vowels, practicing elocution. With each fork, she found that the lessons stuck more firmly. Perhaps what was in her mind before became more cemented in place.

Finally, Praiseworthy had True Name practice forking into a Michelle-form for situations where a skunk would be out of place, and then they worked on perfecting that version of her, as well. It was surprising, at first, that she could even make so great a change with one fork, but then, she remembered precisely what it had felt like to be Michelle, just as she remembered what it felt like to be Sasha.

Eventually, when the practice and modifications had wrapped up, nearly two hours later, the two skunks sat at the top of a low raise in the landscape, and True Name discussed the other reason that she had sought out Praiseworthy.

``I need help in spreading ideas. I know that you have settled back into acting and directing, but I do not have the time or energy to guide emotions and reactions to news while still working on this political angle.'' She plucked a few blades of grass, rolling them into little balls between fingerpads. ``I know that propaganda is not the same thing as theater, but would you be willing--''

``Yes!'' Praiseworthy laughed. ``Of course I would be willing to help. There is more than a little propagandizing in trying to get actors to do their fucking jobs, even when the actors are yourself. What precisely do you need? Speeches? Words whispered here and there? Posters?''

True Name laughed and shook her head. ``Not quite the answer that I was expecting, but yes. Speeches and letters specifically. Some geared toward phys-side, some toward the Council, and probably a few towards other groups sys-side. I would not turn down a few words whispered here and there, though that will take some strategizing. There will be an instance of Jonas who will be working with you in shaping sentiment, as well.''

``I will look forward to it, then.''

They sat for a while in the sun, each looking out into the fields. At one point, Praiseworthy took off her glasses and set them on the bridge of True Name's muzzle, shook her head, and slid them into a jacket pocket.

It was good to be around oneself, True Name realized. There was none of the pressure involved with interacting with others, none of the careful maneuvering required when talking with Jonas. They could just sit there, side by side, and understand that there was nothing between them that the other did not also, at least to some extent, understand.

``Have you talked to many others in the clade?'' Praiseworthy asked.

She shook her head. ``Here and there. I have a meeting scheduled with Life Breeds Life, but that is about it. You?''

``You were the last I had yet to speak with. It is interesting to see how we have each decided to focus on different areas. You dove hard into the political angle. I tried to get back to theatre, but enough of that desire remained in me that your propaganda job sounds fun. Life Breeds Life is quite strange. He has been focusing--''

``He?''

Praiseworthy shrugged. ``I guess. He has been focusing on historical stuff. Documenting this and that, digging into old things. I have no idea where that came from. Loss For Images is writing, these days. May One Day is fiddling with reputation markets---or at least as much as Debarre will let her---and last I heard, Hammered Silver has just been either chilling here with Michelle or sim-hopping.''

``How is she, anyway?''

``Michelle?'' Praiseworthy frowned, ears tilting back. ``Much the same. I think the last of her energy went into us, and she is\ldots I do not know. Empty? She spends a lot of time sleeping, a lot of time sitting and thinking. She came to a play, but left partway through. She is still of two minds.''

``And she still has not explained why she never fixed it?''

The skunk shook her head.

``Any guesses?''

``Nothing solid.''

True Name nodded and turned her gaze back to the rolling plain. So much grass. So many dandelions. ``There is a time and a place for dwelling in memory,'' she said. ``But Michelle does nothing else. It is no wonder she is stuck. When\ldots when ey died, I think she began to as well. When she she dumped the last of herself into the Ode, she sealed the deal.''

Praiseworthy said nothing.

``She is dead, I think. There is no more life in her. There is nothing to be done but let her enjoy that death as long as she would like. I do not expect that she will come back.''

The other skunk drew her knees to her chest and folded her arms across them. ``I think you may be right in that. Let her do what makes herself happy while her shade remains.''

``I wonder if she knows it, yet,'' True Name said, then let silence fall again. The two sat together, watching as afternoon slid carefully into evening.
