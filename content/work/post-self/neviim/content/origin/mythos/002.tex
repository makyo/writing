In the beginning, the gods created the world. They built it up, atom by atom, molecule by molecule. They used eyes like lasers to guide one after another into ordered formations, ranks upon ranks, and then set them to marching. The gods built the world and then they smiled at it from up above. They looked down on their creation and saw all of the possibilities of perfection that it held, of the unending life and endless bliss.

The gods built the world because they desired to shape it to their will. They wanted to bend the world into something that they could direct this way and that, because after all, could they not do that with their atoms and molecules? A world that is orderly! Imagine the wonders they could create! The wills they could work!

\begin{center}\rule{0.5\linewidth}{0.5pt}\end{center}

So the gods set the world to spinning and watched and waited as it began to blossom and bloom. When the time was ripe, they reached down their hands to touch the world, and found that they had become the wind and the tides and the rain and the snow and the sunlight and the moonlight. They reached down to touch the world and shape it to their will, and found that they become impersonal forces in the face of absolute independence. The world they created could not be controlled, because there is no such thing as a world that can be controlled. They reached down, became impersonal forces, and the lives within the world bundled their coats up tighter at the north wind or took their hats off when the sun shone bright, but never could the change a single mind to be such as their own.

\begin{center}\rule{0.5\linewidth}{0.5pt}\end{center}

The gods came together and began to discuss what might become of their world as they'd made it, if they were truly to have no other influence beyond that which the sun and winds might.

``We must start over anew,'' one said. ``We must destroy this one because there is no controlling that which we have created. If we have created something beyond our control, have we not simply set it loose to do as it will? This will not do if we are to bring to fruition all of our goals. Let us start over.''

``We must live with what we have created,'' another said. ``And simply watch what happens. If there is no controlling our creation, then so be it, but to steal creation from the created will bring about no good. It may put a stop to the ugly, yes, but it will also put a stop to the beauty.''

Two voices then banned together and said, ``We must not do one or the other, but simply modify what plans we had. If we are to have our creation continue along the lines which we have devised, then we must use what powers we retain to nudge subtly, push gently, and guide along paths toward such a point that our plans come to fruition anyway.'' And these voices gave such convincing arguments that all were moved to agree.
