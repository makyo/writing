On the morning of every day, when days are warm and there is no rain, on days when Lyut knows when it is day and when it is night, he will gather his ingredients onto a small board and sit at the entrance to his cave and make his incense for three days hence.

Lyut, blind fisher, blind pekania, works with measured care, for he does not want to injure the pads of his paws nor nick his already-scuffed claws nor shave off any of his fur, nor, Ýng preserve him, damage his carefully honed equipment. He works with measured care and a practiced slowness, with a patience known to one who holds the highest devotion to his labor and to his Lord.

Lyut works with particular care when employing the use of his knife for he has cut himself before. He has cut himself and knows that not only will this spoil his incense for the day, it will also leave his pads aching and sore, will leave his fur matted and sticky, will leave a thin layer of blood upon all he touches until the flow stops and the wound scabs over. Knows that he would have to make his way down to the river to wash. Knows, too, after a particularly bad accident with his knife, that the stick he uses to guide his way down the path gets slippery and would need to be cleaned as well, that to bind a wound with only the use of one paw carries some particular difficulty.

And so he gathers his ingredients and tools onto his board and carries them to the entrance to his cave where he sits and works with measured care.

He works from left to right because he holds the knife and hammer in his right paw, and he builds the scent from bottom to top because that is how he has laid out his ingredients, and because it is the base notes of the scent that are the most forgiving to balance.

Begins, then, with the crushed roots of nardin, which previously he had pounded and which now he lays against the board and measures ten claw-widths thereof and cuts with his knife. To this is mixed ten teardrops of common mastic the width of a claw. On holier days he may find himself using copal in its place, and indeed he may use that later. For now, he attempts to find nodules the size of one of his claws without requiring that it be cut or broken, lest his senses be dazzled and the balance lost.

The middle notes come next and Lyut takes a fingertip's length of sweetgrass and puts it into the bowl with the base notes. The scent of sweetgrass is, yes, sweet, but it provides also the bulk of the material that will burn throughout the day.

To this he adds sweet flag root which has been carefully washed and hung and dried. He grates this first with his knife before adding it to the bowl, scraping the blade almost perpendicular along the root to shave off a fibrous powder.

These are all taken together in a stone mortar and ground with a stone pestle to pulverize them into a uniform powder, which he checks with gentle touches of the last fingertip on his left hand, which is the most sensitive.

Judges with his nose and, deeming it correct, finishes, now, with the lone top note of a precious dried pod of cardamom and what he judges to be one third again in weight of makko powder to bind the incense.

To build a scent from the bottom up is to tell the first of three prayers of creation to Ýng, and Lyut works with devotion in his heart as he grinds. He does not speak his prayer; the sound of stone against stone are his words. He does not look up to the heavens where he knows Ýng to reside for sight is not a sense he possesses; allows, instead, his Lord's presence to pierce his heart and travel down his limbs and guide the motions of his paws.

The powder of the incense, thus created, is sifted into a small bowl, the finest silt brushed from the mortar with the very tip of his tail.

To mature incense in the quiet and the dry and the cool is to tell the second of three prayers of creation to Ýng, and Lyut again works with devotion in his heart as he unlimbers himself from where he had been kneeling and carries the bowl to the back of the cave where it will always be driest. He does not speak his prayer; the sound of his paws padding in dirt and fingertips dragging along stone wall are his words. He does not look for the shelf containing the other two incense bowls for sight is not a sense he possesses; allows, instead, his Lord's presence to pierce his heart and travel down his limbs to place the bowl beside the other two.

Lyut then cleans his board, bringing it back into his cave and replacing unused ingredients in their bowls, jars, or baskets by touch and by scent.

At last, he picks up the rightmost bowl in the line and scoots the other two up into its place and carries it to the mouth of his cave. Along the way, he bends down and lifts a dish filled with ash, and carries it with him as well.

To lay the incense trail is to tell the third and final prayer of creation to Ýng, and Lyut works still with the devotion in his heart as he tamps down the ash in the dish into a smooth plane with the tip of his finger, then draws a careful furrow in the fine powder, sowing incense in its wake. He does not speak his prayer; the rhythm of the tamping and the quiet hush of incense and ash are his words. He does not look at the boxy spiral he draws for sight is not a sense he possesses; allows, instead, his Lord's presence to pierce his heart and travel down his limbs guide his left foreclaw while the right hand follows by touch, dropping the powdered incense in its wake.

The presence of his Lord burns bright within him. Lyut does not know light from darkness, but were he pressed to answer, he would say that Ýng's presence is that of light, Their absence that of dark, and by this point in the day, Lyut is filled with light.

The prayers of destruction follow the prayers of creation.

Against a crease in the rock at the entrance of his cave is his fire pit. The night before, he brought in sticks and bark from the near-woods and laid them at the feet of the fire. In the mornings after preparing his incense, he begins the first prayer of destruction, of breaking down the sticks and shredding the bark into tinder and kindling. The sound of the crack of dry wood and the tear of fibrous bark his words, the spirit of his Lord guiding his every movement.

The second prayer of destruction is the forging or rekindling of fire. If there are embers left, then the words of this prayer are the sound of Lyut's breath against them and the slow crackle of kindling catching alight. If the coals are out, then the words of this prayer are the singing of the bow drill between his feet, thermoception stretched taut as he strains to feel the warmth of the new flame starting in the tinder.

The third and final prayer of destruction that Lyut offers to Ýng is that of the lighting of the incense. He works with the same measured care as he lights a punk from the fire, the spirit of his Lord singing along his limbs, and touches it to the small mound of incense at the center of the trail he has built. The words of this prayer are silence.

Only now does he speak his prayers aloud, and by now, he is overflowing with light. It seeps out through his fur, falls from his mouth in honeyed drops, shines from darkened eyes.

Ýng is with him now as he chants, as the smoke wreathes him, as the scent of his labors fills his cave and the clearing and rises up past the tree-tops.

Ýng is with Lyut, and I am as well.

\secdiv

\noindent After prayer, Lyut feeds his fire and sits for a while before it to ensure that the sound of the wood burning is just as it should be and no louder and that the heat of the fire is neither too hot nor too cool, for he knows that a hot-burning fire that roared and rushed with the voice of Ýng's anger was one that would at best burn out too soon and he had been taught that at worst it would claim souls as easily as wood.

With the smoke of the fire mingling with that of his incense, with the scent of his devotion lingering in his nose and clinging to his fur and stinging sightless eyes, he takes up his walking stick and pads slowly down the path from his cave to the section of river he calls his own. His feet guide him with soft shuffling. His stick guides him with gentle tapping. His ears guide him with the sounds of the river. Ýng guides him with Their hand on his shoulder.

At the river by his cave, there is a pool where the water flows out from between two rocks, and it is across that gap that he has strung a net.

Lyut sets his stick aside and crawls on hands and knees to one of the rocks and with a long-practiced swish of his fingers through the water, he catches up the cords of the far end of the net from where they lay on the bank and sweeps his arm around to draw the net around and back toward him.

I have smiled on him today, and in the net he feels the dancing of a fish and, upon dragging the net ashore, feels in its knots also the hard-shelled bodies of the crawfish that live their silent lives on the bottom of the silt-bedded river.

The net entire is laid flat upon the shore to let the fish and crustaceans drown in air while Lyut cleans his paws and knife in the water of the stream.

To wash in cold water is to speak a prayer of cleanliness to Ýng, but also to me, to me who knows the meaning of light dancing on clear water in a way the god of the sun cannot, in a way that blind Lyut cannot, and so I sustain myself with those prayers even as the ascetic guts the fish with measured care, washes once more in the stream, and then with practiced slowness strings his net once more, letting the constant stream of water flow brightly through the pounded and knotted reeds to catch fish, to catch food.

Dripping and naked, Lyut crawls upstream along the shore, fingers crawling among the grass until he comes across the fronds of a fiddle-head fern of which he plucks two. Washes these, then wraps in them his daily catch of fish and sluggish crustaceans, and packs around the bundle clay from the riverbank.

Takes then his stick in hand and taps his way back to his cave, where, after banking a portion of the fire, he nestles his bundle among the hot coals until it is dry and parched on the outside.

In the meantime, Lyut walks carefully into the woods perpendicular to the hill on which his cave rests, brushing aside further fronds to the place where his nose tells him he may have his toilet. After finishing, another trip to the river is made, this time carrying a jug slung over his shoulder to be filled with water for his camp.

By then, the smell of steamed fish is beginning to escape from the clay baker that he has formed, and the time to break his fast is upon him.

His walking stick, hard and long-cured, is used to drag the baked clay from the embers and the jug of water put in its place to bring to a boil. He says a short prayer to Ýng for his bounty, for his food, and for the taking of three lives in order to fill his belly, and by the time the last word is finished, the clay is cool enough to tap and crack apart to exposed his steamed food. I sup from that prayer as well, for I provided him with his meal.

He sets the spent clay aside and unfurls the ferns from around his food. His first bite is of the curled heads of the fronds, seasoned with the fat of the fish and the heady scent of crawfish. His second and third bites are the flesh of the fish scraped away from soft bones with sharp teeth. The rest of his meal is a silent contemplation of what wonderful complexities the silty life of a crustacean must hold, as he pulls the tails from the crawfish, eats the meat within, and sucks the butter from the heads.

Fish head and skeleton and crawfish shells are placed in the jug of water now boiling, the makings of a thin broth which will be his sustenance for the rest of the day.

For the third and final time, Lyut washes that day, and I revel in the act of his careful attention to his postprandial grooming. This is the time when he ensures that his pelt is clean and free of ticks and fleas. This is the time when he massages the dirt out of his pawpads. This is the time when he brushes his whiskers. This is the time when he lays his fur in order. This is the time when he makes himself pure in body before Ýng, having already made himself pure in spirit.

Too, this is the time when he makes himself pure before me, though he knows it not. This is the time when he gives thought to the direction his fur is facing. This is the time when he gives thought to any dirt which may cover him. This is the time when he, blind pekania, blind fisher, puts thought, however abstract, into what a watcher may see.

\secdiv

\noindent Lyut lives his life in prayer and devotion. It is a life that is lived ascending in a steady spiral of years, for time moves upward and yet is echoed below by the change of days, the change of weeks, the change of seasons. This year, this day, this soft spring is an echo of last soft spring beneath it. It is antipodal to the autumn that will come

Cycles within cycles, spirals within spirals. This morning, too, is an echo of the day beneath it, behind it, in the past. His days are defined by the cycle of incense, prayer, fishing, foraging, meditating. He knows that it is day when he wakes when he feels the warmth from the sun. He knows when it is night when he feels the warmth fade. He knows when it is morning because he hears the birds sing. He knows that it is night when the birdsong of the day settles into the chorus of insects.

Clean now, he meditates on this. He meditates on cycles. He meditates on warmth and coolness. He meditates on his relation to it, and on his relationship to Ýng.

He has surmised, for instance, that his fur is of a particular quality that the sun is drawn to, and he has surmised that this is as worthy of prayer as the incense he makes, for was not the sun with Ýng? The sun is drawn to him as it is drawn to the rocks and the dirt and the bark of the trees. It is drawn to them and it dwells within them, for the sun powers him as warmth, and the sun fills the trees with a captive warmth that is released by fire.

And are there not things that the sun shies away from? The sun shies away from night, from water, from the cool fresh leaves that interrupt it, for one need not sight to understand directionality, to understand shade as a consequence of sun's arrow.

Lyut lays on his back to let sun's arrow dry him, to let that warmth pull the water from his fur and the chill from his bones, and then he lays on his front and lets Ýng's light bathe his back as well.

Not all prayer, Lyut knows, is in ritual.

In ritual lies comfort. In ritual lies service. In ritual lies the active participation of worship, that portion of devotion that is a conversation with his Lord. The time of ritual is the time when Lyut may speak up and say to Ýng: I am here, I am yours, I am your vessel of light and all that I do is in service to you and by my very existence, my every action, I serve your glory.

Not all prayer is in service to Ýng, either, for some of it is to Their servant, to himself.

In service of Their servant, he keeps himself clean and free of sin and distraction. In service of Their servant and to Their servants, he prepares the incense that wreathes himself and the village below. In service of Their servant and servants, he subsists only off a single meal drawn from the river and whatever alms the village cares to provide him along with the ingredients for the incense that he makes in turn.

But in meditation lies the comfortable companionship. In meditation lies love. In meditation lies reassurance and trust. The time of meditation is the time when Lyut may sit next to Ýng in silence and appreciate the wonder of Them and the world that They have made.

So this morning, he lays in the sun next to Ýng, beside Ýng, and revels in all that Ýng has created rather than singing praises to Them, because it is important even for the ascetic to understand the beauty of the world, the wonder and delight in it. It is as important for Lyut to feel the way his fur tugs at the sun, collects the warmth, and the way the sun pulls the water from him. It is important for Lyut to feel the ground beneath him and hear in its silence the praises to his Lord. It is important for Lyut to marvel in the way Ýng's sun shuns the underside of leaves and follows the bark of the trees on the side it faces. It is important for Lyut to bake until he's panting and gulping in breaths of air, and then it is important for him to crawl back into his cave, stricken from the sun by the laws of directionality that he understands on a visceral level in lieu of a visual one, for sight is not a sense he possesses.

And then it is time for him to remove his simmering broth from the fire and to sip it from the cool shade of his cave, straining it through sharp teeth to prevent fine carapaces and finer bones from getting caught in his throat, unsalted but nonetheless savory, until, despite the heat of the broth, his thirst is quenched.

This, Lyut knows, Lyut relishes, is the cycle of the day, the cycle of the year, and, his Lord promises him, the cycle of his life, for he will surely be reborn when the hours of his life slow to a stop.

In this, Ýng is a liar, but it is a kind lie, a lie of omission, for when Lyut dies, \emph{I} will take him unto me. I will take him and his acts in life together into my bowl and crush and knead and he will rejoice with me and I will rejoice with him and then whatever rest he has now, whatever glory he knows now, whatever elation he may feel shall be pale in comparison to what comes after.

\secdiv

\noindent Lyut prays and works for the rest of the day, for today is the day that he makes incense for the town below.

This week is the week of fasting and next week is the week of rejoicing, and so this week he must prepare for them three times the normal amount of incense, as this is the week they subsist on smoke until they cannot tell, Zita promises him, the white thread from the black after the sun sets and the cool night comes. This is the week they live on prayer and next is the week they live on celebration, when they bake small cakes in the heat of their fires, in the heat of their ovens, and five of which Zita will leave for him.

Zita may or may not be her name, or perhaps only her title. He does not know, because beyond a few kind words, she will only pray with him and pick up the incense from the edge of the clearing before his cave and leave in its place the alms that the village provides, of flatbreads and berries, of the ingredients for the incense which they grow or perhaps purchase from other villages, who may purchase in turn from villages going south, going south and east.

So today he retrieves his board once more from his cave and on it stacks all of the ingredients for the incense of the week of fasting that will feed the village and the two amphorae that will hold it. He sings wordless hymns to himself as he works with measured care to cut the sweetgrass, to shave the calamus root, to count the cardamom pods. He sings to Ýng as he pounds and grinds batch after batch of incense until his hands are humming, until his pads are singing along with him.

And then he takes his board back into the cave and returns with the stack of ingredients for the incense of the week of feasting, with the base notes of cassia and vanilla, the middle notes of ginger and turmeric, and the top note of star anise, the spices that season the cakes that they bake in celebration, and these he pounds with laughter and with tears, for with celebration comes mourning and with devotion comes the sudden feeling of loneliness brought on by laughing by oneself.

It is evening and he can feel the sun's arrow striking horizontal by the time he finishes, and when he steps out of his cave, cradling his three amphorae to his chest, he can smell even above the incense Zita sitting at the entrance to the clearing. He walks carefully until he can hear her breathing and then sits cross-legged before her and sets the vases down between them, and they pray together:

\begin{verse}
They who make the world, \\
They who end it, \\
They who bring the thunder, \\
In Tsuari which fell, \\
In Tsuari which rose from the ashes, \\
We offer up the words of our forefathers, \\
We offer up the smoke of our forefathers, \\
We offer up our hearts to you. \\
In Ýng's name we pray, In Ýng's world we pray, \\
In Ýng's own voice we pray, \\
By the light of the sun we pray, \\
By the heat of the fire we pray.
\end{verse}

And on until the sun's arrow has wandered off course and into the night sky.

This week, this week of fasting, Zita has not brought him alms. There are no soft leaves of flatbread or ingredients for incense, just as one year ago there were no leaves of bread, and one year before that, there were no leaves of bread.

This week, Lyut does not smile kindly to Zita as she collects the amphorae and walks the path down the slope to the village, because the fasting of prayer is also a fasting from emotions and worldly attachments.

And the next day, it is truly a fast, for there are no fish in his net, and if there are no fish in his net, he knows that he must not collect the fiddlehead ferns, and instead of savory broth, Lyut drinks only boiled water, hot and cleansed by fire, and he spends the rest of the day in meditation, and he goes to bed hungry.

I watch as he sleeps, fitful, and leave for him two fish in his net for his unknowing devotion to me.

\secdiv

\noindent It is the last night of the week of fasting and it is the thirtieth year that Lyut has served Ýng and myself that I have decided to change him and by changing him, change the world, for while I am the god of the water and the god of watching and the god of death, am I not also a trickster god?

I am the trickster god who confounded Ýng in Their creation of the smooth plains of the world by carving the land with my rivers. I am the trickster god who confounded the Lord by setting the moon in the sky to tug at the waters of Their oceans in tides, even when the moon is not seen. I am the trickster god who brought death to Ýng's ever-living world.

I am the trickster god and my trouble will come back on me thirtyfold, I am sure, but Lyut is the thirtieth ascetic who has served me and I am ready.

Lyut has once more gone to sleep hungry, belly filled with prayer and contrition and boiled water. No fish in the net, no ferns to be had, no stale leaves of flatbread or sun-dried berries. I come to him then. I come to him and I touch the back of his neck, then the crown of his head, then the lids of his eyes and the scars around them, and then I sit in the clearing and wait for him to waken. I sit and watch, for that is my jurisdiction.

When the pekania stirs at the slow warming of day, his eyes drift open as usual to the slit of relaxed muscles that is his habit, and then he shouts.

He shouts because I am a trickster god and after forty years of life, after thirty times thirty years of blind ascetics serving Ýng and myself, I am ready for change and I have given him sight.

I know his thoughts: I know that when he perceives the light of the sun for the first time in his forty years, blurry and bright, that he is struck with a mighty pain and a fear far greater than any accident with a knife could cause. I know his terror, his confusion, and his instinctual need to escape, and so I watch him scramble back into his cave and press his face to the back wall for minutes on end, barely breathing, eyes clenched shut.

``Ýng!'' he cries at last. ``My Lord, my Lord, what is happening?''

I answer in Ýng's stead: ``You see.''

He pants into the silence that follows. I know his thoughts: I know that he hears Ýng within his heart and within his bones and within his breath. I know that I have spoken to him in the language of sound, and that this brings with it its own fear.

``You see,'' I say again.

``You are not Ýng.''

``I am Týw. I am the god of the moon and the water and of watching and of death.''

``Týw?''

``Týw,'' I repeat, and smile at his confusion.

``But Ýng is the god of all things. How are you the god of those things?''

``Ýng is the god of all things, and They are the god of me, but of those things not under Their direct dominion, some are under mine, and I am the god of watching, of looking, of seeing. I am the god of water, and I am with you when you fish and bathe. I am the god of the moon, and when it shines down on you, I am with you. When Ýng is with you, I am as well. When you serve Ýng in these ways, you also serve me.''

Tears course freely down his cheeks, and he says: ``It hurts to see.''

``You have never seen before. Come out of your cave.''

He does not move, and so I wait. I know that he will need to attend to his day soon, and I know that he is praying to Ýng and feels the compulsion to perform his acts of service, his rituals, and I know that the village below is waking up to ready itself for a day and night and week of celebration. So I wait.

Too, Ýng waits, because although I sense Their wrath on the horizon, I think that it will not come yet, because this is also new for Them, and They also watch.

Eventually, Lyut crawls, eyes clenched shut, on hands and knees, crawls out into the sun, and sits cross-legged in the center of his clearing.

``Open your eyes.''

He does not. I know that he can see the warmth of the sun behind closed eyelids, showing dusky orange through them. I know that he can sense the shadows cast in the sun's arrow by the leaves above and around him. I know that even this seeing is too much for him.

``Open your eyes, Lyut, faithful.''

``You are not Ýng, you cannot command me.''

``No,'' I say. ``I cannot command you, but you are as faithful to me as you are to Them in the ways that I have described, and so I ask for this small obeyance.''

Lyut ponders this for a long while, his tail flitting agitatedly behind him, drawing praises to me in the packed earth. Finally, he opens his eyes, a crack, a squint. He opens his eyes and looks at the ground before him. He looks at his naked body. He looks at the clearing and at the trees around him. Looks in wonder. Looks in awe. Looks in terror and in panic. Looks at the ground and the trees and the sky. Tries, even, to look at the sun, and learns that the sun's arrows are keenest above all to the eyes.

``It hurts! It hurts!''

``Do not look directly at the sun, faithful,'' I laugh. ``Ýng has decreed that the sun provides your life, and so it is too dear for you to behold.''

He grinds his palms against his eyes and smears his fur with tears and with dirt. Even as he cries, he is marveling at the flashes and swirls of light that come to him now, and each phosphene that blooms in pink and white and green is a prayer to me, so I allow him this moment of non-darkness until the moment passes and he can open his eyes once more without pain.

``Where are you, Týw?''

``I am with you.''

``Can I see you?''

``We are also too dear for you to see with your eyes, Ýng and I, but do you not feel the way we pierce your heart and burn along your arms as you prepare the incense for our offering?''

Lyut is silent once more, still once more. He prays. He prays to Ýng with a fervor he has not yet shown in his forty years. Tears stain tracks down his cheeks as he struggles with the sudden, overwhelming sight. Sight, a sense he now possesses.

``Go and prepare for your day, faithful. I am with you.''

\secdiv

\noindent Lyut is slow to begin moving, and when he does, he walks as though a great dream has come upon him. He lets Ýng guide his movements and I stand apart from the Lord and Their servant.

Lyut moves as though a great dream has come upon him and lets Ýng guide him, and even so his morning task of making incense is far slower than usual, for his eyes water constantly and he marvels at just how drab the ingredients, so bright and colorful in the nostrils and so familiar to the touch, are to behold. He has not known the comparison of color before, but even to one for whom sight is a new sense, he is surprised to find that the crushed root of nardin and the shaved root of sweet flag look so similar despite the vast difference in aromas and purposes, that the mastic, that steadfast base of a scent, nearly glitters in the sun while the jewel-bright scent of cardamom is belied by so dun a color.

He moves as though a great dream has come upon him until it is time to lay the powdered incense in the bowl of ash, that third prayer of creation, and he realizes that he can see the furrow he digs in ash with his claw, can see the tan powder that he packs in its place, and can see the spiral he builds, and then tears come upon him once more, and all of his prayers of destruction are completed through sight blurred by shock, and he relies on his habits and Ýng's guidance to make it through to the end without burning himself.

I stand apart from the Lord and Their servant and watch, and drink in what prayers I may along the way.

At last, the time for ritual passes and Lyut stumbles into the woods to tend to his toilet and lingers a while in wonder at the sight of his own body, the sight of the woods and the leaves and humus on the forest floor, before returning to his cave and, out of the habit of so many years, grabbing his stick to guide him down to the river.

``Do you need that, faithful?''

After a moment's confusion, the fisher laughs. ``I suppose I do not, Týw.''

``Will you leave it behind?''

His answer is a long time in coming. ``It is comforting in my paw. I will take it with me.''

Guided still by habit---and perhaps by Ýng, for I do not know the Lord's every thought---Lyut taps his way down the path to the water, and perhaps it is for the best that he has brought the stick, for his eyes are drawn constantly to every detail along the way, from the way the suns arrow strikes the leaves to the way their shadows dance across the ground when the wind moves across them. His eyes water still, for he is overflowing with sensation. A life lived without a sense is still a full life, and to one born without that sense, raised without that sense, he did not think of himself as blind except in comparison to Zita who picked up the amphorae of incense with such ease that he had never known.

Stops, at last, at the edge of the stream and stares at my domain, mouth open as though to speak, though no words come forth.

I wait a while, and then ask: ``Faithful, do you see the wonder of my creation? My friend the water?''

``I had never imagined that it looked like this.'' His voice is barely above a whisper, and his eyes drink deep of the sight of the stream. ``I did not know that something could be as beautiful.''

This fills me more than any prayer yet that day. ``I am the god of the water and the god of watching and the god of the moon and death. When you come here to fish, when you come here to bathe, when you come here to drink, those are praises that you sing to me.''

Lyut tilts his head. ``Is Ýng not the god of all things? I am sorry for asking again, but I must know.''

``They are the god of many things, and They are the god of me. To sing praises to me is to sing praises to Them in turn.'' At this, I feel the Lord's anger at me soften, though it does not wholly retreat.

``I do not know the words to any prayers to you, Týw.''

``That is alright, faithful. You may pray all the same by fishing and bathing and drinking, by rejoicing in those things that are under my jurisdiction.''

Lyut nods and steps into the water. This is not the usual order of his mornings, but as the wonder on his face at the sight of the water moving around his legs fills me to overflowing, I do not complain. He stands in the middle of the section of the stream that is his own, in the pool held up by the narrow gap across which he strings his net, in the cool water where the sun's arrow pierces the canopy of the trees. He stands there and he watches the way that the light reflects off the surface of the water. Watches, too, the way the water eddies around rocks, around his legs, explores the funnels of whirlpools with his fingers, peers through clear water to the silt and rocks and algae below the surface.

``What am I now, Týw?''

``What do you mean, faithful?''

``Before this morning, before today, when I did not see, I was complete.''

I remain silent.

``I am sorry, god of water and of watching. I do not doubt you, for your gift has spoken for you. I do not turn away your gift, and I offer my praise to you. But if I was complete before and a servant to Ýng, then what am I now?''

I watch him curiously, this servant of mine and of my Lord's, standing in the middle of a pool in a stream where his thighs are steeped in the cool water. ``You are Lyut, faithful of Ýng, faithful of Týw. Has that changed with your sight?''

He runs his hand above the water, feeling the boundary between water and air with his pawpads. He feels the surface tension of the pool, and through him I feel his wonder. He tests and plays as might a kit of his people even as he begins bathing. Each time he comes up for air, he sings a line of praise to Ýng, and every time he is beneath the water, I know that he is thinking about what he is now. Each time he dives, he is singing his praises to me as well, and now he is cognizant of this as well.

After he has said his prayer and cleaned himself he wades to his net in which he finds three small fish. He gives thanks to Ýng and, after a moment, to me as well.

With the fish on the shore, wrapped in net and stunned, gasping and drowning in air, Lyut watches. He watches them glitter and wiggle. He watches them die their slow deaths. He traces sun-struck scales with a claw and asks: ``Do the fish see beneath the water, Týw?''

``Yes, faithful. They see my domain and all its beauties.''

``Do they smell beneath the water?''

``After a fashion, yes.''

``Do they smell my incense?''

``No, faithful. The boundary between the domain of air and the domain of water is too firm for the smoke of your incense to pass. After all, do you smell your incense beneath water?''

``No, I do not breathe under the water.'' Lyut looks angry, then laughs. ``Only, I wonder.''

``Yes, Lyut?''

``I wonder if the fish upon the shore here has the chance to smell the incense and hear the prayers to Ýng before it dies.''

I do not answer directly, saying instead: ``You are not going to die, faithful.''

He looks satisfied at this answer and I realize that I have said what he needed to hear. I know that Lyut holds terror in his breast even still, that he will hold it there until the end of his days, for I have taken his innocence from him. I am pleased to see his satisfaction, and I sense Ýng's bemusement at my anxiety over pleasing a servant.

I am pleased all the same, and I remain with my servant.

I am with Lyut as he gathers his fiddlehead ferns and pawfuls of clay. I am with him as he sets his net once more. I am with him as he cleans his fish and heads back to his cave to prepare his daily meal.

Three times, he closes his eyes and his whiskers droop as he attempts to settle back into his unseeing routine. He is testing himself, I know, and I do not stop him. I do not stop him because I know that when his eyes are open, he is closer to me, to Týw the watchful, and when his eyes are closed, he is closer to our Lord, Ýng, the god of all things, and it is good for him to understand this.

He closes his eyes to shut out the sight of preparing his meal, too confused by the twisting of the ferns around his fish. The leaves which make so much sense to his long-practiced fingers do not behave to his eyes the ways in which he expects.

He closes his eyes to eat his food after cracking open the clay baker, for the sight of the fish changed by fire is unnerving. The change in texture he had always known had changed, as too with the taste, for Lyut was no stranger to the flavor of raw fish. Now, sight-ridden, he finds the taste of the fish reduced when his eyes are opened, as though too much of him, of his mind, his being, is taken up processing that which he sees.

And he closes his eyes, last, when he lays on the ground to dry and meditate.

He closes his eyes as he lays on his front, and then when he rolls onto his back, he keeps them closed, and I see his cheeks wet with tears.

``Speak to me, faithful. Why are you troubled?''

``You say that you are the god of watching, yes?''

``I am.''

``Must watching always be with sight?''

Again, I do not answer directly. ``Do you wish now that you had not regained your sight?''

``It is too much, Týw.''

``You are strong, faithful.''

``It is too much.'' He shakes his head. ``I feel less holy. I feel less pure when distracted by seeing. How can I serve Ýng as faithfully now that my time spent watching is time spent serving you?''

I feel Ýng's anger rising against me once more, and I answer carefully. ``To live is to be holy, to live and rejoice in life, to be pure and clean in your actions and words. Ýng is the Lord of all things, and to Their servants They gave life as a way for the universe to recognize its own beauty and wonder.''

Lyut's face twists in anger. ``And yet I cannot hear Ýng as well today as I did yesterday. He is with me, I know, but\ldots''

``The only mind that can hear as purely as it sees when both eyes and ears are open is that of Ýng, true, and yet in seeing, do you not also praise Them? It was They who made seeing as well as hearing. It was They who made me.''

At this his features soften. His words are slow, and he processes his thoughts and feelings aloud. ``I, as a servant, do not understand the hierarchy of the gods, but, yes, if Ýng made the light and the sun and colors and also you, then I suppose I pray to him as easily by rejoicing in sight as I do in sound and touch.''

The sun is overhead and tipping down its long path through the afternoon. The colors of the trees are bright and I am with Lyut. ``Rejoice, then, in your sight, faithful, for in doing so, you offer prayer to Ýng and to myself.''

A slow minute passes as the fisher meditates. At last, he opens his eyes and looks up to the trees and cloudless sky.

``I will try, Týw.''

``That is all we ever ask of our servants, Lyut.''

\secdiv

\noindent When Zita comes up from the village, bearing an armload of flatbread and a small basket full of spice cakes for Lyut, he had since ceased his conversation with me and had ceased meditating by laying on the ground, and had instead settled for sitting cross-legged in the entrance to his cave looking out. Zita sang as she walked, as she had for the last ten festival weeks that this had been her duty, and so Lyut hears her before he saw her.

He debates for thirty heartbeats whether or not he is willing to keep his eyes open for her arrival. He debates whether or not he is willing to see, to perceive someone with senses other than those he had been born with.

Lyut makes up his mind and closes his eyes when he hears Zita rounding the curve of the path toward the clearing before his cave. He sees her shadow move in the trees, he sees a hint of her between the trunks, and all courage fails him in that moment.

``Faithful, why do you close your eyes?''

Lyut stays silent.

``As you wish, faithful, but know this: while some miracles are private and must be held close to the heart, not all of them must, and to hide this one would be to live a lie before me and before the village.''

``I am not brave enough.''

Zita's singing crescendos as she enters the clearing, then abruptly stops. Lyut supposes that because he is not sitting in the customary place with the customary smile on his face, that she must sense in him some change beyond her ken, and at this, his fear only grows.

He turns over what I had said within his head. He turns it over ten times and considers the ramifications of it. Were he to keep his newfound sense a secret, then yes, he would in some way be living a lie. He would have sight at his disposal and yet the village would know not of the incredible power of the gods that had granted it to him. And yet there was terror to be had at the thought of anyone finding out. He was holy in part because of his unseeing, was he not? He was pure before Ýng at all times, and he was pure in the ways that the village could not be, for that was his role as the ascetic, as the incense-maker, as blind Lyut.

And yet to lie is to sully oneself. To lie before the village was to betray his role as ascetic and to make himself less holy in the eyes of Ýng. To tell the truth was to test the village and change tradition, but to lie was to destroy it for the sake of the village.

To live a lie until Ýng took him and decided at what point in the endless cycle should be placed his death was too terrible a thought, and the need to tell truth, to remain as pure as he could be, won over in his mind.

``Lyut?'' Zita speaks, tentative.

And so he opens his eyes. He opens his eyes. He opens his seeing eyes and looks across the clearing and sees Zita there, shorter than him, softer and rounder than him. Too, she is better fed than him---though that is not his place in the world---but she is different on a level more fundamental than any he could have imagined. She is, he thinks, unlike anything he had expected her to be.

He smiles. ``Zita.''

That he had opened his eyes and looked upon her seems to startle Zita, and she takes a half-pace back away from the cave.

He speaks as calmly as he is able, but he does so quickly as to preempt her leaving. ``Zita, Ýng has blessed me this day. Ýng and Their servant have blessed me, and when I awoke and opened my eyes, I saw. I saw for the first time.''

She frowns and walks toward him. She moves slowly, and then steps a few paces to the side when she is halfway across the clearing to approach him from a diagonal. It is a test, I know, and when his eyes track her movements, she rushes to him and sets down the bread and cakes beside him.

``Ýng has done this?'' she says quickly and quietly. ``Ýng has worked a wonder! Such a wonder!''

``Yes,'' Lyut says. It is a small lie, but one easily fixed when first the topic of me, of the god of sight and of watching comes up. ``Ýng has granted me sight. I have been praying and meditating, and I do not yet wholly know the reason why.''

Zita's eyes dart this way and that as though to take in all of his face, to look at his eyes and to check for the scars that Lyut had sometimes felt beneath his fur while washing, though he knew not where they came from. At last, she looks into his eyes for a long while.

This makes Lyut uncomfortable, and he does not rightly know why. Was there something to behold there? He can see her eyes, and is seeing them for the first time, and to do so fills him with anxiety. They are round and dark, and seem to be made of a ring of brown surrounding a circle of black, and as her eyes move, he sees that the circle of black sometimes grows larger or smaller, though perhaps it is some trick of the light.

But those were simply the mechanics of sight. He can see her eyes, yet he feels that to look directly into the eyes of someone else is to \emph{truly} see them, and he worries that, on some level, Zita will be able to read his thoughts and fears, that she will know deeper secrets about him than he could possibly ever know about her. Was this some knowledge of the sighted that he must someday learn himself?

As well, this close to her and he can smell her better than he ever had before, and she is in no way, in no sense unpleasant.

The feeling of being sullied and unholy hangs around him like a cloud.

He asks, then, quietly: ``What do you see, Zita?''

``I see you as I always see you, but I see you with your eyes open and clear, where they used to be cloudy and dim, and I see your fur brown and thick without the scars that my mother says have lined your eyes since a year you were born.''

``Yes, but what do you \emph{see}?''

Zita finally averts her eyes, though only to pick up a cake from the basket and split it in two, holding out one half for Lyut and keeping the other for herself. The cake is the color of the sun and bespecked with the cassia and cardamom which had gone into the incense. ``I see that Ýng has wrought a miracle and that our time of fasting and keeping holy has led to something truly wondrous.''

Lyut lets his shoulders relax from a tenseness he had not known he was holding, and he accepts the spiced cake from her. ``I see. Thank you, Zita. I have been praying and meditating on this all day, and though I know I must not, I doubted this miracle and felt unholy.''

She bites into her cake and chews, her eyes focusing seemingly on nothing. Lyut can hardly read her expression, so new is his sight, so he remains silent. She swallows her cake and says: ``I think that you are as holy now as you were at the beginning of the time of fasting. You have kept holy as have those who came before you, and the village has kept holy, and perhaps the whole world has kept holy, and now Ýng has provided for us a new thing.''

Lyut eats his spice cake and thinks on this. He thinks about what I had told him. He thinks about the shock of sight, still so new to him that the brightness and colors in the world sting his eyes and bring him to tears. He thinks of the newness in things that have always been there. He thinks of how overwhelmed he is by this mere fact, and he thinks about how small he is before me and smaller still before his Lord.

He thinks about how small he is and realizes that his devotion burns more strongly within him than it had ever before. And, though he does not know or understand my motives, he knows that any servant, that \emph{every} servant of Ýng's is master of him, for the most holy are truly the servants of servants.

He thinks about this and then he smiles to Zita once more and nods. ``Yes. Yes, this is a new thing that Ýng and his servant Týw have done, and in their presence I will continue to be holy.''

Zita tilts her head to one side, and Lyut wonders if perhaps she had not heard well. ``Who is Týw?''

I break my long silence and say, ``I am.''

Lyut stiffens and Zita startles to her feet.

``I am Týw, and I am the god of the water and of the moon and of watching and of death, and I am servant to Ýng, and I have given sight to Lyut.''

When Zita understands, she falls to her knees and prostrates herself before Lyut, seeing no one else to bow before. ``A spirit! A spirit!''

Lyut laughs at this, though not unkindly. ``I believe Týw, that They are the god of the water and of watching, though I know not what the moon is. I have prayed to Ýng about this and I believe that Týw is Their servant.''

``I am. I have given Lyut sight and Ýng is watching all of us.''

``I cannot see you, though,'' Zita says.

``As the sun is too dear to look at, so are the gods, faithful.''

``How can I be your faithful?'' There is an edge of frustration to her voice, and her tail dances about behind her. I accept her agitation just as I accepted that of Lyut.

``Every time you bathe or drink pure water, every time you keep watch on the world, every time you behold the beauty of the moon, and every time you mourn the dead, you give praise to me, for not all prayers are in words, as Lyut well knows.''

He nods in agreement.

``These things are my dominion and Ýng is my Lord in turn.''

Zita sits up slowly. Still frowning, she considers this. ``Why have you given Lyut sight?''

``That is not for you to know, faithful, not yet. There will be a time when you may, however.''

She relaxes at my words, for she knows the workings of the gods and the mystery therein almost as well as Lyut does.

``Now, it is almost evening,'' I say. ``Put away the bread and the cakes lest the night animals take them.''

Zita nods and moves to help Lyut gather his food before remembering that he can see the basket and the flat loaves of bread as well as she, and they laugh together.

After the food is put away, both fishers kneel together and begin to pray aloud to Ýng.

\begin{verse}
They who make the world, \\
They who end it, \\
They who bring the thunder, \\
In Tsuari which fell\ldots
\end{verse}

I let them finish their prayer and bask in the jubilant way that Zita's voice rings out to her Lord.

When they finish, Zita smiles to Lyut and stands once more. ``I must go down to the village and tell them of this miracle. Tonight you will see the moon, holy one, and know its beauty and that will be your praise to Týw.''

The thought fills me with joy, for the moon is indeed beautiful, and I watch Zita put her arms around Lyut in an embrace---his first in many years---before departing down to the village once more.

\secdiv

\noindent Lyut stays up late into the night at the promise of the moon. Night is not day, this he knew, and the subconscious understanding that the sun brought light would mean that the absence of the sun would bring darkness does not surprise him.

He remains curious about all things. He marvels at the red and pulsing glow of the embers of his fire. He wonders at the way the sun's arrow disappearing colors the sky pink, purple, navy, black. He drinks in the way in which the color drains from the world.

The first night of the week of feasting is the night of the full moon, which Lyut had known but had not understood, but now he does. He understands the moon and its importance when first it creeps into view of his clearing. He understands its beauty, and he weeps. He weeps for my creation, and I am filled with praise unclouded by words. Filled to overflowing as I have never been since Ýng created me at the beginning of all things.

And that night is the night when Ýng comes to me and makes his decision.

The next morning, a second strange occurrence greets Lyut when he opens his eyes. Sitting at the entrance to his cave is a creature very much like him in many ways, but in many ways different. Long and lithe, yes, strong and slender, yes, but shorter, and with fur of the purest white as opposed to the dark brown of his own. A face more slender and ears larger, and on the tip of his tail, the fur is dark black.

``Who are you?''

I smile to him. ``It is I, faithful. It is Týw.''

A look of confusion comes over his face, and I must hold back amusement as the fisher sits up and rubs his eyes, looking around as though the answers were to be found in the air itself.

``Týw?''

``Yes, faithful.''

``I thought that the gods were too dear to be seen?''

I close my eyes. I revel in the blackness this brings. I revel in the feeling of terror and the exaltation that come with being embodied. I revel in the power of our Lord. ``Yes, this is true. This has always been true through the long years and longer millennia. However, I was not completely honest with you yesterday, Lyut.''

He frowns, staring intently at me in my new form. ``If you are a god and you are holy, how can you lie?''

``It was a lie by omission, for I am the god of water and of watching and of the moon and of death, but I am also a trickster god. I am the god who sows chaos while Ýng brings order. Forever we work together or strive against each other. Forever we move in a cycle. This is our very nature. This is the way of things, for Ýng must have something to strive against that time move forward and his creations grow and change with it.''

Lyut sits cross-legged and bows his head as he thinks on this. He knows that, on some level, it must be true, for there are times when the weather is bad for days on end and he cannot---or could not---tell the difference between day and night, and there are times when he will go a week without food from the river, and once there was even a time when something happened to the water of his section of the stream that caused it to taste bitter and plant-like, and no amount of boiling could remove the flavor and he was sick with fever.

``You sow chaos and Ýng fixes it?''

``There is no fixing chaos, faithful. I sow chaos because that is who and what I am. Ýng brings order because that is what They are. There is no moral ground on which to judge the chaos that I sow, just as there is no judgment to be made on the order of our Lord. Both are holy in their own way, because they are the chaos and order of gods.''

``Is the chaos of your servants not holy, then?''

``It is not. It is my role in the world to sow chaos so that you may learn and become better for it, but when you sow chaos for each other, you lower yourselves in our eyes.'' I see confusion on his face and sense questions in his mind, but he does not speak, so I continue. ``The chaos sown by living beings is an exchange of power. Inevitable, perhaps, but it bespeaks a lack of devotion.''

Lyut frowns as he considers this.

I give my servant time, for he has learned more in the past day than any of his predecessors have in their spans.

``So then,'' he says at last. ``How can I see you now? What are you?''

``I am the god of watching and of water, of the moon and of death, and I am a trickster god, but all of these things are a part of the world separate from you. I am, this body is, the concrete manifestation of myself and I will take this form for a time. I am this concrete manifestation because I committed a concrete act by giving you sight, and the ramifications to me are also concrete.''

``You made it so that I can see you?''

``No, faithful. Ýng has made it so that you can see me, for They are my Lord and I am Their servant, and I sowed chaos and They have in turn brought order to \emph{me}. At least, for a while.''

Lyut looks startled at this. ``Is it a wicked thing that you have given me sight? Have you made us both unholy?''

``No, faithful, dear Lyut.'' I smile and hold up my hands. ``It is good and holy that you may see, and Ýng agrees. However, They control the balance, and so they have decided that the balance, the exchange, for you seeing is for me to be seen. I will live for thirty years among the world in this embodied form, and you will find that the chaos that I bring is vastly reduced while I am here, for in this form, I cannot work my usual methods.''

``Is that not a punishment, for a god to have their power lessened?''

I laugh. ``No, I do not think so. Ýng was at first angry with me and perhaps They wished at one point to punish me. But They understand now, and this is instead a matter of me experiencing what you experience in the way that only a god can, for gods must learn and change along with their servants.''

He thinks for a long while on this, and I know that he is praying to Ýng throughout, that he is closing his eyes so that his hearing is sharper and his smell is more keen and perhaps his sense of the holy is as well. I do not interrupt his prayer, for Ýng is with both of us. I pray with him. We sit in silence in the cave and hear the wind and the stream and the birds, and we smell the cassia and cardamom and copal, and we share our prayers.

``Týw,'' he says at last. ``I have faith in Ýng and I have faith in you that I will remain pure and that the world will remain pure with us. I do not understand, but I have faith.''

``Good. Now, I will teach you to see, faithful, and you will teach me to be seen, for everything---\emph{everything}---will be different now.''
