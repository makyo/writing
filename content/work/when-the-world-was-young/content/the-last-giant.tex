The golden torc, covered in feathered designs, felt heavy around Prince Ennachu's neck and across his shoulders. He had to resist the temptation to scratch at the lime encrusting his hair, turning his usually brown fur white and forcing it into spikes. He was surrounded by the hushed murmurs of the sailors and soldiers his mother had handpicked for this raid. Their susurrus could not quite drown out the soft slap of the waves beneath the gentle creaking of the boat they rode and the two that followed in their wake. They were mostly wolves, the derbfine of Ennachu and his mother, a few others trusted enough to accompany them, and the druid. Ennachu found himself watching the cloaked skua where they stood, impassive, at the stern of their ship. Their hood was up and all Ennachu could see of the druid was their black, hooked beak. It clicked impatiently from time to time.

``My son.'' Queen Katodirean laid a hand on his shoulder, startling him out of his reverie. She and he looked very much alike, though she had the wisdom of many long years of experience, reflected in her eyes, golden like his, and the streaks of white along her muzzle. Her hair, a little longer than his, was also spiked with lime, but it had long ago gone white of its own accord.

``Mother,'' he said, his voice low. ``Has the druid sent you?'' He could not entirely conceal the suspicion in his voice.

She tsked at him. ``Soita wants only the best for you. One day, you will have to take up the throne and\ldots''

``And defend our land from our enemies. You have said, Mother.''

``And you have not listened, my son,'' she said, her voice as sharp as the biting winds over the dark and temperamental sea. Ennachu tugged his blue and white cloak a little tighter against the chill of both. ``Soita says that only a tusk from one of the giants of Albiyu can protect us from the southern invaders.''

Ennachu scowled, his ears twitching back before he could summon the wherewithal to hide that betrayal of his unease from his mother. The giants of the island to the south of their home on Iweru were renowned for their fierce prowess in battle and their territorial natures. They had woolly hides capable of turning back even an iron sword, and long, curved tusks that could disembowel a band of warriors as easily as a turn of their head. Ennachu looked over the company rowing across the sea that separated his home from Albiyu. It had seemed a mighty force, twenty strong, when they had left the tribe's lands on the cliffs of Mena. In the face of even a single giant, it seemed paltry.

``And how are we to defeat such a foe?'' Ennachu asked. ``Will the druid offer some of their hidden magics for once?''

He jumped as a soft voice hissed in his ear, ``Do not make light of that you do not understand, child.'' Ennachu whirled to find Soita, the druid, standing just to his left. This close, Ennachu could see the druid's red-gold eyes, gleaming like fire under their dark hood. A faint, almost cruel smile graced their short, hooked beak.

``I'm not a child,'' said Ennachu, full aware of how childish it sounded. It was true that he had passed his majority some seasons before, but, in truth, it had not been that many seasons. It was very likely he was the youngest person on the boat, save for one or two others.

Soita snorted. ``You will take the giant's tusk. I have seen it.''

Ennachu stared.

``You have trained for this, Enni,'' said the queen. ``You are ready. I have faith that you will carry out this most important task.''

``Mother,'' Ennachu turned, hoping to impress on her how foolhardy the druid's task was. If he had had any belief that Soita was capable of such simplistic plots, he would have assumed that they were trying to get him killed. ``This is madness. I cannot kill a giant.''

``It is foreseen, in smoke and ash and entrail, child,'' whispered the druid. They swept to the bow of their ship and Ennachu, in spite of himself, followed, straining his ears to hear the skua's soft words against the wind and slapping waves. ``Again and again, I have cast the bones and spoke with the sídsat. The wind whispers that you will, you must, do this.''

``Madness,'' Ennachu repeated.

``Silence, child,'' Soita was bent low, their beaked face and burning eyes were inches from Ennachu's muzzle. ``Silence and listen. The giant awaits in a camp in the forests overlooking the coast. It does not know why it waits, but it will wait for you. You must strike a tusk from its head. With that, we can save our home, save all of Iweru from the invaders from the south. The ocean we cross will be forever closed to them.''

``Can you guarantee it, Soita?'' asked the queen.

``They will slaughter my siblings, there,'' the druid pointed towards the sandy shore, ``and their governor will stand at that spot and stare with jealousy at your home. You will live to see it, Ennachu. But they will never cross this ocean. Their empire will crumble and wither and die, and Iweru will stay safe for more than a thousand years. \emph{If}, princeling, you recover that tusk.''

The shore, for most of their journey just a distant green line capped with grey-white mountains, loomed large now. Before them lay a soft, sandy beach and rocky hills covered in thick green forests and then the distant mountains beyond.

``Why me?'' Ennachu asked, running a hand through his spiked hair. It just made the lime more uncomfortable.

Soita answered his rhetorical question. ``Child, great forces are in motion. They move in all of us at times, and at this time, they move in you. The choice you make over the next few hours will have great consequences across these isles. Now, there, you see? The giant lights its fire.'' Once again the old skua pointed towards the distant shore.

The faint, flickering light grew so slowly that at first Ennachu thought he was imagining it. But as the flames grew, and the boats neared the shore, it became impossible to deny. Unconsciously, he checked the beautiful bronze sword he had inherited from his father, which hung at his waist. He brushed his fingers against the iron axe his mother had given him, a gift for the raid, tucked safely into his belt.

``It is your command, my son,'' Queen Katodirean told him, as they reached the shore. Several of their clansfolk hopped into the shallow water to pull the boats onto the sandy bank. When they had beached, and one of the warriors had handed Ennachu a wicker shield painted with the blue and gold sigil of their clan, Ennachu leapt down. The sand felt cool and soft under his paws, but as shifting and treacherous as the sea it abutted.

He turned to face his warriors. They stood, silhouetted by the setting sun, against the salt-scented waters. Ennachu could just see, half-shrouded in mist, the distant cliffs of Iweru. He thought of his home, conquered by the southerners who had already landed on Albiyu, who had demanded tribute in the name of an emperor who had come and conquered and left, as uninterested in the wild lands of these northern isles as he was of their people. Ennachu thought of life under the heel of a foreign lord who knew nothing of custom, of duty, of what the land demanded of its people.

He drew his sword and raised it above his head, so that it flashed red-gold in the light of the dying sun. Every eye turned to him.

``We seek the tusk of a giant!'' he announced. The warriors fell silent, a mixture of eagerness and fear flashing across the faces before him. ``Our druid says that only with the giant's tusk can we protect Menu and Iweru from the invaders of Rhow. That makes our mission here a sacred duty, in the eyes of the gods and our ancestors. The creature awaits us there,'' he swung the sword around to point at the glimmering distant fire, ``and by all of the gods, in the names of our forebears, we will take that tusk!'' He thrust the sword into the air once again, and the warriors cheered and shouted oaths that they would do whatever it took to save their home.

``You did not mention that it is you that needs to slay the beast,'' the queen murmured as Ennachu turned and began to lead the warriors into the forested hills.

``The druid says I must take the giant's tusk. They said nothing about me striking the final blow.'' Ennachu caught Soita giving him an appraising look, and he stiffened. ``I will recover the tusk, but this is not an honor fight between clans, Mother. I will not risk the safety of our home in vanity.'' The druid tugged their hood down a little closer, but Ennachu thought he caught a smile on the skua's beak.

Ennachu stared at the druid, but they merely turned and stalked towards the distant flickering firelight. Ennachu glared at the back of the druid's cloak as he followed, the muffled sounds of his small army marching behind them.

The land was wild, the ground overgrown with grass amidst rough, rocky hills. Nonetheless, the hike up the mountain was not too difficult, though the sun had fully set by the time they reached the scrubby woods. Even there, the trees were spread apart and the undergrowth sparse. The comfortable smell of pine filled their nostrils, blocking out other scents, and the soft needles of the trees quieted their steps. By the time they had ventured deep enough that they could no longer see the coast, Ennachu had begun to forget that a world existed outside of the forest. Everything seemed muted, faint, a lifetime away from the tiny existence he found himself in. He focused on Soita's back, because every time he tried to look around, a wave of vertigo washed over him while tiny will-o-wisps flickered in the far distance. The druid's steps were sure and unwavering, and Ennachu, against his better judgment, followed them.

From the size of the fire they had seen from the beach, Ennachu had estimated an hour, perhaps two, before they reached the giant's camp. It was closer to three before they found the clearing. Ennachu realized immediately that his sense of scale was entirely off---this was no puny campfire, but a roaring inferno that stretched almost to the tops of the trees.

And there, sitting on a log nearly as large as the ships upon which they had sailed to this island, her back to a cliff down the other side of the low, hilly mountains, was the giant. She was bigger than Ennachu had been expecting. He doubted that, had she been standing, his head would have reached her midriff. Thick legs ended in heavy, round feet, and equally dense arms with beefy hands that Ennachu thought could have grabbed him around the torso as easily as he could pick up timber. Her head was unlike any creature he had ever seen before, with a long, blunt face terminating in an even longer, prehensile trunk. Two ivory tusks jutted from her upper jaw, curved like scythes and bound in burnished metal, perhaps copper, though it was hard to tell in the firelight. The giant was covered in a dense layer of shaggy brown fur, enough that she didn't require even the hide tunic and kilt she wore to protect her from the elements.

In her trunk, she held a branch thicker than Ennachu's leg, and she was whittling at it with two bronze knives, one in each hand, shaping a tool that Ennachu did not recognize. She had not yet seen them, hunched as they were in the darkness at the forest's edge.

Ennachu pulled his sword free, slow and gentle, and it slid from its scabbard with no more than a whisper. It was enough, though. The giant's large, flap-like ears twitched and she lifted her head to stare into the darkness. Across the flames, Ennachu couldn't see what color they were, only the burning red of the reflected fire.

He raised his hand and, with a gesture, ordered the attack.

Ennachu saw instantly that it was a mistake. They were too slow and too far away. Not by much, but by enough that it would matter.

The giant was on her feet, roaring in rage tinged with something else, something Ennachu didn't quite recognize. Dropping a knife, she pulled a long spear of ash from a cache he had not seen behind her. In one smooth motion, she launched it. The spear, copper head reflecting dull red, passed within a whisker's breadth of Ennachu's head to impale one of the derbfine---Aidi, a close cousin, some still-rational part of his mind thought. A second spear followed, another of Ennachu's kin killed, pinned to a tree, before the band reached the giant.

The giant struck out, one spear held in her hand, another in her trunk, and the gleaming bronze knife in her other hand, and Ennachu's warriors fell back, frightened or wounded or both. The giant tossed her head and Ennachu saw another cousin thrown aside by the metal-bound tusks as though they were a grass doll.

Ennachu shouted, bronze sword raised high, and the giant turned towards him, hatred and death in her eyes. She roared and raised one of her spears as Ennachu brought his blade down, aiming not for the giant but for one of the gleaming tusks. Sword met ivory with a bone jarring shock that numbed Ennachu's arm. His sword fell from nerveless fingers, clattering to the ground as the giant's spear drove towards his heart. He met her glare with his own, prepared for his end.

A rough shove knocked him sideways and, as he fell, he saw the druid Soita standing where he had been moments before. Soita flung a hand towards the giant and there was a bright flash of light. The giant's spear caught Ennachu in the shoulder instead of the chest, and he spun as he fell, twisting around. He put a hand out to brace his fall, but the cliff's edge was looming, the faraway coast swimming past wild mountain forest. His hand touched soft soil and for the length of a breath, he thought he had been saved.

Then the ground crumbled away and he fell into space. He shouted, his voice lost to the roar of the giant and the ringing clash of bronze and iron and wood and flesh. He thought he saw the druid's eyes watching him fall, thought he heard his mother shout his name as he pinwheeled through open air. Something hard, a branch perhaps, caught him just behind one of his ears, and then the entire world narrowed to a point. He felt his shield catch on the canopy of a tree, thought as he slowed that it might be enough to stop him, before the shield was roughly whipped away and he was crashing through branches.

He had no memory of hitting the ground, nor any real sense of the passage of time. He simply went from spiraling into space to laying on his back on the loamy forest floor, surrounded by broken tree limbs and foliage. He ached from head to claw, there was a sharp pain halfway down his tail and another behind his ear. When he touched the back of his head, his fingers came back wet with blood.

Conscious that there might be more injuries he couldn't see, Ennachu stood up gingerly. None of his bones appeared broken, at least, though his cloak and shield were lost somewhere to the grasping branches above him. The axe his mother had given him was mercifully still secure behind his belt, but his bronze sword was still in the giant's camp where he had dropped it. His head hurt a little, but he didn't feel sick or dizzy when moving, which he took as a good sign.

Ennachu squinted up the way he had come. Aside from the occasional broken bit of foliage, there was nothing to mark his rapid descent. He caught a flicker of light from the clifftop, but without tools he had no hope of climbing back the way he came. He strained his ears but couldn't hear any sounds of battle. Sniffing at the air didn't reveal any information, either. As far as any of his senses could tell, no living thing had passed through this part of the forest in living memory.

He stood alone beneath the trees, the ground soft and damp with a recent rain. No paths gave him a clue as to which direction to go.

``Which way, then?'' he asked himself, speaking aloud simply to break the silence. Addressing his band, he had felt confident, or at least had been able to hide his fears. Alone, in the dark, in a foreign land of giants and monsters, he felt like a boy again, not yet old enough to sit at his father's table. He pulled his iron axe free, the comforting weight of the balanced applewood shaft a welcome feeling in his paw. He dug his claws into the wood, its solidity bringing him a sense of security.

``Left or right, Enni,'' he prodded himself. He looked to the left, something vague tugging at his memory, then turned right.

Ennachu crashed through the sparse underbrush for twenty minutes before he remembered that he was deep within foreign territory and that dangers, potentially including more giants, lurked around every corner. Taking more care, he chose his path, slinking through the shadows, keeping close to trees where he could keep his footing on thick roots. He tried to keep the cliff to his left, never straying far from it, hoping to find a path leading back to the summit, the giant, and his people. The hills rolled underfoot, though, and soon he was crossing through rocky terrain and merely guessing at where the giant's camp was.

A half hour's walking and a gust of wind brought the scent of smoke and campfire to his nose. He waited for a moment, hand on a gnarled old oak, sniffing curiously at the air. He had not seen any other fires from the beach, he was sure of that, and he certainly hadn't smelled any others. The scent wasn't coming from his left, though, and he was fairly certain that was the direction the giant's camp had been.

Indecision gripped Ennachu. He wondered if he had been wandering, lost, in the wrong direction this whole time. How long had he lain on the forest floor? Even if he found the camp, what would await him there? Ennachu had a vision of his family and friends, his mother, dead at the hands of the giant they had challenged, his home conquered by the legions of Rhow.

Strangely, it was his lost sword that most affected him. It was all he had left of his father. Old-fashioned, out-of-date in this age of iron, he nevertheless felt its absence most keenly. The thought of it becoming a trophy for the giant filled him with resolve. Trusting to his nose, and not a little to luck, he followed the scent of the fire.

It was definitely the wrong way. Ennachu recognized that within minutes of setting out. The fire was coming from downhill, not up, away from the cliffs and into a softer, less wild forest. He thought about turning back, but the woody scent of fire was the only clue he had to anything, and he preferred something, even the wrong something, to endless wandering in the dark. He kept his axe gripped tight, though.

Soon he found himself haunting the edge of a clearing. A large fire, though not nearly as large as the giant's, pushed back the darkness, illuminating the camp's lone occupant. He was a hare, certainly no older than Ennachu, with a long, lanky frame and piebald black and white fur. He wore brightly colored clothes in blues and greens, not very different from Ennachu's own, except for woolen breeches instead of a more fashionable kilt, tucked into a sturdy set of leather boots. The hare was whittling a long length of bone with a thin knife. Next to him was a rucksack, and leaning against the boulder was a longbow. Even strung as it was, it was nearly as tall as Ennachu.

The hare brushed the bone idly, then lifted one end to his lips. Ennachu recognized it as a flute before the stranger had played the first faint, ethereal notes. The music was beautiful, haunting as mists and moor, and it tugged at Ennachu's loneliness, but the hare seemed unsatisfied. He tapped the instrument against his paw, then took the knife to the end, shaving a sliver of the pale white end away.

Ennachu watched, taking in more details. The hare was slim but broad-shouldered, and moved with a kind of restless, nervous energy. He shifted and readjusted his seat almost constantly, sometimes sitting up, sometimes slouching almost to the point he was laying down. Ennachu had never met anyone who seemed so incapable of sitting still. He was mesmerized.

He was clearly a native, Ennachu decided, and that meant he might know the best path back to the giant's camp and, hopefully, his kin. He steadied himself, then stepped into the firelight.

In a flash, the hare was up and reaching for his bow. Without thinking, Ennachu flung his axe, which whipped through the air and sliced neatly through the bowstring before embedding itself in the tree, the hare's paw still a breath away. It left Ennachu weaponless, but all the hare appeared to have was the thin knife.

Ennachu, grinning, straightened and approached. The hare held the now-useless bow in his hands, his gaze travelling back and forth between the snapped string and the axe.

``I'm sorry,'' Ennachu said, and the hare looked up at him. This close, Ennachu could see that the hare was a handsbreadth shorter than him, and rather frail in comparison. He didn't appear weak, in fact Ennachu found his physique rather appealing, but he didn't have the burly body of a warrior, at least in Ennachu's opinion. ``But I need your help.''

The hare said something, his speech fast and accent thick, so that Ennachu understood little beyond the accusatory tone, and something that might have been an insult.

``There's no call for that sort of language,'' Ennachu said, adopting his most royal demeanor and the cool, detached mien that his mother had taught him to affect when sitting as a prince of the tribe.

The hare pointed a finger at Ennachu's chest and told him, in barely understandable language but unmistakable meaning, that he had best clear off.

Ennachu laughed. ``Or you'll do what, fling an arrow at me?''

The hare looked down at the bow, gritting his teeth, then glared up at Ennachu. Ennachu put his hands on his hips and couldn't resist a smug smile. The hare snorted, then swung the bow as hard as he could. Ennachu attempted to duck out of the way, but he was too slow and the last thing he remembered was the loud cracking noise as the ash bow slammed into his temple, a bright, blinding light, and then darkness.

There were no dreams in the darkness that Ennachu found himself in. At first it was just a vague sense of discorporeality, but then a sharp, throbbing ache made itself known. Once Ennachu acknowledged the pain, he found that it had a place to exist, just behind his left ear. That realization forced Ennachu painfully back to reality, with all of the hurts that he had picked up in his short time in Albiyu. He blinked a few times until his eyes focused again, and found himself stretched out in front of the hare's campfire, the hare himself kneeling anxiously by Ennachu's side, a rag in one paw and a small bowl of some liquid in the other. This close, Ennachu could see the hare's eyes were sea green, burnished gold in the firelight and looking like sunrise over the ocean.

``You're beautiful,'' Ennachu said, before he realized what he was saying.

The hare blushed and shook his head. He told Ennachu that he was having trouble understanding the wolf's accent as he washed the cloth in the liquid and then began to dab at the side of Ennachu's head.

Ennachu winced and pulled away. He started to sit up, but the hare put a restraining paw on his shoulder.

``What?'' Ennachu asked.

``Don't move,'' the hare told him, then explained that he was worried that Ennachu had more serious damage than just a broad cut. It did, the hare noted, look like he had taken more than one nasty hit to the head.

``I'm fine,'' Ennachu growled. ``And I don't have time to be nursed. I need to find my people.'' He tried to push the hare's hand away.

``Stay,'' the hare said, speaking slowly and firmly to ensure that Ennachu understood him. ``Let me clean you, at least.'' He added something in an undertone that sounded a great deal like stubborn fool.

With a sigh, Ennachu stopped struggling, and let the hare clean the side of his head. ``What's your name?''

``Weithli,'' the hare responded. He tsked as he examined the cloth, sodden with what Ennachu recognized as blood. ``You?''

``Ennachu. And you shouldn't worry, head wounds always bleed like crazy.''

Weithli raised an eyebrow at Ennachu and asked him if he got injured in the head often.

``Very funny,'' said Ennachu dryly. He lapsed into silence while the hare finished cleaning his wounds. When Weithli was done, Ennachu asked, ``May I sit up now?''

Looking sour, Weithli granted him permission, then scooted back a few paces while Ennachu propped himself up on his elbows, then into a full sitting position. He felt sorely abused, and he wondered if his head would ever stop aching, but none of the symptoms of internal injury that Soita had warned him about---sickness, dizziness, confusion---manifested, which he took as a good sign.

``Why care for me?'' Ennachu asked. The hare shrugged and looked discomfited, then muttered something about acting rashly. Ennachu wasn't sure if Weithli meant himself or Ennachu. ``Well, thank you.''

``You're welcome,'' Weithli said in his peculiar accent. ``You said you needed help? Before?''

Ennachu nodded. ``I have to find my people.'' Briefly, he explained about the druid's prophecy, and needing the tusk of a giant to protect his land. By the end of his tale, Weithli was looking as stormy as ever.

``What?'' asked Ennachu.

Weithli sniffed, and asked Ennachu a question the wolf did not quite understand. He seemed to be asking who mattered more, but Ennachu couldn't figure out whom Weithli was talking about.

Ennachu shook his head in response. ``Will you help me find my people or not?''

Weithli was silent for a long time before he finally said, ``No.''

Ennachu gritted his teeth. He should have known better than to even ask the hare. With a grunt, he stood, then went to retrieve his axe, still buried in the tree.

``Where are you going?'' Weithli asked, still sitting.

``To find my mother and my kin,'' said Ennachu. ``With or without you.'' The axe was buried deep and it took a bit of effort, in silence filled only by his soft growls and the crack of the fire, to pull it free. He tucked the weapon back into his belt, then whirled on Weithli. ``My people need me. I won't abandon them.''

Weithli watched Ennachu cross the campsite, aiming for the tallest peak he could see over the trees. Before Ennachu got more than three paces into the forest, the hare called him back. Ennachu turned to see Weithli dousing the flames with dirt.

The hare didn't meet Ennachu's eyes, but spoke as if to the dying fire. He asked if Ennachu would come with him to see something. If Ennachu did, the hare promised, he would take Ennachu to where his tribe most likely was.

Ennachu glanced towards the mountains. He was, he had to admit, thoroughly lost. But he also had to find his people as quickly as possible. It was, however, an easy decision. As much as he hated the idea of a detour, he couldn't promise himself that it would be quicker to retrace his steps by himself than it would with Weithli as a guide.

``Fine,'' Ennachu said. ``Lead on.''

The hare offered Ennachu a smile which seemed, to the wolf, strangely melancholy. Picking up his useless bow, Weithli gestured for Ennachu to follow him into the darkness.

There were no paths, at least that Ennachu could see, but Weithli moved with the surefooted confidence of a native who had spent his whole life combing these lands. They were almost immediately out of the deep forest and into rocky scrub hills, their way lit only by the shining silver light of the moon. Weithli didn't march so much as bounce from spot to spot, careful not to get too far away from Ennachu, but unable to stand still. In the few moments when he did stop, it was only to point out pitfalls or hidden sinkholes, before bounding away.

In ten minutes, they were on the mountain proper again, and ten minutes after that, Weithli veered them off onto the first path Ennachu had seen since the battle with the giant. They rounded a cliff and passed into a copse, revealing the landscape all the way to the sea. Here, Weithli stopped, actually turning as still as the mountain stone behind him, to stare out over the distant waves.

``Is this what you wanted to show me?'' Ennachu asked. The sight was beautiful, certainly, but he had grown up on the hills of Mena and seen such vistas all his life. There was a kind of magic to it---a different country, connected by the same ocean, a reminder of what he had left behind---but if this was the trade for Weithli's aid, it hardly seemed worth the price.

``No,'' said Weithli, shaking his head as if clearing away an enchantment. ``It's just\ldots'' He trailed off with a sigh.

There was that inexplicable sadness again. Against his better judgment, Ennachu felt an urge to hug the hare, to comfort him. He settled for placing a paw on Weithli's shoulder.

``Are you alright?'' he asked.

Weithli startled, looking up at Ennachu in confusion. ``I'm fine,'' he said hastily, then dropped his gaze, adding something about how he could see his home.

Ennachu frowned, then turned back to the view. ``Where?'' There were no lights, no fires, to mark a settlement, no sign of movement amongst the dark hills. He supposed a roundhouse might blend into the rolling land, particularly at night, but to his eyes the whole country, from the mountains to the sea, was devoid of habitation.

Weithli, his eyes still downcast, pointed. There was nothing there, not as far as Ennachu could see.

``Do you not light fires at night?'' he asked.

``There's no one left to light them,'' said Weithli. Ennachu froze, then turned to face the hare. With a sigh, Weithli explained how his tribe had been caught between a band of invaders, what he called Sassonu, and a derbfine of giants who had lived by the ocean's edge. There had been few survivors, none from Weithli's immediate family, and their little village had been razed to the ground.

``None of the Sassonu survived, either,'' Weithli told Ennachu with bitter satisfaction. He took the wolf's hand off his shoulder but didn't release it immediately. ``And only a few of the giants, but it didn't matter much.'' He gave Ennachu's hand a squeeze and then released it. ``I forgot that you could see it from here, that's all. It surprised me.''

Ennachu didn't know what to say, even less so when Weithli raised his eyes and Ennachu could see they were filled with tears. Under the light of the moon, they were large and liquid, wild and sorrowful as the sea.

``It's been two years,'' Weithli told him. ``But I miss them all.'' He took a deep, shuddering breath, then nodded up the path. ``This way.'' Without another word, nor a glance towards the sea, he pressed on. Ennachu lingered a moment longer, staring out over the hills. With Weithli's story ringing in his mind, he thought he could see the dark shapes of broken houses, hidden in two years of verdant growth.

When Ennachu caught up with Weithli, he asked, ``You don't hate the giants? For their part in this?''

Weithli shrugged and said that his people had gotten on well with the giants, or at least that particular group of them. Closer inland, giants quarreled with the villages and kingdoms, and the Sassonu, whoever they were, hunted them like animals. Here, along this wild coastline, where the Sassonu and Rhowanu were only distant threats, and the sea provided more than enough food for all, there was little to fight over.

``It's the Sassonu I hate,'' Weithli told him, and his voice was cold with that hate. ``The giants lived peacefully until they came, thinking of sport and treasure.'' He fixed Ennachu with an icy stare.

``It's not like that,'' Ennachu said. Weithli snorted. ``No, listen.'' He grabbed the hare's wrist to pull him to a stop. ``I'm here, my people are here, to protect our home. The druid told us what we have to do. Weithli, I promise, I'm not here for a trophy.''

``How do you know?'' Weithli asked. ``Do you trust this druid?''

Ennachu hesitated for a moment. ``I\ldots'' He sighed. ``Yes, I think I do. I don't like them, but I've no reason to doubt them.''

``But do you have reason to trust them?''

Ennachu opened his mouth to say yes, but he found himself unable. Weithli gave him a self-satisfied smile, then nodded to a path half-hidden amongst wild ivy.

``This is what I want to show you,'' he said, then held a finger to his lips to caution the wolf to silence. They crept through the ivy to a small ledge overlooking a deep chasm. There, below, was the unmistakable sign of habitation: a well-worn firepit, surrounded by benches the size of a canoe, a lean-to with a pallet, big enough for all of Ennachu's derbfine, even a trough filled with half-made beer.

``Is this\ldots{} a giant's home?'' Ennachu whispered into Weithli's long ears. The hare nodded, and pointed to the far end of the chasm. There were three cairns, tumbled rocks not yet worn smooth by the wind and rain. Two were barely bigger than Ennachu and they flanked one bigger than any creature he had ever encountered. Sitting atop the biggest mound, gleaming like fire in the moonlight, was a glittering bronze sword. Ennachu had the wild thought it was his lost weapon, before scale reasserted itself. The sword was as long as he was tall.

``The last,'' Weithli whispered back. They were crowded close, the ledge not truly large enough for them to put any significant distance between each other, and Ennachu was aware of Weithli's warmth. There was no fire here, and when the hare turned to look at him, his eyes were the deep sea color of midnight from the cliffs of Mena. Ennachu thought he might get lost in them.

The wolf forced himself to look over the empty camp. Wordlessly, he pointed at the cairns.

``Her mate and their children,'' Weithli said. He was still staring at Ennachu. ``They survived the Sassonu, but then the children fell ill. She came to me for herbs and medicines, there wasn't anyone else, but I don't\ldots{} I mean, I've learned a great deal since then.'' His voice failed. Ennachu reached out and squeezed his hand. Weithli looked startled, but didn't pull his paw away.

``I'm sure you did what you could,'' Ennachu said, not unkindly. ``What about her mate?''

Weithli sighed. ``There was a storm, a terrible storm, while they were out hunting. I don't know the full story, I can't speak her language well, but she said there was an accident. Her mate, they fell and were hurt.'' Again, his voice quivered and died, but Ennachu didn't need to hear any more. Wordlessly, he pulled the hare away from the ledge and back along the path.

They paused on the main path. Ennachu stared up at the moon, wishing it could offer him some guidance.

Something tugged at his hand and he jumped before he realized that Weithli was simply pulling his paw free. He had forgotten he was still holding it. He grabbed the shaft of his axe simply to have something to do with his paw, which felt strangely empty without Weithli's own in it.

``You can't hurt her,'' the hare told him.

Ennachu sighed. ``I don't want to,'' he confessed. ``But I can't abandon my people. I can't let them face what the druid has foreseen.''

Weithli scowled and turned away from the wolf. Ennachu found that hurt as much as any of the blows he had suffered that night. Weithli asked him how he could be sure that the druid was telling the truth, and Ennachu had no answer.

``What about us?'' Weithli added, after Ennachu's silence stretched on. He asked if taking the giant's tusk would leave Albiyu unprotected, to become another territory in Rhow, another jewel on the emperor's crown. He added a lot of what Ennachu thought was unnecessarily flowery language which the wolf found hard to follow.

``I don't know,'' Ennachu said, plainly and plaintively. ``Gods help me, Weithli, I do not know. My mother trusts the druid, and that has to be enough for me.'' He took a step towards Weithli, who stiffened but did not move away. ``If I could save the world, I would. But I have to do what I can for my people.''

``Even if it means adding pain to someone already so hurt?''

``I don't have answers,'' Ennachu said. He realized his voice was shaking. ``Maybe no one can have answers to questions like this.''

Weithli sighed. ``I don't like to hurt you.''

Ennachu couldn't help but laugh. ``Says the hare who cracked my skull with a bow.''

Weithli grinned reluctantly over his shoulder. ``Your accent was so thick, I wasn't sure you weren't Sassonu.''

``\emph{My} accent?'' Ennachu said in incredulous tones, and Weithli's grin widened. He had a beautiful smile, Ennachu realized, at the same time he realized how full of pain Weithli was. It melted away from his face when the hare smiled, and it was only then that Ennachu could see the weight he carried. He offered the hare his paw again.

``If you ask it of me,'' Ennachu told him, ``I will leave the giant alone.''

The smile slid away, replaced with wonder and a hint of confusion. All the same, Weithli accepted the proffered hand. ``I want\ldots'' he started, then stopped. ``Gods above and below, Enni,'' he said more to himself than to Ennachu, ``I've known you all of two hours.''

``What does that matter?''

The hare shook his head with a laugh, or maybe it was a sob, Ennachu couldn't tell. ``I can't ask it,'' he said, his voice soft. He let Ennachu pull him close, almost to the point their bodies were touching. Ennachu found himself wanting to be closer. ``I know what it means to lose a home. I can't ask it.''

They stood, not quite touching except for their hands, and let the silence wash over them. Neither seemed to have anything more to say, but neither did they want to step apart. They existed alone together, a world of their own under the watchful eye of the moon.

Eventually, Weithli pulled away, his ears folding back. He pointed up the mountain path.

``The camp you fell from, it has to be that way,'' he said. He waited the length of a heartbeat, then started up the path, not looking to see if Ennachu was following. Confused, unsure, Ennachu trudged after Weithli, his thoughts torn between home and the sad cairns of the fallen giants.

They walked in the comfortable silence of a living forest. In the lengthening quiet, Ennachu noticed how alive the silence was, with the sound of insects and night creeping lizards, the soft whisper of the ocean winds rustling leaves and tugging at branches. Weithli was quieter than the forest, his booted feet making no sound that even Ennachu's keen ears could make out. For his own part, Ennachu moved softly enough that he was unlikely to draw attention, but in this foreign wood, he couldn't help but make a thousand tiny missteps, cracking twigs and kicking up pebbles.

They did not talk, both lost in their own thoughts, cloaked in the forest. Near the peak, Weithli led Ennachu off the path and into untouched woods. The moon reached its zenith before the landscape started to seem familiar to Ennachu, and it was noticeably moving towards the horizon when they found the giant's camp.

The bonfire had burned itself down to embers, and there was no sign of the giant. Four of the Ennachu's warriors were still present. Someone had rearranged them, moving them from where they had fallen and laying them respectfully on the edge of the cliff, their eyes closed and their hands crossed over their chests. He recognized three: Kentu and Aidi, wolves like him, were siblings a few years his senior. Aidi still had part of the giant's spear protruding from their chest. Beside them was Sentichin, a rat barely old enough to sit at her father's table. The fourth was a stag that Ennachu didn't recognize, an old and scarred warrior with a complicated tattoo along his neck. He had to be from a different tribe---Ennachu's didn't tattoo, at least not frequently. He wondered at the veteran's story.

``Were you close?'' Weithli asked. He remained at the edge of the small clearing, reluctant to get closer to the dead.

``To her,'' Ennachu said, nodding at Sentichin's still form. ``Him, I didn't know at all. Those were cousins.''

``I'm sorry.''

Ennachu shook his head. ``They came willingly. They knew what the risk was.'' A small voice asked him if that was true. He had not known their mission when they had sailed out from their home. Had they known they might never return again? He knelt down and took Sentichen's paw. It seemed so small, so frail. It was cool but not cold. The muscles were rigid, and he dared not attempt to rearrange her. Better to let her, to let them all, be in peace. ``I will see you again, my friends,'' he told them, ``In the utter west, in the Silver Fortress. Wait for me.''

They did not respond, but he hadn't expected them to. He stood suddenly, no longer wanting to be near them, and had to force himself to walk, not run, back to Weithli. He scrubbed his hand against his kilt without realizing it.

``Tracks,'' Weithli told him, pointing at the ground. The hare's eyes were evidently better, at least in the dark, but now that they were pointed out to him, Ennachu could follow the churned dirt easily. He knelt down, careful not to disturb the marks.

The giant's tracks were the easiest to follow. They were large, heavy ovals, almost circular, much deeper than the boot and paw prints of his kinspeople. Overlaying the giant's were the raiding party's. She had fled, he guessed, and several of the warriors had followed. Others must have stayed to check on the wounded and tend the dying and dead.

Ennachu skirted the edge of the clearing, especially careful around the trail the giant and her pursuers had taken. He found another set of warriors' tracks, heading back down the way they had come. They were heavy, too. Clearly they had been carrying something, or someone. Injured, he supposed. Unless his mother or the druid had been among those killed. They would have to be carried back to Mena to receive a full funeral.

Ennachu frowned. Soita had never, in his experience, worn shoes, and the skua's strangely shaped feet should have left clear sign of their passing. He saw no sign of the druid at all in the tossed earth.

``Enni?'' Weithli asked.

``Some of my people went back to the boats. They might still be there, or they might have set sail again. Others followed the giant.''

``Which way will you go?'' Ennachu had expected some hint in the hare's words to urge him to go back to his boats, to sail back to Iweru and leave this land, to let the giant live out her life in peace and then join her mate and children in the west. But the hare's tone was carefully neutral.

Ennachu took a deep breath. He found that he did not want to disappoint Weithli.

``I will follow the giant,'' he said, unable to meet the hare's eyes. ``My people are counting on me.'' His gaze fell on the four warriors. ``We have sacrificed so much already.''

Weithli nodded. ``If I asked you, would you leave the giant be?''

Ennachu hesitated. ``I\ldots{} I would.''

``Please look at me.'' Reluctantly, Ennachu lifted his eyes to meet Weithli's. They were sea-bright again. ``I won't ask. But I will ask you, before you kill her, to consider what her death would mean.''

``I will.''

Weithli nodded again. ``I think I know where she's gone. There's a faster way we can go, if you don't mind a little climbing.''

They took off into the woods once again, moving at a brisk pace. They followed the tracks for a time, before Weithli led Ennachu off the path. Soon, they found themselves skirting a ledge that arced around the side of the mountain, ending in a rough wall twice Ennachu's height. Weithli bounded up the wall as easily as a ladder, turning and perching on the edge. He touched his finger to his lips, and then tapped one of his long ears.

Ennachu, taking Weithli's meaning, pricked his ears and listened. After a moment, he heard the shouts of fighting, the sound of bronze and iron and wood clashing. Then, the unmistakable roar of the giant challenging his warriors. They were close enough that the giant's voice caused the leaves to tremble on the branches above his head.

Ennachu leapt for the wall. He was not as sure as Weithli, nor as practiced in climbing, but determination drove him on. His claws dug into the rocky face of the cliff and he pulled himself, inch by inch, up the wall. Weithli reached down and, when Ennachu was close enough, grabbed the wolf's arms and helped him the last few feet up. Ennachu had half a mind to stop and catch his breath, but another roar echoed through the forest, seeming much closer than it had when he was boxed in by the mountain's wall.

``This way,'' Weithli whispered, before pressing his finger to his lips again. He turned and slipped into the undergrowth, nearly vanishing before Ennachu's eyes.

There was no path here, and wild mountain heather and bilberry clogged the space between the holly and elm that grew close together. Weithli moved with practiced grace and the experience of his young life spent wandering these mountains. Ennachu crashed through brush, one hand over his axe to prevent it from being tugged out of his belt, the other held in front of his muzzle to protect his face and eyes from scratching, clawing branches. He would have blundered straight into the middle of battle had Weithli not grabbed him and pulled him to the side.

He lowered his paw to see another clearing, thinner and covered in the dead leaves of many past seasons. Several of his warriors, led by his mother, had trapped the giant. Queen Katodirean held Ennachu's bronze sword in her left hand, her right hanging uselessly by her side, the fur matted flat with blood. Two of the derbfine were huddled at the far end of the clearing, both clearly too injured to fight. The remainder, eight in total, were spread out in a ring around the giant, preventing her from running. There didn't seem to be a one who did not have some injury. The giant herself was bleeding from a hundred tiny cuts. None seemed life threatening on their own, but her thick, wooly fur was matted and tangled with her own blood. She held a spear in two hands, her wicked knife in her trunk.

The giant roared a challenge, then rushed towards Katodirean. The queen, her teeth gritted, raised Ennachu's sword. He could see the sluggishness of her movement, a tiredness born of exertion and injury both. She parried the spearpoint awkwardly, the bronze blade visibly deforming under the giant's strike.

``Mother!'' Ennachu screamed. He pulled his axe free and ran full to her as the giant swung the spear down. Katodirean caught it again, but the sword slipped free, swinging end over end and landing somewhere in the dark woods. The giant raised the knife in triumph, then brought it down, aiming for Katodirean's chest.

Ennachu collided with his mother, sending her sprawling to the ground. The giant's knife took him in the shoulder, biting deep. For the second time that night, he lost feeling in his arm. He would have dropped his axe, if not for the deep gouges in its haft which caught on his claws. Switching the weapon to his left hand, he faced the giant. She snorted and glowered down at him, brandishing spear and knife and the cruel ivory tusks.

``I don't want to hurt you,'' he told her. She snarled something at him, then stabbed at his stomach with her spear. He batted it away with the axe, then ducked another swing of her knife. His axe lashed out, forcing her back a step, then he had to dance out of the way of her swinging tusks.

Out of the corner of his eye, Ennachu saw Weithli rushing to Queen Katodirean. The hare helped her into a sitting position, then began to examine her wound. They were speaking to each other, but Ennachu couldn't risk turning his attention from the giant to focus on them. She swung spear and knife, her hot, strangely sweet breath washing over him as she growled out a challenge or a threat. Maybe it was both, or neither. Ennachu darted forward, axe lashing out, coming within a hairsbreadth of her throat.

The image, unbidden, of those lonely cairns flashed into his mind's eye. He froze for just an instant, but it was an instant too long. She caught him with her tusks, driving the breath from his lungs and sending him sprawling onto his back. He rolled away just in time to avoid being stomped on. She stabbed down with her spear, and he hacked it away with the axe, sending the head skittering away into the darkness. She jabbed down into his ribs, hard enough that he heard them crack, but his thick shirt stopped her from skewering him. He swung up at her, gaining a precious handsbreadth of breathing room as she dodged out of his way. Her knife bit into his thigh as he scrambled to his feet, and he nearly collapsed back to the ground.

Instead, Ennachu swung the axe downward, leaving a long, dark wound in the giant's trunk. Surprised, she jerked backward, leaving the knife buried in his leg.

``Enni!'' Weithli shouted. Ennachu waved him away, his gaze locked with the giant's. She snorted, pawing at the ground like a bull. He knew what she intended even before she lowered her head and rushed him, the ground rumbling under her feet, tusks swinging scythe-like to mow him down. His injured leg barely held him upright, there was no chance he could dodge out of the way.

Instead, praying to each god individually, he held his ground and waited. He swung at the last minute, aiming for the tusk. His axe hit almost exactly the same place his sword had an eternity before. Between the force of his swing and her momentum, the blade cut deep into the ivory. With a crack, the tusk broke along the damaged point, shearing off and skittering across the ground. The giant roared in anger, and her other tusk caught him in the stomach, sending him flying again. He hit a tree with a bone-jarring finality. His strength fled completely, and he crumbled to the ground. He half wished he had struck his head again, if only because he wouldn't then be conscious to watch the giant charging him.

And then Weithli was between Ennachu and the giant, cracked bow held in both hands. Ennachu had the surreal experience of watching the hare swinging the bow, just as he had when he had cracked the wolf's skull, into the giant's face. She was stronger and denser than he, though, and Weithli didn't knock her unconscious. She recoiled, though, shouting in pain and, unless Ennachu's imagination was running away with him, confusion.

He struggled to his feet, letting his axe drop to the ground, and limped to Weithli's side. He yanked the knife from his thigh and tossed it to the giant, who snatched it out of the air, looking suspicious.

``Can you translate?'' he asked Weithli. His leg buckled and he would have fallen if Weithli hadn't caught him. Leaning hard on the hare, he looked up at the giant. ``We were wrong.''

Weithli stared at Ennachu for a moment, but translated dutifully into the giant's language. The giant hesitated, and Ennachu waved his warriors to lay down their weapons and step back.

She spoke, her voice, when she wasn't roaring battle cries and challenges, deep as mountain roots but gentle as a stream. She still held her knife, but dropped the broken spear and took a step back. Progress, Ennachu supposed.

``She asks what you mean,'' Weithli said.

``We came here and attacked you, without provocation. We had our reasons, but that did not give us the right to treat you as we did. And for that, I apologize on behalf of my people.'' Weithli translated, and the giant responded.

``She wants to know if you think that makes what you did acceptable.''

Ennachu shook his head. ``No. But if it will assuage you, in exchange for the life of my people, I offer you mine.'' He heard his mother, now being supported by two of their warriors, hiss. The rest of the band stiffened, then turned to watch the giant.

Weithli hesitated. ``Tell her,'' Ennachu said. Weithli searched his eyes for a moment, then turned back to the giant to translate.

She was quiet for a long time, weighing his words. Finally, she said something curt, then turned and strode off into the darkness, away from Ennachu and his people.

The assembled band let out a collective breath. One stooped to grab the shorn tusk, as tall as Ennachu was.

The world was starting to spin, and Ennachu worried for a moment that he had, in fact, hit his head, but it was just Weithli lowering him to the ground to tend to his wounds.

``These are deep cuts,'' he said, as Ennachu's mother was helped to the hare's side. ``But I don't think they're life threatening, so long as they don't catch an infection.''

``Can you tend him?'' Queen Katodirean asked. Weithli nodded. ``Then do so, please.'' While Weithli worked, she turned on Ennachu. He was surprised to see she was smiling. ``That was foolish, perhaps, but it was very brave, Enni.''

``Yes, well,'' said Ennachu, feeling embarrassed. He was uncomfortably aware of the band watching the hare. ``I had good advice.''

``I see. And how did you happen to meet your friend?''

While Weithli worked, Ennachu told Katodirean what had happened since he had fallen from the camp, leaving out only the part where Weithli had cracked the bow over his head. He caught the hare's smile as Weithli tended his wounds.

``It seems we owe you a debt, Weithli,'' the Queen said as the hare helped Ennachu up. ``You have saved the prince's life at least twice over tonight, I think, and mine as well.'' She touched her bandaged arm. ``You have the gratitude of our tribe.''

``It was my pleasure,'' Weithli said, bowing as best as he could with Ennachu leaning on him. The Queen graced him with a faint smile, then waved for the band to return the way they came. As he helped Ennachu follow, he mouthed, ``Prince?'' at the wolf.

Ennachu shrugged. ``What did you think the torc was for?''

The journey back to the beach took much longer than the trek up the mountain had, owing to the slow pace of the injured and the need for the weak and weary to take frequent rests. By the time they reached the shore, the first grey-pink light of dawn was creeping up over the mountain.

The rest of the band, except for the druid and the four whose bodies still lay on the mountaintop, were waiting for them by the boats, most dozing or resting around low, smokeless fires. The sentry gave a great shout when she recognized them, and a second when someone raised the tusk up high enough for her to see. The rest jumped to their feet and rushed forward to greet them. Ennachu was deluged with questions about what had happened, where he had been, and how he had managed to defeat the giant. Queen Katodirean waved them all back.

``There will be time enough for stories on the trip back to Mena,'' she said in her most imperious tone. ``Where has Soita gotten themself to?''

But everyone seemed to have lost track of the druid in the chaos of the first battle. Her muzzle set in a grim expression at her missing advisor, the queen began ordering the boats readied to leave while the injured were given a moment to rest.

Ennachu found himself seated on a log next to Weithli, his head on the hare's shoulder while they watched the least hurt prepare sails and push the boats off of the beach.

``Thank you,'' Weithli told him, his voice pitched low.

Ennachu smiled. ``It was the right thing to do.''

``Will I see you again?''

With an effort, Ennachu sat up to look the hare in the eyes. ``Come with me.''

``Leave my home?'' Weithli asked, his eyes wide. But Ennachu saw the faint twitch of a smile play on the hare's muzzle.

``You could build a new home. If you want.''

``With you?''

Ennachu blushed. ``Would you if I asked you to?''

It was Weithli's turn to blush. ``I\ldots{} Let's not get too ahead of ourselves.'' He squeezed Ennachu's paw. ``But you've seen my home, it seems fair, perhaps, for me to see yours?''

Ennachu squeezed back as he leaned against Weithli. They sat together in silence and watched the sun rise fully above the mountains. And, when the boats were ready, Weithli climbed aboard with Ennachu.

