\begin{quote}
  \emph{Author's note: Between England and France, there once was a lush valley of lakes and rivers, now lost to the rising seas. We will never know many lives were lost or changed forever when that land vanished beneath the waves; only that they were there. Now and then, along Dogger Bank in the North Sea, some diver or fishing boat brings up bits of bone, wood, antler, and stone with the unmistakable marks of having been shaped by careful hands.}
\end{quote}

\noindent I still remember the morning the first wild cherries ripened, all those many years ago.

Half Moon, the seer who lived at the edge of our village, had abandoned her hut and was carrying as much as she could gather into a bag of horse skin. She was following the trees marked with three horizontal bands of white earth pigment, the way westward toward the Chalk Cliffs in the Land of Many Deer.

The old she-wolf's eyes were deep and uncanny, always ablaze like the fire pits we cooked our fish in. But on that day they blazed with a terror that chilled me nose to tail.

Her steps were quick for someone so old. I tried to keep pace with her a while and asked, ``Half Moon, where are you going?''

``Away, Starling,'' she replied, stepping quicker still, nearly breaking into a run as I followed her.

``Away? Where?'' I asked.

``The Land of Many Deer! Didn't I tell you?''

She had, in fact, told us. Four moons before, we had killed a deer and saved Half Moon the bones for her broth. But before she boiled them, she cast the bones on the ground and read their signs by the light of our fire.

The bones told her the Great River would flood and sweep our home away forever, some time soon after the first wild cherries had ripened.

None of us had taken her seriously. The little village of horse skin huts our pack lived in had been there since Half Moon's grandmother's grandmother's grandmother's time, and our village was on a rise a fir tree's height higher than the river, high above the highest floodwaters anyone had ever seen.

``Supposing you are right,'' I whined, my ears back. ``What good are you doing us by running away?''

Half Moon stopped in her tracks. ``Child, you must understand, the Land of Lakes and Rivers will be swept away. Our home will be lost beneath the sea for all time and if the pack stays there, they'll be gone too. If the pack needs me, come and find me. I'm sure some of you will get your wits about you and when you do, I'll be waiting to help you. I'm of no use to you dead!''

I clung to her waist. ``Half Moon! You can't go! We need you!'' I begged.

She reached with spindly arms of remarkable strength and pried me off of her. ``Then come to the Ancestor Oaks in the Land of Many Deer. That's where I'll be, waiting by the Old One. Go tell your stupid father!''

She flung me away none too gently and carried on, almost running, until she was far out of sight beyond the aspen wood.

I ran back to my village to tell my father, Raven. He was with my younger brother, Bright, knapping flint outside our hut.

``What's the matter?'' my father asked, setting his work aside a moment.

``It's Half Moon. She's left in an awful hurry. She says a great flood is coming. The sea will swallow up the Land of Lakes and Rivers,'' I explained.

Bright sighed and rolled his eyes.

``When has she ever been wrong?'' I challenged him, baring my fangs a little.

``Don't you bare your fangs at your brother like that!''

I jumped. My mother, Gentle, spun me around and handed me a wad of grass. ``I need you to help me with these fishing nets,'' she said. ``Come on, stop bothering your father and brother!''

``She's not bothering us,'' Father chuckled.

``Well she's got work to do too. Everyone does their part for the big hunt! Come on,'' she chided, grabbing me by the wrist and dragging me back to the hut.

We sat cross-legged, on deer skins across from each other, the hearth in the middle of the hut between us burning in warm glowing embers. ``Half Moon is leaving,'' I said as I took a handful of grass from a corner of the room and began splitting the blades into ever finer strands with my claws.

``Why?'' Mother asked, picking up a half-made cord and braiding the fine strands one by one.

``She said the Land of Lakes and Rivers will be swallowed up by the sea soon,'' I told her. ``Remember? She said the same thing four moons ago.''

My mother stopped braiding. Her eyes went wide. ``Starling\ldots{}last night I dreamed about this! I'd forgotten all about what Half Moon saw in the bones, but last night I saw the river rise and rise until there was no more land, and I saw our village swept away. I saw wolf folk and fox folk and badger folk all running for higher ground. But the waters kept coming and coming\ldots{}it was such a horrible dream! It felt so real!''

My chest tightened. ``We have to leave,'' I whispered.

Mother's ears splayed, and she gazed down with a long sigh. ``We can't. Your father and brother are leaving for the Big Hunt tomorrow.''

Every year, near the start of autumn, our pack and the neighboring packs would all go to the east side of the Great River for a Great Hunt, catching fish and hunting horse and deer and gathering together for a feast. Meat-eating folk from all the clans, from the tiniest stoat to the biggest wolf, would all come and share what we'd caught and gathered. And we would give a great gift of nuts and fruits for the deerfolk, and the promise that we would never hunt a deer that walks upon two legs like the wolf folk did long ago, in the wandering times. There'd be gifts and songs and stories, with the hunters acting out the hunt in deer skins and antlered headdresses. And at the end of the feast we would all honor the land together by dancing until we could dance no more.

To interrupt the Great Hunt? No. Never. It meant too much to the clans of this valley.

Mother didn't say another word after that. She put all her attention into braiding, her fingers moving fast, nervous, fumbling at times but never stopping, never lifting her eyes to meet mine. I forced myself to do my part, splitting grasses until the day was done.

We had a supper that night of rabbit flavored with summer raspberries we'd dried and stored in pots some moons before.

``I'm proud of you, going on your first Great Hunt!'' Mother said.

Bright grinned. ``Father says he's happy to have me along!''

``Well, I'm getting old, son!'' he said. ``I need someone to help me.''

``You're not that old,'' my mother said. ``And you've got my brother to help you. And your friends.''

``Starling, you're being awful quiet. What's the matter?'' Father said.

What could I say? If they wouldn't listen to Half Moon, why would they listen to a silly young pup like me?

``It's nothing,'' I said. ``I'm tired, that's all.''

I devoured the rest of my food and excused myself from the fire. I wandered down to the river, sitting on the wooden planks my father and uncle, Singer, had laid by the riverside to slide their boats in and out. They'd worked hard on this, knapping for days to make great axes from flint and cutting each tree down to size with the help of the whole pack. We'd been here so long and done so much that the land itself had our mark on it, signs of who we were and what we'd done. Would it really all be gone forever, lost beneath the sea?

I cried, my eyes blurring like the moon's reflection on the rippling waters. No. No, it couldn't be. It couldn't be over, my world, everything I knew! I didn't want to leave!

But I knew where to find Half Moon.

I would watch the river, wait for it to rise\ldots{}and if the water rose one day, I would be ready.

\secdiv

\noindent On the day of the Great Hunt, we gathered at the water's edge. Mother held pots of yellow and red ochre dampened with river water and, with a brush made from a boar's bristles, she closed her eyes and waited for the shapes to show themselves. After a moment she opened her eyes and dipped the brush into the ochre and painted those shapes on my father. ``A successful hunt and no harm to you,'' she sang over and over in that ancient plaintive melody I'd heard fifteen autumns in a row.

I smiled, thinking of the day when I would have a husband to bless before the hunt. I was smitten with Kingfisher, the gold-furred hunter one year and a summer older than me; and Kingfisher shared those feelings. Our parents had discussed marriage but never on any serious terms; we were still a year or two too young.

Like me, Kingfisher was a Sky-Child; that is, his heart and soul were different from his body. And two Sky-Children could have children, the same as two Earth-Children like my mother and father.

He held his head high as his mother painted circles of yellow on his chest and arms, proud to be a hunter, but when he saw me his dignity and gravity melted into a silly grin. He ran up to me as his mother finished painting her visions on him, taking my hands in his.

``Kingfisher\ldots{}listen\ldots{}I\ldots{}uh\ldots{}Look, we're in danger. Half Moon said the river was going to rise and\ldots''

Kingfisher tapped a finger on my nose to silence me. ``Oh shush now\ldots{}that old bone boiler! It's been how many days since the wild cherries ripened? We'll be fine, Starling.''

He planted a kiss on my muzzle and my tail betrayed me, wagging all of its own.

Kingfisher smirked. ``That's more like it. I'll see you after the hunt, gorgeous.''

And for one precious moment I forgot all about the danger bearing down on me. I was young and in love, and that's a powerful herb.

The leader of the hunt, Bright Horns of the Moon, blew a blast on an aurochs horn. Some of the hunters, like Kingfisher, headed into the hills above the village, away from the river; while some of the hunters, like my father and brother, walked to their boats, sliding them into the water as we cheered and beat drums. We watched them all the way to the edge of the wood at the far side of the river, where whole groups of them became tiny dots, and one by one we stepped back from the river.

``Let's go forage,'' my mother said. ``I want to see if we can add something to the feast.'' Of course. It was my brother's first hunt and if he couldn't provide enough to the feast on his own our family could keep our dignity by foraging the best late fruits, nuts, and mushrooms.

We went back to our hut and fetched our baskets, walking to the higher ground beyond the river valley. A wood, rich with many kinds of trees and bushes, stood at the top of those hills. It took a little while to get there. They looked so close from our bluff by the river but by the time we arrived the sun had already passed the highest point of His journey and He was headed down the hill of the sky into the seas beyond sight.

``Will we have enough daylight left?'' I asked as I looked back on our village, now small as my basket.

``Our trip is already worth the bother,'' Mother said with a smile, shaking a low-lying branch from a hazel tree and watching the nuts pile inside her basket.

A moment later I spotted a large circle of delicious morels and I picked all of the biggest and best. That alone filled half my basket. I even helped myself to one or two.

We were filling our baskets with wild cherries when we felt the earth tremble and heard an unearthly sound, a mighty roar.

``What's that?'' I asked, setting down my basket and scrambling up a beech tree for a better look.

Along the Great River there surged a wall of water, spreading out far beyond its banks.

``What is it?'' Mother called from the ground below.

The great surge moved so fast. It carried trees and bushes and the destroyed remains of huts from further up the valley as it churned along, horrible and brown. It tore the carefully-laid timbers of our slipway and tossed them aside. It ripped the huts we had built, repaired, and tended for so many generations from the earth as easily as a wind tears the seeds from the black poplar tree. And half our pack, though too small to see, went with the village. I couldn't hear their screams. I couldn't see them swallowed up. But they were there. No one who was in the way of that brown wave of death could possibly escape.

I screamed. The loudest I had ever screamed in my life. The only time I have ever screamed from sheer mortal terror.

I clambered down and threw my arms around my mother. ``Half Moon was right!'' I sobbed, shaking with dread. ``The village! The village is gone!''

Mother grabbed my hand, dragging me away from the river. ``Where did you say Half Moon was going?'' she asked.

``To the Ancestor Oaks,'' I said.

Mother stopped in her tracks. ``That's a very long way,'' she said. ``And it's getting dark.''

I pushed forward, brushing past her. ``Then we'll walk in the dark! I know how to find my way by the stars! Let's go!''

\secdiv

\noindent The sun was nearly gone for the day when we heard a voice behind us. ``Wait! Wait for us!''

It was Kingfisher and his mother, Swan. Their fur was caked in mud and they were panting hard. In the dark, a few yards behind them, I saw the glow from the eyes of about half our pack.

``Are these all the survivors?'' My mother asked, scanning for my father and brother.

``I don't know,'' Swan sobbed.

``We ran, but the river kept rising,'' said Turtle, a brown wolf who would have been on his third hunt. He was Great Horns of the Moon's nephew, spindly and lean and a bit awkward, but not a terrible hunter.

``Raven? Bright? Are you here!?'' my mother called, frantic.

``The last time I saw them they were in their boat,'' Kingfisher said. ``But that was a while ago.''

``My uncle didn't make it,'' Turtle whimpered. ``None of my family did. I'm alone.''

A wolf so muddy I couldn't recognize him stepped forward. ``I'm so sorry, Gentle! I\ldots{}I lost sight of them. I have no idea where Raven and Bright are.''

``Singer! I'm so glad you're safe!'' my mother sobbed, throwing her arms around him, not minding the thick, sticky mud all over him.

``Is it safe to stay here?'' I asked.

``Go and look,'' said mother.

I ran to the top of a nearby hill, the highest one with trees on it, and climbed up the tallest tree I could get my claws into.

In the blood-red glow I could see the river that once ran through the middle of our land was now a vast sea, with only a handful of islands. But it didn't seem to be rising. At least, not very fast.

I watched it a moment, my eyes on a stand of trees cast long in shadow and hard to observe. I squinted, trying to make out the movements of the water.

Yes. The water was still rising. It was rising knee-high every minute or so, not nearly as fast as it had been but still alarmingly fast.

I scrambled down the tree and down the hill and ran back to what was left of my pack.

``The water's still rising,'' I said. ``We have to keep walking till we reach the Chalk Cliffs.''

There were groans of protest and howls of despair. My heart twisted. What could I possibly do for them all? But we weren't far from the cliffs. And from there we would be another two days from the Ancestor Oaks where our seer would be waiting for us. And so, with night closing over our heads, we forged on, guided by the stars.

\secdiv

\noindent By morning we had reached the Chalk Cliffs, but we were too tired to go on. We made camp there, with a clear view of our valley, or what was left of it, and slept together, piled on top of each other to keep warm.

There were other folk escaped from the valley there too. Some fox folk had a camp not far away, and a band of otter folk had a camp further west. They huddled much the same way we did, miserable and hungry and fewer in number. In the glistening morning sun I could see, here and there, bodies floating in the water. But I couldn't tell what sort of bodies they were, let alone whether they were any of our pack. But I begged Moon and Sun that Father and Bright weren't among them.

We woke near sunset a few hours later, still tired, and gathered round a fire to share what little we had with us. Mother and I had some of the berries and nuts we had gathered and the foraging wasn't bad near the white cliffs, but we were too tired to gather too much.

``I think we should wait here, in case anyone else from our pack shows up,'' Mother said.

``I'm out of ideas,'' said a voice so hoarse and strained I couldn't recognize them in the gloom of the late evening.

``Fine, fine! As long as we find some food while we're here!'' Kingfisher complained.

``Enough whining! I know what we had wasn't much, but we shared it all and it's enough to keep us alive another day. That's all that matters right now,'' his mother scolded him.

That was when I became aware of a sound I had never heard by the Chalk Cliffs: the sound of waves. I shuddered. I had heard waves only once before, when our family went to the sea in our boat and fished by the shore. To hear that rhythm, to hear the slow rumble of the sea instead of the gentle song of life in the valley, chilled me to the core. I burst into tears, shaking and sobbing.

``Starling! Are you alright?'' Kingfisher asked, resting a hand on my shoulder. Not thinking, I threw my arms around him and buried my face in his fur, crying for all we'd lost. He caressed me and I stayed there a while, in his arms, comforted like a newborn pup.

We stayed awake until the fire guttered, then one by one we nodded off, sleeping away the last little while before dawn. My sleep was a dreamless void, full of only feelings. Terrible feelings. Sorrow, longing, and the dread of being keenly aware that one day, I would die.

\secdiv

\noindent Nobody from our pack came, from dawn til the time the sun was a little less than a quarter of the way through the sky, and with mutters of agreement we shambled north to the Ancestor Grove to try to find Half Moon.

The way to the Ancestor Grove, marked by trees and rocks painted with three red circles, had once followed cliffs and ridges above our land but now, there was only a series of high cliffs and desolate shores and mud\ldots{}so much mud! The rising waters had made the ground so sodden in places we were afraid we'd be mired there when the tides came in, but we carried on, helping each other cross.

The sun was three-quarters through His course when we heard panting and footsteps. ``Hey! Hey! Wait! Wait!'' a voice called.

We turned to see a wolf, about my brother's age, running toward us. His gray-brown fur was matted with mud and he seemed to be favoring his right leg a bit. The pain must have been too much for him, for he broke stride and hobbled the rest of the way toward us.

``Hello, friend,'' my mother said. ``Do we know each other?''

The lone wolf paused and folded his ears back. ``No, I don't think so\ldots''

``What's your name?'' Mother asked him.

``Dragonfly,'' the exhausted wolf said. ``I lost my pack. I\ldots{}I don't know if anyone's alive. I'm hungry and scared. Can I come with you? Please?''

Mother looked over the wolves in our diminished pack. ``Should we let the newcomer follow us?'' she asked.

There were nods and murmurs of approval.

``We're going to the grove where our ancestors live to set up a new camp,'' my mother said.

``You're going to live in your pack's sacred grove?'' The young wolf tilted his head at the thought of it.

``Our ancestors spoke to our seer in a dream and said we must go there,'' my mother explained. ``You can come with us and stay as long as you need to. But once we get there, we will expect you to work hard with us as soon as your leg is healed.''

``I'm an excellent fisher. I'm strong and I'm quick with my hands and\ldots{}it doesn't hurt too bad to stand I guess,'' the wolf said, flashing a friendly smile, his tail betraying a slow, submissive wag.

``Do you need any help walking?'' I volunteered, seeing how much pain the wolf was trying to hide from that leg. Sprained, most likely, or he wouldn't have been able to run on it for any distance.

``Thank you kindly,'' he said. He was a bit taller than me and his arm rested comfortably on my shoulder. ``I'm glad I found you all.''

``It's bad times,'' I said. ``We have to take care of each other.''

\secdiv

\noindent We made camp and foraged by the Ancestors' River. The foraging there was better, and Dragonfly helped us catch a few fish for our meal. We stuffed the fish with some herbs and wrapped them in long grasses from the riverside, placing them by the fire to cook.

It was the first proper meal most of us had eaten in two days, and it did our spirits good to have something to eat again.

Refreshed and our bellies full, we sat warming ourselves, huddled around our fires.

``So, where do you come from?'' Mother asked Dragonfly.

``My village wasn't far from here,'' Dragonfly said, helping himself to another handful of salmon and gobbling it up. ``When the waters came I ran up to the top of a nearby down. That down is an island now. I was the only survivor. My pack is gone.''

``How did you get here?'' Singer asked.

``After a while on that island I decided to take my chances and head west. I found a good sturdy bough drifting by and I grabbed onto it, and I kicked and I paddled it all the way to the shore.''

My heart leapt at his story. ``You mean there are still islands above water out there?''

Dragonfly nodded. ``Quite a few. Some of them had creatures on them. Some of them were clinging to the tops of trees just above the surface of the water. They begged me for help but\ldots{}I couldn't. I couldn't save them.''

Dragonfly burst into tears and almost out of instinct, I put my arms around him to comfort him. ``I don't know where my father is either. Or my brother. I'll probably never see them again. But we'll be your pack. We'll look after you.''

``Thank you,'' Dragonfly sobbed.

I lay on my back in the grass and stared up into the night sky as the fire's last embers glowed dark orange. The night sky is a terrible and mysterious thing. There are so many strange objects out there, fires that burn in the far-away. Gods perhaps. Or put there by gods. Every pack, it seems, has their own story told from generation to generation.

I felt insignificant in that moment, and my heart lurched as if caught in a snare. Did those gods care if I lived or died? Or how? Did those bright points of light even know my name? Or were the gods all dead and gone, and the bright stars above me simply the dust and debris that showed they had once been there, long ago?

I gazed into the black beyond and slept another dreamless night, beneath the indifferent stars.

\secdiv

\noindent We walked along the river, westward into the hills from the first red glow of dawn over the horizon until the sun was high in the sky when we came to the Ancestor Oaks. We knew them by the paint of chalk and ochre pigment of red, yellow, and white in bright patterns all over the trees and rocks at the edge of the grove, and the image of the Old One carved into a tall pole that had once been a tree in this grove many years ago that marked the boundary of our space.

And in a clearing at the center, the tallest of the ancestors stood alone. She was called The Old One, the Daughter of Thunder himself and the mother of the Firstborn of our pack.

Those who died easy and with no worries became tall oak trees in this place; but those who died fearful or in despair became the wild sort of wolves who go on four legs and cannot speak, but they can sing. When a wild wolf died, they too could become an oak after one hundred lifetimes; but when an oak died, they were reborn within the pack.

Half Moon was the Firstborn reborn, after the tree she had once been fell sixty winters ago; and when the Old One fell it was said that she too would be reborn among us. But although we came to the edge of the grove every year to honor the ancestors, no one had seen the Old One in seventeen years, and being fifteen summers old I had never seen the Old One with my own eyes.

We stood outside this holiest of places, milling about, all of us trying to think of what to do next.

``Where is Half Moon?'' Kingfisher asked.

``She must be camped further in,'' my mother said. ``But we mustn't go in without her.''

I sang out a call for her, a long howl to say ``I am here.'' But my howl was lost in the dense trees where no sounds echo. Would she even hear me?

A faint howl in an elderly voice replied a moment later. A short howl to say ``I'm coming.''

We waited, listening for the sound of her footsteps. They came, advancing slowly. Tired footsteps. Nothing like the footsteps I'd heard falling away from me a few days ago.

Half Moon came into view, haggard and weary. ``Child, is that you?'' she called.

I ran to her and threw my arms around her. ``Half Moon! You're alive!'' Tears ran down my face.

Half Moon caressed my head like I was a young pup. ``Yes, yes\ldots{}I'll be with you for\ldots{}a few more years, I think,'' she murmured. ``I'm so happy to see you, Starling! I thought I had lost all of you!''

``We lost about half the pack,'' Mother said.

``I lost all of my pack,'' said Dragonfly.

``Oh, Half Moon, this is Dragonfly,'' I said. ``He's joined our pack.''

``The spirits told me about you, Dragonfly,'' Half Moon said. ``They told me a newcomer would cross our path with important news about the flood. Do you know anything, friend?''

``You mean about the islands that are still left?'' I asked.

Half Moon's eyes grew wide. ``There are still islands out there!?''

``Many,'' Dragonfly said. ``And there are creatures trapped on some of them.''

``Child, follow me,'' said Half Moon, leading us deep into the grove. ``Don't be shy! This is important, come on!'' she ordered, once more walking like a creature half her age.

We came to the sacred clearing where the Old One stood\ldots{}or had stood. Her trunk had broken at the base and an ugly burnt scar ran down her side.

``Lightning must have struck her,'' Half Moon murmured, pointing to the blackened scar. ``Probably happened years ago.''

``Oh no,'' my mother sobbed.

``Don't weep for her. The Old One told me in a dream that she has been reborn. She said, `I will make a boat of this old husk and sail to the island where the son of my pack waits for rescue.' And I did not know the meaning of this. I couldn't see a single island from the shore. Starling, my child, you're sixteen summers old, yes?''

``I am.''

Half Moon took my hand. ``And the last time our pack came to this grove was seventeen springs ago! My child, I believe you are the Old One reborn! And you will be the one to rescue Raven and Bright.''

My heart stopped a moment. ``No...''

The Old One? I couldn't be her. I didn't feel old. I didn't look old. I didn't know what anyone would expect of the Old One. I couldn't be her. I didn't want to be her. I wasn't ready!

``I believe you are,'' Half Moon said. ``I may be wrong but\ldots{}I've always been right before, haven't I?''

There was no sense arguing with Half Moon; the flood had proved her point for her.

``Then what should I do?'' I asked.

Half Moon examined the great bough of the fallen oak. ``This wood is pretty far gone on the ends, but this bit here in the middle, you see, it's exactly the right size to make a stout boat that will ride the waves, isn't it? So build a boat out of The Old One and find your father and brother!''

``All by myself?'' I asked, ears back and tail between my legs.

Half Moon shook her head. ``You can have some help building the boat. But when you set out to sea, you must be all alone.''

``But I don't know how\ldots''

``You do\ldots{}and you will,'' Half Moon said. ``Our lost sons need you. Start building!''

\secdiv

\noindent For three days we gathered supplies, made tools, and cut and gouged at the wood, shaping it slowly into a boat fit for the open seas, as long as five tall hunters and as wide as my arms spread side to side. Half Moon prayed and burned sweet herbs in a small fire nearby, wafting the smoke with the paddle I would use to propel the boat as we turned my old body into the largest boat that had ever been made.

How could I possibly travel in such a large boat all alone? We kept the hull as light as we could but this boat would be enormous and heavy. I could see the doubt in Kingfisher's eyes too as he eyed the size of the thing.

Uncle Singer did his best to make the boat as light and agile as possible. ``If we make the front point just so,'' he explained, ``the water will part before you and you will glide forward.''

``You're good at this,'' Dragonfly said with a smile.

Singer's ears splayed. ``So was Raven\ldots''

``IS!'' I growled. ``My father IS alive!''

Singer gazed at me, then at Half Moon. The old seer shook her head in disapproval at him.

``If Half Moon says he's alive, then I believe her,'' Singer relented.

I gave him a disdainful snort. ``She was right about the flood, wasn't she? Really, do none of you believe her, even now?''

``If she's wrong we'll lose you,'' Kingfisher said.

``And if you're wrong, we'll lose Raven and Bright,'' Half Moon shot back.

``No more arguing!'' Uncle Singer growled. He struck up a working song:
\vspace{-0.5em}

\begin{verse}
The sky makes the earth, \\
The earth makes the stone, \\
The stone makes the blade, \\
The blade cuts the wood, \\
The wood makes the boat, \\
The boat travels far, \\
Travels far, \\
Travels far!
\end{verse}

\vspace{-0.5em}
We joined in the singing, our disagreements set aside for a moment, the pack working as one to save our lost sons.

\secdiv

\noindent At the end of the third day, we took a supper of rabbit stew cooked in skins.

I sat with Kingfisher, off a little way from the rest of the pack.

``I just want you to be safe,'' Kingfisher whimpered. ``I may have lost my father out there. We lost so many friends and loved ones. Please, I don't want to lose you too!''

I held his hand. ``I promise I will come back to you. I\ldots{}I trust Half Moon.''

``You don't sound so certain,'' Kingfisher pressed on. ``Are you sure you don't want me to come along, just in case?''

I sighed. ``Kingfisher, I love you. But I have to trust her.''

Kingfisher's eyes went wide. ``You\ldots{}you love me? You really mean it?''

I grinned, rising to my feet. ``What do you think, silly?''

That was all. I turned in for the night, letting him sleep on that.

\secdiv

\noindent We were up again at the first light of dawn to finish the boat and by the time the sun was halfway down, Half Moon had carved a protective sign into the bow of the boat.

The whole pack admired our work. ``The finest boat ever built!'' said Bull Roarer, one of the few hunters to make it to the grove with us.

``You've honored your ancestors,'' Half Moon declared. ``This will be big enough to carry you and your father and your brother, and plenty of supplies!''

``I guess this means I'm leaving soon,'' I said, running my hand along the sides of the boat.

``You must leave at dawn tomorrow,'' Half Moon said. ``We can't wait any longer. You have to save our sons!''

\secdiv

\noindent The next day at dawn, eight of us, myself and Kingfisher in front, carried the heavy wooden boat to the river. As we carried we sang the song our pack had sung each time they would leave the Old One at the heart of the grove:

\begin{verse}
Goodbye, Old One, mother of our kin \\
Goodbye, Old One, we will remember \\
Goodbye Old One, we will return \\
Goodbye, Old One, you will be here to greet us \\
Twenty winters come and go \\
Twenty springs come and go \\
Twenty summers come and go \\
Twenty autumns come and go \\
Goodbye Old One, we will return \\
Goodbye Old One, you will be here to greet us.
\end{verse}

We came to a shallow inlet of the river with a nice flat shore to slide the boat in. The pack heaped food, tools, and some bone, antler, and hides to work into whatever I needed into the boat, bundled into horse skins and tied in place with sinew strung through holes in the upper sides of the boat.

``Please come back safe, Starling,'' Kingfisher begged me, his soft brown eyes heavy with sadness and his ears splayed. He took my hand and squeezed it.

``I'll be back,'' I said with all the confidence I could muster. ``Also yes\ldots{}I did mean it.''

Kingfisher tilted his head. ``Mean what?''

I kissed him between his splayed ears and held him to me. ``I meant it when I said I loved you.''

``I love you too,'' he sobbed. ``Please come back. So we can\ldots{}maybe one day\ldots''

``...Start a family?'' I grinned.

``Y\ldots{}yeah. Start a family,'' he mumbled, the insides of his ears turning bright pink.

With that I climbed into the boat. Half Moon hobbled up to me, looking very old again, and whispered in my ear:

``Child, listen well\ldots{}If you don't succeed, it means you were not the Old One reborn. This means nothing bad about you! If anything it means I'm a failed seer. If you die on this quest, don't let your soul be fearful. You will not become a wild beast. You will become a great tree, like the Old One and we will honor you as our hero all the same. But if you succeed, then one day you will become a great seer. I see something amazing in you, Starling. I know you'll make me proud no matter what.''

With that she walked round to the back of the boat and pushed with all her might. The heavy wooden hull slid just a little in the soft mud. Then the whole pack gathered round and pushed, and I was adrift, moving out to sea with the current and waving goodbye to my pack, wondering if I would ever see them again.

I paddled along, the current helping me glide fast down the narrow river. I saw wild deer grazing on the riverbank, trees and fields, and the burial grounds of the dead of other packs and their sacred places all along the river. Every clan---the foxes, the wolves, the deerfolk, the otters and stoats---had their own markers and they all seemed to be here, but their markers were a little different than the ones in the valley.

It was a little after noon when I arrived at the sea, at the wide mouth of the river where the current propelled me into treacherous waves that tossed my boat every which way but would not capsize me. Further out there were great swells that I rode, up and down, feeling sick to my stomach after a little while. I wasn't made for this! Why couldn't I have been otterfolk? I struggled to keep my stomach from heaving as I grew used to this sensation, this lack of a clear, straight horizon. I had only a general sense of up and down and it was miserable.

As for paddling, it was of little use. The wind and swells took me wherever they wished, and I was alone, bobbing like a little acorn in a raging river.

But by and by the heavy swells subsided and the sea was calm again. Small swells, maybe as tall as my ankle, now rocked my boat gently in the late afternoon sun.

Then I spied it. The first island I had seen. What had once been the top of a high down was now a small island surrounded by dead trees and debris that still floated around it.

There were bodies. Bloated, discolored, shedding their fur, and stinking hideously. Could I even stay here?

Well, the ground I could see looked dry enough. I turned my boat toward the tiny island.

All at once a mighty swell, maybe as tall as my boat was long, rose up from the sea beneath me and crashed into the island. I was hurled into the treetops where I clung for dear life only to see my boat washed out to sea along with much of the debris around the island.

I was trapped here, no food, no warmth, and no escape. And as the waters subsided from that island I understood who these bodies were. They were the bodies of those who had sought shelter here.

I grew tired as the sun slipped below the horizon, and I wedged myself into a crook in the tree just tight enough to stay put should I fall asleep.

I lay staring into the blackness, my eyes on the shimmering fires of the far, cold heavens. And the tree rocked me in her arms and sang a lullaby like my mother sang when I was young.

I woke refreshed to find no more swells had reached the island. Some of the debris that had been around the island the previous day had returned to the island, once again tangled in the boughs of submerged trees, a twisted mass of broken trees and bodies and things I scarcely recognized, things from the depths of the rivers and seas. But my boat was nowhere to be found.

I clambered down from the trees cautiously, watching the waves, careful for any sign of another swell. I reasoned they must not hit this island often, as I scanned through the debris left among the trees and on the ground. Enough wood I could probably build a fire, if there was time for one. But if another wave came? I couldn't risk staying down from the trees past sunset, let alone for very long at all.

One wave, slightly larger than the ones that had lapped the shore most of the night and day, crashed hard into the shore, and I jumped. No\ldots{}that wouldn't work.

What was I to do? Wait for a boat that might not return when I was surrounded by wood?

But I had no flints with me. They were in the boat, in a sack that might be beneath the waves now.

I climbed back into the trees, waiting for my boat, my throat now horribly dry. I needed water. My water skin was in the boat too, and the water around me was briny and brown and full of death.

Rain. I needed rain. I was so thirsty. So cold\ldots{}

From beyond the seas, Thunder stirred and sent his fire down to earth with a mighty crack. The sky grew dark, and the wind, sharp and cold, whipped the waves to white crests as the treetops I clung to swayed perilously.

The frothy white waves became heavy, foamy swells, churning the dense, foul soup of debris around the island, removing some, adding some, macerating bodies and splintering boughs.

The wind cried as Thunder struck his fury across the sky in sharp, jagged bolts of fire.

``Thunder!'' I cried in the growing dark as the storm raged on. ``What do you want of me!?''

The wind died all of a sudden, and all was deathly silent. Rain fell, sparse and gentle.

Far to the north the sky was too dark to see very far. A distant rumble and a bright flash lit up the horizon and I saw a monstrous wave headed right for my little island.

I held tight to the tree, bracing for the impact.

The wave loomed, slow, unstoppable, and so very high. It picked up speed as it grew nearer, towering almost as tall as the tree I clung to. And no sooner had I judged its true height when it crashed into the island and my tree uprooted with it.

I was thrown from the tree into the water, and struggled a moment before I found a piece of another tree's trunk to cling to. Now in the light of Thunder's fiery bolts I could see my island, far away.

But much closer by was my boat. If I dared swim toward it.

Thunder challenged me with a great fork of fire above me lighting the dome of the sky with a mighty crash. This was it. This was my chance.

I let go of the bough I had clung to and swam as hard as I could across the water. I have never been a strong swimmer and the boat was further away than it seemed, but by and by, my hand rested on its wooden side.

I climbed inside my boat to find my water skin and my sack of food and tools secured in place just as they had been. I drank from my water skin, careful to save some for later.

``Thank you,'' I whispered to Thunder. ``Thank you. I will make you proud!''

That was when I noticed there was no paddle. I was all adrift on the cold, indifferent seas.

\secdiv

\noindent Two days passed with no land in sight, and my food stores were looking bare. My water skin had been replenished with a little rain water that had fallen in the boat but even that wouldn't last me long.

For all I knew, I may have been far out to sea, far away from any land that wasn't swallowed up by water long before I was born. Perhaps I would die out here, all alone in horrible pain, and never see my family again.

But I still had a little food, a little water, and a little hope, and I still had my strength. If I was going to die, it wouldn't be for a while yet. And I kept my eyes open for any sign of land, or anything I could use as a paddle.

It was difficult to tell where I was. It felt as if I was heading eastward, or perhaps southward. But there was nothing to tell, only the general direction of the setting and rising sun and even that was difficult to tell when the sky was cloudy, or when mischievous water spirits kept making false suns on the horizon.

At least Thunder had given me back my boat; a reward for my courage. And I knew Thunder was watching me still, that He never abandoned those He favored. And I felt my last reserve of hope grow just a little on those forbidding swells in the middle of nowhere.

And just as the sun was setting on my third day out to sea, my food and water nearly gone, I saw a tree branch floating near my boat that had just the right shape, long and straight with an end that widened into a flat, broad surface.

A paddle made by the hands of the gods.

And with that paddle, I turned my boat eastward, keeping the setting sun to my back and the bright star the elders call the True Light of the North to my left by night until finally, exhaustion from lack of food, water, or rest overcame me and I collapsed in the back of my boat, face up to the pink glow of dawn.

CRASH!

I woke to Thunder's roar and heavy rain on my face. Rain so heavy I filled my water skin and the boat was still filling with water, riding low, threatening to sink though I scooped it out as much as I could.

But in the gray mist I could see the land. A large amount of land, in fact. And by and by I recognized it. This was one of the ridges on the far side of the Great River from where our pack once lived!

Getting closer, I saw that the high water mark on the plants, trees, and rocks was much higher than the water level now. At least some of the waters had receded.

I bailed and paddled the best I could at one time. But once I got close enough to the island I paddled for all I was worth and felt the boat come to rest on the gentle slope of the ridge. Using a rope of spun grass, I moored it to the snag of a beech tree and waded ashore.

A good portion of land had remained above water here. In fact, I couldn't see the other side. This wouldn't be like the Island of Drowning. This would be an Island of Safety because I knew this land and I knew there would be food and water here. I would live another day.

I set up camp immediately, in a spot that was nice and dry. I built a fire and warmed myself by it. With the tools in my sack, I made a very basic shelter of sticks and leaves and fell blissfully asleep as the day drew to a close.

I woke to a strong hand reaching from behind me and clamping my muzzle shut. ``Don't move!'' a gruff voice from in front of me said, and I felt the tip of an antler spear point at my throat.

Then a familiar scent hit my nose, and I heard my attackers sniff in disbelief too and I knew it wasn't just my imagination.

``STARLING!?'' said a younger voice behind me, and in an instant he let go of my muzzle.

``BRIGHT!'' I cried, throwing my arms around my brother.

``Starling! What are you doing here?'' asked my father, the one who'd been holding the spear to my throat only a moment earlier.

``I came to save you,'' I said. ``There's a lot I need to tell you.''

``And there's a lot we need to tell you,'' Bright said. ``Come on, let's go to our camp. We'll tell you more.''

\secdiv

\noindent We hid in a cave where, in happier times, our tribe had taken shelter during the hunt. It was near the top of the ridge, but hidden by dense thickets and used as a breeding den from time to time by wild foxes whose pungent scent masked ours. We hunkered down amid the bones of voles and squirrels and piles of orange and white fur, only a dwindling supply of animal fat in a dim stone lamp for light.

``A member of our pack went mad. He killed and ate one of the deerfolk stranded here,'' Father explained. ``The peace between the clans of wolf and deer was broken and they have declared war. They've already killed several of our pack.''

``Wait, who else is here?'' I asked.

``It's just us now,'' whispered Father. He sighed. ``Gift of the River was here with us, and Carp and Cormorant and Wasp.''

``The deerfolk got them, one by one,'' said Bright. ``We couldn't save them.''

``How did you get here, Starling?'' Father asked. ``Were you on one of the islands nearby?''

``No father, we made it to the Land of Many Deer,'' I said. ``About half of us did. And we went to the Ancestor Oaks. Half Moon was there. She told us that the Old One was taken home by Thunder around the time I was born.''

Father gasped. ``Then you're\ldots''

``...Here to save you,'' I interrupted. I didn't want him to say it. I didn't want to think about that. I only wanted to get away from here as fast as I could. ``I have a boat. It's by the shore.''

``Did you hide it?'' Father asked, his eyes wide with alarm.

I shook my head. ``No, I didn't know\ldots''

``We have to leave tonight,'' Father said, gathering up his and Bright's few remaining belongings. He and Bright grabbed their spears and hurried out of the cave.

``Show us where it is before the deerfolk get it,'' Bright whispered, eyes flashing with fear.

Another storm was brewing in the north as we hurried down the hill toward my boat. We were just past the camp I had set up when we heard heavy footsteps and a voice shouting ``There they are!'' and the horrible screeching, bellowing war cries of maybe six or seven stags.

``We have to hurry!'' Father cried. ``They've seen us!''

We hurried to the snag where I moored my boat and discovered the water level had gone down a fair bit. Was it the tide? Was the water receding? I couldn't say. But my boat was now slightly out of the water, hung up on the snag on a taut line.

``I'll cut the rope. Starling, take my spear. You're strong as your brother. I believe in you.''

Father handed me his spear and Bright and I stood guard, backing toward the water as father hacked through the thick rope with a flint chopper.

``Get in! Get in!'' Father cried.

Bright and I half-waded, half swam to the boat. It was teetering side-to-side at the end of its tether and if we climbed in we knew we'd upset it, so we clung to the sides, on our tiptoes to keep our heads above water.

The line gave and the enormous boat splashed into the water, nearly knocking us on our heads. We hoisted ourselves into the boat and Father swam after us, catching hold of the stem of the boat and dragging himself in. I grabbed the gnarled branch that had taken me all this way and paddled for all I was worth, out into the darkening seas.

With a whoop and cry the deerfolk lobbed their spears at us. Most of the spears hit the water but one kept going, headed straight for us. My heart sank. Time moved in slow motion and I was keenly aware of a cold wind at my back like an omen of something unspeakable.

Thunder crashed, far in the distance, and in that instant I reached out my hand, grabbing the spear just as its point came inches from my father's back.

I gave the most feral snarl I could manage, teeth bared, ears pinned back, eyes ablaze, and snapped the shaft of the spear over my knee, throwing it into the sea.

``I'll paddle for a while,'' said Father. I collapsed into the bottom of the boat, hungry and exhausted.

\secdiv

\noindent We hadn't gone far when the seas became choppy and dangerous, and we made shelter from the storm on a nearby island, this one smaller and with no other packs or any creature of two legs. The rain replenished our water skins, and a lone hazelnut tree filled our bellies enough to live another day. We made the best paddles we could with the tools we had and slept with our upturned boat for a shelter, the best sleep any of us had enjoyed in many a terrible night.

We set off in the morning on calm seas, singing old hunting songs as we rowed together, and by nightfall we could clearly see the white cliffs on the horizon; the seas were not so wide nor the islands so far apart in the south. From there it was only a few days hugging the coast til we reached the river that led to the Ancestor Oaks, and to our new home.

There was no one to greet us when we pulled our boat ashore near the grove. We pulled the boat as far onshore as we could and carried what was left of our provisions with us into the grove.

In the days since I'd left, a village had sprung up around the stump of the Old One. I recognized the faces of a family I thought for sure had been lost in the flood.

``Heron! Cattail! Is Beetle with you?'' I asked.

``She's in the tent sleeping,'' Heron said.

``So glad you made it,'' Father said, throwing his arms around Heron and slapping his back. ``Where were you?''

``We were on a small island about a day's journey from here,'' said Cattail.

``One with a large hazelnut tree right at the highest point?'' Father asked.

``That's the one,'' said Heron.

``We stopped there on our way back. We were stranded on the highest ridge we could reach west of the Great River,'' Father explained. ``If not for Starling we might not have made it.''

``The Daughter of Thunder has returned and brought our sons home!'' a voice cried behind me. It was Half Moon, her scruffy old features twisted into a big grin. ``Child, you've done it! You are the Old One reborn!''

Mother ran to Father and they threw their arms around each other, sobbing. The young hunters gathered round Bright and cheered for him. Our cousins and uncles and aunts and more distant relations all crushed in, and I was overwhelmed. They were screaming, yelling, reaching to touch me, lifting me up\ldots{}And my heart sank. I didn't think I could ever get used to this life, to being the reborn Old One. I wanted to go back to being Starling. The funny, awkward Sky Child nobody expected much from.

The celebrations continued well into the night. Song and dance and prayers sung to the Old One and to Thunder\ldots{}all for me. It was too much.

But at last everyone had their fill of celebrating, and somehow I kept my composure the whole time. But when the time came to retire to our little horse skin hut, when we had closed the door of woven willow branches that kept the wild animals out, I curled up and sobbed.

``What is it, Starling?'' Mother asked.

``I\ldots{}I can't be the Old One! I'm still young!'' I sobbed.

``I think you and Half Moon ought to talk about this tomorrow,'' said Father. ``She'll teach you everything you need to know.''

``I just want to\ldots{}I just want to start a family with Kingfisher! He didn't even get a chance to get near me tonight! No one would let him get close!'' I cried my eyes out. I didn't care if they knew how I felt about it. I loved him. I didn't want to have to part ways with him to go live like some hermit.

``You may not have to,'' Mother reassured me. ``Please, try to get some sleep. You can talk to Half Moon tomorrow. I'm sure it will be alright.''

\secdiv

\noindent Half Moon reassured me that I could start a family; in fact, she insisted I did start a family. The Old One, after all, had to learn the ways of motherhood and Half Moon, with no child of her own, could never teach me that. But first, I had to promise to let her teach me everything she knew.

So it was that for three years, Half Moon taught me the secrets of the seer. Every incantation, every root and mushroom that lets one speak to the gods or heals sicknesses, how to read bones cast on the ground, how to gaze into water by fire light or by moonlight and read the future in the ripples.

She taught me the language of dreams, the tales of the gods, the tales of our pack going back thousands of years.

Then one day, I came to her hut for my lesson, but she wasn't waiting outside for me. I stepped inside her hut and found her lying covered in furs and skins. She looked frail, her eyes were dim, and her breathing was shallow.

A bit of the old light returned to her eyes as she saw me. ``Starling,'' she whispered with a serene smile. ``My time has come, child.''

I took her hand. ``I still have so much to learn from you,'' I said.

``Nonsense, child,'' said Half Moon. ``I taught you everything you need to know. This is why I've decided it's time for me to rest.''

I felt tears burning in my eyes. She'd been so spry just the day before, no sign of illness or distress. She was so old she could let go any time, and she had chosen her own time to go. There was no sense in trying to convince her otherwise.

``I love you, Teacher,'' I said, stroking her forehead. ``I'll miss you.''

``One day\ldots.'' Half Moon paused, her eyes wandered and her breath stopped a moment before she sucked a gasp of air. ``One day a child will be born in our pack, and you will see me\ldots'' Her eyes became dim. Her feeble grip on my hand relaxed completely. One final gasp, and she moved no more.

\secdiv

\noindent We buried the stump of the Old One, her roots now raised to the sky, a little ways from the ocean in the middle of a wood palisade. And in the hole from the roots of the Old One, we buried Half Moon and laid a great stone on her grave to honor Thunder.

Kingfisher and I started a family the very next spring, with gifts of food and tools and skins from the pack as we moved into the new hut we had built for ourselves.

And some months later, when our firstborn's eyes opened to the world, I gazed into them and saw a very familiar light.
