\begin{quote}
\emph{You must undertake a pilgrimage of seven days, because seven is a number of ill omen. Of all the things the gods gave us, none can be broken apart evenly into seven: the twelve knuckles and five fingers that we use to count, and the thirty days in a month. It is under ill omens that you must live out your days with grace.}
\end{quote}

Tehq felt the muscles in her shoulders burn as she carried the boy's body out of the nameless, rain-soaked vassal canton. The winter rain had soaked through the thin burial wrap, and the leopardess could smell the mint leaves that she had placed over the young boy's chest. Under normal circumstances the boy's father would carry him---he was undoubtedly stronger than Tehq herself, whose years had been spent hunched over manuscripts and clay tablets. His ankle still swollen from the very avalanche that had taken his son's life, he merely hobbled beside her, mute, his head lowered as the rain battered his neck.

They reached the family's burial land after a few short minutes, the plot marked only by the traditional red strips of fabric, tied to low-hanging branches in the surrounding trees. She set the boy's body down while a young girl, now an only child, opened a sack of fern seeds Tehq had asked her to bring. The father took a shovel from his wife, who held it back from him for a moment, as if to protest on account of his condition, but she soon relented. Tehq, stiff and awkward with inexperience, watched as he shakily fought with the black mud that covered the forest floor. Tehq noticed only a few burial mounds around her, covered with the lush green ferns that hastened the dead's return to the soil: two less than a decade old she guessed, and one mere months old, but much smaller. The rhythm of a shovel piercing mud carried Tehq away like a morbid lullaby, the lyrics known to her since childhood. By the time the father was done, the sun had turned her head and left them for the horizon. She reluctantly steeled herself for the prayers that were to follow.

After the burial was complete, Tehq returned to the family's cabin, where the family patriarch practically strong-armed her into accepting an offer of salal wine. It was tradition to offer a funerary priest such trifles, even if she could not stomach the company of a grieving family. The boy's mother, scarcely older than Tehq herself, but whose figure and unkempt fur testified to the hardships of motherhood, fell into a cushion opposite the fireplace and began to quietly sob.

Tehq was too cold, and too sore, to make eye contact with anyone. She had never done this before---not on her own, at least. She had never seen any of these people before yesterday, and this was about as awful a way to meet anyone as she could imagine. The boy's sister, just a few years younger than Tehq, sat down beside her, and put a hand on her mother's shoulder before she turned to Tehq.

``I feel a bit guilty, asking a funeral of you when you haven't even finished your pilgrimage yet,'' she said to Tehq.

Tehq shook her head. ``Death herself is unconcerned with such formalities, so neither am I.'' The remaining formality would be satisfied in two days' time, she thought.

``This is not our first death this year,'' the girl said.

``I'm sorry,'' Tehq said stiffly.

``I had my first son last spring, but the goddess turned her head at him, and she did not breathe air into his lungs once he was born.'' The girl paused. ``When a priest came to bury him, I asked him why the goddess would do such a thing. He said that the gods make all things so that they may one day die, for without death nothing would live. But why would the goddess not let something live before it dies?''

Tehq recalled the freshest mound she had seen in the burial plot. ``By the time I was your age, I had spent years of my life trying to make sense of what had happened to me during the war. As a young girl I couldn't fathom why the gods would permit such an excess of suffering. After so much philosophizing and thinking and praying, after enlightenment eluded me, I realized the truth: neither reason nor faith would ever provide me with answers.''

``Then where does the answer come from?''

Tehq forced herself to finish up the last of her wine in one swig. ``The fault lies with the question. Just as reason and faith allow us to conjure the phantasms of greed, pride, and loyalty that drive us to war, it carries us towards that question of yours that is no less illusory. Truthfully, there is no \emph{why}.''

\begin{quote}
\emph{There will be hardships across your seven days. Some your inexperience may create for you, some the gods will select. How you overcome them is not important: as in life you may succeed in many things and fail in others, but the grace of death levels all equally.}
\end{quote}

By the time Tehq had returned to the tavern where she was staying, she felt about as wretched as she had ever felt in her life. The rainwater had soaked her clothing along with every inch of her fur, and every muscle above her chest ached.

She had stripped herself down to a loin-cloth and a chest wrapping and collapsed in front of the tavern's fire pit, along with a dozen other merchants and traders who were in similar states of undress. Liquor flowed freely, drunken conversation mingling with the scent of burning peat-moss and cider. Tehq seemed disinterested in all of it, at least until her fur dried. Her first unaided ceremony was about as unpleasant as she had imagined it would be. She felt as though she had somehow failed, although she wasn't sure why. She always looked up to the older priests who had raised her, shown her grace, wisdom, and dignity in the wake of tragedy. Why did she feel so far beneath them?

``You look like you've had a harder night than most.'' The voice came from a tigress who sat down beside her. She was beautiful, with flowing elbow-length black hair and bright red robes that wrapped tightly across her generous figure.

``I think the merchants can offer you more for the night than I can,'' Tehq said, her eyes unfocused as she watched flames lap up the sides of the wooden logs.

``What they have in coin they lack in conversation,'' the tigress said with a smirk.

``What makes you think I'm any different?''

The tigress paused, and Tehq could feel the woman's large brown eyes sinking into her fur like the rainwater.

``Priestess in training, probably a war refugee. Hardly the dull type.''

Tehq's brow furrowed and she turned to stare at the tigress. ``How did you---''

``You have a southern accent, but you're not a trader or a nomad. Only people like that who end up this far north are here because they have no home to return to. As for the under-priestess part, one of you passes through here every few months. You're easy to spot.''

Tehq gave the woman a glance that said all it needed to. The pair sat in silence for a number of seconds, the pain in Tehq's shoulders drowning out the din of the drinking traders who surrounded the rest of the fire.

``I lost my father during the siege,'' the courtesan continued. ``He was a merchant, selling weapons to the provinces in the south. A week into the battle, the king tracked him down, and killed him.''

``I'm sorry,'' Tehq said, for the second time that day. The phrase felt just as useless this time.

``Sometimes I take comfort in the fact that all the men responsible for his death were either killed or had their hands marked after the siege failed. Do you think it's right to find solace in such violence, priestess-to-be?''

Tehq couldn't tell if the woman was making professional small-talk or genuinely asking, although the delicate way she rested her muzzle on the back of her hand suggested a bit of both.

``We shouldn't guilt ourselves for the emotions our hearts drive us to feel. At the same time---feelings that are born from pain will only beget pain, if we're not careful.''

``Spoken like a true priestess,'' the tigress said, beginning to look just as lost in thought as Tehq was. ``I have a habit of checking the hands of any older male clients before I take their money, but I feel much less compunction about that.''

Tehq laughed mirthlessly. When she was young, she remembered hearing stories of captured warriors who chose to fall on their swords rather than be branded with the usurper's mark. The thought still sent chills down her spine.

``I suppose I would do that if I was in your position.''

``What about you?'' she said, her voice soft like a caribou foal's coat, despite the subject matter.

``What about me?'' Tehq said coolly.

``Has the priesthood tempered all that bitterness?''

Tehq thought, unsure if she felt like answering. ``One time when I was fifteen, I saw a man with the mark in the middle of town. I had this thought---so immediate and all-consuming it nearly swept me off my feet---that I wanted to kill him. The only reason I didn't was because I knew I would never get away with it. Later that day I told one of the priestesses about what had happened. She said that great men go their whole lives without letting go of that anger, even if it ravages them.''

``Perhaps that anger will be what you confront in the woods, then.'' The tigress said.

``Confront?''

``The locals think under-priests go into the woods to fight a monster at the end of your pilgrimage,'' the woman said. ``I know it's just children's stories, and the fight is metaphorical, but the priests I've met always talk about the end of their journey in such a way. Like it's a battle.''

``I'm surprised you pay such close attention to your patrons' pillow talk.''

``It's my job, but I'd be lying if I said I didn't find it interesting.''

``A less cynical person might think you're taking a shine to me,'' Tehq said. She couldn't help but let the corners of her muzzle curl upwards in a smile.

The tigress laughed. ``No need to temper your cynicism. Like I said, your types come through here every so often. To the surprise of no one, a dreary funeral priest tips better than a drunk merchant.''

Tehq couldn't help but chuckle. The tigress's attention felt like a summer breeze across bare fur, and she couldn't help but notice the fluttering in her heart.``You're right---about the battle, that is. However, no one tells us what we'll be fighting.''

The tigress stood up and offered her hand to Tehq. ``How about some tea, perhaps somewhere more peaceful where you can tell me more? You look just as cold as you do lonely.''

Tehq took the courtesan's hand, and lifted herself out of her seat.

\begin{quote}
\emph{There is no removing oneself from the flow of the cycle, that which changes all things, brings the living closer to death and the dead closer to life. However, you may find moments of reprieve on your pilgrimage; they will humble you, as like all things they are temporary.}
\end{quote}

When Tehq woke, dim slivers of moonbeams pierced the clouds and poured in through the window in one of the tavern's private rooms. She instinctively reached a hand across the bed, but the tigress had left. She felt her heart drop into her stomach---nothing good came from late nights where she had nothing to distract her from the pull of her wandering mind. Through the murk and moonlight, Tehq saw a silhouette by the door.

She bolted up and stumbled out of bed; without the moose skin sheets, the cold ate through her bare fur and chilled her skin. The silhouette was gone, leaving nothing but a cold spear of fear lodged between her ribs. She stumbled towards the door, hand outstretched, as though her claws might make contact with the ethereal shadow she had seen seconds ago. Her knuckles brushed the door, at which point she gave up her search and turned around.

The shadow stood between her and the bed.

Its head was down, unkempt black hair glistening in the moonlight. Tehq felt her breath catch in her throat, her feet nailed in place as it hobbled forward, into the light.

For a moment, Tehq thought it was the boy she had buried hours before. The shadow was the right height, with the same hair. The clouds shifted, bathing the room in pale light. It was not the boy.

``Ikarih?'' Tehq asked.

The young girl did not respond. She was wearing a moose skin jacket, little more than a cut-down section of pelt with a button at the top and a hood roughly stitched on. Tehq had one just like it as a child. The girl looked up at her, face dirty, her eyes bloodshot and wet with tears.

``Ik---''

Ikarih's lips did not move, but Tehq heard her voice.

\emph{You haven't grieved at all. Why?}

The girl's face contorted in agony. Blood poured down from her hairline as she stumbled forward, her bloody hands grasping at Tehq's bare fur. Ikarih started to fall, and Tehq could see the arrows sticking out from her back, the tanned leather of her coat drenched in blood. Tehq tried to scream, but the air froze in her lungs.

She awoke to an unpleasant tingling radiating throughout her body, her heart beating with such speed it felt like it might burst. The cold air stung her lungs as she pulled the blanket over her face, her claws digging holes in the fabric.

By the time Tehq crawled out of the tavern in the morning and hobbled to the stables, fog was rolling over the canopy of the black-green forest that surrounded the little canton, as though the uniformly gray sky was falling to earth to wash the town away.

``I hope you slept better than I did, Tik,'' Tehq said. Her caribou did not respond.

She threw her pack over the animal's haunches and hauled herself onto his back. Tehq had convinced herself to eat an elk sausage for breakfast; she was simultaneously hungry, but couldn't summon the motivation to eat. It was already late in the morning, and Tehq thanked her god that most of the merchants had disembarked long ago; she was hardly in the mood for conversation.

``On your way?'' a voice said from across the stables. It was the tigress from last night. She was wearing an expensive seal fur coat over her robes and leaning against the wall of the stables with her arms crossed.

``I have two days of travel ahead of me. Places to be,'' Tehq said tersely. She couldn't remember the tigress's name, and although the consort was probably used to such things, Tehq couldn't help but feel slightly ashamed.

``One of the girls I work with said she heard a scream coming from your room last night. I said it wasn't my doing.''

Tehq pushed on the caribou's antlers, egging him forward. ``Rough night.''

``Hey Tehq!'' the woman shouted as the caribou canted out of the stables and towards the muddy cobblestone road out of town.

``What?''

``Do you know where the word \emph{trauma} comes from?''

Tehq felt thrown by the question, even if she should know the answer. She rolled the root over on her tongue, the way one might taste wine. \emph{Hʷaušuš.} It took her a second, her skull still foggy from lack of sleep. \emph{Hʷaušuš}. From \emph{ihʷišauš}.

``That which ravages itself,'' Tehq answered.

The woman said nothing, just nodded and bade her farewell. By the time Tehq realized she should be courteous and say farewell herself, Tik had turned a corner, and they were nearly out of town. Her heart sank as she realized it would be the better part of a week before she slept in a proper bed again.

\begin{quote}
\emph{The battle you shall undertake on the seventh day is unique to you, and thus the high priest sends everyone on a different journey. No matter where you are sent to, the sixth day remains the same for everyone. Your revelations at the steles of the old temples will be your last chance to anticipate what awaits you the seventh night. The meditation of the previous five days will moor you to the earth. The terrorflower of the sixth night will unmoor you.}
\end{quote}

The forests were as vast and as dark as the seas that surrounded all four corners of the gods' lands. Roots roiled the mossy, wet earth, tripping Tik's experienced hooves. His antlers scraped against the tree trunks that appeared to squeeze them in from the sides of the dirt path. Before they had entered the woods proper, Tehq had allowed the sun to guide her, but the canopy of the gods' conifers choked all but the faintest slivers of sunlight. To the degree that her path was predetermined, it was with the graces of the gods.

``You've gone this route before, haven't you?'' she said to Tik. She had never spoken to a caribou like it was a person, although she could understand where the term \emph{mount-mumbler} came from; they merely listened, without judgment.

``I've only seen the old steles once before, not long after I was taken in by the temple as a young girl. A priest took me up the hills to one not far from the temple. He said that one day I would understand them.''

The beast snorted.

The dirt road became even narrower, and they soon turned a corner to meet a series of rocks jutting out of the dirt, improvised stairs leading upwards to the top of a hill. She dismounted the caribou and let him traverse the steps on his own weight, the leopardess walking just ahead as the cervine awkwardly climbed up the rocky stairs, errant tree branches scraping his thick winter coat. The air was heavy with the scent of camphor and rain.

They reached the top of the hill, where she tied his reins to a tree and stepped forward, mossy soil sinking under her feet. The clearing reminded her of the stele she saw as a young girl---not from location but from feeling, the way her breath caught in her throat, the way her heart strained against her chest.

The petroglyph set into the stones was only a head shorter than her, spanning a smooth section of the rock face that was only an arm's length wide. Swirls and jagged symbols lined the edges of the carving, where they framed a human figure, its arms hanging down from narrow shoulders, its head set with two sharply-carved eyes, countless antlers protruding from its skull and mingling with a series of six-pointed stars. Above the figure, a circle with six marks was drawn in the center. If she squinted, it looked like an outline of the very statue that rested in the center of the temple where she lived: Irrḫaukt, goddess of death, her arms outstretched, with the body of a leopardess and the head of a caribou.

Tehq recalled something she read on a tablet as a child, when she was still learning the scribal arts: there were ten such steles of the goddess known, their age a mystery, but in the dawn of writing, when the god of wisdom gave Tehq's ancestors the knowledge of words drawn into clay, the first scribes wrote of the steles. They said the carvings depicted the goddess as she showed herself in countless sleep visitations to those who did not know of cities nor kings, in the shards of the nightmares of the guilty and the soothing dreams of the martyred, before the rivers split all lands, and before the ice sloughed to the south.

It was a full minute before Tehq could summon the confidence to move her feet from the ground. She returned to the caribou, his snout digging into the soil as he foraged for food. She reached into her pack and pulled out a coiled length of goat hair twine, crow feathers tied into the string at regular intervals. She tied one end around a tree, walked across the clearing until she wrapped it around five other trees, then finally tied it off where she began, forming a lopsided hexagon in front of the stones. She untied the caribou's reins and moved him outside the circle, pulling the twine upwards to make way for his antlers.

``Sorry, friend. I don't know if anything rests in the soul of a farm caribou besides dreams of salmonberry and lichen, but I don't think you'd want to find out.''

She lifted up her robes and sat down on her knees in front of the petroglyph, her hands shaking as she pulled a goatskin flask of salal berry wine from her bag and uncorked it with her teeth. She libated it onto the earth, the dark red liquid swallowed up by the soil.

``I suppose one doesn't ask for niceties from the goddess of death,'' she said as she returned the empty flask to her bag, ``so I won't deign to do such a thing. If you have seen fit not to bring me to submission yet, perhaps you will grant me something more.''

If the goddess heard her, or cared, she did not know. After all, a man does not justify himself to a caterpillar. She pulled a little leather pouch from her bag and dumped the contents, a coarse green and black powder, onto her fingers. She tried desperately to steady herself as she opened her mouth and rubbed the powder on the underside of her tongue. It tasted earthy and bitter, like seeds of the thorny-bulbed flower from which it was extracted. She closed her eyes, lowered her head, and started to chant.

\vspace{1em}

\setlength{\tabcolsep}{3pt}
\noindent\begin{tabular}{p{0.4\textwidth} p{0.6\textwidth}}
\noindent\emph{aküśunån asšåqåk aqʷitülåk} \newline \emph{ikśånån sšaquk äqʷitüluk}
&
\noindent\emph{I survive, I pass into darkness, I fall silent} \newline \emph{The Shadow traverses the silenced\ldots{}}
\end{tabular}

\vspace{1em}

The words stuck to her soft palate until they became a mumble, and before she noticed, she had stopped speaking. Her mouth felt like it was full of dirt. When she opened her eyes again, the forest was darker, broken apart, as though she was viewing it through the reflection in a murky lake. She tried to pull herself to her feet, but her limbs forced her to the ground, muscles leaden and inflexible. Shadows flexed and flickered in the corners of her vision, the shapes pulling into a dozen crawling monstrosities the moment her eyes did not focus upon them. Terror thrummed in her veins as the monstrosities moved around her, but she could not bring herself to run.

With great effort, she lifted her head, and met the shadow.

She was not sure if it was god, or man. Thoughts passed through her head like rainwater through woven cloth. It was a child.

``Ikarih?''

The edges of the shadow sharpened. It was not Ikarih, the hair was wrong, she was too old. Tehq understood.

``Why am I seeing you?'' Tehq said to the blurry image of her younger self.

``Do you not always see me? Sometimes in a dream, sometimes in your reflection in the water?'' The shadow's lips did not move as she spoke, the words instead rising up from the earth.

``Why now?'' Tehq could not force her throat to pull language from the mire of her mind for more than a second at a time.

``You ask the wrong questions, as always.''

Her heart felt as though it was trying to punch its way out of her ribs. ``You're\ldots{} you're\ldots'' she tripped over her own tongue, ``\ldots{}supposed to have a clue for me.''

``Clues cannot save you from yourself.''

``You're supposed---'' Tehq repeated herself.

``Tomorrow you will face what you have yet to mourn.''

Something sour turned over in Tehq's stomach. She swayed on her knees, the world tilting around her.

``You're wrong,'' Tehq spat.

``You speak from anger, not wisdom.''

Against her better judgment, Tehq pulled herself to her feet. She towered over the shadow of her younger self, but she did not feel like it.

``Of course I do.'' She could not tell if she was raising her voice, her head was a rock sinking to the bottom of a lake. ``Ikarih died in my arms. I saw my parents put to arrows.'' She stumbled backwards, her legs beginning to fail her. The monstrosities writhing around her moved in, like spiders crawling towards their prey. ``I still have nightmares. I cried for weeks. I felt nothing for years. I had no choice but to mourn them!''

``Your sister and your parents were the first to fall under the soil of the earth, but you have not mourned all.''

``Then tell me what I have forgotten! My sister, my parents, my innocence---''

The shadow of herself pulled towards her, as though the ground between them had disappeared. It grew, surrounding her from all sides, it did not speak, it did not open its lips, but she heard it screaming, the sound of a hundred avalanches. It no longer looked like she did, it was a monster, a nightmare, with pale eyes like moonlight beaming through a blizzard.

She fell to the ground.

``The day the war arrived dozens of futures did not come to pass, some died, others walk as shadows. You must meet the eyes of the one you have forgotten, and you will see judgment in its gaze---be it judgment of forgiveness, anger, or capitulation,'' it said.

Tehq tried to scream, but she could not; her tongue remained frozen, and she felt her lips babble like an infant's.

``You must look into the eyes of the unmourned, and with grace you must make peace,'' it roared.

The monstrosities in the corners of her vision moved in, and forced her into darkness.

\begin{quote}
\emph{The gods have decreed that all things must suffer. Therefore, there can be no enlightenment without suffering. On the sixth night you will acknowledge your suffering, on the seventh you will not be victorious unless you know how it has shaped you.}
\end{quote}

Tehq was awoken by Tik's rough tongue licking the morning dew from her cheeks. The sunlight, even dulled through the clouds, burned her eyes as she opened them. She did not know for how many minutes she lay on the ground, the veins in her head throbbing.

Eventually she summoned the power to grab Tik's reins and pull herself shakily to her feet. She wrapped her arms around his neck, unable to stand on her own as the earth wobbled beneath her.

``Time. What time?'' Tehq mumbled. Tik snorted in response, lowering his head to pick at the mossy earth.

Tehq squinted up at the sun, the dim overcast morning making her eyes water. It was mid-morning. She let her arms fall to her side, and immediately stumbled backwards onto the ground, where she decided to remain until her body told her otherwise.

After she regained control over her legs, she stumbled over to where her bag rested by the stele, and rummaged around for a folded cloth of rations. She took a few nibbles from the corner of a slice of acorn bread, testing to see if her stomach would allow her to keep food down. After a few more minutes of waiting, the acorn bread was joined by elk sausage and dried salal berries, washed down with more wine. After the world no longer tilted every time she stood on her legs, she pulled herself gingerly onto Tik and consulted the guide that the high priest had written for her.

Written with ashy ink on finely-pressed cattail paper, the final set of instructions were no more than a sentence:

\emph{Leave the clearing of the stele via the path to the northeast; half a day's travel on the main road will bring you to a western trail, where your journey will end at a lake.}

Typical high priest behavior; extracting meaning from vagueness is left as an exercise to the reader.

As Tik canted down the pathway down to wherever the high priest was leading her, shadows of last night flickered in Tehq's memory. Part of her knew she should prepare for what awaited her at the lake, but she couldn't bear the thought of uprooting whatever memories the shadow was directing her to; it seemed to her that her memories did not need further examination. Whenever she closed her eyes, she could still see her sister's body in her arms.

The pathway down the hill led to a main road that was part of a trade network, as indicated by the heavy wheel tracks carved into the hardened dirt. The hours passed in bitter, exhausted silence, the cold air biting at Tehq's hands as she relentlessly gripped Tik's leather reins. She stopped on the roadside once for lunch, although it was mostly to give poor Tik a rest, after he had spent the better part of two days carrying her. Trade caravans creaked by them as Tehq listlessly nibbled her elk sausage while Tik rummaged through the moss and mud with his snout, foraging for mushrooms.

By late afternoon, a light drizzle of sleet had set in, blanketing the dirt road in glittering white ice. Tehq unconsciously clenched her jaw, wishing for her journey to end. The shadow still murmured to her, words creeping across her thoughts. She had done enough mourning to last a lifetime, what wound could be left that she had forgotten to heal?

Tik stopped in the middle of the road. He lowered his head, bearing his antlers at the road ahead, just as the two were about to turn a corner.

``Tik?'' She put a hand on his head, petting his sleet-covered fur.

She heard a commotion, followed by a scream, then a hoarse, piercing roar. Bears.

Tik snorted angrily, threatening to buck her off. A war-mount he was not.

She dismounted and pulled a bow and quiver from her saddle bag, stringing the bow as Tik backed away. Another scream. She nocked an arrow, readying another in her mouth. The priests had taught her archery once when she was sixteen---and she was not particularly good.

She ran forward, paws slipping on the ice as she turned a corner.

She didn't even see the source of the screams at first, just a dead caribou, and a massive brown bear wrestling with \emph{something} on the ground. The bear shifted, and Tehq caught a glimpse of a bloody red hand between folds of fur, clutching a knife buried in the animal's side. The knife was too small to make a difference, but just large enough to make an angry bear even angrier.

Tehq yelled and stomped her foot in vain, the bear disinterested in her as the man---an older tiger---pulled out the knife and weakly slashed it across the bear's face, angering it further. The two briefly made eye contact before Tehq raised her bow and let the arrow fly, bronze tip striking the bear firmly in the shoulder.

The bear roared furiously, lashing around to charge at her, great paws slipping on the blood and ice. Her hands shaking, Tehq nocked another arrow, the bear closing over half the twenty paces of distance between them by the time the second arrow flew, grazing the bear's back. Left with no other choice, Tehq hit the ground, rolling off the road and colliding with the mossy tree trunks that separated the trail from the forest. Pain bloomed in her back as she hit the trees, numb fingers grasping for another arrow before she had even registered where the bear had come to a halt.

Disoriented and bleeding, the wild animal spun around to face her, its limp slowing it just enough to spare Tehq from the claws as she had dived. She pulled the arrow back as the animal charged, the bow's string scraping her cheek just as she let go.

The arrow embedded itself in the bear's neck with a \emph{thud,} its feet giving out from under it as it slid across the mud, massive body thrashing uselessly in a growing pool of dark blood.

Tehq pulled herself weakly to her feet before nocking another arrow. She still remembered a prayer her father had taught her when he took her seal-hunting with the nomads, before the war. It was the same phrase said by the corvée warriors who hunted the men who killed her sister:

\emph{I am sorry. To the soil you must return.}

She aimed for the animal's skull, and released the arrow. The bear became still.

For a brief moment, she felt an inner peace she rarely felt, as though the bear's death had loosened something inside her.

The tiger behind her groaned. Ears perking up, she spun around and ran towards him just as he was wrapping a torn fragment of his coat around his bleeding leg. His eyes were unfocused, and most of his lower torso was smeared in blood, although she was unsure if it was his or the bear's.

``You know, my dad always told me it was a good omen to see bears this time of year. It means winter is over,'' he said shakily as he tightened the scrap of fabric around his leg. Blood oozed from a gash just visible under his riding trousers.

He was a good head taller than Tehq, approaching middle age, with a wiry black-and-white beard. His muzzle was dotted with scars and scratches, and a tip was missing from his ear.

``You should get that taken care of,'' Tehq said awkwardly, kneeling down to examine his leg.

``And the sky is gray,'' the man said. ``The name is Šōhen, by the way.''

``Tehq,'' she responded, hoping her caribou hadn't run too far.

``You saved my life.''

``And the ground is wet,'' she responded, smiling weakly.

Tehq stuck her fingers in her mouth and whistled, summoning a very nervous Tik from the woods. He trotted towards her reluctantly.

``And if you had arrived late, I would have had a funerary priestess to bury me anyway,'' he said.

``Do you have anything to boil water? A bear's claws are heavy with sickness, you must wash it with boiling water,'' she said, ignoring his earlier comment.

``Trust me, this isn't my first scrape,'' he said as he pulled himself up with his hands and pushed himself to the side of the road, bloodied leg dragging across the dirt. He pulled a flint from one of the pockets of his oversized jacket and stuck it in his mouth while he reached for any dry tinder on low-hanging branches. ``I come prepared these days.''

Within a few minutes, Šōhen had testified to his survival skills by starting a fire while Tehq gathered sleet in a small copper pot. The fire crackled warmly as the sun kissed the horizon, and Tehq couldn't help but rest her hands in front of it as she waited for the water to boil.

``I'm sorry about your caribou,'' she said.

Šōhen grimaced, the shadows in his haggard face dancing by the fire. ``If the gods spare my leg from rot, I can mourn my mount.''

Tehq's ears perked up. Between the bear and Šōhen, she had completely forgot why she was here. ``Do you know if there's a lake here?''

He nodded. ``Big one, just off the main road. Not long by caribou.''

Tehq breathed a sigh of relief. The water started to boil, and Šōhen removed the bowl from the fire, bracing himself as he poured the steaming water over his leg.

Tehq could see the muscles in his jaw clench as the water poured over raw flesh.

``That won't be enough to stop the bleeding,'' she said.

He grimaced. ``I was hoping you wouldn't say as such. My family has a small cabin nearby---I keep a supply of chokeberries there. A simple salve should staunch the flow of blood.''

Tehq got up and adjusted Tik's saddles. ``Do you think you can make it?''

``I'm afraid I don't have a choice.''

Tehq wordlessly bent down and let him wrap his arms around her shoulder, muscles in her legs straining as she laboriously crawled onto Tik's back with Šōhen holding onto her. Her thoughts returned to her pilgrimage, and she wondered if the gods would put such an incident before her if they did not intend her to intervene. Šōhen's next words, however, answered her question.

``There's a trail leading west around the next bend. It'll take you to that lake; my cabin is there.''

Tik slowly canted forward, unaccustomed to a doubling of his passengers. A chill ran down her spine as she realized that Šōhen's life had not collided with her own on accident.

Sure enough, only a few minutes later, the road split in two, with a smaller path heading downwards into the woods.

``That one,'' Šōhen said, motioning weakly, his other hand firmly gripping Tik's saddle. The path leading down to the lake was almost as narrow as the path she had taken to the stones, the two enveloped in darkness as the woods swallowed them again, errant branches tugging at Tik's antlers. Tehq could smell her passenger's blood mixing with the scent of conifer sap and wet earth.

Šōhen adjusted his grip on Tik's saddle as he tried to steady himself, the sleeves of his jacket catching on Tik's saddlebags and riding up his arms. Out of the corner of her eye, Tehq saw the mark on his forearm. An \emph{X} carved deep into the flesh, rippling scars visible through the striped orange and black fur.

A priest's voice came to her through the twisting inky-black woods like a crack of lightning.

``Our enemy is not our teacher. If we turn our swords to unarmed men, by what measure are we better than them? Still---something must be done with them.''

Tehq, wearing robes that were at least two sizes too big, watched in horror from the window of the temple along with a half-dozen other orphans and the priest. The warrior was as naked as the day he was born, ribs stretched against thin and patchy fur, his lanky frame dragged through the snow by two men who held him by his arms. A crowd gathered, watching silently as the men pulled him into the center of the town, where a blacksmith's forge waited. One of the men wrapped his hand in leather and pulled a knife from the forge, the bronze hot and glowing like the sun. A few of the children gasped. The warrior screamed as the other man held him to the ground, forcing the warrior's arm out into the snow. With two quick cuts, they drew the \emph{X} into his arm with the glowing blade, the naked warrior screaming with a ferocity that made Tehq's blood run cold. She pulled her robes up to her eyes as the man thrashed helplessly on the ground just as another was dragged out, the condemned warrior thrashing fruitlessly against his captors.

Tehq buried her head in her robes as she felt bile rise to the back of her throat.

``Are you alright?'' Šōhen asked weakly.

Tehq pulled herself from the riptides of memory, her mind's eye burning like an open wound splashed with sea water. For a few seconds she did not respond.

``Yah.''

``You sound almost as bad off as me. Traveled long, I assume?''

The currents pulled her under. Her stupid, weak little hands tugging uselessly at the arrows embedded in Ikarih's back. The splattering of blood on snow as swords cut through flesh.

The tide came out, the dark forest around her seeming nothing more than shapes, as unreal as a child's ink brush drawing.

``Seven days.'' Tehq felt her lips move, but the words seemed to come from across a great distance.

Šōhen mumbled something in acknowledgement, but the words were lost on her. As reality seeped back into the clamor of shapes and sensations around her, she felt as though she might crawl out of her fur. Every muscle in her body rippled as her soul thrashed against her flesh like an animal caught in a trap. He was one of them.

The tides swallowed her again, and Tehq tried to moor herself with the distant voice of a high priest. \emph{You must live out your days with grace. Death will hang over you always, demanding inner peace as the earthen world rages around you, as the nights become deep, as the soil of time is watered with both rain and blood.}

A pristine lake stood before them, still surface glimmering with the orange-red haze of the sunset.

``Here,'' he said. ``There's a few hunting lodges halfway down the lake, one belongs to my father-in-law''

``It's a beautiful lake.''

``As good a place to bleed to death as any,'' he said sarcastically.

She pulled on Tik's antlers, bringing him to a stop before she dismounted, awkwardly stumbling onto the ground.

The cabin was more a small hunting lodge, the inside populated with little beyond a bed of animal furs, a fireplace, and a table. Tehq sat on the ground, staring at the patterns in the rough hewn wood. She didn't notice that she had her hand clenched around one of the arrows in her quiver until Šōhen said something.

``Don't worry, I'm not going to attack you, even if I was in a state to do so,'' he said between clenched teeth, holding a dripping cloth sack of chokeberries over the gash in his bleeding leg, the astringent purple liquid dripping down onto exposed flesh.

``What?''

``Your hand. On the arrowtip,'' he said.

``Oh.'' She loosened her grip, palm drenched in sweat.

``Don't blame you. Young priestess traveling the woods with a strange old man. I'd have a hand on a sharp object too,'' he said, his tone suggesting humor, but his words became a jumble of sounds in Tehq's ears.

``No, it's just---your arm.'' She spoke before she could decide if she even wanted to say it.

Šōhen was tying a fresh strip of linen around his arm. He stopped, his jowls turning downwards to a grimace. ``I was hoping you wouldn't notice,'' he said before tying off the fabric. ``That was a long time ago.''

``It was six years ago,'' she said, her voice barely a whisper.

He leaned against the bed, nursing his leg, his eyes focused on some distant object behind her. ``Sometimes it feels like a lifetime ago, sometimes it feels like yesterday. Particularly the nightmares. Those---those don't feel like ancient history.'' He briefly made eye contact with her, before his eyes darted back towards the floor.

Tehq opened her mouth, a small instinctive part of her wanting to mention the nightmares, but she couldn't bring herself to say such a thing, to bring voice to the unconscionable idea that something might connect the two of them.

``I saw my sister die.'' The feeling of words in her throat told Tehq she was speaking.

``And I my nephew. Sometimes I'm not sure if I'm being haunted by ghosts or by memories. Even in the moments when you forget, the memories are still there, not as images or sound but as pain. Aren't memories strange like that? What happened has passed, the beast has been slain, but still it ravages; but then I wonder, is the beast still ravaging me, or is it myself?''

``She was ten.''

``He was seventeen.'' Šhōhen gazed at something beyond the walls of the cabin. ``The last time I saw him, a sword had opened him like a butchered goat. I remember standing there, wondering if I was dreaming, because none of it seemed real. I spent so much time mourning him I forgot that a part of me died as well.''

A part of Tehq wished he would stop talking---the more he talked the harder it became to be mad at him, and she preferred to be mad at him; if her anger was just, there was nothing to her experiences beyond good and evil. They were victim and victimizer, caribou and bear, black and white. The thought of a world beyond that duality gripped her with sudden and unexpected terror.

``Do you expect me to feel sorry for you? Did you not choose to pick up a blade?'' she said.

Šōhen looked grimly at the floor. After a moment, he shook his head. ``I was a debt laborer. I didn't choose anything. I killed a farmer's caribou. They dragged me into court, I didn't have the silver to pay.''

``I lost my parents. My whole family. A temple took me in when I was thirteen. And you---you went to war for a caribou?'' She didn't realize she was speaking through clenched teeth until she felt the pain in her jaw. ``Why should I believe you? Why should I feel the same sorrow for you I feel for my family?''

``I'm not asking for sympathy. I've had to forgo such wants years ago. As for the belief, they said I didn't have a choice. I was released two months before the war ended. The court gave me a copper seal to prove I paid my debt---''

He moved his hand to reach behind him. Like the rush of an avalanche, Tehq's mind flung itself back into the present. She thought back to the road, with the bear, to Šōhen sheathing his knife just behind his right hip.

Her hand squeezed the arrow again, arm swinging around to point the glistening bronze tip at his throat. The motion was violent enough to undo the leather strap holding her hair in place. He stopped. She looked at him through the unkempt locks of black hair that had scattered across her face, her pupils wide like a cornered animal. Her hand shook, the dull metal less than a thumb's length from his skin. He looked at her, but his expression was not one of fear. For a brief second, as the moonbeams pouring in from the window rippled across his face, Tehq thought she saw a look of understanding, sympathy even.

``I'm not going to \emph{hurt} you,'' he said, his bloodshot eyes locked on hers. ``I told you, I have a coin\ldots''

His hand reached under his robes, he shifted his weight, and Tehq saw a glint of metal, pale and red like the bronze body of a dagger. The world around her turned to shapes, the way images dissolve right as one is falling asleep, sense disappearing in the murky waters of unreality. She felt her arm swing around, muscles flexing and then stabbing forward just as he reached his hand behind him. She felt her hand connect with fur, the knuckles holding the arrow connecting with sinew and muscle. She felt warmth on her hand, saw the river of red, scented blood in the air, just like when she was a little girl, when she held Ikarih in her arms.

She pulled the arrow away. More red, more than she had seen in years.

She scrambled to her feet, backing herself into a corner. She heard a clattering sound. The arrow had fallen from her bloody hand onto the floor. Šōhen was covered in blood, one hand reaching up at her weakly, the other clenched into a fist. Tears rolled down his cheeks as his hand weakly grasped for his neck.

The hand fell to the ground. His eyes rolled back, lifeless gaze settling on the ceiling, mouth agape, cheeks sinking into his mouth dotted with missing teeth. As his muscles loosened, Tehq heard a clattering sound, and looked down just as a large bronze coin came to rest by her feet.

Outside the cabin, the moon shone brightly in the cloudless night sky, the first she had seen since last summer. Pale light blanketed the lake as she stumbled outside and fell to her knees at the water's edge. She forced her hands into the water, letting the blood wash away into the clear, sparkling water. Tears fell down her face and into her mouth, where she tasted the warm brine, a distant sensation among many. Tik trotted up beside her, snorting and resting his head beside hers. Tehq envied the ignorance of animals.

Her whole body was jittering and alive with fear, her heart frenzied as it beat within her, the furious rush of blood in service of nothing as she stared blankly at her hands in the water. As her body steadied itself, she could only think that the chaos of her childhood had slipped a yoke over her. The war was over, the blood it had spilled had long since sunk into the soil, but still it controlled her, sometimes in dreams, sometimes in feelings, and now in violence.

She leaned forward and squeezed her hands together under the water, forcing the blood out from the white and yellow fur on her hands and wrists, little streaks of crimson spreading out across the lake. She pulled her hands from the water and dried them with the front of her robes, watching as the droplets of water sent ripples along the surface. The water calmed, and Tehq stared down at her reflection, her own gaunt and tired eyes staring back up at her in quiet, resigned disappointment.
