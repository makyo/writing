The Isle of Ocrit was not a mighty power. No great battles had been fought there. No tales were told of it. No riches were to be found there. It would be a half-century or so yet, ere the cultivation of grapes would make its slow migration from the mountains beyond the east horizon, so it had no vineyards. It would be nigh onto a dozen centuries before the great empires rose and matured enough to crave silphium, so it remained only a pale yellow flower in the woods and a savory tang in meat broth. Anyway, Ocrit did not have so very much silphium, either.

What Ocrit had was a true and genuine Oracle. And a true and genuine Oracle, even if it were not so famous or prestigious as the Cybeles of Delphi or Iliun or Cumae would one day be, was worth a journey.

Let us imagine some tyrant, brooding on next season's harvest of conquest. Or perhaps an heiress, seeking an interpretation of a troubling dream. Or an oratorical nobleman, seeking an advantageous marriage for his favorite nephew. They would come to the small harbor, make their way up the winding path through the sighing cedars. They would begin to question, more than once, if they had lost their path, only to soon see, ahead in the dim forest, an oil lamp guttering in the constant cool sea wind. They would hear ahead in the darkness of night---for the Oracle would speak to none unless the moon were overhead---something wild and dark played on a haunting pipe, invisible beyond the trees. And just when they would consider going back---surely a seasoned war leader can make do without prophecies, surely a village wise woman can interpret a dream, surely one's dear nephew can find his own way into a marriage bed---the trees would open up, and the sky would reappear.

On the left hand would be goat-grazed cliffs down to the lights of town and harbor, where the sailors who had brought them here sat in the taverns in the company of that most enviable of lovers: a full belly, plenty of beer, and a warm fireside. But on the right hand would be the sanctuary.

I have heard it claimed it was a lightless palace of obsidian columns, where chthonic vapors from the underworld took the shapes of shades of the dead and vengeful spirits. But I would not believe such things. Obsidian makes poor columns. And what vapors there were would have come from the braziers, not the underworld. Such things are for oracles as they are imagined, not as they are---or were, rather, I know of none today. More likely it was like any other shrine to a little god among the little islands in those little days, magnified to terrifying and august by the darkness of the hour and the loneliness of the journey.

Now follow our hypothetical pilgrims in. See them startle as a masked acolyte steps forth from the shadows. The warrior blusters, perhaps, and the heiress pouts, but they are told to enter of their own free will, to speak only once, and that only to ask the question they have brought.

Then in through black curtains, to an inner chamber. The only light, the moon through a single high window and the smoldering of something sharp-smelling and smothering in the braziers all about the room. Perhaps the orator is curious, steps near to examine one of them, and staggers back at a sharp word from a masked acolyte. The same one who led him in? Ah, impossible to say! They are all about the room he now can see, and all hooded and cloaked the same. And something in the braziers made him lightheaded.

Best to be done with this unworldly business and away. So they speak. They ask,

``Shall I see victory when I lay siege to the clan of the northwest?'' says the eagle.

``I dreamed that my grandmother, dead these thirteen years now, pierced me through the heart with a spindle, and left me hanging in the fig tree. What does it mean?'' says the goat.

``Young Tammuz is of a marrying age, will he make an advantageous match?'' says the stag.

And then they wait, nervous, while their question rolls away into the close heavy darkness. They cannot even see who it is they are asking. Until they can.

Across the room, hunched over one of the smoldering braziers, smoke and darkness wreathed about their head---and a fearsome head it was indeed---Behold! The oracle would turn toward them ominously just after they were noticed---and how it was the oracle knew when to turn, when to stay statue still, who can say---then rise to an impossible height and stand in the shaft of moonlight. A great mask, some primeval and forgotten monster of a bird, cruel bulwark of a beak and imperious eyes, like a vulture god risen from the underworld. It would stare, yet how could it stare, with eyes of blank stone? Yet not a one, of all the years of pilgrims, had not felt the weight of the oracle's regard as surely as they felt their own breathing.

And when the tension was just a hair from becoming unbearable, the oracle would speak:

\begin{verse}
Thou would be conqueror, thy snows fall red. \\
The oldest mountains turn their backs on thee. \\
The serpent shall lie burning in thy bed \\
Ere you return to set it laughing free.
\end{verse}\pagebreak

Or,

\begin{verse}
Go ask the spindle what the fig tree means. \\
Go ask the beehives what the spindle spake. \\
Go seek for thirteen months of sleep and find \\
The spindle cracks, and all the skeins unwind. \\
Thy grandmother has left her shawl behind \\
And somebody must wear it to the wake.
\end{verse}

Or perhaps,

\begin{verse}
The golden apple, lovely to behold.\\
But green is sweeter on the tongue than gold.
\end{verse}

And that would be that. And then an acolyte---did it matter if it was the same one?---would usher them out again, to ponder on what it could have meant---the eagle sullenly over a now-less-likely campaign, the goat pensive as if picturing herself in her grandmother's place at the table, the stag uneasy and now less sure if that favorite nephew really is his favorite after all. Perhaps, for some additional alms, beyond what they had given to be admitted? Ah, but of course, the acolyte would confess, the deep meaning of the oracle's wisdom could be explained, by one who had initiation and familiarity, perhaps on the way down the path to the village, for it is dark now indeed\ldots{}

And that was how the art of an Oracle was practiced, in those days.

Historians today, I suppose, would shrug, would say it was some drug, to induce trance in a practitioner who would then believe themselves possessed by Apollo---or at least something close enough it might as well be called `Apollo' in the history books---and deliver whatever pronouncements bubbled up in their brain.

And they would not be wrong about this, in the way a man who calls a fresco only plaster and paint, who says a mosaic is only a great many fragments of broken glass, who says a sacrificial cake is only corn and honey and fat, much as one eats every day, so what is the difference really, is not strictly speaking wrong.

But what the historians do not, could not know, is that after our pilgrims left, and the doors were locked for the night, the oracle would fumble his way out the back. Would slip off the trailing androgenous robes, pull off the top-heavy bird mask, and take deep breaths of clean clear air until there was no oracle there at all. Only a young dog called Talzu---brindle coat, darkened to an ash-brown mask around his face---naked and shivering till he staggered exhausted into the bed in his little hut, hidden behind the oracle's sanctuary.

\secdiv

\noindent Talzu remembered very little before his life on Ocrit Isle. And that which he did remember, he held of no consequence.

His mother---perhaps as oracle he could have said where she was now if he cared to, but he did not---had seen some signs in him. Perhaps she had a touch of foresight herself, perhaps there was some omen or augury, perhaps she saw he showed no fondness for girls at the age it was to be looked for, and any oldwife will tell you that means a man's path in life leads to the mysteries of the half-world.

Perhaps she merely needed to rid her house of another mouth to feed.

Whyever it was done, Talzu had been sent to the nearest oracle---for `nearest' was the oracle of Ocrit Isle's most notable quality---in the midst of his tenth summer. He had served as an acolyte. He had lit the lanterns in the woods, he had played the haunting pipe in a hidden alcove, and had never thought to go farther. Most did not.

But he did. And when he did, his life truly began.

\secdiv

\noindent ``You are dwelling on the past again,'' Histuman would have said, had he been there.

Talzu walked among the cedars on the upper slopes, down the path the orator, the heiress, the general would have trod last night. It was late morning, and in the light of the sun the woods were less aweful and numinous. Working at night meant he could sleep as late as he liked, do whatever he liked in the day. Some of the acolytes, he supposed, expected him to remain in his room. Perhaps he would have, were he a grander oracle, if anyone cared to pry after his secrets, if there were luxuries for him to send acolytes to fetch. But on Ocrit Isle the only luxury to be had was freedom, so it was what he took, when he could.

``You could have more,'' Histuman would have said. ``Those who come for what we do, they pay well.''

``You're chatty today,'' Talzu muttered and picked up the pace toward the market.

``You could send an acolyte to the market for you,'' Histuman would have maintained, ``as you used to go for me. You could send a ship to fetch you fine linens and rich spices.''

``There is none,'' Talzu navigated the stepping stones across the gully where, during the spring rains, there would flow a busy creek. ``Who could bring aught I want.''

To that, Histuman would have had no answer.

\secdiv

\noindent If Talzu gets to dwell on the past, tis only fair we do as well. So let us look back at Talzu the acolyte, when he has been tending the pomegranates in the side garden, where pilgrims were led if they were too overcome with fear or sorrow on hearing their prophecy. It did not see much use, and the pomegranate bushes never set fruit for they did not grow well on this side of the mountain, but there was little else to do to keep busy until the next pilgrim arrived.

But when the step of a bare foot sounded on the wind-worn stone behind him, Talzu straightened and froze.

``Somehow,'' the voice of a wolf, dried to gravel with irony, had said, ``you knew I was no pilgrim, eh lad?'' Had Talzu not already straightened and froze, he would have done so again. Histuman the Oracle sat himself on the stone bench just beside one of his sworn-silent acolytes, the only ones to know his identity, and looked Talzu up and down like a farmer considering whether to buy a prize bull at market.

``What would you have of me?'' Talzu had said. It was forbidden for him to kneel, or use any titles of reverence, for that might reveal who it was to whom he spoke. But he could not wholly keep a tremble of fear from his voice.

``Much, boy, I fear very much indeed,'' the old wolf chuckled. ``How many years do you wear?''

``This coming harvest will be my nineteenth.''

``And in all that time,'' Histuman's eyes strayed to the sunset over the edge of the sea, ``have you yearned to leave this island? Forsake my service and secrets? Go out, see the world, and win glory in it, eh?''

The dog shrugged. ``I know of nowhere I would go.''

If Histuman were disappointed at that, he showed it not. ``And there is no woman whose favors you seek?''

``I\ldots{}no, I don't\ldots'' Talzu blurted.

``No ruddy maiden at the market? No plump matron? No prosperous widow?''

Talzu had felt a growl rise in his throat. ``I have never lain with a woman, if that's what you mean.''

``But you have,'' Histuman glanced at the young dog slyly, ``disported between Kuruhdu's knees of a night, when the two of you were on vigil together. Aye, and warmed his bed of a morning after long sleepless watch.''

Talzu's breath had caught in his throat. But then\ldots{}this old wolf tormenting him was Oracle, was he not? What man could have secrets from him?

``You will not have long to love him. Within a moon, a trading ship will come by, unexpectedly, bearing a hired blade who hails from the far Cimmerians. A great sabrecat of a woman, and Kuruhdu will be so enamored of her, though she will be twice his height, that he will seek release from his vows and my service. Which I will give him, of course, why not? He will sail away with her, and that will be the last we ever see of him. Now, when you went to the market, a morning ere last,'' Histuamn had been suddenly done with that line of discourse, ``You brought back chickpeas, leek, beets, and a good bit of lamb.''

Talzu had nodded, confused. The wolf had tossed all of these into a stew---and Histuman almost never cooked himself---with a good helping of barley to make it go round, and it had been a very fine meal indeed for the oracle and his four acolytes.

``How did you know,'' Histuman had pressed him, ``to bring back the makings of the equinox stew my mother used to make, when I was yet a boy among the cities of the plain? Which I have not tasted since?''

``I\ldots'' Talzu had had no idea what Histuman was talking about. ``I did not.''

``Oh, was there nothing else to be had in the market that day?''

``No, there was\ldots'' the dog had wracked his mind to remember, ``there were pears and salt cheese and tunny fish\ldots''

``And why is it, boy,'' the wolf had interrupted, had risen to his feet, had suddenly and softly stroked Talzu's cheek with the back of his hand, ``that I have never once heard you speak a lie?''

The memory of that still put cold fingers between Talzu's shoulderblades.

``You know the answer to your own questions,'' Talzu had said. Had been unsure if he were begging an answer, or making an accusation.

``An oracle who knows his business,'' Histuman had answered, ``rarely asks a question to which he does not know the answer.''

\secdiv

\noindent And that had been the day Talzu had been no more acolyte, but apprentice. Every moment, waking and sleeping, he had spent with the old wolf. Learning not just signs and omens, in the stars, in the clouds, in the patterns the curls of smoke from the sanctuary braziers wrote on the air, but how to stand amid them and see the whole story they told as a woodwise hunter may stand amid trees and understand the whole forest. Learning not just the rites, but the multitude of meanings behind the rites, how they meant the moon journeying to the underworld and returning with its secrets, but also meant the man hung living on a sacred tree as sacrifice from whose heart's blood all harvests sprang, but also meant the unnamable watchers who stand motionless and stern at the gates of death in the uttermost west keeping secrets no mortal mind could know and live.

Learning, aye, but also practicing. Tasting with his breath the bitter secret mix of incense and oily seeds and sacred herbs burned in the braziers and fighting to keep his head steady amid the warm and weightless rapids that would then swell within his ears as he held that fume-filled breath, a little longer each time. Memorizing whatever verse the sanctuary could remember, much of it nonsense, and then changing it, a word at a time, so that nothing remained but the rhythm and rhyme, that whatever visions might come, his tongue would have the agility to tessellate into verse on the spot. Balancing the heavy mask, until he could sit with it, stand with it, turn with it as if it were his true face.

By the time the ship had come, just as Histuman had said it would, and Kuruhdu had disappeared to sea with his saber-mawed warrior, just as Histuman had said he would, Talzu had begun to see. Or rather, had begun to realize he had been seeing, all along. How the present burned down the future like the wick of a lamp, fixing it into the past even as it consumed it. He understood, though he couldn't have put it into words, how the old wolf did what he did, and therefore, he supposed, how the young dog did also what they did. He began to know the answers to questions before he asked them, once he had learned to ask them the right way round.

And when he awoke just before dawn, with Histuman's arms around him, with the old wolf's taste on his breath, with the oracle's kiss on his forehead, Talzu would feel the future rolling towards him, clearer every day, like a great island-swallowing tide.

\secdiv

\noindent But now, of course, that future had dwindled to mere present. Histuman had been among the things its tide had swept away, and Talzu had taken his place as oracle, just as foreseen. And who was to know the difference? The other acolytes were just as sworn to secrecy as ever, and anyway by now they had been all replaced. Arikimra, now keeper of the island's smaller tavern, was the only one still to dwell on Ocrit Isle, and he had been released from his vows while Histuman yet lived. As far as he, or anyone else on the isle could say, Talzu was just another of the acolytes.

``You miss me,'' Histuman would have said, as Talzu approached the buildings clustered about the harbor, ``more than usual.''

``So what if I do?'' Talzu returned an idle salute from a sow driving a herd of goats toward the upper pastures. She had no idea to whom it was she offered an upraised hoof in casual greeting. Two moons ago he had told her that either her next child would be stillborn, or the birthing of it would cost her life, and now rumor said she had put off her husband, he had gone, with great bitterness, to be a fisherman with his brother on the next island southward.

``You know very well,'' the old wolf's voice would have been peevish, but he would have nuzzled the young dog's ears as he spoke, ``what. Find yourself a distraction. A cheese dumpling or a bowl of beer or a handsome fellow, well, handsome enough, here.''

``All the same distractions,'' Talzu shook his head, ``as have distracted me less and less with each passing of winter. You know what it is to grow old, old man.''

``I am glad I did not live,'' Histuman would have said, ``to hear a mere lad of single score and seven winters call himself `old.'\,''

Talzu stopped at the pile of stones at the crossroad---the only crossroad on the island, in truth, worth piling stones at---to add a pious-enough pebble to the windward side. He paused a while there, and before continuing spoke.

\begin{verse}
``A lifetime is a road before us each. \\
The man who sees his road laid plain unto \\
The gates of death, though yet so far away, \\
May call himself an old man, and speak truth.''
\end{verse}

Talzu stood in silence a while before his feet found the will to walk again.

``How long have you been brooding on that one?'' Even in death, the former oracle would not have asked a question to which he did not already know the answer.

``Long enough,'' Talzu whispered.

And it may be the old wolf would have taken pity on Talzu, had he been there. It may be his teacher would have appreciated the passive despair he had not meant to teach. It may be that is why he would not have said anything as the dog he'd loved in life walked to the harbor, for it may be he would have known that even an Oracle needs to be surprised from time to time.

It may be that is why Talzu had no warning, either natural or unworldly, of the proud ship with the saffron colored sails and the burnished copper prow beached comfortably in the harbor, nor of what twist of destiny it had delivered to the Oracle of Ocrit Isle.

\secdiv

\noindent You will not have heard, I daresay, of Ouanaxes, whom some called Pirate King. The kingdom of which he was both prince and exile has no name in the remembrances of mortals. He lived too soon for the invention of history. And though epics indeed were sung of him, and tales told, the only one to make its way, limping and exhausted, to these cold latter days is this.

Ouanaxes was not such a man as to have any care for whether you and I had heard of him.

Imagine him, then, as Talzu first saw him. Begin with a lion, give him all the strength and royalty a lion ought to have, but take from him all concern, and all dignity, for he is free. His silt-brown fur knows well the touch of sunlight, the indistinct pebble-grey stripes are acquainted with the storm-streaked clouds they resemble, the dun mane smells of salt spray.

Rather than princely finery, give him a kilt of toughened leather, the kind divided for easy movement, a sash of brightly woven cloth across his chest, and a trusty sword in a worn scabbard, with the hilt of which, just as you catch sight of him, he has gestured some of his men up toward the woods. In his other hand put a bowl of ale, brought out by the tavernkeeper, whom he pays by tugging free one of the dangling golden beads sewn to his sash. It is the most wealth she has ever seen at once, in her life. Then around him picture a whirl of activity, sailors and pirates, fighting beasts all. The ram barters for provisions, the cormorant fills jars of fresh water, the ibex seeks carpenters to repair those bits of the ship, the hoopoe seeks smiths to sharpen these spears, or the rats and foxes and seals merely look for a comfortable bed and a willing wench. A chaos of seaborn manhood, at least by the standards of Ocrit Isle, and at its epicenter is Ouanaxes, as if it emanates from him as the philosophers claim the true natures of things emanate from the gods, with heavy sandals undone and bare paws at ease, as if he had no more cares than an innocent shepherd in the golden age of lost Arcadia.

It may be he did not.

It took Talzu some time to make his way through the storm of activity to the lion that was its eye. This was good. It gave him time to consider what to say.

But he would consider in vain, for the pirate spoke first. ``Well met, honored sir. I presume you king of this fair island?''

``We have no king here,'' Talzu said, cautiously, ``in all honesty, we have not folk enough even for a chieftain.''

``Why, it's a thousand pardons I must beg!'' Ouanaxes' eyes sparkled, Talzu would come to know, whenever he grinned like that. ``I took you for a king at least, for it's myself and you alone who take a breath of leisure amid this bustle. Come then, take that breath with me? If your fair isle has offered my poor band hospitality, why, it's only fair I offer it back!''

And that was a better opening than Talzu guessed he would have been able to plan. ``If you but saw us when no pirates had made harbor,'' the dog took a seat beside the lion, ``you would find little else but leisure here.''

``Pirate, you say?'' Ouanaxes affected great innocence and drained his bowl of beer.

``An islander,'' Talzu shrugged, ``knows a pirate when they see one. You would not see a hubbub like this for a fisherman!''

``I daresay not!'' Ouanaxes laughed.

``Nay, for a fisherman brings no wealth from the treasure barges. As never a one of them thinks to stop here themselves, we islanders are not like to see any of it save what a pirate comes to spend.''

``Ah.'' The lion seemed, for the first time, less than perfectly at ease. ``You have trade with pirates often, then?''

``Here? Never.'' Talzu accepted a bowl of ale from the tavernkeeper, who then gave another to Ouanaxes and bustled away before the dog could pay her. ``Other islands, to be sure, but Ocrit is overlooked by all save those who seek the Oracle.''

``Oracle?'' The lion perked his ears, and oh the sound of his voice was stirring like promise of a journey begun just at dawn.

``If you come not seeking the Oracle, you are the first,'' Talzu huffed.

``In good faith, I heard not there was such a thing until now.'' Ouanaxes ran claws through his windblown mane, and oh the roll of dusky fur over the muscles of his bare shoulder was a perilous thing to see. ``I saw only an island where an honest captain might rest his crew, patch his hull, and fill his belly.''

The dog glanced down the shore at the slender ship beached there. Several oars were broken, and more than a few arrows, hafts snapped off, heads buried, studded the starboard side like the stubble of an old boar's chin. ``How long it will be safe to do that,'' Talzu said, very carefully, as he finished his bowl of ale, ``may take an Oracle to say. If you will excuse me, captain, I must be about my business.''

``Perhaps we two can share drink and speech again?'' Ouanaxes stretched to his feet as Talzu rose to go, and oh the possibility of laying upon that fur seemed more comforting than any bed. ``If I might ask to know you better, of course.''

``They call me Talzu,'' the dog said. ``Any of the islanders, I trust, can tell you where to find me.''

``Then find you I shall,'' smiled the lion.

And oh, that smile was more intoxicating than the fumes of a dozen oracular braziers.

Talzu strode away from the harbor, back toward the heights and his sanctuary.

``Are you then in love so quickly as that?'' Histuman would have asked, greatly amused.

Talzu saw no purpose in replying. An oracle rarely asks a question to which he does not already know the answer.

\secdiv

\noindent It was a few days yet, ere Ouanaxes visited the oracle.

Betimes Talzu met him every day in the market, or the tavern, or the shore. Every day the lion had another task in hand---his men were scouting the coast for a cove where a ship might anchor out of sight, or seeking a woodcutter to see about felling a cedar for a new mast, or trading necklaces strung with amber and lazuli beads for flatbread and dried fish, or merely all heaving stores into the hold, naked and sweating and unashamed as primordial gods at their world-shaping labor, before he charged with them, laughing, splashing, into the surf to bathe. And every day he would set whatever task aside long enough to smile at the dog whom he knew as nothing more than a fellow man, and talk, and share a bowl of beer.

Every day Talzu felt his heart become a little less his own, and foresaw that tomorrow it would be even less so.

When finally Ouanaxes took him by the paw, pulled him without a word onto the ship, and led him to the stern where cushions were laid under the canopy, Talzu accepted that his heart was lost entirely.

``How did you know?'' Talzu asked, when their muzzles had parted. The taste of the pirate's lips clung to his.

``I am not a fool, my friend,'' Ouanaxes gently unfastened the dog's tunic, pulled him free of it and down into his arms. ``I know enough to know when a man wants me.'' The same motion of the lion's paw somehow contrived to run up the dog's side, explore the shape of his flank and underarm, then take him by the wrist and lead Talzu's paw down Ouanxes's chest to rest between his bare thighs. ``And I know,'' it took only the smallest motion of his hand to touch the other man fully, but the lion had left that motion to Talzu to make, ``that I want someone who wants me.''

No man had touched Talzu, had held him, had loved him thus since Histuman had died.

``Tonight,'' the lion whispered, after, to the dog who lay in his arms, head on his breast, clutching him tightly. ``I must at last go and seek your Oracle after all.'' Ouanaxes gently stroked Talzu's ears, and if the lion felt the dog freeze, for just a moment, he acknowledged it not.

``What is it you seek to know? I may be able to answer it myself.''

``Alas, I ask not after your heart, my friend. That,'' the pirate kissed him, slowly, gently, ``I mean to win wholly myself. I'll not suffer fate's interference there. No, someone advised me, when I first arrived, that the oracle might tell me how long it were safe to remain here. I needed to be ready, if it's an unfriendly answer I'm given, to go at once.''

``And you lay with me now.'' Talzu's brow furrowed, ``knowing you might be about to leave?''

``It was the last thing,'' Ouanaxes smiled, ``I had need to see completed, ere I could bear to depart.''

\secdiv

\noindent Talzu had but barely enough time to make it back to the sanctuary, don his robes and mask, and calm himself before Ouanaxes came to seek the Oracle.

``If you falsify prophecy, boy, because you wish to keep him\ldots'' Histuman would have whispered ominously in his ear, had the former oracle been there, ``then may your spirit never again know peace.''

``I know!'' hissed Talzu, without moving. ``Distracting me will not help!''

The two nearest acolytes shared a worried look, behind their masks. But there was no time for concern, so the one went to the entrance, to meet the pirate, to command him to speak but once, and that to ask the question he had brought.

It twisted Talzu's heart within him to see the lion, so near, yet be unseen, be unknown. But Talzu's heart was not what was wanted here, was it? Talzu was not who this man had come to seek. He sought the Oracle, did he not? And Ouanaxes needed the Oracle, no matter what the Oracle had to say, not Talzu.

That thought proved enough to stiffen his will and empty his mind. He breathed in the fumes, steadily and silently, and felt fate fill him like a rising sap in spring fills the unfurling leaves. Just in time.

``If I make harbor here, if I return here, from my raiding, when my ship and men have need,'' the lion wore a cloak across his shoulders, drab and rough. Perhaps he meant to be disguised, seem less the warrior, more the peasant? Or perhaps he meant to seem humble before whatever divinity moved in the darkness before him? ``will it remain for us a safe refuge?'' But in fact it only made him seem the larger and more solid, like a wall hung with tapestries, ``Or will we be discovered here?''

He was not the only one in this room, was he, whose concealing garments made him into something larger than he was? But of the two of them, the dog could see through the embellishing disguise to the man beneath, tense and uncertain, and the lion could not. That was, perhaps, the burden of being an oracle.

This was the last thought in Talzu's head before prophecy chased it out, to rattle about the inside of the mask, while the oracle declaimed,

\begin{verse}
``Thy throne upon the sunset's pillars calls \\
In vain. From these obscure haunts you shall flee \\
No more. Forsake you all that you could be, \\
And frail old age you may yet live to see \\
Beneath the hand of these oracular halls.''
\end{verse}

Surprised relief flashed across the pirate's face. Whatever he had expected to hear, it was not that. Talzu's head swam, his senses returned, and he first credited them not, for he thought he saw Ouanaxes on his knees, arms spread and palms raised in supplication. And only when the lion raised his face again, opened his mouth, eyes shining, for effusive thanks that never came because he remembered, just in time, he was not to speak again, did the Oracle understand that somehow his soothsaying had indeed been in both their favors.

There had been no need to adulterate or bend it to keep his beloved. Fate, at least the piece of it Ouanaxes had asked, the piece Talzu had spoken, had been on their side.

The lion left the sanctuary the one way, his step lightened, his eyes lifted up. The dog, after a time, left it the other way, shaken and scarce able to believe what his own mouth had said.

Talzu found himself collapsed beneath the fruitless pomegranates he had once tended. One of his acolytes pressed his shoulder, gently, relieved to see he yet drew breath.

``You spoke true, lad?'' Histuman would have helped him to his feet by now, if he had been here. ``That was prophesy indeed?''

``I did,'' Talzu croaked, and the acolyte startled, for he knew not to whom Talzu spoke, ``It came on me so strong that I couldn't have resisted if I'd tried.''

``Why then, rejoice,'' Histuman would have said.

``Please, can you stand?'' whispered the acolyte.

``I will try.'' Talzu answered both of them.

``Some good fortune even we do not foresee, so it may as well fall to you, eh?'' Histuman would have shook his head, baffled, as he was left behind in the tiny side garden to think on how strange the ways of fate had become.

And as Talzu let himself collapse into bed, into sleep, he was grateful Histuman was not there to ask what else he had seen, of which he had spoken not a word.

\secdiv

\noindent Now step forward, in your imaginings, a month or so. The season had turned, and Ouanaxes announced the winds had turned with them. Those ships which went north and east, he said, bearing gold and incense from the God-Kings in the south, have weathered the summer becalming and now mean to bear back cargos of rare metals and jewels, from the unknown shores of the north and whatever nomad warlords they could find to trade with there. So the season was come for piracy.

``I will bring you back,'' Ouanaxes bid farewell to his dog, on the shore, with a great abundance of kisses, ``a gold ring for your tail. Set with amethysts, maybe.''

``I would rather,'' Talzu returned every kiss his lion gave him, ``you bring me back your self, safe and unhurt.'' But there was little fear in him. Three among his crew had visited the Oracle, the night before, and all had asked if any among the pirates would be slain. Each time the answer had been no.

``Still, amethysts would look most striking against your fur!'' Ouanaxes laughed, and his eyes glinted, and he went aboard.

Once the boat took the surf, and passed the breakers, Talzu went to the high bluff, to watch it drive west on a score and six oars until the sail caught the wind to carry them toward the sunset.

``You said yourself,'' Histuman would have reassured him, ``he will not be harmed.''

``Aye,'' said Talzu. ``But it will be wearisome, waiting for him to return.''

\secdiv

\noindent Over the next fortnight, an architect came to ask if the hill on which he planned a fortified place for a local despot were firm and stable, a rich matriarch came asking to which gods she should make sacrifice so that her yet-to-be-born grandchildren would live healthy and prosperous, and a lovesick young fool came wanting to know if a woman to whom he had never spoken loved him.

Each night, after he had answered them, Talzu's dreams were a torment.

The first, he dreamed of Ouanaxes, robed and crowned, seated in a high place to deliver verdicts both just and merciful.

In the next, he dreamed of Ouanaxes bearing a sacred torch, on a quest through haunted mountains, to relight the altar fires at an abandoned temple and appease the curse of an angry god on a whole people.

On the last, he dreamed of a city fully in celebration, dancing and singing in the marketplace and on all the rooftops, as their prince, long promised, returned from exile to take the throne and restore peace and plenty. And below, Ouanaxes's ship drew into the harbor, stately, on sea as smooth as beaten metal and clear as glass, under showers of silver apple petals cast upon the breeze.

``Sleep has failed you, lad,'' Histuman would have said, if he could have sat beside Talzu, ``and this is a poor place for breaking your fast.''

``True,'' the dog clutched his breakfast cup in the sanctuary garden as the stewed grains and sweet herbs in it grew cold, ``but it faces west.''

``When other men are troubled by dreams of ill portent,'' Histuman would have sighed, ``they consult an oracle.''

Talzu scowled at where the old wolf would have been sitting.

``Break your fast first, lad.'' Histuman would have said. ``What will your pirate think if you waste away to nothing before he returns?''

Talzu's scowl deepened, but he gulped down his gruel and curds. ``Did you ever,'' he said, ``know more than what you were asked?''

``Aye,'' Histuman's voice would have grown cautious and grave. ``Rare it was, but from time to time there would come one on whom the fates had laid a finger. Those with great and noble destinies, or monstrous and horrific ones. And whatsoever they actually asked, some part of the deeds they would someday do would bear down upon me like a deluge.''

Talzu bit his lip.

``I have heard, indeed, I have seen, what may happen if it be too much.'' Histuman would have relaxed easily into lecture, ``I was not apprenticed here, you know. I learned at a temple on the mainland, and that land is thick with heroes. When they would come, my teacher, an old and august woman, a leopard, she would sometime snap, deliver them prophesies unasked for, that she had not the strength to hold back. Many was the time they could not even speak their question entire. It became, I think, a part of her fame---that you might be told not what you wanted to hear, but what you needed to know---but it broke her in the end. Her soul could bear the weight no more. And that is why, when I came to the mastery of my foresight, I sought out an obscure sanctuary, to unknown gods, where few would think to bring anything so pestilent as a hero's destiny.''

Histuaman would have fallen silent, then, on first noticing how tight Talzu gripped the cup, how wide the dog's eyes were, and how fixed on the horizon toward which Ouanaxes had gone. And the old wolf, who would have known better than to ask what his student had seen, would have only put an arm around the dog's shoulders and held him close.

\secdiv

\noindent The day the ship returned, Talzu was awake before the sunrise, and down waiting at the harbor hours before he sighted it.

Ouanaxes was standing on the prow, leaning forward. He was too distant for what he shouted when he saw Talzu waiting to be heard, but he dove off and swam ashore without waiting for the ship to make land, so his feelings were not difficult to infer.

It would perhaps be thought very shocking, in these days, for man to kiss one he loved in full view of all the island and a shipful of his sailors, but those were simpler times.

When they at last lay, peacefully, blissfully, in one another's arms, all appetites sated---which had taken no little doing to accomplish---Ouanaxes kissed Talzu again, on the side of the neck, and said ``I suspect it's as sorely as I missed you, that you have missed me.''

``That may be,'' Talzu said. ``But it's also that you are a man whom it is a joy to welcome.''

``Oh, I am welcome, then?''

``Must I welcome you still further, to make you understand?''

``Let it never be said,'' the lion nuzzled him, ``that I rejected offered hospitality.''

The raid, indeed, had been a brilliant success. They had come upon a barge heavy-laden with tribute, bound for a warrior queen---who purposed to build a palace that outshone her father's in splendor---in an attempt to win the allyship of her armies. Because these armies were so desperately needed, no warriors had been spared for the ship, and they had taken the whole cargo with but little bloodshed. They unloaded all manner of rich and comfortable furniture---as well as the to-be-expected gold, silver, fine patterned linens, incense and spices, and all manner of jewels---and the homes of Ocrit Isle were suddenly all more gracious than they had ever before dreamed of being.

And there was indeed a tailring of amethysts set in gold for Talzu, as promised.

But for all the time the pirate and the oracle spent in eachother's arms, rather than seeing to the treasure, you would have thought neither of them cared a bean for any of it.

\secdiv

\noindent The next three years passed much as has been described. There was plunder and victory on the sea, and there was love and comfort on the return.

For the dog's part, when Ouanaxes was gone the dreams of his beloved's glory and heroism, if he but left him and his isle, would haunt him. Then when the lion was in his arms again, they would recede like the tide, always threatening a return.

``So, when I am away,'' Ouanaxes said, ``you are some manner of priest at the sanctuary of the Oracle?'' His head lay in Talzu's lap, in the whitewashed brick cottage the pirate had taken, a half hour's walk from the harbor, to be his dwelling on Ocrit Isle.

``If I were,'' Talzu stroked the lion's ears, ``I would be bound by sacred oaths not to reveal it.''

Here discourse was obliged to wait for a time, while Ouanaxes's tongue attended to more important matters.

``I do recall,'' the lion nuzzled the belly that cradled his face, ``a number of mysterious fellows, their faces all hidden, who attended my audience when I went. If I were to ask if you had been one among them, what then might you say?''

``I suppose,'' Talzu laughed, ``I would ask you to tell me about your country. Where did the journey that brought you to my bed begin?''

So Ouanaxes, who was no fool and could see plainly what was plain enough, moved up beside Talzu on the bed, and told the dog of an entire city that was a palace, of the topless towers, and the temples on the high places. Of eating melons cooled in springwater and meat skewers hot from the grill in the market square, of the warm and steaming public baths, and the festival parades on the holy days dedicated to the queen of the night sky, and the lady of the underworld, and the sacred king of the harvest between them. And if his voice grew low and wistful, heavy with nostalgia, and if he slowed to a halt, and shied away from any mention of why he was not there now, or how when folk spoke of the absent prince they oft used the words `banished' or `exile,' then Talzu mentioned it not.

They each understood what it was for the other to have secrets.

\secdiv

\noindent ``I understand what you are doing, lad,'' Histuman would have said, ``but do you?''

Talzu turned not away from Ouanaxes's ship, departing on what the pirate said was likely the last sortie before winter storms came to shut all the merchants in their ports.

``You have not said you saw more of his fate,'' the old wolf would have followed Talzu as he strode up the path, past the cottage where he meant to spend the winter with his lion, and into the forest toward the heights and the sanctuary. ``But neither have you made it hard to guess. Will you tell me, at least, what grim future you fear in your dreams? What keeps you from restful sleep every night you are not with him? What does my shade linger with you for, if not to give you counsel?''

Talzu strode faster.

``If some danger awaits him, or even death inescapable,'' Histuman would have been snarling by now, ``what good does it do, to keep this from him?''

``I saw he was going to leave!'' Talzu turned on his heel in the sanctuary gates to howl back at the empty forest. ``I saw the grand and glorious destiny---throne, triumph, and a hero's renown---that awaits him if he leaves this place and never returns! Fate means him to be much more than mine, and by the gods, if any man knows he can indeed be much more, it is I!''

Histuman would have been too shaken to reply.

``Yet as long as I do not tell him, as long as the oracle stays silent,'' Talzu shot a disgusted look at the hall where he had stood, masked and robed, to tell Ouanaxes it would be safe to dwell here, ``he has no wish to leave! He is happy with me, I am happy with him, and I'd be glad to count whatsoever glory might have been as worth nothing, as a thing that will never exist and therefore matters not, if it were not that I cannot unsee what I saw!'' The dog could not keep a whine out of his voice, ``Every dream grows clearer. In each of them he is more glorious. And in none of them am I anywhere to be found.''

``Is that not his choice to make?'' Histuman would have drawn near, tentatively, as if trying not to startle away a frightened animal, ``If you lay the two futures before him, and let him decide?''

``He will decide to stay, because he will decide not to hurt me.''

``You have foreseen this?''

``I do not need to.''

``Then,'' Histuman would have said, ``all will be well. Why this woe?''

``Because he should go! It is an unjust thing to deprive a rightful king of his kingdom, is it not? Is that not what I am doing, old man?'' Talzu retreated into the sanctuary proper, hushed the concerned acolyte with a gesture, and strode into the hall. The braziers were unlit, the mask set aside in an alcove shrine. ``And what of the fates? What plagues will they send on my head, or on his, if I continue to defy them?''

``If you wish,'' Histuman would have stood by the mask he had worn in life, one paw on it, wistfully, ``I will play the oracle for you, lad. You journeyed to the sanctuary, you came within, you asked your question.''

Talzu could hear his own heartbeat in his ears as he nodded.

Then, without ceremony---perhaps the dead need not the things, to see fate, that do the living---Histuman would have recited,

\begin{verse}
``Trade crown for heart for crown, and be forevermore alone. \\
Lose all thy self within the masks you did not ask to bear, \\
But none but those outside of them can read what masks fate wears. \\
Let him that speaketh fate to men have no fate of his own.''
\end{verse}

If there had been an acolyte who had the gift to hear what Histuman would have prophesied, then perhaps for some additional alms, beyond that he had paid to be admitted to this place---and Talzu indeed felt he had paid much, by now---someone could have offered an interpretation.

But there was none but himself.

``What did that mean?'' he asked, quietly.

``I suppose,'' Histuman would have sighed, as he gestured for Talzu to follow him into the garden, ``you have not foreseen wrongly. If he leaves, if he returns wherever it was he came from, he finds glory there. And aye, that may be what the fates intended for him.''

``But if he does,'' mulled Talzu, ``he loses himself in kingship, in the mask of it? The same as I was becoming nothing but the oracle, ere he arrived.''

``A likely reading, lad,'' Histuman would have nodded, ``what make you of `those outside' who `read the masks fate wears'?''

``I suppose that means us.'' Talzu said. ``Means me. In order to foretell fate, I had to shake loose of it, to be without it. That is why all that can be interpreted of what you said is about him, not me.''

``If all I can do is foresee of him, then I shall tell you what I foresee.'' Histuman would have taken a good breath, gathered his thoughts. ``On the one path, he leaves you, and all is as you have foretold. Glory and a throne, the kind of destiny all men dream of and few attain to. On the other, he remains with you all his days, and those are unremarkable. Eventually the petty kingdoms know better than to send their ships past here, they will have learned to fear the peoples they meet on the sea. By then he will scarce care. He will have brought wealth enough to make Ocrit Isle a comfortable place for himself, for the one he loves, to live out the rest of their days.''

Talzu wore the face of a man who expects a trap.

``The dreams, on this path, either fade, taking much of your foresight with them, or they grow until your mind snaps under the strain. And one day,'' Histuman would have growled, ``some strange and foolish people may discover your forgotten tomb, look on your bones and his, lying paw in paw and arm in arm, and say `they must have been brothers.'\,'' He would have pointed a finger at Talzu without looking at the young dog, ``And it is you, lad, that must choose, not he. He came to you, the oracle though he knew it not. And aye, he had a glorious destiny before him, but if keeping him is what you choose, and all that comes with it, why, is that not a destiny too? Is that not a path the fates have set before the man, just as much as is the glory you saw?''

``And perhaps,'' Talzu whispered quietly, ``I would rather be broken in his arms, than whole and alone?''

Histuman would have had nothing to say to that. Which is hardly to be wondered at, since he was not there. He was dead.

The dog squeezed his eyes shut against his tears, managed to contain them. ``I would you had not died. That I were still only your apprentice. That I could know, if I let Ouanaxes go, I would still have your bed, and your arms, to take comfort in.'' And Talzu hoped Histuman would have said something like, `But then you would not be Oracle. And you are a greater Oracle than I.' But there was nobody there, save himself, to say it.

Thus did Talzu set his shoulders, and dry his tears, and turn to do as a great oracle would do: To choose the future, by choosing which prophecy to say, and which to leave unspoken.

\secdiv

\noindent So it was, alas, that I must tell you: when Ouanaxes returned---empty handed, as he had said, the season of storms when none could safely set sail was all but upon them---it was to see Talzu waiting, as ever, at a high place above the harbor. But this time it was without eagerness.

``The oracle has summoned you,'' he told the lion, his face all concern, ``they say there is something they must say.''

``You cannot warn me what it is?''

The dog shook his head.

``It is uncommon, is it not,'' Ouanaxes ran a paw through his mane, and oh the way the fur rolled over the muscles was a precious and bittersweet sight, ``for the Oracle to call for a man? Usually tis pilgrims who seek them out.''

``I have never known it to happen before,'' agreed Talzu.

``When?''

``As soon as can be.''

``Very well,'' Ouanaxes breathed in his courage, like the hero Talzu had foreseen him into, ``lead the way.''

When they reached the sanctuary, Talzu stopped him. ``You must wait here. An acolyte will come to fetch you, in the Oracle's own time.''

``Will that acolyte be you?'' Ouanaxes asked, very earnestly.

``I\ldots'' Talzu shook his head, ``...cannot say.''

\secdiv

\noindent The oracle was lighting one of the braziers when the lion was admitted. He had not had to wait very long. The room was lighter than usual, for rarely was a pilgrim permitted to see it during the day.

``Hail. I was told,'' Ouanaxes went to one knee, ``you would have words with me?''

The oracle nodded, slowly, for they had to be careful with the enormous mask. No moon was overhead, no sacred herbs burned in the braziers, no rites had been performed. But it seemed, today, such things were not needed. The oracle spoke, quietly, casually, as would two citizens who met in the street:

\begin{verse}
``Why do you tarry, King Ouanaxes, here? \\
Thy house sits empty, and thy crown unclaimed. \\
Thy uncle is unseated and undone \\
And, jeered out of the orphan's gate, is fled. \\
His treachery can no more threaten thee. \\
The goddess waits, upon her lantern hill, \\
To crown again her sacred king, and cries:\\
``Why do you tarry, King Ouanaxes, there?''
\end{verse}

The lion flinched back, as if he had been struck. He opened his mouth, thought better of it, closed it again.

Someone observing very close might have seen the oracle's mask tremble.

Finally, Ouanaxes gathered himself again, bowed graciously, and made for the door. But when he reached it, he stopped. ``I know it's forbidden to speak more than once, but it would not be the first time I dared do what I must, for I knew it was banishment for me already, and nothing had I to lose. So I will say: may I know if Talzu is here among you?''

The oracle turned their back.

``Well, whether or not he is, I would say this: I will never forget him. I swear it.''

The acolytes all looked to the oracle, who whispered ``It will be made known to him.''

If any noticed the tail, visible beneath the oracle's robes, with a ring of amethysts set in gold, none dared remark on it.

\secdiv

\noindent Talzu walked the path down through the cedars utterly alone.

It would be well, he supposed, to retrieve all his worldly possessions from the whitewashed cottage in which he had meant to spend the winter, return them to the hut behind the sanctuary. Without his beloved, what use had he for the place?

The dog froze as he stepped through the doorway. ``You are yet here?''

``You thought I would go,'' replied the lion sitting on the bed, awake with a lamp though it was after midnight, ``before I saw you at least once more?''

``I feared-'' was all the dog managed to say before the lion was upon him, clutching him tight, kissing him with a desperate hunger.

``Nay,'' sobbed the lion, between kisses, ``never. To be with you is the last thing I must see completed, before I may leave.''

\secdiv

\noindent In the morning, the Pirate King gathered his men on the shore.

He spoke to them of his homeland, which many of them shared, and told them the tale of his banishment, as a youth, by his mother's brother. He warned them they might face dangers---for it was nigh to winter, and the season of storms where only fools and desperate men set sail was upon them---and battle at journey's end, for who could say how many of his uncle's party might yet remain? But any man who sailed with him, he would regard forever as a hero, and if the gods were with them and he did reclaim his throne, their names would be etched in stone to be remembered for as long as his house endured.

Alas, no, I can tell you none of those names now.

He would not command any of them go. ``Let any man speak,'' said Ouanaxes, ``and then I will lead those who will follow, and I shall think no less of any who choose to remain, for aye,'' and he could not keep his eyes from straying to the high bluff above the harbor, where Talzu watched, ``Ocrit has been a home to us indeed.''

In the end, some stayed, and some of the islanders of Ocrit left. For such is the way of the lives woven for mortals by the fates: they intersect, they tangle with eachother, and never do they meet but some go their separate ways. And yes, it was a hard voyage. The storms were dire, but some god of the winds must have been with them. For yes, they arrived safely. The lookouts on the lantern hill spotted the burnished copper prow and saffron sails. And yes indeed, Ouanaxes entered into his city, amid rejoicing, under showers of silver apple petals, and he relit the altar fires, and was crowned, and ruled both justly and with mercy. Just as the oracle of a distant island had once foreseen.

\secdiv

\noindent And some have said that when the time and signs were right, the dog left the island. Left behind the oracle's mask. Another acolyte took it up but had not the gift, and the Oracle of Ocrit Isle was no more renowned, faded into curiosity and mere fortune-telling, until it was forgotten. But Talzu, they say, journeyed across the seas and found his hero once again, found the city he ruled, and there they lived as many years of destiny and noble deeds, in eachother's arms, as mortals might dare to have.

But others have said not. Have said that is all lies of poets, a drop of honey at the end, to make the tale more palatable. They say Talzu remained at his duty, passed the rest of his days as Oracle, though from that day on when he took off the mask he went not to the market or the tavern, but to the high bluffs to watch the sun set over the western seas. And he slept no more in the hut behind the sanctuary. Though it were a longer walk, each night, he made his dwelling instead in a whitewashed cottage, about half an hour's walk from the harbor.

And still others say they both wander the earth to this day, seeking one another. For being reunited is the last task they must see complete, ere they depart this life together.

But I cannot tell you which of these, if any, is the truth.

I am no oracle.
