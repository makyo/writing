The grass rustled.

Bont flicked his ears, and his beady, gentle eyes blinked. Just the wind, blowing across the steppe. He kept walking, aiming for the village below the mountains. His steady three-toed plod had brought him little by little across the vast distance, and he was nearly at his destination.

``Hairy one!''

Bont turned to face the voice, snorting. A young hyena, all legs and teeth, erupted from the grass. Bont snorted again and lowered his head. The forward and greater of his two horns, as long as the hyena's body, sliced his view of the enemy in two. He blew out through his lips.

``Hairy one!'' The second attack came from his left flank, but as he wheeled to face it, a third and fourth hyena closed in on his right. The four formed a box around Bont, galloping in to snap at him whenever his ample rear was exposed.

He wheeled and stamped, stamped and wheeled. The hyenas danced around him, and although they backed off when he lowered his head, they were gradually wearing out his strength. Bont tightened his grip on the skin sack he carried. It would have held all four hyenas comfortably, if he could have gathered them into it.

The biggest hyena, growing bold, sprang on his back and clung to his wool. Its teeth gnashed at Bont's hump, but couldn't penetrate the fur and fat. Bont shook it off. It rolled on the ground and jumped back up, thrashing its tail.

Bont broke into a lumbering run. Hyenas bounced off him, scrabbling for purchase. When they grew too close to his front, he swept his horns low, tripping them. In his time he had gored lions with his primary horn; broken skulls with a blow from the shorter secondary. Yelps and hoots accompanied every swing he made. His eyes, made for seeing all around, caught movement everywhere he looked.

They were so fast, moving over the grass like clouds in the sky! The spots and shading of their coats let them hide in clumps of grass or flat against the earth, bamboozling Bont so he could no longer count their number.

The village was close. He aimed his horns at the collection of skin shelters and snorted.

He stumbled as the smallest hyena rushed between his legs, and fell on the trampled pathway with a crash that shook the ground. Dust filled his eyes and his nose. The hyenas, triumphant, bounced on his hump and chased each other up and down his back.

``You cubs stop that!''

The mother hyena spoke in grunts and gestures, but her meaning was clear. The four cubs scattered, but the largest was not quick enough, and she smacked it with the side of her muzzle. Bared teeth clonked against its skull. Bont, who reared a single calf each year with his wife and would have died before he hurt it, winced.

``Welcome, hairy one,'' the mother hyena smiled.

Bont stood, brushing earth from his knees, and followed her to the largest of the skin shelters. The hyena village was bigger than last year, and it had pushed further north as the steppe receded. It was hard for Bont to see pictures in his mind, but as he looked at the sheets of sunbleached skin, flapping stiffly against branch props, he thought of the deer people, their soft eyes, and the graceful dances they performed.

He lifted the entrance flap and ducked to enter. He sat, and was brought water in a skin bag. The other hyenas entered one by one: the toolmaker, the hunters, and the ones who looked after the cubs. Bont thought there were more hunters, more cubs than before, but he soon reached the limit of his counting.

``I would like some of the small things that hold small things, and a new grind stick,'' Bont said when he was refreshed.

``Skin pouches and a pestle, got it, got it.'' The mother hyena nodded enthusiastically.

Since he started his trading visits as a young woolly rhino, the hyenas had changed leader more times than Bont had toes and horns. Back then, he had understood their speech as well as they understood his. Now, they used noises when they spoke to each other that Bont could not follow. The old, simple words and gestures were for cubs, and for him.

``And what do you have for us?'' asked the mother hyena. The cubs had come sneaking in along with the adults. The four that were hers, the ones who had hunted Bont, lolled panting around her.

Bont opened his skin bag. It was hyena-made, and it held many things, more than he could carry even though his hands were large. Inside, smaller pouches held herbs and mosses so they wouldn't get all mixed up.

The hyenas were so smart! They fitted flint to wooden shafts, like teeth that could bite their prey from afar, or chew the branches from trees. They made containers so they could carry and store their food and water. They made shelters for sleep, because their hides weren't thick and woolly like his. But they couldn't seek out plants the way Bont could, and they didn't have his nose for separating the ones that harmed from those that healed.

He showed the contents of his pack, spreading the dried plants out or holding the pouch of seeds up to be sniffed.

``For stomachs,'' he told them by pointing to his own, ``for heads, for bones. This one stops bleeding. This one stops wounds going bad. This one stops pain. And \emph{this} one---`` His hand hesitated over the small, dark leaves. ``This one\ldots stops. When there is nothing more you can do.''

They ate, when the bargaining was done. Bont was brought fruit in a stone bowl. The hyena mother conveyed that it was the four cubs who had gathered them for Bont's arrival, when the wind brought his scent to the village. Bont rumbled his approval and patted heads.

The toolmaker brought him what he wanted, and a new thing, too: like a spear, but with a bigger point and short shaft. The toolmaker showed that it was for digging, instead of Bont's horn. Now he could pull up delicate plants without crushing them.

Bont was fond of the toolmaker, who spoke little and watched everything. Like Bont, he looked for things and he found them, but unlike plants, the things the toolmaker sought did not exist until he found them with his mind and made them real.

The smallest hyena cub crept close to Bont and snuggled up against his side. His fingers worked the soft fur of its neck ruff. The cub stretched, splaying its overlarge paws, and laid its chin in Bont's hand.

When stomachs were full and eyes were closing, the mother leaned close to Bont.

``And the other plant?'' she asked. ``The danger one?''

Gently, so as not to disturb the cub at his side, Bont reached for his bag and brought out the greyish moss with its strong, bitter smell.

``Not too much,'' he cautioned, breaking off a tiny piece that would suit a hyena, with its light body and quick heartbeat.

``Or die,'' the hyena mother confirmed.

``No. Too much and come back without, without\ldots'' he tapped his head, where his mind lived.

She took the supply he gave, and left to conceal it in some secret corner.

The hyena cub slept, its head in Bont's palm. Its ears flicked and its closed eyes crinkled as it dreamed. So many thoughts in a little soft head that Bont could break like an egg!

It was warmer here than in Bont's high, far home; warmer than it had been when he first began visiting. Too warm to lie under skins, surrounded by fur and hot breath. He slid his hand out from under the sleeping cub and moved carefully between the sprawled bodies, back through the skin flap into the breeze.

He turned his head so the curved length of his horns caught the setting sun. The forward, longer than his skull, was chipped and worn from a lifetime of uprooting trees, fighting lions, and keeping other men away from his wife.

The horns were heavy, and his wool was heavy too. He lay on the ground and closed his eyes against the orange sun.

\secdiv

\noindent He woke to the sound of many paws running, and to yips and yelps of concern. Every hyena was in motion---in and out of the shelters, sniffing and calling. Trying to follow them all with his eyes made Bont dizzy.

When the mother hyena loped by, her eyes wide with worry and her tongue hanging out, Bont stopped her with an outstretched arm and pressed her haunches to the ground. He made her drink from his water bag and he asked what was wrong.

She conveyed that her cub---yes, the smallest, no, not the biggest or the middle-sized ones---was gone. Lost. No scent to follow. The hairy one will help look, yes? He is big and he is good at finding things. The smallest cub likes him.

Bont's eyes and nose were not as good as the hyenas', nor could his legs cover distances quickly. He could travel further than they could, yes, but too slowly to be of use to a lost cub. How long had it been missing? Hyena bodies, warm and small, cooled more quickly than Bont's large frame. And the cub was smaller than small. He wiggled his fingers as if he could still feel the flowerhead weight in his hand.

Bont was not good at seeing pictures in his mind. But he was good with plants.

``I will help,'' he promised.

He found a quiet spot in the shade, behind the largest shelter. The calls and footfalls of the hunting hyenas were fainter here. He opened his pack and took out the danger moss.

He had warned the mother hyena about it: how it could take you on a journey and return you changed, a shell with the nut gone. Bont had used it as many times as he had fingers and he had always come back, but he could feel that a little of himself was taken away each time, as the pestle that grinds the herbs grinds itself away too.

He broke off a section of the dried moss. Crumbs fell from it to the ground, and he carefully collected them on a damp finger so no cub could lick them up. He placed the moss in his mouth and chewed.

When it was reduced to a wet, dense clump, he tucked it with his tongue between his bottom lip and his teeth. His mouth tingled from the chewing. He settled into a comfortable sit and felt the numbness spread from his lower jaw to his neck and spine. His arms and legs got heavy, and the weight of his horns forced his head down on to his chest.

First he was falling. The ground was above him and he fell into the sky, and it was more frightening than falling the other way because there was nothing there to end the fall.

The world flipped and he was looking down from above. Had a bird of prey taken the cub? Was that the message of the moss? He saw the hyena village below him, and his self, still and calm in the middle of all the activity. With a tug to his stomach, he was lifted higher. Now he could see the whole steppe. There were his wife and this year's calf, tiny as insects but clear in every detail. He reached out to touch them, but they disappeared into the grass.

The steppe shrank. The large people, like Bont's kind and the mammoth, retreated with the grasses, huddling together, dropping in number until none remained. Meanwhile, the hyena village grew, pushing up into the former cold places, and the dancing deer people ran from the hyenas. Soon the hyenas were so many, the deer so few, that the hunters became fighters who killed hyenas from other villages and took their food.

Now the tents in the village lay empty. They fell and disappeared. The hyenas were gone like the mammoth and woolly rhino.

Too far. I need near. I need \emph{now}. He thrashed his limbs, trying to swim back to the hyena village and his body. Instead, the sky over the vanished steppe went from blue to purple to black. Bont was closed in by walls. He stretched his arms and touched rock. At his feet, a small whimper told him he had found the hyena cub.

``I am coming for you,'' he told it. He tried to pet its ears, but his hand passed through. The cub shivered and tucked itself up even smaller. Its nose was to a crack in the rock, where air and a little light came in.

The cub was alive, but where was it? He pressed himself to the cave wall, trying to push through. Something caught at his throat and he knew he had swallowed some of the moss.

There was light, so much light. He was outside the cave. With the moss fizzing inside him, there were more colours in the world than he could normally see. Everything was bright and clear, as if the noon sun shone on it.

Where was this place? He didn't recognise it. And how had the little one got in? He looked hard at the rocks, thinking. There was no sign that a cave was there.

Look up, the moss told him.

He saw a fresh pale scar on the side of the mountain, where no plants grew. The moss in his mind told him that part of the mountain had fallen to block the cave entrance and trap the littlest cub.

Who knew that rocks could act that way? The hyenas could picture many things, like skin shelters and stone bowls, but none of them had pictured this. And now he had to get that picture from his mind into theirs. He stared at the mountains. He must hold on to the shape of them, to the landscape around, so the hyenas knew where to go. This was their land, not his steppe.

Bont felt that he had left himself a long way behind. He struggled as if something sticky was holding him down. He sat behind his own eyes, unable to move.

Too deep.

Too far.

Too\ldots{}

High, faint yips found his ears, and he felt himself being nipped and nuzzled. He opened his eyes and saw a blur, which shaped itself into the mother hyena.

``Hairy one!'' she said, nudging him anxiously with her nose. She spoke with movements of her head and paws, and with small noises: she'd thought he was dead! Did he take the danger moss? What about her cub? The scent of her, behind her words, told Bond she feared the cub was dead also.

Alive. Trapped. Let me tell you the place and you can find it. He used his arms to make the shapes he had seen. The picture was already fading from his mind as the moss left him, but the mother hyena was nodding.

``Yes. Yes! We know it! We will go!''

\secdiv

\noindent The hunters came with them, although Bont tried to explain there was nothing for them to hunt. The toolmaker, too. And the other cubs could not be kept away. Before the journey was halfway done, Bont was carrying two in his arms. How had the smallest come such a way by itself? It must have walked all night.

There was the mountain with the fresh scar, and there was the cave, and the rock that blocked the entrance. The mother called out, and put her head to the rock. From her face, the smallest cub had called back, but it was too faint for Bont's little ears.

The hunters prowled and sniffed, but could find no other way to get inside. It must be the rock. They put their paws on it. It did not move. They looked at Bont.

Hyenas could not do everything, and that made Bont feel happy. He looked at the rock. It was bigger than he was, and when he put his arms around it, he could not shift it either. It didn't even wobble.

He dug the tip of his forward horn into the earth, as if the rock was a plant he could uproot. The strain as he pushed made his jaws go numb and rigid. This way would not work either.

Bont walked backwards from the rock. He lined it up so his forward horn divided it into two equal parts in his view. He lowered his head.

The mountain echo turned the noise of his feet into thunder as he charged. The rock loomed up, and he closed his eyes.

He struck where he had aimed, the base of his forward horn smacking into the rock. The shock shuddered through his head and neck, and his hump trembled as it absorbed the blow.

Bont stepped back, panting. The rock had not moved. He must start from further away, run faster. He scraped the ground with a foot and paced away, counting one step for each of his toes.

The second strike sent pain shooting through his head, and little bright lights. The rock had not moved. He shook himself and tried again.

The third time, he struck wrong, and the rock tricked him so he fell. His lip was cut and bleeding. The rock had not moved.

It was harder to raise himself after he fell again. The rock had not moved. He staggered as he ran. His ears were full of bird noises and the rock wavered into two rocks. When he struck it and fell once more, the hunters leaped upon him to stop him trying again.

They poured water from a skin bag over his face, and squirted it into his mouth. The mother licked the cut on his lip with her tongue and it stopped hurting. But Bont's mind was full of the cub in the cave. Had it been scared by the banging and crashing? Or did it know Bont was trying to rescue it?

The hunters crowded the rock, trying to move it between them. But there was no room for them to push all at once in the same direction. The toolmaker barked at them to stop. When they paid no attention, he came and lay by Bont.

He had the same eyes as the smallest cub, Bont noticed, brown and gentle, and brimming with thoughts. He gazed at the rock, then back to Bont, looking up and down the curved length of the rhino's forward horn. His forehead wrinkled with the strain of all the thoughts behind it.

The mother hyena put her paws on the rock wall of the cave, as if she could touch her cub through it. Bont watched her, his chin on the ground to relieve the weight of his horns and his aching head.

The toolmaker's paw hovered above Bont's forward horn. The gentle brown eyes asked permission, and Bont nodded. He had no feeling there, so he could not sense the rough pads as they traced the dip and rise of the horn. With gestures, the toolmaker asked Bont to root up a tuft of grass from the soil in front of them.

Bont couldn't see why, but he obliged. Why not amuse the toolmaker? He was useless for anything else. He dug the tip of his horn into the earth and pulled the grass up from it with a tearing sound.

The toolmaker seemed delighted. He jumped up and began searching for something.

While his back was turned, Bont replaced the grass and covered the roots over.

The toolmaker returned with a smooth stone that filled both his paws. He placed it under Bont's forward horn, in the middle where the curve was lowest. He pointed to another tuft of grass.

The job was easier with the stone to rock his horn up and down. Bont understood, and he wanted to try the big rock immediately, but the toolmaker made him wait. He tried the stone closer to Bont's head, then closer to the rock, until he found the place where the balance was best.

Then, with the help of the hunters, the toolmaker found a bigger stone. They rolled it in front of Bont. He rested his forward horn on it and dug the tip under the rock that blocked the cave.

So many rocks and stones and people and places to keep track of! But all Bont needed to do was lift and push. The toolmaker had given him what he needed.

He rolled his shoulders, sending strength from his legs and weight from his hump into his head and horn. Resting on the stone instead of the ground gave his horn more room to swing upward, and the weight of the rock felt less.

It was still a great weight, though. Bont pushed until lights floated in his eyes again, bracing his hands and feet against the hard earth. When his feet slipped, the toolmaker scrabbled little holds for them in the ground. He pushed as if he could see the cub on the other side, as if it was his own small family trapped in the dark.

There was a dull crack, and Bont's head jerked. He fell on his side, and in the flashes in his eyes he saw his wife and calf. Something was different and wrong, so different and wrong that he could not tell what it was. He blinked away dust and saw the smallest cub scamper from a black gap in the mountain where the rock had moved.

He was not hurt? In spite of the shock and the noise, his limbs were straight, and he could turn his head.

His head. The white line of his forward horn, like a tree with no branches dividing his world, was gone from his sight. He saw it lying on the ground, a dead thing bloody at the base where it had snapped away from his head.

Bont had never seen the toolmaker so agitated. The hyena indicated with pats and waving paws that perhaps they could stick his horn back on? Or make Bont a spear, such a good spear, longer than his horn and straight instead of curved?

``No,'' Bont said. He lifted his head. It felt light, and he raised it higher. He stood straight. ``I don't need. I will last as long as I last.''

The other cubs were already playing with his horn, jumping over it and chewing it. The toolmaker moved to stop them, but Bont reached them first and picked up this piece of himself, the familiar become strange.

He handled it for a few moments, running his fingers along the chips and scratches. Then he held it out to the cubs. They took it in their mouths, fought over it and ran about with it, their tails high.

The mother hyena's voice was full of love and fear and anger. As she held and licked the smallest cub, she was asking it why. What had made it run from the village and cover such a huge distance? The squeaked and snuffled response was not one Bont could understand.

``She says she wants to be a finder and travel with things to trade,'' the mother told him, her voice gruff with a break in it. ``Like you.''
