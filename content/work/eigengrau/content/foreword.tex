\chapter{Foreword}

What is the color of a moonless night? Of closed eyes in a darkened room? Of the edge of sleep?

One might be tempted to say ``black'', but this is not so. In those lightless places on the edge of vision, strange biophysical effects take hold. After all, the mind and the eye would rather reserve purest black for contrast --- that black is ``that which is the darkest thing I can see''.

But what do they do when everything is the same dim hue, when even those hue-sensitive cones have too little light to work with? In such a liminal state, even the occasional staticky hiccups of the rods' excitement look like color to an input-starved brain; in the absence of information, it over-interprets noise. This color is called Eigengrau, which we might calque as ``own gray'', or perhaps more poetically as ``intrinsic gray'' – that liminal shade that the mind turns to when all other color is lost to it.

This collection, then, is named after a liminal color, unknown to pigment and light; in some sense an impossible color. Such has been the pleasure of my experience these past five years with Madison. From the very earliest days that I have known her, I have felt something slightly slippery, something intense in some difficult-to-express way, about her presentation of herself. As I have taught her of Eigengrau, she has taught me of liminality, and I cannot help but feel that this wondrous exchange is a microcosm of our friendship. Who and what I would be, not having known her, beggars my imagination; certainly I would be impoverished by the lack.

And just as she has left her beneficial marks on me, so have many others left marks on her; scattered throughout these pages are crystalline memories in ink. They are fingerprints left in clay by acquaintances and lovers and absent friends and ideas and moods, little slices through reality preserved in glass. They are a quiet glimpse into carefully curated parts of her soul; should rescue simulations ever come to pass, I have no doubt that this book will form a crucial part of the reconstruction for hers, and thus perhaps for mine, for scattered among those glimmering memories I recognize fragments of myself.

Close, then, your eyes, and look upon the color you carry always with you, that lightless, liminal color barely known to art or to science. Gaze upon Eigengrau. Then open your eyes and gaze with delight upon \emph{Eigengrau}.

\vspace{1.5ex}
\noindent --- Lorxus

\newpage
\thispagestyle{empty}
\null
\vfill
\begin{center}
    \emph{For Margaras}
\end{center}
\vfill
