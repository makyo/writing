\chapter*{Fables \& Follies}

Fables are the first section of this collection, mainly because fables, such as in the vein of Aesop, were the literary starting point for the poet. Four of the six poems included here in this section will be doubtlessly recalled by any aficionados of that teller of tales, and are my original verse adaptations of his fables. The opening poem, Little Vixen Red, and the final fable, Agricola \& Avarice, are the only narrative poems here of any true originality from the poet, but it is hoped that doesn’t detract from reader enjoyment. Much indebtedness is given to Aesop, and his various successors throughout history, for helping this poet get to his true beginning.

\vspace{1em}

\textasciitilde\ Michel-Vincent Corbeaux

\newpage
\section*{Little Vixen Red}
\vspace{-5pt}

\begin{verse}
Clever, cunning, slyly sneaking; \\
Clever little vixen red. \\
Fix your gaze upon her peeking; \\
Whoops! The little vixen’s fled! \\
Where’s she going? No-one knowing \\
That the little vixen red. \\
Far from prying eyes, is spying, \\
Laughing as her games now spread. \\
Clever, cunning, slyly sneaking; \\
Clever little vixen red. \\
Hunters in confusion speaking, \\
Wonder where they have been led. \\
Where’s she going? No-one knowing \\
That the little vixen red. \\
Far from prying eyes, is spying, \\
Laughing as her games now spread.

Clever, cunning, slyly sneaking; \\
Clever little vixen red. \\
Hunters in their frantic seeking \\
Miss the paths where she has tread. \\
``Where’s she going?'' No-one knowing \\
That the little vixen red. \\
Far from prying eyes, is spying, \\
Laughing as her games now spread.

Clever, cunning, slyly hiding; \\
Clever little vixen red, \\
With the hunter’s will subsiding, \\
Pounces on his weary head! \\
After playful trick, she dashes \\
Through the Autumn leaves all dead, \\
Far into the woods she crashes, \\
Laughing as her games are spread.
\end{verse}

\newpage
\section*{The Lion \& The Boar}

\begin{verse}
Two beasts arrive to drink at summer's height, \\
The boar and lion find a flowing stream, \\
But then they clash with brutal words and fight \\
To be the first to drink and reign supreme.


But vultures perch within a nearby tree \\
To watch them fight and wait for one to die; \\
The boar and lion turn around to see \\
These birds above that circle in the sky.


The morbid thought of vultures at a feast \\
Upon their flesh defused the tension there, \\
So peace returned unto these once mad beasts \\
And soon the birds took flight to other air.


Recall in hostile times the sage instruction: \\
Resist the urge to rage and sure destruction.
\end{verse}

\newpage
\section*{The Peacock \& The Crane}

\begin{verse}
The peacock spied a crane of duller shade; \\
He deemed her less among the creatures made. \\
‘Twas then he flared and flashed his plumage fine. \\
``Your plumes may catch the eye far more than mine,'' \\
The crane replied. ``But when it comes to flight, \\
``My wings allow for me to soar in height. \\
Although I may be dull compared to you, \\
My feathers serve a greater purpose too.''
\end{verse}

