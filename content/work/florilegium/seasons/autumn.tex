\section*{Autumn}

Autumn bears a strange dichotomy of plenty and impending naught. In Autumn, we harvest. We think of squash and gourds. We think of wheat, rye, corn, those fields all tan and gray. Those rattle-dry stalks we met in spring are born here.

The grain is in the silo. The gourds and potatoes are in the cellar. The fruit has been canned, the hay mown and baled, and we have never seen so much food, it seems.

And yet now is the time we consider empty stomachs. There is a particular Autumnal anxiety\footnote{Or perhaps a fear. Halloween lies there, doesn't it? There is a terror to your work, something existential, but you were also a fan of horror. You always asked for `Halloween music'. Your story was going to be the one that started that other fiction podcast we were planning on, where bummers were welcome to complete the dichotomy\footnotemark~with The Voice of Dog where there were none.\par
I don't know why I associate you so heavily with both terror and horror. You were a delight to be around, and your work is not \emph{all} terror or horror. I wouldn't call your personality dark, or at least no darker than fallen leaves-- but I am getting ahead of myself.}
\footnotetext{``I had read the sign,'' I wrote for one of my only attempts at horror/terror \parencite{plu}. ``And had immediately fallen down into the space defined by that dichotomy, the gap between had-to-be and could-not-be. Dichotomy? Dialectic? There was no telling anymore, no matter how many times I'd tried to paste one word or the other onto the two phrases. Were `dichotomy' and `dialectic' a dichotomy or dialectic?''\par
Clearly, I'm still shaky on the difference, despite those seven weeks in DBT (the D stands for `dialectical', after all), but at least I recognize it; I can just dwell in that space between two truths. Best I can do when I'm about to write however many hundreds of words on dialectics/dichotomies.} that lays bare future hunger and says, ``See? It doesn't matter how much you have stored away. This is Winter.''

It's easy to lean on one or the other. Keats, for example, is impressively himself about the whole season:

\begin{verse}
\emph{To Autumn}

Season of mists and mellow fruitfulness, \\
\vin Close bosom-friend of the maturing sun; \\
Conspiring with him how to load and bless \\
\vin With fruit the vines that round the thatch-eves run; \\
To bend with apples the moss'd cottage-trees, \\
\vin And fill all fruit with ripeness to the core; \\
\vin \vin To swell the gourd, and plump the hazel shells \\
\vin With a sweet kernel; to set budding more, \\
And still more, later flowers for the bees, \\
Until they think warm days will never cease, \\
\vin \vin For Summer has o'er-brimm'd their clammy cells.\footnote{I know that this line has little to do with cells in the biological sense, but how poetic a description of cancer!\footnotemark~Cells living in eternal summer, growing and growing, over-brimming in unchecked autolysis.}
\footnotetext{\Warn~They said it was just a lipoma, and then they stopped looking. Even though we told them she'd had a lipoma removed from atop her head back when we adopted her, back when she was a puppy, they stopped looking. They stopped looking! They said she was too fat, said as they peered over their imagined glasses at us, as though it were our fault that she was no longer so svelte, and then they sent us home. They sent us home! They said it was a benign lump and that German Shepherds just get those sometimes, that she was just too fat because they can be such couch potatoes, and then they stopped talking to us because they were too busy, too busy, too busy. A year later, she had slowed down to the point where she refused to go outside. She began spending all day, all night in the bathroom. That last day, her gums turned white and her belly was visibly swollen. That last night, she died\footnotemark~in my arms.}
\footnotetext{\Warn~I know that I'm trying to square what I have of you with your death, but when Falcon died in my arms less than six months later, then I really, \emph{truly} knew what death looked like, and now I have to square that with your passing as well. Did you, too, cry? Did you, too, try to hide? When you breathed your last, did you slump over to the side and stay warm far longer than one might expect? There was no one there to chide us and send us home that I can blame; there's no cancer, if that ephemeral mention from your girlfriend is to be believed, that lurked beneath the surface. You were and then you were not, and the only referent I have is a dog who died too young. I'm ashamed that I can't help but make the comparison.}

\parencite[249]{keats}
\end{verse}

While Stevens is much more austere about the whole season:

\begin{verse}
III \\
The blackbird whirled in the autumn winds. \\
It was a small part of the pantomime.

\parencite{blackbird}
\end{verse}

In Keats's work, we see the lush language that we expect out of a romantic poetry. Even in a free meter, there is a sharp focus on technique that one expects from Keats in particular, with well-balanced assonance of both nasals (/{\DisplayFont m}/, /{\DisplayFont n}/) and sibilants (/{\DisplayFont s}/, /{\DisplayFont z}/, /{\DisplayFont ʃ}/) leading to a sense of fullness, or perhaps the final warm breeze of the year.

The winds in Stevens's verse are not warm, though. With the aforementioned austerity, we are given one of the first cold winds of the year, and we see that the trees have lost their leaves already, miming against the sky as they are.

While I hesitate to say that Dwale walks a middle path here, its work does feature elements of both plenty and paucity. By establishing these two poles, we can then begin to triangulate where the poet believes Autumn lies.\footnote{This, after all, is what I'm trying to do, I think. I can't ask you where Autumn lies. I can't ask you if you feel the same way about the onrushing cold that I do, about saying farewell to the heat of Summer. I can't ask you if your moods are still defined by the school year, as mine are, these many years gone, with stress peaking around what used to be the end of term and depression creeping in around that first week of school. I can't ask you many things. I can't ask you anything.}

\begin{verse}
\emph{Face down in the leaves}

We crawl through moist humus like millipedes, \\
Feasting on dirt and dead, crumbling leaves \\
While striped skies cycle through violet hues, \\
While time's kisses take the shape of a bruise. \\
Endeavors wear the warmer years away, \\
Reduced at last to heaven's dormant clay. \\
Alive, I lick brambles until my tongue \\
Tears, despairing ever being so young.

I think of you.\footnote{By your absence, I feel your presence, and yet I continue to try and gaslight myself into believing that you never existed. Are you gone? You must be. Were you ever there, though? Were you a real person?\footnotemark~Were you someone so grounding that I felt childish before you? Were you someone I had the chance to meet back in 2015, where I stared longingly at your kosovorotka in gold-trimmed black, wishing I was brave enough to wear something like that? We'll never know, I suppose. One more thing I'll never be able to ask you.}\footnotetext{\begin{quote}
There was no more Codrin in the L\textsubscript{5} System. Ey was only here. Ey couldn't remember being there, for were the sims not the same? And if ey had never been there, had ey ever really existed there? Ey was only memories, and perhaps that is all ey had ever been. Navel gazing and existential crises mixed with the glee of having actually *done* something. No longer just the passive amanuensis, but now the active participant.

\parencite[51]{toledot}
\end{quote}

Clearly a perennial fear.}~I don't smile when I do.\footnote{Maybe I will, some day. I'd sure like to think so.}

A moment more and then the day is gone, \\
In evening grey, we mourn the vanished dawn, \\
And so on, maybe waiting for someone \\
To come drag us back to where we belong.\footnote{After all, ``Would God that I had died for thee'' (2 Samuel 18:33, KJV) is a sentiment at least 2,400 years old.} \\
In dreams we interred, with your pure throat bare, \\
I know your breath, your jasmine-scented air. \\
Alive, a god to mites and mud-daubers. \\
The harvestmen scuttle and bob onwards.

\parencite[9]{leaves}
\end{verse}

For Autumn, we are greeted by the vision of plenty and naught in the form of fallen leaves. The bare trees speak to a lack, and so the leaves on the ground bear testament to this. And yet the leaves themselves are someone's plenty, are they not? The millipedes, the mites and mud-daubers, the harvestmen all have a place to live, have food for the season, even if we have already collected ours. Everything is always food for something.\footnote{Even if that something is time.} The leaves are food for the insects, and they leave behind the humus, which will be a slow food for things too small to see.

And we, perhaps, are food for that ground.\footnote{Were you buried, Dwale? I realize that I don't actually know. When Idun passed on news of your passing, she also asked what observances should be made for a Muslim who has passed. I know that expressing one's wishes for when one dies is not always something does with one's partner --- hell, I don't know that any of my partners and I have talked about it, though it \emph{is} in my will --- but it does make me wonder: were those customs upheld?\footnotemark~I realized, also, that I don't know how much of your identity was known by your family. I have to interpret your life only to the extent that I can interpret your poetry: I haven't the ear, I have only the words, and you are not around to ask.}
\footnotetext{Every time I take the long way home from the store because traffic sucks or highway 2 is too much, I think about stopping by the mosque that I pass and asking about this. It's always also couched in that selfish desire to also ask after a framework for dealing with grief.

When I was talking about lack of framework in the context of this essay, a friend sent me a link to a tweet wherein the poster states ``An american \emph{(sic)} is told a thousand different ways that experiencing grief is abnormal, improper, and something to be done in private on your own time.'' \parencite{grief1} This is stated in contrast to the Jewish practice of sitting shiva and the following sheloshim which provides a structured procedure for engaging with grief. Another user replied that this might just be a white, middle-class American thing: ``White Anglo Saxon Protestant based communities may lack rituals for mourning. I don't know that world. But everyone from Black Americans to Latinx to AAPI to ethnic white  communities (Polish, Italian, Ukrainian etc) have ways to mourn that aren't exactly hidden.'' \parencite{grief2}.

So here am I, bathed in white cultural protestantism and puritan work ethics, having nothing to hang my grief on but a desire for resolution, for even a hint at a framework. Five years after Margaras's death, when I was still trying to process what life without him would actually be like, I wrote:\begin{verse}
  \textit{Yit'gadal v'yit'kadash sh'mei raba}\\
  Would that I had the faith\\
  To pray daily.\\
  Eleven months to let you go,\\
  And an amen to end the sorrow.\par
  \parencite{uvaip}
\end{verse}
I still wish for that. I wished it then when I was trying to figure out why I was less of a person even five years on, and I wish it now that I have to mourn both Dwale and Falcon at the same time. I have nothing to lean on but confusion and words.} This idea that we, too, might be a feast of plenty to someone is not a new one --- `food for worms' is an idiom for a reason. It isn't for the world at large, and it isn't for poets. Even Dwale tackles this in the poem that will be used for Winter.\footnote{The me who is writing this from top to bottom is dreading this. I applied to grad school with the poem I plan on using, and have already bathed myself in it once, and to do so again feels exhausting before the fact.}

And yet there is another layer of lacking here: we lack the absent interlocutor. \emph{We} have buried \emph{our} dreams, here, those dreams where \emph{I} know the scent of \emph{you}. This, as before, features a turn from the external and impersonal to the internal and personal. Toward the end of the first verse, after language surrounding the world around us, we get not only an action that we take (and how delightful, that homonym in `tears'), but the feeling of despairing that comes with it.

Autumn is, it seems, a dialectic: two things can be true at the same time. Plenty and paucity. Alive and dead. Impersonal and personal. There is an eternity between each of those sets of truths, as though Autumn, more so than the rest of the seasons, holds on the longest. ``How hard the year dies: no frost yet,'' Graves writes in \emph{Intercession in Late October}. \parencite[23]{graves_intercession} ``Spare him a little longer, Crone / For his clean hands and love-submissive heart.''\footnote{Who knows how much of my skittishness around winter is a me thing or an us thing. Spare me a little longer.} 

Issa says,

\begin{verse}
\begin{multicols}{2}
\emph{Akatombo} \\
\emph{kare mo yubo ga} \\
\emph{suki ja yara}

\columnbreak

Red dragon-fly --- \\
He's the one that likes the evening, \\
Or so it seems.
\end{multicols}
\vspace{-1em}
\parencite[65]{issa}
\end{verse}

Despite being the in-between of Summer and Winter, something that seems as though it ought to be a smooth transition between hot and cold as Spring tried to be, Autumn steadfastly refuses to be anything other than its own entity. We are unsure\footnote{After all, I think our well was out into Autumn, or maybe it had \emph{just} recovered. We were borrowing water from the neighbors for the dogs --- Falcon, who was dying, and Zephyr, who probably knew. I had burnt out so hard at work I had to take a leave of absence, had to spend sixteen hours a week in therapy, and on going back to work realized I still hated everything. I'm unsure even now whether life would have been easier without that grief. There is now dialectic between you being alive, of course, but there is this dialectic within me being unsure of whether or not I've processed your death.\footnotemark~Sometimes I have, and sometimes I have to stop writing this essay for five days because looking at it makes me cry.}\footnotetext{\Warn~Ditto with Falcon. Sometimes I'm able to make it an entire day not thinking about her, and then I'll be laid low by an evening of flashbacks, the way she slumped to the side, just how long her body stayed warm\ldots} of whether or not we like Autumn; surely some seem to, but this duality makes it elusive. Rather than shy away from it and decide to let it sit or cleave to it and enjoy every minute, we always have a little bit of that space between ourselves and the season, a little bit of that eternity.

Issa says,

\begin{verse}
\begin{multicols}{2}
\emph{Akikaze yo} \\
\emph{hotoke ni chikaki} \\
\emph{toshi no hodo}

\columnbreak

O winds of autumn! \\
Nearer we draw to the Buddha \\
As the years advance
\end{multicols}
\vspace{-1em}
\parencite[11]{issa}
\end{verse}

We think of it. We don't smile when we do.
