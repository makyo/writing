\begin{center}
  \footnotesize \emph{Content note:} this document contains frank discussions of death and grief, including descriptions of the euthanasia of a pet (marked with \Warn~).\normalsize
\end{center}

\vskip1cm

\begin{verse}
\centering
  \emph{What means death or grief} \\
  \emph{In the face of endless time?} \\
  \emph{Slow-turning seasons.}
\end{verse}

A year spirals up.

A day, a week, a month, they all spiral, for any one Sunday is like the previous and the next shall be much the same, but the you who experiences the differing Sundays is different. It is a spiral, proceeding steadfastly onward. A day is a spiral, with each morning much the same as the one before and the one after. A month, following the cycle of the moon

But a year, in particular, spirals up. It carries embedded within it a certain combination of pattern, count, and duration that delineates our lives better than any other cyclical unit of time. Yes, a day is divided into night and day, and those liminal dusks and dawns, but there are \emph{so many of them}. There are so many days in a life, and there are so many in a year that to see the spiral within them does not come as easily.

Our years are delineated by the seasons, though, and the count of them is so few, and the duration long enough that we can run up against that first scent of snow\footnote{Scientists have described the `scent of snow' as the air being too cold for the olfactory system to register scents.} late in the autumn and immediately be kicked down one level of the spiral in our memories. What were we doing the last time we smelled that non-scent? What about the time before?\footnote{Seasons being a handy way to count the years, I am, at time of writing, more than `the time before' years gone from when I last smelled this back in Colorado, taking a walk to clear my head after yet another argument in the Writers' Guild chat, leaving it to the mods. There is perhaps something to be said about the inevitability of a spiral.}

Or perhaps one thinks across the spiral. One, stuck in Winter, thinks back to Summer --- ah, such warmth! --- and tries to remember what it was one was doing then. ``Only silhouettes show / in the billowing snow,'' Dwale writes \parencite[19]{leaves}. ``Remembering months, now / gone when new blooms would grow.''

The power of the cyclical nature of the year is of an importance that draws the heart onward, and that which moves the heart is fair game for poetry. The demarcations for this cycle are the two solstices and two the equinoxes. One finds oneself at the longest night of the year and knows that, from there onwards, it is downhill into summer.\footnote{I am not sold on this metaphor; both uphill and downhill bear positive and negative connotations, and it is difficult to say which to apply when. Ask a poet.} One finds oneself at the longest day of the year and before oneself lies cooler times.

Dwale (1979--2021; it/its) was a poet living in the Southern United States. It was moderator for and, for a term, president of the Furry Writers' Guild, and was known for facilitating the `coffeehouse chats', hour-long lectures surrounding various writing topics that took place twice a week. Its work is described as focusing on ``altered states of consciousness...poverty, addiction, subjectivity, and the transience of existence'' \parencite{dwale}, though to reduce its body of work to any or all of those provides an inexact picture of its writing. This will be touched on in a future section on translation, but needless to say, this paper will focus on its work through the lens of seasonal progression. 

The concept of seasons and seasonality is well trod within poetry. Exploring that is beyond the scope of this paper.
To rely on synecdoche is the best one can manage with a topic so large. To that end, it is worth exploring the poetry of Dwale in such a context.
