\section*{Summer}

As the year continues on its upward spiral, we come to one of those strange apogees of the longest day. Strange because yes, of course it bears meaning as the longest day, and yet the start of Summer never seems to fall directly on that day, does it? There is doubtless some good reason that, at least here, that is the first day of summer rather than midsummer.

And yet even that isn't always accurate, is it? Some years, summer doesn't feel like it has truly hit until well into July, when the temperatures climb and the rain becomes a distant memory.\footnote{And perhaps your well dries out when you head out of town for you husband's surgery, so your dog-sitters to have to figure out water, leaving you to fret and pace around the hotel room, and maybe that's the time you decide, ``You know what? Work is so terrible that I think I'll apply for grad school.'' But you have to provide a sample of analytic writing to do so, so you pick one of your friend's poems to analyze, and two weeks later --- when you've come home to no water and a dog whose health is steadily declining though you don't know it yet --- your friend is dead.} You're left feeling miserable for weeks on end, wishing for even a drizzle to quench your thirst, or even a bit of cloud cover at night, enough to maybe knock the temperature down into the low seventies so you can finally, \emph{finally} get some sleep and yet the days spiral forwards through heat-haze.

\begin{verse}
Summer, season of hot insomnia, \\
That much never seems to change at all. \\
Laying awake in the red desert night, \\
I shape forest from shade and wait for fall.

Ten years now gone,\footnote{It's 2022 as I write this, which means that, come September, it will have been ten years since Margaras died.\par His was the first death that really hit me. The first one I was really able to comprehend. Koray came into the bar, asked if this was the place he would have frequented, passed on the news, and then left.\par It was crushing. It destroyed me. I am still not entirely sure why, since we were friends, yes, but we were hardly so close as to warrant the reaction that I had, and yet I did.\footnotemark\par And yet I did and now, a decade later, I only think of him on the anniversary or when I come across the notifications I have from him and from Koray. Maybe that's why there's that worry about the box labeled `regrets'. I have my regrets for Margaras, and the amount by which those are outweighed by the good memories is too small for my liking.}\footnotetext{Not unlike Dwale, I suppose. Perhaps a good chunk of this --- of both of their deaths --- is due to just how little I interacted with them through anything other than text. I met Dwale once in person, and never met Margaras. I listened to Margaras's music and listened to audio versions of Dwale's stories, but other than that, they were relegated to words on a screen.} and who thought I would miss \\
Cricket songs, cicadas and katydids? \\
Then I'd gladly have grabbed a big hammer, \\
Smashed them flat as Pinocchio's conscience.

Testing palisades of clocks and yardsticks, \\
No advent waits for the restive dreamer. \\
I bandage my tattered, bitten left hand \\
And shed the smoke rings on my cloven finger.

\parencite[8]{leaves}
\end{verse}

The poem follows a similar structure to that chosen for Spring: three stanzas of four lines each, often falling back into a stressed-unstressed (or vice versa) meter, though far more free. We have a few more near rhymes, (`at all' and `for fall', and, to a lesser extent, `dreamer' and `finger'), plus a few pleasing instances of alliteration (`\emph{cri}-\emph{cket}\ldots{}ci-\emph{ca}-\textbf{das}\ldots{}\emph{ka}-ty-\textbf{dids}').

Also as before, there is a volta in the third verse. Whereas with Spring, we switched point of view from Winter to Spring, here, we switch away from from the concrete world and into something more abstract. Where we start with hot deserts, forest shade, katydids and hammers, now we are confronted with unknown tools of measurement, dreams, and smoke rings. We have that which defines itself in the external world and that which we define internally, and with those two poles, we are left to extrapolate what is between them.

Issa says,

\begin{verse}
\begin{multicols}{2}
\emph{Natsuyama ya} \\
\emph{Hitori kigen no} \\
\emph{Ominaeshi}

\columnbreak

On the hill of summer \\
Stands the slender maiden flower \\
In a solitary humor
\end{multicols}
\vspace{-1em}
\parencite[65]{issa}
\end{verse}

The slender maiden flower is the slender maiden flower. We have no say in its existence except that we might pick it, trample it, or leave it be. It is itself, in all its glory --- or at least all its solitary humor. The flower defines itself and though we may take action on it, may think it beautiful or ugly or lonely or austere, that doesn't matter to the flower.\footnote{For a while, I was quite caught on the idea that others have agency of their own. Of course they do, I mean, I just found it marvelous that this was the case. There was no way that they could not, right? They live and love and feel just as much as I do, so I can't say that this same applies to people; they define themselves, sure, but they can actively change how I create meaning from their existence.\footnotemark}\footnotetext{Of course, having written this, I feel bad for the flower. Perhaps it desperately wants to be seen as austere instead of lonely, as beautiful instead of ugly. Ask a botanist.}

``Summer, season of hot insomnia / That much never seems to change at all'' speaks well to this. Summer is Summer. It is the season of hot insomnia and it doesn't care how tired we are. It's not that it is inimical to us so much as existing within its own external nature. It exists in that floating world that is separate from us. It does not know us, it knows only itself. It's hyperreal, perhaps, only casting its shadow into our reality.

``Sleep, or don't.'' Summer yawns, lingers beneath the eaves and between still branches, bothers not with such as us.\pagebreak

Issa says,

\begin{verse}
\begin{multicols}{2}
\emph{Mi no ue no} \\
\emph{kane tomo shirade} \\
\emph{yusuzumi} 

\columnbreak

Heedless that the tolling bell \\
Marks our own closing day --- \\
We take this evening's cool
\end{multicols}
\vspace{-1em}
\parencite[39]{issa}
\end{verse}

This is the inverse, the other pole of our spectrum. Whether or not the bell tolls for us and our day, whether or not the evening's cool is of that floating world, we still can define ourselves and our actions in the face of it. We are the ones who can take that cool as some small respite from the hot insomnia that the Summer might otherwise offer. We can define ourselves in that context, and by that, we can define the world around us.

In this sense, the cool evening and the end of our day --- indeed, the season of hot insomnia that never changes --- is something over which we can layer an artificial definition. The semiosis in play allows us to turn Summer into a sign that we can interpret. Our artificial definitions apply to us, even if the heat of the day doesn't give a damn about us. ``Testing palisades of clocks and yardsticks, / No advent waits for the restive dreamer'' because we restive dreamers are only able to measure by our artificial definitions.

But that cannot be all. There has to be more than the external and natural, that which defines itself, and the internal and artificial, that which is defined by us. We smash the insects flat with a hammer, correct? We build air-conditioned bedrooms to be able to get our sleep, correct? What is in the middle is agency. It is the permission we give ourselves to form these definitions in cooperation with the world around us. We can cry out at the sight of blackbirds bursting from the trees, because that is a thing that we have the power to do, ourselves:

\begin{verse}
X \\
At the sight of blackbirds \\
Flying in a green light, \\
Even the bawds of euphony \\
Would cry out sharply.

\parencite{blackbird}
\end{verse}

It is the act of taking meaning from each other, as well, for each of us has our own agency: we can interact with each other and influence each other's definitions of ourselves.\footnote{Viz. me meeting Dwale in the writers' guild and deciding --- actively deciding --- that I would like to be its friend. It wasn't lacking, and neither was I, but something about someone who might choose `it/its' as pronouns, someone who could engage with poetry in a way that had always eluded me. Doubt nips at my heels, though. Is ``deciding to be someone's friend'' a normal thing to do? Was that weird? Did it resent me for-- but I shouldn't be thinking like this.}

As that golden bloom of Summer\footnote{Of dandelions:\begin{quote}``Of course. They are a weed, yes. Or often thought of as one. The leaves make a good salad, though, and I was told that you could dry, roast, and grind the roots to make a coffee substitute.''\par
\parencite[161]{toledot}\end{quote}\par
They are death in Summer, I've always felt. I was always supposed to kill them, and they were always the sign of a dead lawn. Still, I read all about them on realizing how good they smelled and grew my little obsession. I passed it on to the characters in my books, and let them feel out that connection to death so that I could do so from a distance.} defines itself as all things must, and we have to take it at its word. We can kvetch about the insomnia of Summer, that which makes us sweat through the sheets so that the thought of touching someone else makes one feel clammy and disgusting\footnote{Just me? No? Maybe just me.} all we want, but that doesn't mean anything to Summer. It just also doesn't stop us from layering our own definitions atop that.
