\section*{Spiral}

To return to Spring, to make it through that cycle of growth, of insomnia and harvest and frost, is to stand at a precipice. It is to stand right up against the edge of that spiral, lean over carefully, peer down into the depths from however many storeys up, and wonder. It is to confront memory in the form of heights. It is to regard the spiraling days, weeks, and months to either side of you, give them the acknowledgement they deserve, and then return to peering down into the depths.

To return to Spring is to hit that vernal equinox, look down, and feel the steam of memory, the heat of the last year, washing up over your face. What was that album again? ``Memories Come Rushing Up to Meet Me Now''?

We remember last Spring. We remember Autumn, because there it is across from us. We remember Winter and Summer through some haze --- Winter is still too fresh, Summer so long ago --- by peering through the haze of the days around us.

It is nearly a cliché to try and find the spirals in everything. A snail's shell! A nautilus! Find the Golden Spiral in every hidden cloister.

A cliché is a metaphor, though, and a metaphor is a framework upon which much can be built. We can look at the cliché that is the spiral and say to ourselves, ``Hmm, what else fits this pattern?''\footnote{Our relationships, perhaps. My relationship with Dwale spiraled, after all. One long spiral from when we met, however many years ago, and then that spiral made its full cycle when it died, and now I'm on to the next loop, able to look down on what time we had together.}

Fit, then, a poem --- a particular poem, yes, but also the idea of a poem --- to this framework. A poem is a spiral. Poetry is a spiral. Writing poetry. Reading poetry. Burying oneself in words too rich to taste --- it is all a spiral.

Elliot Weinberger, in his survey of translations centuries of translations of one of Wang Wei's poems, does an admirable job of this. Throughout the years, he views the ways in which translations move: the views of the poem, the views of the time the poem was written, the views of the place in which it was written. Orientalism stains so many of them, especially those so early on. Even those most contemporary run into certain levels of inexactitude that miss the ways in which the languages translate. Is it `shine' or `reflect'? Simply `sound', or something more complex such as an `echo'?

``{[E]}very reading of every poem, regardless of language, is an act of translation: translation into the reader's intellectual and emotional life. As no individual reader remains the same, each reading becomes a different --- not merely another --- reading,'' as he so succinctly puts it \parencite[46]{wangwei}. ``The same poem cannot be read twice {[\ldots]} the poem continues in a state of restless change.'' It is all very Heraclitus.

By virtue of the reader's ever-shifting state of mind, they constantly re-translate otherwise static text, even from minute to minute, and build up a library of meaning from a single work. Reading a poem is as much a form of self-definition as it is of entertainment.

The spiral of the year, of the month, of the day applies as well to the poem, the stanza, the line. We've spiraled our way from spring to spring. We've spiraled our way upwards, using on its seasonal poetry as synecdoche for seasonality in poetry as a whole, through each of those seasons, and so the only fitting end is to use one last poem of Dwale's as a synecdoche for poem-as-spiral. We can read a microcosm of the spiraling year into a single poem. Start at the beginning, and when you get to the end, start over because you're already a different person.\footnote{And that's not so bad, is it?}

And so, one more time before setting aside the topic to steep for another year, let us address one of Dwale's poems:

\begin{verse}
\emph{Poem for a Deceased Lover}

Seven days\footnote{The instances of seven and eleven in this work may call back to \emph{shiva}. As a Muslim, though, the periods would have been different for Dwale.} had passed when I heard you died, \\
A message in the warm morning hours. Dawn \\
Rose, and no one said how I should go on, \\
Or wade this mire without my only guide.

Flown to space by what callous earth destroyed, \\
I chase the long-flying radio waves. \\
Far away from grief and a potter's grave, \\
I sift to find again your breathing voice.

Teacher, my every thought was yours to thresh,\footnote{A counterpoint to Spring, perhaps; thresh in Autumn, harrow in Spring.} \\
So now what sure course would you recommend? \\
Your kind words turned to shrapnel in the end, \\
Pieces of you left here in my heart's flesh.

Lover, did you mean to leave this deep wound? \\
I would sell my world to kiss you farewell. \\
Eleven years facing perpetual Hell, \\
And all I can say is, ``Too soon, too soon.''

\parencite[14]{leaves}
\end{verse}

If we are to tackle this as Weinberger does (and as we have touched on before), a good place to begin would be the prosody and sonority. We are again confronted with lines that follow a unique meter of iambs/troches interrupted towards the end with a spondee: ``Seven days had passed when \emph{I heard} you died'' works out as three troches, a spondee, and then an iamb (we could call it a `third epitrite', apparently, or we could be realistic), and ``A message in the warm morning hours.\footnote{Again, accents may complicate this, as `hours' may be one or two syllables.} Dawn'' as three iambs, a troche, and a spondee. 

This type of analysis may at times act as a desiccant, drying out an otherwise lush poem, but it does serve its purpose in giving us a glimpse at just why a poem makes us feel the way we do. When taken with the familiar half- and almost-rhymes (this time in \emph{ABBA} format), we are once more faced with a stumbling feeling that, in this case, perhaps speaks to trying to make one's way through the day with tear-clouded eyes.

Upon returning to the top and reading the poem through, one is struck by a sense of distance contrasted with the particular intimacy that comes with a wound. `Seven days', `flown to space', `long-flying radio waves' all speak to the impossible gulf between life and death, while `pieces of you left here in my heart's flesh', 'this deep wound', and `kiss you farewell' describe a closeness that crosses boundaries, a breach of an integument.

The grief is shown in the freshness and immediacy of the words. `Wade this mire' feels impossible in so low a place. ``I sift to find again your breathing voice'' shows the urgency that follows loss, the hasty need to find what is no longer there.

And yet, even within the span of the poem, we see that urgency lessen. We hear uncontrollable, gasping sobs calm down into mere crying. We are not yet at sniffling, at the dull pressure in our head that follows actually crying, but we are at least able to speak, by the end, our sorrow. ``Too soon, too soon,'' we say, and it is no soft platitude,\footnote{Platitudes are for others. They are for those trying to convince each other that they are saddened by this change. They are mere performance.
\begin{verse}
{[\ldots]} \\
``Good man, good man,'' they mutter, \\
doing all they can to convince each other \\
through well-rehearsed performances, \\
that this must be the case. \\
The silently bereaved already sit graveside.\par
\parencite{penguins}
\end{verse}\par
But grief, true bereavement, is almost reflexive. It is \emph{performative} in that way. By grieving, we grieve. Add in the fact that I'm helpless before my compulsive explanation and beholden to my graphomania, and this was my grief over Dwale. I could not sit, silent, by the graveside. I could not sit \emph{shiva}. I could not bury myself in a community that is willing to support me, but what I could do is use the framework of words to pull meaning from that which feels too big to make sense. I \emph{do} have tools, even if it may not feel like it when grief burns particularly bright.} but our meager attempt to put into words what we are feeling when what we are feeling is still too hot.

Despite mentions of Hell,\footnote{And I sure hope that the torment of plagues and politics doesn't last eleven more years, much less for perpetuity.} it is comforting to see here that grief has transmuted into sadness. We have climbed that year-long spiral eleven times,\footnote{And while this may have been longer than Falcon lived, longer than she made our lives a joy, we got to make her entire life a good one.} we have had our period of lamentation, the soul has been purified, and we can see what it is to live life without them.\footnote{And it will live on at least as long as I do, will it not? I would that it had not died at all, but as it had to, at least I have the ability to think about it, love it from across that infinite gulf in my own, awkward way. I have the privilege of being able to memorialize it. I have my threnody, and through that, its works are set for those to see who might not otherwise.} Sure, we will always hunt their breathing voice, their kind words remain with us, we will never kiss them farewell, but it is now comprehensible. We can intellectualize their loss. We can pull it into words and set it before us. We can read our grief from top to bottom and then start once more at the top. We know it well, our sadness, and each time we take our trip\footnote{This is not a new idea, of course. In my choral conducting courses, we talked about taking `the seven trips through the score' in order to tease it apart so that we could put it back together with our students. Again, though, that Madison has passed.} through the text, we can feel its impact soften. It does not leave us, but it becomes  a part of us.

And now, when we spiral around once more to the top of the poem, we can look down over that perilous edge and see what we were. We can see the way we bury our face in a pillow we hug to our chest so that the gasping, choking sound of our sobs is muffled --- from whom? Perhaps even this version of us, here in the future --- however many levels down. We can look down to the level just below us and see how we're starting to come to terms with that loss. It was not a smooth transition, this integration of loss into ourselves, but now that we've once more reached the first line, we are no longer ``I, who grieves'',\footnote{Or perhaps ``I, who writes paeans to grief in the footnotes of an essay and worries that this is not doing the actual Work''. Just me? No? Maybe just me.} but perhaps ``I, who has grieved''. We can think about how our love is borne out of the solar system on those radio waves (for what else is WiFi?) and, even if we do not smile, we do not cry.

We can look up, too. We can look up and see all of the other times we \emph{will} read the poem and imagine who we might be. Might we be someone who can read through this poem and only \emph{remember} the us who was so torn by grief that they couldn't breathe for sobbing? A hazy memory, one where we remember that us as some different person.

And so we read the poem again and see something new --- aha! Is ``I sift to find again your breathing voice'' an anaphora? --- and it all becomes a little softer, a little more abstract. We read and read. We come back to our poem years later and it inspires nostalgia in us. Nostalgia! Simpler times for simpler versions of ourselves. A little younger, a little dumber, but no less capable of feeling.

Issa says,

\begin{verse}
\begin{multicols}{2}
\emph{Ro no hata wa} \\
\emph{Yobe no warai ga} \\
\emph{Itomagoi}

\columnbreak

Around the hearth --- \\
The smile that bids us welcome \\
Is also a farewell!
\end{multicols}
\vspace{-1em}
\parencite[101]{issa}
\end{verse}

A year passes and we look down through the haze of time, down along that spiral.

We read the poem again, re-translate it for ourselves, and spiral through the lines and verses.

A year spirals up, and so, too, does a poem.

% (Probably some gentle self-deprecation around writing a paean to grief in an attempt to get out of doing the actual work required to process it)

% (Positive outlooks: even if she only made part of our lives good, we made Falcon's whole life good, she had that last delicious meal; Dwale is honored and will live on in its writing and our memories, how lucky am I that I get the chance to be a part of that?)
