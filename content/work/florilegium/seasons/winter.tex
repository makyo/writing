\section*{Winter}

``Now Winter comes slowly, Pale, Meager, and Old,''\footnote{I wish you had died in Winter, Dwale. I wish you'd lived to comfort me through Falcon's death. Hell, I wish you'd lived to comfort me through your \emph{own}. I wish you'd lived to the winter of your life, not to a mere 42 years. The very beginning of Autumn, for you! You had your plenty and your paucity. I wish you'd made it to `Pale, Meager, and Old'.\par Of course I wish you hadn't died at all, but I wish you'd died in Winter if you had to.} Winter sings in \emph{The Fairy Queen} \parencite{purcell}.

Winter creeps. It eases into place. Even if there is a sudden, blustering storm to begin the season, that is but the first noise you hear. It creeps and crawls in because it cannot but creep and crawl. It's old. It's tired. ``Dying sun, shine warm a little longer!'' \parencite[206]{graves_poems} we may beg,\footnote{And perhaps do. I am not --- most of us are not --- immune to that simple desire that we have a little more time together. Another year, another month, another day. Even another hour together,\footnotemark~enough time for me to tell you that I think of you often, that you mean a lot to me, that I hope you understand it.\par 
The last time we talked one-on-one, you was a month before you died, June 5\textsuperscript{th}. You pinged me in the Guild chat asking me to DM you because you couldn't find my chat in your client: ``Could you DM me right quick?''. I sent you a surprised-looking sticker and you said, ``Uh, just wanted to say that although I don't know what I ever did to deserve having your support, I do see and appreciate it.''\par 
``You just strike me as an earnest and well-spoken. You do quite a bit to buoy others up, and that ought to be returned in kind,'' I said, but honestly, how the hell was I supposed to respond to that? That I think we both wind up in that spot where impostor syndrome becomes more dire? That any praise, any validation becomes almost too hot to touch?\par 
It's a haunting sort of message to be left with. That I should praise your work (for I think it was in response to my review of \emph{Face Down in the Leaves}) and inspire that terrifying ordeal of being seen has me confused and upset, but I don't know that there's much to be done about it. I wanted to type so much more than I did, but this wasn't the place. The conversation wasn't open after that. It had closed up, the point having been made --- inadequately by either of us, I suspect --- and then I never got to say anything about it again.}\footnotetext{\Warn~I never really got that with Falcon, either. She was sick that morning, and then we drove to the emergency vet, and she was inside for perhaps ten minutes before the vet came out and said she had perhaps six hours to live, and it was time to make a choice. Six hours of pain and a death in agony or one hour of pain, dampened by drugs to lessen the shock, and then a blissful sleep. A slump to the side, a last breath, eyes open, mouth open, warm there on the floor.\par I never got that with her. I never got it from you, only a tweet the day after you died from your partner, but it was somehow less real, less immediate, because of course it was. ``I do not know which to prefer,'' Stevens writes \parencite{blackbird}. ``The beauty of inflections / or the beauty of innuendoes, / the blackbird whistling / or just after.'' } but all it wants to do is lie down and blanket the land.

Much of the imagery in poetry around Winter picks up on this, and we commonly see instances of blankets, of beds, of rest.

Issa says,

\begin{verse}
\begin{multicols}{2}
\emph{Arigata ya} \\
\emph{fusama no yuki mo} \\
\emph{Jodo yori}

\columnbreak

A blessing indeed --- \\
This snow on the bed-quilt, \\
This, too, is from the pure land
\end{multicols}
\vspace{-1em}
\parencite[46]{issa}
\end{verse}

Perhaps it is because we so often experience Winter through the lens of contrast. We experience Winter through the warmth of fire. We experience Winter because it is \emph{out there} and we are \emph{in here} (or, failing that, we are experiencing Winter directly because we are \emph{out there} and would very much rather be \emph{in here}). We think of snow on the ground, we think of blankets because, yes, of course it looks that way, but also because we are primed to think of winter in terms of the contrasts to how cold such a blanket must be.

Or perhaps we think of Winter this way, of snow as a blanket, of sleepy silences, because the world really does seem to be asleep. It goes beyond mere hibernation; the whole world --- the Earth, the sky, the rivers and lakes --- all seem to be asleep. ``I wonder if the snow loves the trees and fields,'' Lewis Carroll writes \parencite{carroll}. ``That it kisses them so gently? And then it covers them up snug, you know, with a white quilt, and perhaps it says, ``Go to sleep, darlings, till the summer comes again.''''

Even the snow itself seems destined for sleep, drifting down in fat clumps or being blown nearly sideways, helpless, only to be piled up in immobile drifts.\footnote{I go back and forth on what death must feel like. There are times when I think it must be like this slow fade to sleep. It has to have that hypnagogic quality. You could try and stay awake, but gosh, it would be nice to get some rest.\begin{verse}
Her eye turns inward, \\
vision dims and movement stills \\
as winter claims her.\par
\parencite{pale_she}
\end{verse}\par And sometimes, I think it must be like a wave, rolling up to subsume you, and it's so, so much bigger than you are that there is nothing you could possibly do to stop it:\begin{verse}
A flash of coppery sweetness, \\
A clearing of the sinuses, \\
A burst of unnamed colors, \\
A rush of creativity, of wonder, \\
Velvety softness, a low hum, \\
And then the wave recedes. \par
\parencite{rush}
\end{verse}\par And I'm sure it has much to do with the way in which one dies. Perhaps there is terror. Perhaps there is relief.\footnotemark}\footnotetext{\Warn~I'm sure that Falcon felt a bit of both. She was in so much pain, and yet she was stuck at the vet. She was delirious with pain meds. She was surrounded by the worst smells, people she didn't know. She was in a cage until we got back.\par I have no idea what you must have felt.} It's destined for sleep because what else would a blanket do?

\begin{verse}
I. \\
The snow is falling, \\
sleeping, \\
whispering, \\
dreaming of water.

{[\ldots]}

III. \\
A single snowflake awakens, \\
shimmers, \\
glows, \\
watches the world with weary eyes, \\
darkens, \\
settles, \\
and disappears.

\parencite{esch}
\end{verse}

Wallace ties in this sleepiness\footnote{And, in sleep, there is that not-knowing, not-caring. There is something of death in sleep, and for one who years for such, that bears much allure.\begin{verse}
Pale she sleeps, sleeps still. \\
Waking her may have listened. \\
Endless winter calms. \par
{[\ldots]} \par
If spring never comes, \\
pale she supposes, that's fine. \\
Winter is for dreams. \par
She'll dream, or she won't. \\
She'll carry on or she won't. \\
Cold has claimed heartwood. \par
\parencite{pale_she}
\end{verse}\par But I shouldn't be talking this way.} with contrast\footnote{Contrast, then, with wakefulness. Contrast with life. ``Find all the happy pictures of Falcon that you can,'' my therapist suggested. ``Tell yourself stories about them.''\par It works some of the time.} --- as we shall do before long --- by contrasting the stillness of a world asleep with our faithful blackbird:

\begin{verse}
I \\
Among twenty snowy mountains, \\
The only moving thing \\
Was the eye of the blackbird.

\parencite{blackbird}
\end{verse}

Similarly, Graves has,

\begin{verse}
She, then, like snow in a dark night \\
Fell secretly. And the world waked \\
With dazzling of the drowsy eye \\
So that some muttered `Too much light,' \\
And drew the curtains close \\
Like snow, warmer than fingers feared \\
And to soil friendly; \\
Holding the histories of the night \\
In yet unmelted tracks

\parencite[143]{graves_poems}
\end{verse}

``As Earth stirs in her winter sleep,'' he writes elsewhere \parencite[173]{graves_poems}. ``And puts out grass and flowers / Despite the snow / Despite the falling snow.'' Winter has crept in and tucked the world away to sleep for a while, and though we might stretch and peek out and, seeing the sun, think to ourselves, ``I really must get up,'' we are helpless to actually do so. Make attempts, sure, but there is no waking from Winter on any terms other than Winter's.

It plays into the timelessness of that serenity.\footnote{Wishful thinking.} It is so quiet!\footnote{Yet more of the same.} The contrast is so high! Morning light is the same as noon light is the same as afternoon light. How could time pass? Winter will not permit it.

\begin{verse}
XIII \\
It was evening all afternoon. \\
It was snowing \\
And it was going to snow. \\
The blackbird sat \\
In the cedar-limbs.

\parencite{blackbird}
\end{verse}

And, of course, perhaps we think of Winter this way because that very danger that keeps us inside. Winter, death-season, can have snow as a funeral shroud as easily as a blanket. We are not \emph{in here} simply because it is cold \emph{out there}, but because that very cold brings death with it.\footnote{Guidelines for reporting on suicide state that you should not report on the method with which someone killed themself. This, apparently, does not apply to the living. Terry Gross asked Allie Brosh during an episode of ``Fresh Air'' how she imagined committing suicide and, rather than keeping that close to her heart, the author explained that she had planned on freezing herself to death through a mechanism I won't describe.\parencite{brosh}\par And, as the guidelines say, that has stuck with me. I think about it every time it gets cold.\footnotemark~I thought about it that night after Falcon passed. I thought about doing just as Brosh said, and finding a way to experience that very sort of blanket, that very shroud.}\footnotetext{Which, I realize, is the opposite of death-thoughts elsewhere in the year. Perhaps Autumn is the season for thinking of fire, Spring the season for leaping, and so on.}

Issa says,

\begin{verse}
\begin{multicols}{2}
\emph{Kore ga maa} \\
\emph{tsui no sumika ka} \\
\emph{yuki goshaku}

\columnbreak

Is this it, then, \\
My last resting place --- \\
Five feet of snow!
\end{multicols}
\vspace{-1em}
\parencite[37]{issa}
\end{verse}

The discursive nature of this section might itself be related to the blunted vision of the world after snow.\footnote{Or maybe just because I'm riddled with memories. They pock my surface, keep me from moving smoothly through analysis. Could I write about the season of Winter in some more cohesive manner had not the ground been covered with that shitty slush the day that Falcon passed?} How can we define the world around us when we can barely make out its edges? We cannot define Winter because it's so blurry around the edges. Obscured. Defined by contrast, but only the contrast of blackbirds or bare tree limbs, rather than one hill from the next, one house from the next. We can only pin it down by walking those paths, one by one, heading up to the top of the hill and looking down from there, walking up the drive to get a better look at the numbers tacked to the side of the house. There are so many perhapses and maybes\footnote{Better, I think, than if-onlys and if-I-had-justs. There's that wishful subjunctive, as always. Would God that I had died for thee. I don't really feel that for you, Dwale, and I don't know whether to feel sorry or grateful for that. I've felt it for others, as is perhaps obvious, and it is one of the worst feelings I've had in my life. I'm sorry that there is something about our friendship that precludes that, I guess, but I'm more grateful that I don't have that feeling associated with our memories.} to be had.

\begin{verse}

VI \\
Icicles filled the long window \\
With barbaric glass. \\
The shadow of the blackbird \\
Crossed it, to and fro. \\
The mood \\
Traced in the shadow \\
An indecipherable cause.

\parencite{blackbird}
\end{verse}

`Indecipherable' indeed.

And so, with an eye warmth and cold, to contrasts, to blankets and sleep, to softness and inexactitude, to death, we come to our final poem:

\begin{verse}
\emph{Dirt Garden}

My garden of foxtails and milk-thistle, \\
Alive and wild, more so than tended rows \\
In growth, has died. I killed them a little, \\
The crab-grass clumps, Datura and nettle. \\
``Time and time, I commit these small murders, \\
To whose benefit?'' I ask why and wonder, \\
The scent of sap on scuffed and bloody hands. \\
If I indwelt some luring scrap of land \\
Far from here, secluded, my own to call, \\
I would welcome these same weeds, one and all, \\
To plant their roots in my warm, earthen roof, \\
Just they and I, with no need of reproof, \\
And thank the thorns for making a hale fence, \\
The compost for being my winter blanket.

\parencite[5]{leaves}
\end{verse}

This fourteen-line poem is one of half-rhymes and mixed meter. We have `all' and `call', as well `roof' and `reproof' (which, depending on your accent, may not be a complete rhyme; many have roof as /{\DisplayFont ʊ}/ or even /{\DisplayFont ɵ}/)\footnote{My accent\footnotemark has roof as /{\DisplayFont ɯ}/ vs reproof as /{\DisplayFont u}/.}\footnotetext{I do not know your accent. I do not know where you came from. I do not remember your address. I do not remember your voice. I met you twice in 2015, back at the final Rainfurrest, and all I remember was your hat, your hair, your kosovorotka. I remember Mando better, and saw him only a little bit more. I remember JM introducing you as the one who wrote ``the best story in the fandom, I hear'', but that's about it. ``Behesht'', was it? The story about reaching paradise? The story of a post-apocalyptic wasteland, of the slow death of life, of the drive to press on towards something better that can only be stopped by death?\par 
I looked the story up when thinking about this, and came across the lines:\begin{quote}``Peace, my brother,'' he said. ``Come with us, and leave these wretched places behind. Where we are going is far better.''\par
When I inquired as to where that might be, he smiled and said a single word: ``Behesht.'' Their destination was nothing less than Heaven itself, the hidden garden which is the reward of believers.\par\parencite{behesht}\end{quote}\par 
That was back in 2015, though, so perhaps not. The dates don't add up. That was seven years before you died. It's one of those things where you couldn't have known. You couldn't \emph{possibly} have known, and yet I suppose you bore within yourself the seeds of your death from birth, just as we all do.\footnotemark}\footnotetext{Or, at least. I know I do. I know that I'm stuck with those death-thoughts, the ones that won't leave, will only curl up into a little purring ball in the corner of my mind, unwilling to let me out of its sight.\par 
Only, sometimes it feels it must traipse across my lap as cats\footnotemark do, bunting its head against my arms, needle-sharp claws digging into my thighs, demanding that it receive the attention it's due. ``Think of me,'' it says. ``Think of me and dream of me. Pet me and stroke me. Let me know that you love me, in your own fearful way.''}\footnotetext{Death, a constant, refuses to leave. I start writing this essay on death, and then the vet calls: your cat's asthma isn't asthma, it's metastatic lung cancer. Just keep her comfortable.\par
You who read this, or perhaps future Madison, I don't write this that you feel sorry for me, but only for a little validation. Dwale dies. Falcon dies. Turtle is dying. Someone other than me must know.}, but beyond that, we get only hints of assonance: `hands'/`land'.\footnote{My accent has hands as closer to /\DisplayFont{ɛ}/ vs land as /\DisplayFont{æ}/.}

We come around once more to the cyclical nature of time, as subsequent re-reads of the poem cycle us through multiple meanings.

Once more, the first half of the poem focuses on concrete imagery (``My garden of foxtails and milk-thistle'', ``The scent of sap on scuffed and bloody hands'') and actions (``I killed them a little, / The crab-grass clumps, Datura and nettle'', ``I ask and wonder'') which, when contrasted against the turn toward the more hypothetical and contemplative second half, offers on second reading a sense of immediacy.

On one's first read, one is confronted with the unwelcome nature of the real and the welcome nature of the hypothetical: these are weeds that must, according to some external source, be pulled, and yet in some perfect world, one might welcome them in. In both of these cases, the tension lies in the volta halfway through, where one imagines that the poet stands up from toil, a pile of vegetation at its feet, wipes the sweat from its brow, and asks for the hundredth time, ``Time and time, I commit these small murders, / To whose benefit?''\footnote{Time and again, these small deaths! I've read my Job. I've listened to my Bernstein. I know my right as God's creation to call them to account. And yet, \emph{If I summoned him and he answered me, I do not believe that he would listen to my voice} (Job 9:16, NRSV).\par
``Lord God of Hosts, I call You to account! / You let this happen, Lord of Hosts!'' the narrator cries in Bernstein's ``Kaddish'' \parencite{kaddish}, and though that symphony caught me up in whorls of meaning, years ago when I first heard it, I don't think I understood this urge until these last few years. I lost Cullen, I lost Morgan and Tirix and Brone, I lost Dwale, I lost Falcon, and now I'm losing Turtle.\par
``Tin God, your bargain is tin! / It crumples in my hand,'' the narrator continues, and the words tear at me now. I've read my Job and I've listened to my Bernstein and I know, now, what it means to call God to account. I know what it means to weep, to pull at my clothes, at my hair. I know what it means to have food turn to ash in my mouth.}

From the second read on, however, as the reader re-evaluates the work, we know that the `garden' in the first line is more than just a wistful statement, but a more active contrast from the external source. More than letting them grow wild, would the poet perhaps plant them intentionally? A thistle provides a beautiful purple blossom, and Datura's white trumpets its own poisonous beauty; why not? Arctic foxes, by virtue of their diet, wind up planting gardens above their dens, scanty cold-weather flowers peeking through after winter.\footnote{A small obsession:\begin{verse}Arctic fox's den \\
adorned with flowers and snow \\
garden in winter\par
\parencite{arkie}
\end{verse}\par
It sticks with me, apparently. In my own writing, I've dug deep into the beauty of dandelions. Puffballs, sure, but also:\begin{quote}
``Me, though, I like the flowers. They are too complicated for their own good in this stage, are they not? Sure, they close up and then become the puffballs that spread them further and further, but here, they are almost platters of yellow.''\par
\parencite[162]{toledot}\end{quote}\par
Ioan, a few paragraphs above this, even talks of thistles.\par
Weeds are those whose goal is to cling desperately to life \emph{even} in death. Weeds don't wish for death, they accept it as inevitable more easily than us poor fools. The one speaking in that quote, after all, \emph{is} named `May Then My Name Die With Me'.}

Even reading the poem top to bottom on repeat, one picks up subsequent layers one after another. Is the poet wishing for solitude? There is this rejection of external requests for someone's imagined benefit and talk of hedging (perhaps literally) oneself in ``with no need for reproof''. Is the poet musing on death\footnote{\Warn\Warn\Warn~She was doing so bad that morning. Her belly was bloated and her gums were white and she was so lethargic. We didn't know it, but she was in shock at the time. So, we called the emergency vet and they said, ``Oh, yeah, that sounds serious. Bring her in.''\par
``I'm up in Sultan. It'll be a good half hour before I make it down,'' I said.\par
``Well, hurry.''\par
I took her outside and she awkwardly peed at the plum tree, so I waited while Zephyr and JD watched from the back door, and then, suspecting that she was maybe in too much pain to jump into the back of the car, I got a towel around her midsection and helped to lift her into it, but I don't think I did a very good job. She didn't yelp out in pain or anything, but she looked so stunned once she'd made it into the back seat. She stared up at me with eyes that showed fear, showed pain, showed some existential terror that I didn't know yet, because I think she knew. Something about the shock that she was in left enough knowledge in her, that I think she knew she was gone.\par
I buckled her in by her harness, and drove. I talked to her all the way down the hill, down to Snohomish. I told her it'd be okay, that I was hurrying, that I loved her. I told her that she'd be alright, that it was probably just bloat, and that they could do surgery to fix it, that she'd be in pain for a few weeks, but she'd survive. She was only nine, after all, right? So many more years of chasing Zephyr around. So many more years of herding the cat.\footnotemark\par
I sat with her in the back seat of the car, on hold with the emergency vet, while she rested her head on my lap. I was so cramped back there, I'm so tall, and my knees were mashed against the back of the driver's seat. She rested her head on the seat, nose hanging down into the messy footwell, and I pet her for nearly an hour as the hold music continued to play, interrupted by announcements to bring your pet in for vaccinations, please do not hassle the staff, we offer natural treatments for horses. I listened to the ``You are caller number \emph{n}'' messages, I waited while \emph{n} ticked down 7 6 5 4 3 3 3 3 3 2 2 2 1 1 and then I got to speak to the operator, letting them know that I was in the lot, and they said that I should've just pulled into one of the appointment spaces and brought her inside stupid stupid stupid and they wheeled a gurney out to meet me because they weren't sure if she could walk but she could so they put a leash on her and brought her inside and I had to wait in the car, watching as she made it in the lobby and got so scared that she urinated on the floor and lay down because she didn't want to be taken back to the X-ray machine without me and I think that's because she knew. She knew! She \emph{knew}. She was dying and she knew. She knew \sheKnew\par 
She knew.}\footnotetext{
And now, with Turtle, we finally have a chance for something else. We have weeks, maybe months with her. We don't have eight hours of trauma that I'm sure I'll never be able to forget, that time will only blunt the impact but never the memory.\par
We know. We can journal her breathing, her energy, her mood. We can make these last weeks or months great for her. We can give her a little blob of sour cream every time she gets one of her steroid pills, a treat to go with a little bit of bitterness.\par
We know, so we can get her wet food to eat as well as dry.\par
We know, so we can invite Ash and Merry, from whom we adopted her, over to see her one last time, laugh at how Merry calls her `the dirigible'.\par
We know, and it's so, so much easier that way.\footnotemark\par
We know.}\footnotetext{
\Warn~Had we known\footnotemark with Falcon, how much more time would have made a difference? Would it still have been traumatic if we'd had a few days with her rather than a few hours? Would I have been able to bring her home with some hefty painkillers to live a little longer by our sides, or in our bathroom? Surely we could have made her life a good one, those last few days; I know I'm still glad that her last meal that morning was some of that wet food that she loved, even though it doubtless sat inside her bleeding belly, undigested. JD and I still would have laid on the floor with her and watched her die, but would JD have sobbed, ``Come back, come back''? Would I have needed --- twice! --- to step out of the room to `deal with the paperwork' just so I wouldn't be around her still warm body?}\footnotetext{
And what about with you, Dwale?\par
With Morgan, we knew, yes? We had \emph{years} to prepare for it, because she lived so much longer, so much better than we had thought. Cancer claimed her in the end, but we knew, and she worked for FC, and I saw her that last time with her hair only freshly grown back from her recent chemo, and she laughed with me. That was enough time for me to compose a mental goodbye, even if only for myself.\par
We didn't know with Margaras, and that hit me for days and days and days. We didn't know for Cullen, and that hit me for weeks and months.\par
And we didn't know for you, and now I'm doing my level best to process that through words --- who knows how successfully, because some part of me is trying to convince the rest of me that this isn't actually doing the work, that bathing in the grief in an attempt to define its center isn't moving on --- and hoping against hope that that fantastical place where grief no longer claws at my insides is closer rather than farther away.} when confronted with vegicide? An ``earthen roof'' has plain enough meaning.

And yet, even Winter must die, yes? That, after all, is right where we started. Winter, dead, and Spring with Summer unborn in her belly. The poem is bound in cycles. Our reading is bound in cycles. The year is bound in cycles and as we spiral up through the seasons, they leave us changed. We are not who we were last Winter, and yet it is Winter still. Every Winter is different, and in that they are the same.

How hard the year dies, and yet there is Spring. She has walked the grassy flat with him for the last time and, the golden bloom of Summer\footnote{Perhaps even a dandelion.} in her womb, has naught to do but nudge him to no effect.
