\hypertarget{interpolations}{%
\subsection*{Interpolations}\label{interpolations}}
\addcontentsline{toc}{subsection}{Interpolations}

Allow a small diversion.

It is important to reckon with two interpolations within the text that appear to be later additions, and it would be nice to address these before coming to the text that they interrupt.

The first interpolation is that of a poem that comprises the entirety of chapter 28. The poem takes the form of a Hymn to Wisdom that Alter describes as ``a fine poem in its own right, but one that expresses a pious view of wisdom as fear of the Lord that could scarcely be that of Job''. \parencite[458]{alter}

The NOAB, however, suggests an additional interpretation of the Hymn to Wisdom, which is that it may have originally been the conclusion of Elihu's speech. For evidence, they mention that this topic, the elevation of wisdom, feels familiar to those chapters of Elihu's, wherein the youngster harps on the topic of wisdom and knowledge at length. Additionally, the editors note the similarity in the final verse of the Hymn, ``And he said to humankind,''Truly the fear of the Lord, that is wisdom; and to depart from evil is understanding''\,'' (Job 28:28, NRSV) closely echoes Elihu's final words as they stand: ``Therefore mortals fear him; he does not regard any who are wise in their own conceit.'' (Job 37:24, NRSV)

The hymn itself is a respectable piece of poetry. It begins in a roundabout way, discussing the acquisition of physical wealth. It describes the ways in which gold and silver are extracted from the earth and copper smelted from ore. It describes paths unseen by beast, ones that require work to acquire. Throughout these few verses (1--11) runs a very clear directionality. From the start, they are heading \emph{towards} something. They are pointing \emph{at} something. Verse 12 illuminates: ``But wisdom, where is it found, and where is the place of discernment?'' (Job 28:12, Alter)

Certainly not beneath the earth! If Qohelet\footnote{That is, the teacher in Ecclesiastes.\footnotemark}\footnotetext{I wrote that silly book, \emph{Qoheleth}, with the knowledge of the misspelling right there at my fingertips and yet still managed to do so. Ah well, I was young, once, and dumb.

Which is not to say that I am not now, of course. I certainly feel it sometimes. Even the young bit. Madison is, what, eight now? Not many eight year olds are smart. I still fumble. I still seem to create those humiliating moments that stick in the memory and make me wince whenever they come up, though they've changed in tenor over the years.} has taught us anything, it is that. Wisdom abides despite toil, despite merriment, despite even riches.

In fact, though many of the same ideas within the hymn are also there in Ecclesiastes, those in the latter tend to be more refined, more fleshed out. This might be due to the later date of composition of the former, but may also be due to the context of the book and the interpolated nature of the hymn. The author of the hymn views wisdom as an ephemeral concept. It is not something that can be held or perceived by man, or, indeed, life itself: "It is hidden from the eye of all living" (Job 28:21, Alter). Even other abstract (though often personified) concepts seem to have difficulty with it: "Perdition and Death have said, "With our own ears we have heard its rumor."" (Job 28:22, Alter)

Qohelet, on the other hand, has a much more grounded view. He says that wisdom is one of those things that you gain by experiencing, something that abides through all of the ups and downs in your life and is only ever strengthened. This is not to say that he is in any way upbeat, however. Wisdom, folly, riches, merriment, these all will go with you to the grave. They, too, will be meaningless.

That is, until, one gets to the end of Ecclesiastes. The second half of chapter 12 is, per Alter, likely an interpolation of its own, where an epilogist rounds out the remainder of the book with some sounder, more conventional piety. ``The last word, all being heard: fear God and keep His commands, for that is all humankind. Since every deed will God bring to judgment, for every hidden act, be it good or Evil'' (Job 12:13-14, Alter) echoes the end of the hymn, which puts it, ``Look, fear of the master, that is wisdom, and the shunning of evil is insight.''

Both of these interpolations seem to be taking the raw feelings of the authors of Job and Ecclesiastes and trying to soften them, shaving off all those coarse edges.

In Job we have a man striving to be heard by God Himself, and in Ecclesiastes, we have a teacher who is bordering on nihilism,\footnote{It occurs to me that perhaps one outcome for Job is that \emph{he} becomes Qohelet. Can one imagine going through the events of Job and not coming away with at least a little bit of nihilism? A little bit more stoic than when one went in? Your family dies. Your livelihood is stripped away. You sit in the pit of ashes with lesions all over your body, and then God comes down in his whirlwind and fixes it all for you. You look back on all of your piety, you look back on all of your wealth, and suddenly yes, it is all a chasing after the wind.} yet both of these editors are trying to fit these texts into the context of a tradition that, while it does include (and even encourage) the capacity to call God to account and to feel that certain sense of nihilism, would still appreciate a somewhat more positive view within its scripture.

((More\ldots))

The second of these interpolations is the Elihu's speech --- and, indeed, the entire character of Elihu, who is never mentioned outside his own chapters\footnote{I think we all must have one, an Elihu. One of those people who enters our lives seemingly at random, sticks around for a while, speaking a little too loud and a little too long, and then leaves again, leaving nothing but a sour taste in the mouth and a sense of bafflement. I know that I have one, though they're back in my past. They slipped in sometime around 2011 or so, perhaps 2010, a friend of a friend at first, and then perhaps a friend, and then disappeared in a huff sometime early on in 2012. Said huff took the form of a few sanctimonious statements that left me so in doubt of my identity that my transition was delayed by at least a year, easy. All that came before my intercalary years, and doubtless contributed to the death of Matthew.} --- in chapters 32--37. Alter holds a particularly dim view of Elihu, stating, ``At this point, in the original text, the Lord would have spoken out from the whirlwind, but a lapse in judgment by an ancient editor postponed that brilliant consummation for six chapters in which the tedious Elihu is allowed to hold forth.'' \parencite[460]{alter} Few seem convinced that the character and his speeches are from the original text. The NOAB, notably bearish on the whole Bible, agrees that this may indeed be the case, though it does so with a sigh and a tone of resignation, adding, ``In any case, the Elihu speeches are part of the book we now have''. \parencite[767]{noab}

The editors of the NOAB offer additional insight, that Elihu's speeches may have simply been shuffled out of order (a problem elsewhere in the text) and that his speeches may have originally come after the final of Job's three friends' speeches after chapter 27. This both lends credence to the Hymn to Wisdom in chapter 28 being the conclusion of his own speech and ensures that God replies to Job immediately after \emph{his} final speech rather than after Elihu's, which would better fit the structure of the book. There is no reason it cannot be both, of course; the two additions could have been both interpolations and inserted out of order through some mix-up or whim in an early editor's haste.

Elihu presents a departure from the rest of the book.

As the framing device draws to a close, we are introduced to three of Job's friends: Eliphaz the Temanite, Bildad the Shuhite, and Zophar the Naamathite. All three are presented as Job's contemporaries. They are wise, they are learned. They have, we can guess, known him for years now. These three friends have seen a lot with Job, rejoiced with him, wept with him, much as they do in the introduction.

Job and his friends have three rounds of arguments, which shall be covered soon, and then, beginning in chapter 32, Elihu is introduced out of nowhere. ``So these three men ceased to answer Job, because he was righteous in his own eyes.''\footnote{Did they give up? Did they see that Job was starting to change, was starting to stand up for himself, and realize that hey, maybe this was for the best? It seems deeper than simply winning an argument.} (Job 32:1, NRSV)

It is interesting to note the differences in tradition, here. Alter has ``because he was right in his own eyes'' but offers no note as to why, which is a little disappointing. JPS agrees with him (``for he considered himself right'' (Job 32:1, JPS)). Both of these are Jewish sources.

Christian sources, however, all lean on righteous, while the HCSB, NIV, and KJV having identical wording for that phrase. This colors the meaning, does it not? JPS and Alter describe Elihu as being angry because he is declaring himself more right than God, where as the Christian sources all interpret the text as Job justifying himself \emph{rather than} God. Interestingly, the 2001 translation of the Septuagint has Elihu upset that Job is declaring himself righteous before God, a sense of uncolored plainness that is missing from the other translations. In this case, Elihu is seemingly upset at Job for being upset.\footnote{And here our very own Elihus return. They return and they roll their eyes and stand, arms akimbo, before us. Why are you angry? Why are you crying? Who cares if you're right? They are in the position of authority, are they not? Get it together.}

If you will forgive further discursion, the next verse is all over the place in translation. KJV and NIV suggest that Elihu is upset at Job's friends because they couldn't find any fault in Job but still condemned him. JPS agrees, but uses `merely' before `condemn' which adds a value judgement. Alter has him upset because Job's friends couldn't show Job to be guilty. Though it is difficult to pin down why, Alter posits that Elihu is angry at Job's friends because they just couldn't actually find a way to condemn him: ``because they had not found an answer that showed Job guilty'' (Job 32:3, Alter) (a sentiment echoed in the footnotes for verse 13: ``In attributing this statement to the three reprovers, Elihu shows them admitting the failure of their own arguments.'' \parencite[548]{alter}), while the NRSV walks the middle path with ``because they had found no answer, though they had declared Job to be in the wrong.'' (Job 32:3, NRSV)

Weinberger continues to be relevant: ``{[}\ldots{]} translation is more than a leap from dictionary to dictionary; it is a reimagining of the poem.'' \parencite[46]{wangwei}

This is where we leave off, and then this youngster, this whippersnapper, this upstart Elihu picks up.

``I am young in years, and you are aged. Therefore I was awed and feared to speak my mind with you,'' (Job 32:6, Alter) he begins, and we are off to the races, or at least some brash exhortations to wisdom.
((On Elihu))

All stories are perforce interpolations within real events, this essay began, and that holds true here.
