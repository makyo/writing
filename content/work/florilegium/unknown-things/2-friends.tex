\section*{Friends and “Friends”}

How do we remember the past? How do we remember all of those countless conversations that make up our friendships, our relationships, our enmities? How do we remember the past?

The Book of Job remembers it through just the discourses. It remembers entire conversations, entire histories of friendship, through the lens of those two weeks Job spent in the cold firepit, covered with ashes and sores. It remembers them all through discourses and speeches and prayers.\footnote{I know, for instance, that my conversations with my husband around transition were many and scattered. We would chat over dinner, or we would talk on that horrifyingly yellow couch that he'd inherited about the fact that I was feeling strange about all these different aspects of identity. But you know what I remember? I remember sitting on that couch and talking in well-formed sentences, in paragraphs and essays, about why it was that I felt like the body I had and the body I \emph{had} overlapped incompletely, or I remember sitting on one of the dining table chairs turned to face the living room in a skirt I had made for myself, explaining to him that I felt like a part of me died when Margaras did.

These were almost certainly conversations. They were full of filled pauses and the backtracking failures of speech that come with just plain chatting, but that's not what I remember. I remember discourses and speeches and prayers.} Perhaps strangest of all, though, it remembers them disjoint and out of order.

Edward L. Greenstein discusses the transpositions, interpositions, and interpolations that go into the book of Job. Take, for instance, Job's first speech. ((end with vision such that Eliphaz can reference it, despite no one else mentioning that.))
