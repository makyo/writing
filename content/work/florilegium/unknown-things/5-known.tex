\section*{Should All Things be Known}

There is a point of least faith. This is the minimum amount of faith required to simply get by in the world. The word `faith', here, is specifically left lowercase:\footnote{However terrifying this large a concept may be, as True Name would have it:

\begin{quote}
But what does it mean to believe in something like {[}the irreversibility of time{]}? Or the sanctity of life or love or art? Or God, for that matter? `Belief' as a word is a stand-in for a concept so broad as to be to be intimidating or impossible. One may say as Blake did, ``For everything that lives is holy'', but encompassing that within one’s mind is truly terrifying.

\parencite{mitzvot}
\end{quote}}faith in God, perhaps, but faith that the world will get better? Faith that the next breath will come, that you and the world in which you exist are compossible?

This point implies for some an ideal of least faith: that one should strive to live their life taking the least number of things on faith as possible, that to rely too much on faith becomes a fault. For others, it is a principle of least faith: it is an intrinsic property that we tend towards the least amount of faith required to live, as is evidenced by the ever-increasing understanding of the world around ourselves.

And, perhaps because of that principle, this point of least faith is always shifting, trending usually downwards --- though some discoveries, if they are to be believed, may make that line tick upwards. Every day, we drift towards some point at which all things may be known.


