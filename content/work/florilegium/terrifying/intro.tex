There is an idea out there that there are three types of prayer: \emph{help}, \emph{thanks}, and \emph{wow}. \parencite{helpthankswow}

Like all discussion around anything so complex, personal, and transcendental as prayer, the trichotomy is incredibly reductive, and yet like all incredibly reductive divisions, it does have its uses. Classifications such as this allow us to sort through our feelings and how we intellectualize abstract concepts such as prayer. They are not hard-and-fast boundaries into which we may bucketize blithely,\footnote{A fact which is incredibly difficult to remember, sometimes.} but perhaps sticky notes we can post next to text, the better to understand it when we work our way through it.

They need not even apply entirely wholesale to a prayer. Take the Lord's Prayer:

\newlength{\tRemainder}
\setlength{\tRemainder}{\textwidth-\widthof{Help or Wow}-1.5em}
\noindent\begin{tabular}{@{}>{\raggedleft\arraybackslash}p{\widthof{Help or Wow}} p{\tRemainder}}
\emph{Wow} &
Our Father, who are in the heavens, let your name be held holy; \\
\emph{Help} or \emph{Wow} &
Let your Kingdom come; Let your will come to pass, as in heaven so also upon earth; \\
\emph{Help} &
Give to us today bread for the day ahead; And excuse us our debts, just as we have excused our debtors; And do not bring us to trial, but rescue us from him who is wicked \\
\emph{Wow} &
{[}For yours is the Kingdom and the power and the glory unto the ages.{]}
\end{tabular}

\noindent(Matthew 6:9--13, \cite[10]{dbh-nt}\footnote{I figured I'd spoken too much on the topic of translation already for it to be worth yet another go, but here I am again. You can surely remember my delight in Weinberger's idea of translator as active participant, and yes, this plays as much a role here as it did with my choice of Alter's translation of the Hebrew Bible, but I'm leaning on Hart's translation here for its distinctly modern and universalist take on the New Testament. Hart himself is a staunch universalist, his book \emph{That All Shall Be Saved} neatly lays his reasoning bare, and much of the focus on ineffable love feels applicable to the discussion of the ineffability of love, no matter how secular.})
