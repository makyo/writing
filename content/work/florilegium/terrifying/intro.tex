There is an idea out there that there are three types of prayer: \emph{help}, \emph{thanks}, and \emph{wow}. \parencite{helpthankswow}

Like all discussion around anything so complex, personal, and transcendental as prayer, the trichotomy is incredibly reductive, and yet like all incredibly reductive divisions, it does have its uses. Classifications such as this allow us to sort through our feelings and how we intellectualize abstract concepts such as prayer. They are not hard-and-fast boundaries into which we may bucketize blithely,\footnote{A fact which is incredibly difficult to remember, sometimes.} but perhaps sticky notes we can post next to text, the better to understand it when we work our way through it.

They need not even apply entirely wholesale to a prayer. Take the Lord's Prayer:

\newlength{\tRemainder}
\setlength{\tRemainder}{\textwidth-\widthof{Help or Wow}-1.5em}
\noindent\begin{tabular}{@{}>{\raggedleft\arraybackslash}p{\widthof{Help or Wow}}|p{\tRemainder}}
\emph{Wow} &
Our Father, who are in the heavens, let your name be held holy; \\
\emph{Help} or \emph{Wow} &
Let your Kingdom come; Let your will come to pass, as in heaven so also upon earth; \\
\emph{Help} &
Give to us today bread for the day ahead; And excuse us our debts, just as we have excused our debtors; And do not bring us to trial, but rescue us from him who is wicked \\
\emph{Wow} &
{[}For yours is the Kingdom and the power and the glory unto the ages.{]}
\end{tabular}

\noindent(Matthew 6:9--13, Hart\nocite{dbh-nt}\footnote{I figured I'd spoken too much on the topic of translation already for it to be worth yet another go, but here I am again. You can surely remember my delight in Weinberger's idea of translator as active participant, and yes, this plays as much a role here as it did with my choice of Alter's translation of the Hebrew Bible, but I'm leaning on Hart's translation here for its distinctly modern and universalist take on the New Testament. Hart himself is a staunch universalist, his book \emph{That All Shall Be Saved} neatly lays his reasoning bare, and much of the focus on ineffable love feels applicable to the discussion of the ineffability of love, no matter how secular.})

There are two faults within this system of classification. The first is the sheer amount of territory covered by each of the categories. Prayers that fall under \emph{help} may cover requests for deliverance from hardship, requests for plenty, requests for salvation, or even the panicked gaspings toward God that come with terror. \emph{Wow} prayers may come from terror as well, or perhaps beauty. I'm not sure there's any clearer explanation than that which Rilke provides:

\begin{verse}
Who, though I screamed,\footnote{I dearly love this translation by Will and Mary Crichton \emph{except} for this word. Will, my friend, 'cry' is \emph{right there}. So many before and after you chose it. Who, though I cried out...but alas, we come round once more to translations and translators.} would hear me among the ranks \\
of the angels? And even supposing one of them took me \\
suddenly to his breast, I would perish within his \\
overpowering being, for the beautiful is right at the margin \\
of the terrifying, which we can only just endure. \\
And we marvel at it so because it holds back in serene disdain \\
and does not destroy us. Every angel is terrible.

\parencite[11]{duino}
\end{verse}

\emph{Ein jeder Engel ist schrecklich}, he writes, and there is perhaps no prayer more powerful than one spoken from the limen of terror and beauty we call awe.

And \emph{thanks}? There is too little bound up in that word for us to hang our hats on. I've marked none of the Lord's Prayer as \emph{thanks}, despite the very nature of its praise.

The second, however, is that there are categories that is misses by virtue of the way it thinks about prayer. It comes at it so literally! How could it possibly hope to encompass the headiness of ritual? The comfort of mantras? The familiarity of well-worn words that linger with us through liturgy? It is a goal-oriented, Christian (indeed, largely Protestant and Evangelical) view of prayer.

And where is the silence of contemplation? Where that ecstasy --- that literal \emph{ekstasis} of standing beside oneself, the one of which the mystics sing --- of simply being? Of pouring love over oneself like an annointing oil?

\begin{quote}
Dear Raspberry,

It's not that I never noticed before how many red things there are in the world. It's that they were never any more relevant to me than green or white or gold. Now it's as if the whole world sings to me in petals, feathers, pebbles, blood. Not that it didn't before---Garden loves music with a depth impossible to sound---but now its song's for me alone.

\parencite[119--120]{timewar}
\end{quote}

There is prayer in the quiet purity of love, and Blue and Red, the main characters in Amal El-Mohtar and Max Gladstone's \emph{This Is How You Lose the Time War}, life a life in prayer.
