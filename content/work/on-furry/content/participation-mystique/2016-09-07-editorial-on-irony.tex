\Post{On Irony}{September 7, 2016}

Sixteen years ago, I was a not-so-wee lad just starting his freshman year of high school. I had grown a foot and a half in the previous few years, and my voice had fallen down the staircase from alto to baritone. I had just come out to my mom as gay. My favorite saying, which my step-mother hated, was ``sarcasm makes the world go 'round''.

And I had just found furry. That too.

Now, today, I'm well on my way to becoming a giant woman working in open-source software. I've not grown any taller, really, though my hair is now quite a bit longer, and I no longer sing, not having an outlet I feel safe in. I recently came out to my mom as polyamorous, married to my cis male husband, loving my genderqueer partner, and looking forward to seeing my trans-girl pup again. I'm still in furry; but I no longer believe that sarcasm makes the world go 'round.

A lot has changed in the last sixteen years of being a furry. The tenor of the fandom has changed as the resources have shifted. MUCKs grew less popular, as did IRC, while things such as Twitter, Slack, and social sites have taken off. VCL still exists, as does Yerf!, in some form or another, but FA grew to take their place, and still others are vying for market share. Skype grew, then started to fade, to be replaced by Discord, while AIM and other direct messengers seem to have been overtaken by Telegram and the ilk.

I finished high school in that time, then started university in biochem, switched to music education, switched again to music composition, left university and started working for a health insurance company, and finally wound up at Canonical.

I dated, I stopped dating, I changed species once (from red fox to arctic fox), I composed and wrote. All of this, or at least most of it, took place within furry. At some point, perhaps around 2010, I managed to get in touch with Klisoura, so that I could snag some of his data from the Furry Survey for visualization practice. These visualizations would lead to {[}a{]}{[}s{]} itself and the {[}a{]}{[}s{]} panels, before long, and {[}a{]}{[}s{]} would lead to Love - Sex - Fur and the guides to safer-sex, relationships, and gender.

I like to think that I've grown more sincere over the years, that I've started to prize earnestness above sardonic humor and honesty above glibness. This started as something that I found myself enjoying more and more in others. Not that I didn't enjoy the occasional bit of snark, and I certainly enjoyed good humor, but I found myself starting to surround myself with people who were able to express the way they felt about something truly, without as much of a mask as I had built up for myself.

The journey to becoming a more earnest person, myself, has been one of the harder things I've had to go through in life. The habits formed when young are hard to break, and just as I still misgender or deadname myself, I still find myself slipping back into those sarcastic ways far too easily. It's a mask I wear - one of many - and it adheres too readily to my face. For so long, it was inconceivable that I feel emotions other than anger and pride. Not forbidden, not even ill-advised, but, for me to have felt despair or elation, joy, depression, or sadness\ldots{}well, that would have been a sign of just how broken I was.

As I put it to my therapist, I took a passage from the book Dune by Frank Herbert and applied it way too literally to my life. Young Paul Atreides is being tested by the Bene Gesserit reverend mother with the gom jabbar, a test which will determine whether he is a human or an animal. A human, the reverend mother explains, is in total control of his emotions and feelings, and can use those to better himself, while an animal is ruled by his emotions and feelings, and can easily be overrun. ``You've heard of animals chewing off a leg to escape a trap? There's an animal kind of trick.'' she explains. ``A human would remain in the trap, endure the pain, feigning death that he might kill the trapper and remove a threat to his kind.''

I was putting myself to this gom jabbar daily, continually. I still do, if I'm not careful. If I'm to be a human, I mustn't let my emotions rule me. I took it far beyond the point where it was healthy, bottling up feelings to the point where they would on escape at moments of crisis. Running away, a suicide attempt, punching a hole in the wall, a fight, a cut, a burn. I would be less than human to feel any emotions but pride in my accomplishments or anger at the shortcomings of others. I would be an animal (and not in the fun way). I was trying to be my view of my father, I was trying to be a support for my mother.

My therapist (perhaps rightly) rolled his eyes, but the meaning got across well enough.

I'm still friends with Klisoura, of course, and had the chance of spending a lovely hour or two yapping with him at Rocky Mountain Fur Con a few weeks ago. As we discussed some trends showing up here and there within the fandom, he said something that knocked me on my tail for a bit with its weight: ``My journey through furry has been a journey of decreasing irony.''

Mine has, as well. Of course, like everything I write for {[}a{]}{[}s{]}. I must caution that this doesn't necessarily apply only to furry: my journey towards living happily has also been a journey of earnestly accepting my emotions and feelings and then expressing them, in not feeling bad about liking the things that I like.

That said, I think it's not worth discounting the ways in which furry is structured to encourage such a shift, from ironic to sincere. The shift may be one that happens in everyone's life, but furry provides the social lubrication to allow it to happen more easily.

The primary means by which furry encourages sincerity is by the obvious fact that we're all really here because we like something. There are plenty of hot takes about hipsters, geeks, sports nerds, and so on, about how uncool it is to be a part of such a group, as though one ought to be sheepish about the things that one enjoys.

I told my story earlier to show just how this is played out. In the competitive nature in which children, especially those in the formative years of the early teens, are so often raised, it's not enough to like something, one has to excel at it. That is, one can't form a portion of one's identity around something lest that leave a spot for weakness. To enjoy the idea of philately or model trains is fine, but there's risk to be found in enjoying them too much, basing a portion of your identity off of themThere's no pride to be had, and it's opens you up too easily to damage and loss, should your stamp be unattainable or your train set ridiculed. Or, to recast in furry terms, forming a portion of your identity around your membership within the fandom opens you to shame as you watch your fandom being derided as a bunch of sad kinksters by an inebriated volleyball parent in the FC convention elevators.

Liking things - just earnestly liking them, without shame or defensiveness - is something of a skill learned over time. By the time that we are able to form a portion of our identity around something that we like, we've already learned the skill of shame. It has already become engrained in us as we start to actively pursue hobbies in our early to mid teens. To take that enjoyment beyond a simple hobby and into an identity, from a fan to a member, means stepping past that shame knowing full well that it's watching your every move.

Contrary as it may seem, through our habit of connecting with each other through created personae and avatars, we are able to construct something of a defense for ourselves. It's a sort of layer of indirection, which allows, e.g, Makyo to be a member of the subculture while, to her coworkers, Madison is more of a fan of anthropomorphics. The internet has proven a boon for subculture membership in this half-anonymized way; the same may as well be said of a gamer with a penchant for playing games as fast as possible with only one hand, or an entire subculture surrounding countless anonymous individuals playing Pokémon.

Mainstream culture doesn't know how to interface with furry culture because furries are the only non ironic people left on the Internet

--- gay victim soul (@tragicgay) August 24, 2016

It's not so much that furry makes one sincere, as it provides so many opportunities to be sincere. Furry didn't make me less ironic, that would be a silly statement for a fandom centered around creating personalized anthropomorphic characters. Furry did, however, make me want to be less ironic. It did help me in getting closer to being a more sincere person.

I think that I'm not alone in this, either. I felt it. Klisoura echoed the sentiment when he said that furry was a journey of becoming less ironic. Twitter user tragicgay felt it when they tweeted about mainstream culture being unable to fully understand furries due to the lack of irony.

@tragicgay I wonder if, despite some ppl saying irony died after 9/11, irony has completely triumphed.

--- BilderstreitKünstler (@Christaphorac) August 25, 2016

Is that true, then? That mainstream culture has so enshrined irony that it's baffling to be earnest? Is that us furries failing their gom jabbar?

I'm not sure, and perhaps am not one to say, given how much trouble I've had in my own new sincerity, but I think that may be it, at least to some extent. There's no small part of me that wants it to be the case, too; that wants furry to be this staggeringly beautiful new way of looking at the world, experiencing enjoyment, showing emotions, just plain unabashedly liking things.

I think that is, perhaps most ironically, what makes us most human.
