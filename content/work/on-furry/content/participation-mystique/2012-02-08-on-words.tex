\Post{On Words}{February 8, 2012}

This is a post I did not intend to write. I certainly did not intend to continue the Participation Mystiquepost into another.

Actually, truth be told, I had planned on taking a week off from writing; coming up with some fluff post pulled together from a combination of responses with some neat witticisms thrown in for good measure, or even just tossing up a guest post. Work's been decidedly hellish, and when I haven't been working, I've been feeling some emotional strain resulting from a large case ofover-commitmenton other projects. Come Monday, however, I'd caught up on sleep, and started rifling through comments and tweets in response to a few statements I'd made over the past few weeks. I eventually decided that I shouldn't be a lazy fox-man and pull together a formal response here in the form of an article.

So. What is furry?

It feels like every website, blog, and even every individual has to take a crack at defining furry. I personally wanted to stay away from it as much as possible because I didn't want my own attempt at a definition to color the views of the readers of the site. There have been a few comments on my last two posts and a few of my tweets, however, that have shown that that's already the case, and that my circumlocution around the issue may have caused more problems than it avoided. That is, for certain definitions of ``problem'': I lovethis sort of discussion, truth be told. Almost as much as I love circumlocution. Or the word ``circumlocution''. Sorry I'm so wordy.

``Furry'' is an overloaded term. One of the most descriptive definitions of ``overloading,'' as I understand it, comes from the realm of computer science. When one overloads an operator, that means that one is changing the way that operator works within a certain class of items. That is, the `+' operator, given two numbers, will add them together, but when given two strings, will turn them into one string by concatenating them. Additionally, when one is dealing with structrured data, one can overload a reference to a piece of that data. `ID' can refer to a student ID, a class ID, or Idaho.

I like this metaphor when it comes to overloading words in language. In particular, I feel that the concept of an overloaded term intended to mean multiple discrete things is particularly applicable, given the response I've gotten to certain posts and tweets. Namely, Altivo's comments to the article on sexism in the fandomand Sparf's responses to my twitter query about one's favorite ``unintentionally furry work of fiction''. In both cases, the difference between one person's definition of furry and the other's is notable. The big discrepancy seems to be whether or not the holder of the definition considers things that are not intentionally furry as furry or not. Put another way, is anything that represents an anthropomorphic animal furry?

The whole concept of anthropomorphism, as I'm sure my (likely 100\% furry) audience already knows, is the attribution of human characteristics to non-human objects, usually to non-human animals, real or fictional. Since I seem to be on a tear of explaining myself, this is what I would call the parent category of what is furry. The fact that Coyote could talk, that Mickey Mouse could stand on two legs, that Garfield hated Mondays, these all fall into the camp of anthropomorphism, without a doubt. However, in each case, the author or authors designed the animal in question without a thought (at least, at first) that they might be subsumed by a fandom that was not specifically related to that exact thing (insofar as there is a Coyote, Mickey, or Garfield fandom).

Both of the commenters I mentioned before appear to disagree with me on this, however, and I know that they are not alone in their definitions of furry. In fact, the number one response to the question ``Describe furry in your own words'' on the {[}a{]}{[}s{]} Census and Surveyfar and awayseems to be ``an affinity for anthropomorphic animals''. However, I'm not convinced that I'm alone in feeling that this isn't exactly the case for many who call themselves ``furs''.

My biggest complaint with simply claiming ``any anthropomorphized animal'' as our own is that the definition is simply too big for a fandom to be able to be structured around it. Specifically, I feel that there is more to the fandom than simply anthropomorphic animals: avatars. It's not so much that we share thoughts or even fantasies about anthro-animals with each other, but that we all create our own avatars consisting of a mix of ourselves and an animal of our choice. I'm not sure that a furry convention would be able to gain multiple thousands of attendees if it simply consisted of many people agreeing loudly with each other that they like talking foxes.

On the other hand, I know that there are many levels of auto-anthropomorphism within the community. Some people find it a fun thing to draw, some think it's pretty awesome to dress up as an animal, and many find it perfectly pleasing to interact with each otheron the Internet as if they were anthropomorphized canines and felines. The main thing that ties all of these diverse individuals together is the fact that they enjoy the connection between man and animal embodied in the concept of anthropomorphism. It is the root of our community, and the base of our interaction with each other. There are two questions that deal with this on the {[}a{]}{[}s{]} survey: ``what is your level of anthropomorphism?'' and ``what is your means of interaction with the fandom?''. That such questions are even part of what could be considered a general census of the furry fandom is a clue that there is something more than the specific concept of having a partially-animal avatar.

This is why I prefer the definition for furry as ``a collection of people who identify as furries''. I think that it encompasses the right amount of people without overstepping bounds. It allows me to say things like ``unintentionally furry'' in order to differentiate between those who do something related to anthropomorphic animals and those who consider themselves members of a group who is willing to focus on anthropomorphics to the extent that many will even create for themselves an avatar for interpersonal interaction that is an anthropomorphic animal. In short, it allows me to step on the fewest toes. Or tails.

I feel that this differentiation is important, not only for us being able to define ourselves to ourselves, but also to the world around us. I've mentioned before the reaction of the writer Steven Boyett's reaction to discovering that his novel The Architect of Sleephad been latched onto by members of the furry fandom (for those who missed it, it was decidedly negative). When we define ourselves to others, we have to take into consideration our own definitions of the fandom, as well as others'. This is something that was elaborated on by Samuel Conway (that is, Uncle Kage) in his Anthrocon panel on interacting with the media (something which I very much recommend watching). Conway neatly breaks this down into a few key points:

Don't define ourselves in terms of what we're not - If you say ``it's not about sex!'', then the first thought that will leap into the minds of your listeners is ``wait\ldots{}why did they mention sex?''

Don't define ourselves in terms that aren't easily understood- This ties into some of my qualms about defining furry as ``people who are interested in anthropomorphic animals'': doing so provides such a broad definition that it becomes easier for the listener to oversimplify than to understand, and that only if they already know what we mean when we say ``anthropomorphic animals''.

Do be aware of first impressions- Conway suggests that you lead with your answer to the question ``what is all this?'' by saying that we are fans of ``cartoon animals''. While this grates on my nerves, I have a hard time disagreeing. If someone's first opinion of you is as a fan of Tom and Jerry and Rocky and Bullwinkle, then there is little harm done before you go on to explain the fact that many of us come up with our own personal characters with which we associate.

Do be aware of the listener's preconceptions- While this isn't explicitly described in detail in the talk, it is implied with Conway's interactions with the ex-military audience member: if the listener already thinks that we are a bunch of sex-crazed maniacs who fetishize getting it on in animal costumes, take that into account in your own interactions with them.

These are just a few of the items mentioned in a lengthy talk on interacting with the media, but I feel that they're important to consider when coming to terms with defining furry. There are many who hold their own vague concept of what we are already in their heads due to either their own personal interaction (or membership) with the fandom or with a media outlet's portrayal of us.

Besides even that, though, our interaction with others within the fandom depends in part on what we consider to be a furry. Some have a more liberal definition of furry, in that it includes constructs that are not intended to be included in a fandom of those who create their own constructs for themselves. Others, however, hesitate to even call themselves furry, so much as furry artists, or eschewing even that, anthro or even animal artists. Put that way, my own definition seems to be something of a cop out: I say that those, whether or not they have constructed their own characters, are furries so long as they identify as such.

In more concrete terms, I think that this is the definition that my readers should take into account when reading my articles and the twitter feed. When I say that there is a focus on sexuality and a certain sort of sexism in furry, I mean within those who identify as furry; similarly, when I ask what is a favorite ``unintentially furry'' work, what I really should've asked is what would be a favorite work focusing on anthropomorphic animals that didn't originate from our own subculture. This is partly in my defense as a response to those who have called me on my use of the term, but also me tossing my own two cents in when it comes to defining furry: it is what you make of it!
