\Post{On Names}{December 25, 2011}

What is in a name, anyway? For us here in the fandom, a name can be several things: a pseudonym, a description, even a whole other being, however fictional. It's safe to say, then, that names are pretty important to furries, and so maybe that's worth taking a look at. You have to start somewhere, so lets begin with how to construct a name. There are, of course, many other ways to construct a name, but we've listed just a few of the best here.

{[}adjective{]}{[}species{]} - This, of course, goes without saying. It's the only method of choosing a furname that's endorsed by an entire website. On the Internet!The pros? Well, obviously, the first impression will go much smoother, now that everyone knows your a SlutFox or an AngstWolf*. There is simply no mistaking what you are, is there? It's also food for subversion! Who knew, SlutFox is a virgin, and AngstWolf is really doing pretty alright in life! As for the cons, well, if you can't change the name, but you wind up changing your species, you could be SOL.

The suedonym - Sometimes, you just can't think of a name. Or\ldots{}well, you can, but they're all taken. Well, that's no reason to stop you! Why, I was once !Xabbu (from a book by Tad Williams), then Ranna (from a book by Garth Nix), who then became Astarael (same book). The pros - don't really need to think too hard; it might already be memorable! The cons - it might already be memorable as something else; it might already be memorable by many others (I wasn't the only Ranna\ldots{}).Subcategory: The they-can't-suedonym- You know, if works even better if you don't have to worry about the problems associated with a stolen name. Like if the author or origin is long dead. I knew a cat named Merlin, for instance.

The appropriation- Why not just appropriate another word for yourself? I very rarely go by Happenstance, which is also the name of a French film (pure happenstance, of course). I have a good friend named Whiskey, too. That's good, I like whiskey and Whiskey! Pros: pretty memorable. Cons: this one's pretty safe, actually. Subcategory: l'appropriation- Bonus points if you appropriate from another language. Just.. be careful of Japanese, okay? There are a few Ookamis out there.

The punny animal- Of course, these are totally memorable for reasons that make people want to hit you in the mouth. My otter-sona is named Macchi. As in Macchi-otter. In fact, the back story is that he's got light fur and his parents weren't very inventive, so they named him Caramel. Caramel ``Macchi'' Otter. Sigh. Pros: totally memorable. Cons: people want to hit you in the mouth. Subcategory: The recondite lingual obfuscation of humorous intent- If the pun of your name takes more than a few words to explain it\ldots{}may actually be a pretty good name, because then people won't hit you in the mouth so much.

The real name- I\ldots{}er\ldots{}hmm. I guess I may have met someone who used their real name once. Maybe? I mean\ldots{}hmm. Hey, was that guy Ty really named Ty? Does Karlhockey count? This could be big, guys, I don't know\ldots{}maybe the new, unique trend in furry pseudonyms would be to just use your real name. I mean, that's pretty inventive, and it's already© you\ldots{} Pros: inventive. Cons: now they know your name.

\begin{itemize}
\tightlist
\item
  See:http://rikoshi.gd-kun.net/furry.html
\end{itemize}
