\Post{Let's Talk About Sex}{August 17, 2013}

I spend a lot of time burying myself in the fandom, reveling in the connections we build and giving back in the best ways I can manage. I'm not a good artist (just trust me on this - it's best for\emph{everyone}if I don't draw), and my music career stalled after I graduated college: I've not yet found the means to jump-start it. When it comes to fiction, I'm afraid I have more ideas than I have motivation. It's not a good combo, really, as I wind up with (quite literally) notebooks full of ideas with all of two stories to show for it. I have a doofy time-travel story that somehow managed to involve choir music,\emph{sigh}, and a short piece of erotic fiction that involves two foxes and some milkshakes.

I'm not an excellent writer. My prose is loose, and the fact that keeping it that way is the only way I can manage to get anything done does not speak well for improvement down the road. I can pull a mean metonymy, I'm average at alliteration, and if I squint for a while, I'm sure I can squeeze out a metaphor, but I have no formal training in writing beyond the minimum required to graduate high school, and whatever it took to get my music degree.

What I do have going for me, however, is words. I've got a lot of words.

So: lets sit down and have a little chat about sex. Or - wait, cancel that. Let's have a grand discussion about it

In my delving within the fandom, I have wound up involved in a few projects, and one of those is the delightful, rather fuzzy book club \href{http://bookmarfs.com}{Bookmarfs!}. The club is a very open affair, though we've got regular participants in the form of myself, my partner Forneus, a most excellent fox, Peri, TabbieWolf, and the inimitable Lunostophiles. When we somehow managed to boil this month's book down from a choice of \textless{}book about sex\textgreater{}, \textless{}book about heterosexuality\textgreater{}, and \textless{}essays on patriotism\textgreater{}, to \textless{}book about sex\textgreater{}, I was a little surprised, though hardly displeased. All three were quite good, and while I do hope we get to \emph{Straight: The Surprisingly Short History of Heterosexuality} and\emph{The Partly Cloudy Patriot} sometime in the future, Mary Roach's\emph{Bonk: The Curious Coupling of Science and Sex} has been a very interesting read, and there's one passage in particular that I'd like to share:

\begin{quote}
The media's ubiquitous coverage of sex and sex research - as well as the genesis and population explosion of TV, radio, and newspaper sex advisors - have chipped away at the taboos that kept couples from talking openly with each other about the sex they were having. Bit by bit, sex research has unraveled the hows, whys, why-nots, and how-betters of arousal and orgasm. The more the researchers and the sexperts and the reporters talked about sex, the easier it became for everyone else to. As communication eases and knowledge grows, inhibitions dissolve and confidence takes root{.}
\end{quote}

I think a lot about the ways in which we, as a culture, move forward. JM and I have both written about it over on {[}adjective{]}{[}species{]}; about the ways in which both conformity and transgression, each in their place, help to advance both individuals and society. I think that the two writing platforms of {[}adjective{]}{[}species{]} and LSF embody this nicely, when we get down to it. Sure, there's some mixing - there's quite a bit of transgression on {[}a{]}{[}s{]}, and there's bound to be a fair bit of conformity here on LSF! Still, though, our whole point is to move things forward, whether it's through reinforcing the ideas that tacitly pervade the fandom, or by challenging the very ideas that many hold dear.

The previous quote from\emph{Bonk} comes from a section about the ways in which Masters and Johnson changed viewpoints with their publication of \emph{Homosexuality in} \emph{Perspective} in 1979. The book is admittedly not without its problems - the second half of the book apparently could be read as a tract on curing homosexuality, though I've not read it for myself. The first half of the book, though, was spent dissecting the differences in sexual interaction between homosexual and heterosexual couples, whether they'd been together for years or simply assigned to be together that night by the researchers. Between homosexual couples, Masters and Johnson noted, there was much more attention to one's partner and much less attention to, to paraphrase Mary Roach, the author of\emph{Bonk}, goals. Rather than focusing on the goal of orgasm, the sexual act shifted on the act of providing pleasure.

As Roach admits in the quote, much has changed in the nearly 35 years since the publication of\emph{Homosexuality in Perspective}. Never mind problems in the\emph{HiP} study, nor even social attitudes towards sexual orientation, the discussion around sex itself, as the author insists, has shifted. We are becoming more open about sexuality (the reasons for which include, not least of which, commercial opportunity), and by virtue of becoming more open in general, we are becoming more open specifically within our relationships. The taboos that dog us through the ages don't stand in the way quite as often as they used to, and are easily knocked down by more knowledge, more discussion, and being more open.

They're not gone completely, however, and that's why, \href{http://www.youtube.com/watch?v=4niz8TfY794}{to paraphrase Vi Hart}, so long as we construct barriers for ourselves, we always find ways to deconstruct them as well; and we will always construct those barriers.

To that end, I'd like to announce the first of (hopefully) several guides published and distributed freely within the furry fandom: \href{http://guides.lovesexfur.com/safer-sex}{TheLove ◦ Sex ◦ Fur Guide to Safer Sex}.

JM and I have been discussing such a thing in various guises for quite a while now, and I've spent the last month or so laying the groundwork for such a project. The idea is simple: to provide a short, accurate guide or guides for distribution within our subculture to accomplish the simple goal of disseminating more information. This is the heart of Mary Roach's point in the quote above, it's the fact that underpins the very existence of the {[}adjective{]}{[}species{]} projects, and it's the one thing mentioned over and over again whenever an idea such as this is mentioned.

Furry is nothing if not participatory, however, and so here's the twist: all of the guides are open sourced. Not will be, but already are, have been from day one. "Open Source" has almost certainly permeated most aspects of our culture by now, but it's still worth clarifying. Anyone with the will to do so is free to make changes to the guides. They are not, of course, open to vandalism - changes must be approved before they are merged in with the master copy - but they are open to everyone to not only suggest, but to be proactive and make corrections, to add information, and to extend the base of knowledge. We'll get to the how in a bit. First, though, let's get into some goals and planning.

Rather than spend time listing what this is not, let me just explain a little bit of where I'm coming from on this. I'm not a very sexually active person, despite my part in a site such as this. Even between my partners and I, acts that even I would consider sex are decidedly rare, to the point where I would be comfortable with the label "gray-aseuxal" at the moment, in that sex sort of falls into this weird gray area. Labels and identity are always in flux, of course, but I mention this primarily as a preface to the fact that despite that, we make it a point to talk about sex a lot. \emph{A lot}. If we could be said to have a favorite thing to talk about, it's probably sex. It's in these discussions that we've come across the most internal strife in values instilled by parents, society, and even some parts of the furry fandom. Sex, like many topics shrouded in taboo, also carries a lot of negative weight for people, and my primary interest is that of sex positivity.\emph{\\
}

Sex positivity is one of those topics almost certainly in need of its own article (or several), but in short, the idea is that sex and the discussion around it should never be a negative experience. That was my tweet-length description, but we asked on Twitter a while back, and got several additional replies: "That each person has the right to express sexuality to their own degree without being made to feel bad about it", and "it means accepting that a complex area of human interaction exists and is full of nuance that requires careful consideration", and "sex positive means treating sex in a mutually joyous, consensual, fun, and funny manner. It's an expression of love, but not the only one by far, nor is it the only physical one" were just a few of the examples.

The primary goals of the first guide are:

\begin{itemize}
\tightlist
\item
  Accurate and up to date information on safer sex, and
\item
  A safe and positive outlook on sex and sexuality.
\end{itemize}

Beyond that, we aim to make it as inclusive as possible, as well as pertinent within the fandom. It should be interesting, fun, and worthy of discussion, just like the act itself. And so with that, have at it! Read, share, and contribute! It's all available \href{https://github.com/adjspecies/lsf-guides}{here}.

\hypertarget{how-it-works}{%
\subsection{How it works}\label{how-it-works}}

The guides are hosted, all together, in one git repository on GitHub. A git repository is simply a folder of files that a program, called git, knows about. Git is a distributed revision control system: it knows how to track changes made over time (the revision control system part), and it lets anyone work with the entire project at once or even host copies of it (the distributed part). GitHub is a website that hosts these repositories in such a way that anyone can contribute to them. This means that you can grab a copy of the entire set of guides (a process called forking), make the changes you want, save those changes back up to GitHub (a process called pushing), and then ask us at LSF to consider merging your changes in (a process called a pull request). This is, by necessity, a very brief overview, but GitHub not only has some excellent documentation on the process at the \href{https://help.github.com/}{top of their help page}, but also a \href{http://windows.github.com/}{client for Windows} that helps make the process much, much simpler. If you need any additional help with git, please feel free to ask! You can contact me directly at \href{mailto:makyo@adjectivespecies.com}{\nolinkurl{makyo@adjectivespecies.com}}.

Once you have made a change and created a pull-request on GitHub, try to get some input on the changes you'd like to make! I'll do my best to review every request that comes my way, but we'd like to have two positive reviews for each pull request to be merged, unless it's something trivial such as fixing spelling or markup. At that point, the branch will be merged into the master branch, and changes will go live soon thereafter on the site. Additionally, once we get print versions of the guides, those changes will be merged and made available wherever the print versions will be distributed. There's more information on the whole process on the \href{https://github.com/adjspecies/lsf-guides}{repo page} on GitHub.

If you don't wish to contribute directly - whether you don't wish to have a safer-sex guide in your list of contributions on GitHub, or you simply don't want to bother with git - that's also fine! If you have a change you'd like to make, you can simply tell us about it (or send us a diff) by emailing the address above. I'll work to create a branch to represent your work under my own account, and then offer it up for review, anonymously if you so choose. If you don't want to go as far as actually modifying the guide itself, you can also \href{https://github.com/adjspecies/lsf-guides/issues}{open an issue}, which someone else can then resolve. The goal is to make it as easy as possible to manage contributions, and we don't want to leave anyone out! Additionally, you can always help review upcoming changes (\href{https://github.com/adjspecies/lsf-guides/pull/1}{here} is an example of a proposing changes), which will be announced here and over Twitter. You can do that through email, as well. Our primary needs for those are:

\begin{itemize}
\tightlist
\item
  Making sure the information is accurate (we use the CDC as our benchmark),
\item
  Making sure the information is relevant (the guide should be clear and concise, and hopefully at least a little furry), and
\item
  Making sure the information is presented well (is it furry, does it look good, etc).
\end{itemize}

There are other ways in which you can contribute, as well - not only will we be having more guides coming up soon (the next planned is on relationships, along with the {[}a{]}{[}s{]} guides for fursuiting and convention attendance), but we'll also need a few pieces of art for each! Finally, you can always help by spreading the word and getting these guides out there. After all (say it with me)...

\emph{More information is better!}
