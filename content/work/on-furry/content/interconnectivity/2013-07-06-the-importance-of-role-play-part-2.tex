\Post{The Importance of Roleplay --- Part 2}{July 6, 2013}

\emph{Previously on Love ◦ Sex ◦ Fur...}

Okay, so we're hardly a television series here. Last time I wrote, though, I spent nearly two thousand words on just how interesting I think sexual role-play is. Once I started nearing the end of the article, though, I noticed that I really sort of forgot to include more than a few token references to just how role-playing fits in with relationships. Sometimes, my writing suffers from the fact that I get so easily focused on a smaller aspect of a larger issue, and it's hard to step back far enough to see things from a broader point of view without losing the train of thought. Also, given that I finished the article literally fifteen minutes before it was supposed to go live due to a ridiculous power outage, I figured it'd probably be best to leave things as they were and instead dub that article Part 1 and save Part 2, this article, for an exploration of how TinySex and role-play in general fit into relationships, both romantic and otherwise.

No small amount of ink (or key-presses, for that matter) has been spilled on the topic of how we, as furries, create these avatars for ourselves and use them to interact with each other, often in the most banal of ways. We role-play being \href{http://characters.openfurry.org/description/19}{advertising agents}, get \href{http://www.furaffinity.net/view/9927219}{pictures} of ourselves running late to work, or write stories about \href{http://www.kyellgold.com/books/oop.html}{football players in love}. Sure, we have our spaceships and magic, but speculative fiction of that sort is hardly solely the realm of furries. We also have our slice of life stories and \emph{bildungsromans}. We role-play full lives as our characters, not just the heroic segments.

Last week, the topic I focused on was primarily that of one of the more banal, yet no less important, aspects of life: sex. I talked about sex primarily in terms of the act itself. I think that the last article suffered particularly from a lack of talk about the relational aspects of sex, as I spent most of my time thinking and writing about psychological aspects such as transgression or mechanical aspects such as orgasm. They're important, sure, but sex is a social act, given that it hardly takes place all by one's lonesome. I think it's well worth taking that step back and looking at the social aspects as well.

There are, I've found some relationships in which role-play Just Works™. That is, even though there are relationships that I've had, friendship or more, in which I've given the occasionally \emph{hug} or \emph{nuzz}, there wasn't any room, need, or desire for much beyond that. However, there were many that were just the opposite. There were relationships that focused almost entirely on role play, and for a variety of reasons. Both types (and everything in between), I feel, are valid. A relationship that consisted entirely of me being a fox person and the other being a different animal person, interacting primarily in those roles, was still a close-knit relationship, one that I felt comfortable calling such even in the romantic sense. Looking into why that's the case is what got me thinking about this in the first place.

Restating that thesis, I don't think it's out of place for me to say that there are relationships that do more than include this context of role-play: they flourish on it. For various reasons, even if the acts involved in the RP within the relationship were possible offline (and we've talked about a few that aren't), the relationship just wouldn't necessarily work the same, if at all, outside of this context.

The opposite is also true. There are people with whom I am good friends, but we only ever just talk online, though we might be close physically or even romantically offline. That's hardly extraordinary, of course: I talk with all of my coworkers online primarily about work, and have at most every job, though we may chat and go drinking outside of that online context. It's as though there's a divide between where role-play will and won't work for us that we pick up on and take into account when interacting, no matter how close we are emotionally.

One aspect that I think is worth reemphasizing is that, although the emotional connection in these role-play heavy relationships may be different, it's not invalid and hardly existent in these role-play. The interactions that I have and have had with others online (as mentioned previously, and worth a caveat again, I'm not as into this sort of thing as I used to be) were often deep, personal connections to another individual or group of individuals. The countless hours I spent talking to and fooling around with others hardly count as nothing, the others hardly faceless automata reacting solely to what I type.

A lot of that has to do with the idea of spaces, or contexts. Sex online is still sex, but it takes place in a separate context, a separate space from sex in person. As a good example of that, it's been my experience that sex in person usually occupies most or all of one's attention, whereas that's hardly the case with role-play online. In fact, beyond just potentially having more than one thing going on, one might have more than one scene going on in order to still have most of what one's up to fall into the category of sex. This is something that fits within the pornographic aspect of the act: rather than being solely a participant, you are also watching or reading something, and as with pornography, that needn't take up 100\% of one's attention.

The social aspects of sexuality can be seen as a collection of innumerable moving parts. We are our own expert systems, in that regard, with certain tolerances that must be taken into account in order for things to mesh in any given context. Taken this way, having multiple contexts - especially when one of those contexts is notably free and lax on rules - may be of use to some. Much of the previous article was about utilizing such a context in a way that proves beneficial to a great many people on an individual basis, but also it works just as well in a relationship setting.

One example where folks benefit is in differing levels of sexual activity. This stems mostly from the disconnect between sexual acts in RP and sexual acts offline. One side of this is that there isn't necessarily a need for one to be sexually active in real life for one to be sexually active in character. Whether it's having a low sex drive, or just the inability to currently take part in sexual acts in person (dorms, work, and so on being common reasons), doesn't necessarily preclude one from taking part in intimacy with someone else. The inverse is also true, of course: one can still be sexually active in person despite being in a relationship with someone online, where RP may be common. Not everyone lines up sexually all the time, after all.

Another important example relates to sexual promiscuity, and the way it is perceived in a role-play context as compared to something more grounded in real life. While monogamy and polygamy are complex social topics most definitely worthy of being explored more deeply on their own, it's interesting to note a surprising number of relationships that are more open in an online context than they are in person.

Part of this, I think, is due to the relative safety of TS: the lack of physical contact and the role that consent plays do help to keep everyone safe. Even when one is not currently in a romantic relationship, however, there is little in the way of physical harm that might come from being promiscuous in role-play, and sexual satiation does help keep people healthy mentally, even if one does wind up having to tone down promiscuity when one does wind up in a relationship.

Beyond that, however, the idea of contexts once again comes into play. What it means to have sex in the context of typing out strings of sentences with erotic content at one another can very obviously mean something different from consensual physical sex. That said, one isn't simply alternating roles of narrator in telling some erotic story. One's characters are extensions of oneself, and so the waters are muddied. Some take this to be equivalent to sexual activity offline, while others separate the ideas totally, and there are those who treat it every way in between.

To round out my hendiatris of examples, a final one is the ways in which basal aspects such as gender and orientation come into play. This is something that I'm not sure is as widespread, but it has figured large in my own experiences, and I find it worth investigating. Both gender identity and sexual orientation can be much more complex and intricate than society often gives them credit. Experience through experimentation help flesh out one's sense of identity and self within a social setting, as they interact quite closely. Gender-play and a safe-space to experiment outside of more simplistic definitions of orientation. JM wrote on {[}a{]}{[}s{]} about this previously, noting the number of members of our subculture who \href{http://adjectivespecies.com/2012/03/19/re-evaluating-your-sexual-preference/}{re-evaluate their sexual orientation}, and I think this is one potential aspect of it (after all, I came into the fandom as a gay male and who even knows what I am now).

A lot of these seem to boil down to relative levels of "okayness". Being okay with oneself, being okay within a given context, and being okay both physically as well as emotionally and intellectually. The crux of the matter, after all, is that the physical is deemphasized while the mental and emotional are emphasized in role-play such as this. This is part of what goes into a safe space, and that's what's often required for maturation: a place in which one is both challenged without reduced possibility for harm, and an easy way out. It's a place to define the boundaries of okayness when it comes to things such as relative sexual activity, promiscuity, or countless other aspects of self.

There has been a lot of discussion over the last year about sex-positivity in furry. How sex and the fandom interact is more than just how sexual furries are in general, but how discussion around sex is framed between furries and themselves, and furries and the world at large. After all, this is a big part of why LSF was started in the first place. However, it's my belief that beyond being one of the more interesting aspects of that intersection, role-play is also one of the more sex-positive ones: interests are discussed and expanded on, websites such as \href{http://f-list.net}{f-list} are created, and tools such as \texttt{wixxx} are used. There's discourse around the topic, and it's not just that it's relatively positive, it's also expected. A culture has built up within and around this RP context. It's no longer just singular acts.
