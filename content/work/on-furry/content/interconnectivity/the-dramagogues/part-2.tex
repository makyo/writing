In the last post about drama, I wondered whether or not we, as a community, really were more dramatic than those around us, and if so, why, or if not, why we seem to think we are. Much of the content of that post came from responses to a few questions on twitter. Perhaps the best thing about our fandom is our willingness and ability to communicate, and that really is the basis of much of these articles, I had asked previously whether or not we were more dramatic and why, and gotten several very succinct answers as to why that might be the case,Beyond that, however, I also askedif our drama is in some way different than that inthe world around us, and got several additional responses to this question, which is the basis for this, the second episode of The Dramagogues.

Yes. Furry drama sheds and gets all over everybody.

--rustitobuck

While this response may have been provided tongue-in-cheek, I think it does illustrate something that we do fairly well: appropriate. Furries are very, very good at deciding what is furry and what is not, and will do so every chance they get. There was a book published several years ago, The Architect of Sleep by Steven R. Boyett, which featured anthropomorphic raccoons as its characters. The author did not intend for them to be furry, andhad originally planned on the book being the first in a series, but the response from the fandom was so outsized and, from the author's standpoint, creepy, that he refused to continue the series with the fandom's response being the reason why (he was not so polite in his wording).

We are so eager to appropriate things around us in the name of our fandom that it could be that, whether or not our drama is indeed all that different, we have made it ours. It may be just drama, but, being involved in the fandom, it becomes Furry Drama. There are, of course, some issues that may be unique to our subculture such as the intellectual property one has in a character, but it feels sometimes as though we could stick cat ears and a tail on any old problem and turn it into a furry problem. If you get short changed at the farmers market, you can complain about it,and if the artist you paid \$5 for an icon takes a few days too long, you can do much the same, but it's now possible to make it into a furry problem.

not particularly. the irony of the furry fandom is that it's more human than humans are.

--\_am3thyst

Another way to look the same issue is to consider that our drama is simply an artifact of us being a slice of humanity as a whole. Humans have their own little dramas that are being played out all the time. However, humans aren't a small, rather tightly knit group of peoplewith many things in common. While all our problems may be relatively human,it could be that we just read more deeply into them because of our commonalities. On the other side, because we read so deeply into them, we do tend to be more focused on the day-to-day human dramas of our fellows. I think that may indeed be why we are so closely knitting the first place, at least in part.

This is one of those good-for-you scenarios. Even though the drama around us is\ldots{}well, drama, it's still an instance of us interacting, which is a good thing, and the fact that we are so emotionally tied to the issues at hand is evidence of our emotional investment within the fandom. I used to wonder what the fandom would be like without all of the drama at seems to come with the package, and I think I've come to the conclusion that I just wouldn't like it that much. It's not that it.s comforting by itself, so much as that it's evidence of how much we care about our hobby. If furry were something where being involved didn'tmean enough for one to get emotionally invested, really don't think that it would be something that I would've stuck with this long, nor something that would've grown as fast as it has, even if it means focusing on our all-to-human problems.

Drama is drama, regardless of who says it or the content.

--Adonai\_Rifki

I mentioned a quote in the last post, ``Minority identity acts as a force multiplier on social dynamics. In-feuds carry the implicit baggage of membership''. Perhaps our drama really is just drama and has no special furry significance, and although the Internet likely has its effects on the issues involved, it could just be that our membership in the community makes us feel obligated to interpret things in a furry context. This quote does well to tie together the previous two in that it brings together the ``content'' being appropriated and the ``who'' of us just being people.

Our membership in this group carries the implicit membership in the drama therein. By taking it onto ourselves and turning it into the fandom's drama, we may wind up blowing it out of proportion (or way, way out of proportion), even though it's still just a little spat between individuals, as would happen between any groups of people. Still, it's comforting to know that we can do so much together, even fighting amongourselves.

I'd say any look at Facebook would say no.

--mousit

On the other hand, perhaps it's not our membership to the fandom that makes us so keyed into each other's drama, and our drama seems different and out of proportion because we happen to be tech-savvy people. The benefit of anonymity provided by the Internet, or at least a lack of direct consequences for our words and actions could be part of why it's so easy to turn any little thing into drama. Perhaps our reliance on such a medium in order to properly express ourselves has its downsides: both an enhanced sensitivity to the language used around us (due to its relative permanence as compared to speech) and the ability to maintain a structured, even institutionalizedfacade presented to those around us.

Before I got into furry, I got into the Internet and some of its culture. I've mentioned before that I started out on some bulletin boards in about 1999, and we were no strangers to drama there, either. With communication on the Internet, it's easy and even encouraged to ``write for your audience'', to steal a term. Speech is very extemporaneous and it's easy to have a slip of the tongue or to say something potentially offensive without meaning to (foot-in-mouth syndrome), but it's much easier to write with a purpose, rather than extemporaneously. That is, even when you're discussing the relative merits of two different restaurants, you are writing with a very specific goal, reading and reread in what you've written, and making sure, even if only subconsciously, that you present yourself at your best. At the same time, however, you know that others are doing just the same and thus tend to pay a good amount of attention to language that's being used around you.

In furry, this structured presentation of self has become institutionalized in the concept of one's character, no matter how tightly associated the individual is with it. Even on visual media such as SecondLife, our interactions take place as structured language intentionally built to deceive, in a way. We intend to show ourselves as our characters and we write carefully in order to do so. Perhaps this is a symptom of furry, but it seems as though it's built into the Internet as a whole. The ability to maintain near-real-time communication using text allows one to build up whatever facade they wish while still coming off as a real person.The drama here comes up when a bit of that facade slips or is let down in order to share an honest opinion with someone, or let loose with some previously hidden emotions. This happened nearly as often in the boards I had been a member of as it does within furry, but seeing as how we were all a bunch of hormone-saturated teenagers, I had chalked it up to that, instead. Having been around the 'net as an adult now, I can say that we're just as childish (if not moreso, sometimes) as we were when we were teenagers when presented with the opportunity for anonymity, however partial.

If I were asked to give an opinion on the spot as to whether furry drama is different than regular drama, I would say no. Within the fandom, we have some very ordinary problems, and I don't think that our membership to this subculture changes the problems we have in any way. However, I would not be able to say that without a caveat: our membership does change the way in which we interpret drama. Our problems may be very similar to those among any predominately text-based culture, but our focus on our characters adds a strange twist to everything we do here, including fighting.

Tune in next time as we look at the way drama changes and fluctuates over time within the fandom, as well as how that is similar and different from the world at large. The Dramagogues, only on {[}adjective{]}{[}species{]}! Wednesdays,12pm mountain! (Okay, so I wouldn't do well in TV\ldots{})
