Many people, I suspect, use the idiom, ``Hindsight is twenty-twenty'' in a way that is better served by other, more appropriate words or phrases.   The sense in which I hear it most commonly used is perhaps more adequately covered by the beautiful portmanteau, ``regretrospect''.  That is, now that things are said and done, I respect a lot of what happened during this adventure.

Also, it's my second favorite portmanteau after ``congratudolences'' and really ought to see wider use.

No, I think ``hindsight is twenty-twenty'' is better reserved for cases when seemingly unrelated occurrences come together to form an outcome that seems to be greater than the sum of the parts.  It fits best when you look back at your life and disparate, unconnected events come together to make the situation you find yourself in now.

I came out to myself and my (at the time) fianc\'ee as transgender over a process of several months.  It began sometime in 2010 or so, when I started to feel like I was able to put words to the things that were making me feel bad.  I began by identifying as genderqueer, and although that label still fits very well, I adopted `transgender' in 2015 as the one that I use in day-to-day life to describe myself, as it leaves the fewest questions as to why I'm a six-foot-two rectangular man-shape in feminine clothing and makeup.

But we're talking about hindsight, so it's worth bringing up that one of the only things I ever stole was the book ``The Boy Who Thought He Was A Girl'', back in second grade.  I'm guessing at the title here, as I can find no record of it through casual Googling, however, I remember that it was a trashy, essentialist book about a boy who wanted to learn how to kiss, which somehow made him girly and, thus, confused about whether he should actually be a girl.  Of course, in the end, his understandings of his gender roles as a boy were firmly straightened out by strict-yet-loving family.

Or perhaps another step in hindsight was sneaking into my step-mom's spare room when I was about twelve and trying on one of her old dresses.  At that point, I had yet to become the lummox that would be my post-pubertal destiny, and so the dress fit, although tightly.

Or hey, skip ahead to 2006, when I had just turned twenty and realized that it felt just as good to role-play online as a vixen as it did as a tod, though I told myself at the time that it was because I wanted to experience more relationships than the male homosexual relationships I'd had to that point.

Each of these things, and so many more, felt like an independent occurrence to me.  It's only in hindsight that I can see that there were aspects of me straining towards some way to feel happy and comfortable.  When I was growing up, they were oddities, but now just another way to see the present more clearly.

I think that it's fairly common that one comes to terms with a portion of one's identity in this fashion.  Before I came out as trans and made the question of orientation at least twice as complicated, I went through the process of figuring out that, despite being born male, I was also attracted to other boys as well as girls.  Those `crushes' in elementary school make more sense, and so on.

There had to be some lever that pushed each of those instances from a collection of loosely related occurrences into the formation of a strong facet of my own identity.  With orientation, it was obviously the rush of hormones that came with puberty: all of the sudden, `liking boys' took on a new tenor.

With gender, it was almost entirely the furry subculture's fault.

\secdiv

\textit{Indi's story}

\secdiv

\textit{The stats on trans folk in the fandom}

\secdiv

\textit{Another story}

\secdiv

\textit{Witnessing, Mirroring, and the gender cascade}

\textit{Why the fandom is important for trans folks}
