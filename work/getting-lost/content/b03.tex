\chapter*{Chapter Three}
\addcontentsline{toc}{chapter}{Chapter Three}

Carter had come up against a unique hurdle.

One of the problems with the genderqueer patient, RJ, was that, although it was notionally feasible for em to have an X in their gender records and all the pronouns ey chose, not everyone had recognized that.  Various pronouns flourished and died, styles of dress had come and gone, but the arcane triad of institutions -- banking, health care, and government -- remaned stodgy and stuck in their ways.

As a result, RJ existed in a unique limbo in some cases.  Although eir passport specified X under the gender marker, although ey went by eir own pronouns (which, Carter was informed, were a truncated version of singular-they known as Spivak pronouns or Elverson pronouns, depending on the source), and although ey functionally lived life as a genderqueer person, without close ties to anyone, ey still had records that marked em as male in the arcane triad.

This, by itself, was not an interesting fact.  Even though folks had been doing similar for going on centuries now, there were still articles and posts about the perceived barriers to entry when interacting with the world as a genderqueer person.  Person after person had complained about the difficulties in changing one's gender marker -- a letter from so-and-so, such-and-such documentation -- and person after person was turned down by clerks of the court for various reasons; take your pick of: ``we don't do that here,'' ``you have to do that at the federation level,'' or ``you'll also need this [unrelated document]'', among countless others.

Everyone had imagined, as medical technology had advanced around the subject of gender, that such changes would become easier as the world moved on past gender.  Some held decidedly more dire predictions in their heads than others, of course.  It hadn't quite worked that way, though.  The fractious nature of identity combined with regime shifts had left quite a large gap for people to fall into.  Nearly a hundred years after transgender rights had been codified, there still existed only the M, the F, and the X.

What Carter was learning was that the X came with a whole bouquet of baggage.

Before one grew up and settled on an identity, one was burdened with either an M or an F on official paperwork; this despite years and decades of campaigning.  The result was, as far as she could tell, a duplication of records.  One wound up with an M or F record that was linked permanently to one's X record.

Carter had only the faintest idea of why it would be so difficult to change records unilaterally.  Having worked for her fair share of time in academia, she had become inured to the committee culture of university life.  Government, she supposed, was like university, only hundreds or thousands of times the size.  Rather than one single database storing individuals embedded within a system -- citizens, that is -- there were likely countless such databases.  Updating one's gender record caused a ripple that propagated through databases in a way that was understood by the makers at the time, but `at the time' often referred to a time when legislation was passed.

The result was that Carter was coming across two records.  There was RJ (X) and RJ (M).

Oh, rather, Avery Croft, her subordinate in the stats and history department, had come across this problem and eventually kicked it up to Sandra, their lawyer.  Finding no clear path forward that could be found under a ten percent time bargain, Sandra had kicked it further up the ladder, and now Carter was dealing with the multiplication of records.

\textit{That's okay,} she thought to herself.  \textit{This is my job.  I'm the one who ties things together, and this is just another one of those.}

She was parked before three decks, delved into her workstation.  She had a deck for RJ (X) of pretty substantial size, a deck for RJ (M) of much smaller size (some of which was, doubtless, duplicated in the first deck), and a deck that she was working on, titled simply `research'.

To her right were two additional decks.  One for Cicero (M) and `research'.  Cicero (M) was an enormous deck.  Even among the countless decks of cards scattered through the shared system that the team was using, it was among the largest.  Cicero had more than just a little bit of past behind him, documented through various channels.

Were Carter to zoom in on Cicero's deck, she knew she would find thousands, perhaps even tens or hundreds of thousands of cards detailing interactions with the DDR.  These were `paperclipped' together, a shorthand to say stuffed in a folder without being separated from the rest of the deck, such that, by leafing through the deck, she simply came across one large entry encompassing all its component entries: a grouping within a grouping.

From what she could tell, however, skimming through the `research' deck, was that the two had encountered very different sorts of interactions on the `net, even taking into account their varied interests.

Collin Jackson was almost a parody of DDR addict, whereas RJ Brewster was a quiet, introverted person online, choosing to interact, when ey interacted in public, on a one-on-one scale.  However, they were both solid participants in the furry subculture.

% the thing is that RJ's unique position as an outlier gives the team more insight into actual problems: ey's slipped through a lot of automated checks based on outdated assumptions of gender
