The world is headed in some pretty interesting directions when it comes
to things like Augmented Reality. ~From little things, like QR codes
next to items to allow further investigation of them, Google Goggles,
which overlays locations of restaurants or other map markers on a
real-time video of your surroundings as taken by your phone's camera to
all of
the~\href{http://vimeo.com/search/videos/search:augmented\%20reality/st/27d7a185}{concept
videos}~coming out from various places around the 'net. ~One of the more
important, if not the most important, uses of AR is the addition of a
data layer over what we perceive around us. ~Need to know more about
someone from their business card? ~Snap the QR code on it and find out
all you need. ~It's that simple, and let me tell you, furries are
totally prepared for this additional layer of information: we're already
pros.

We're used to multi-layered channels of communication, in this fandom.
~With the majority of our interaction taking place online, we talk, role
play, and chat plenty, but we're usually not doing only that. ~There is
still the base layer of our communication online, the words and ideas
going from one person to another, or among several people, but there are
several things that change the way we interact, and especially change
first impressions. ~When we meet someone for the first time online, we
have plenty of subtle ways of extracting information from or about them,
and several of them without the other person's knowledge that we're
doing so.

When you're interacting with others on a MUCK, such as FurryMUCK or
Tapestries, you have several tools at your disposal to tell you more
about the person than you could ever find out in real life without
knowing them for years. ~MUCKs are text-only, so one of the first
commands you learn is `look', which will provide you with a short
description of how someone looks; an obvious addition for the primarily
visually-oriented furry. ~Beyond that, however, there are commands such
as \texttt{wi} or \texttt{wixxx}, WhatIsz, which will show you what a
person is interested in (or~\emph{not}~interested in) in areas both
clean and dirty. ~Some of these are specific enough that they would
likely not even crop up between a couple with no online interaction for
years. ~Another tool that's available is, depending on the muck
\texttt{cinfo} or \texttt{pinfo} - character information or player
information. ~Even more free form than WhatIsz, these commands will let
you know not only about the character, but about the person behind eFox
or iWolf you're chatting up, as much as they'll let on.

It's not just on MUCKs that we have these additional layers of
subliminal conversation going on. ~Even on IRC where such commands are
much more limited, we still have the rest of the internet available to
us, and by far, FurAffinity has changed and helped this the most. ~As
soon as you see someone's name online, there's a good chance that you'll
be able to just look them up on FA and find out a good deal about them,
from where they live to the types of things they're into judging by the
art they favorite there. ~FA isn't the only site out there, of course,
and you can also find out much more explicit detail on sites like F-List
and The Rabbit Hole, not to mention other art sites like VCL, SoFurry,
and e621.

These are so entrenched in the furry fandom that, writing this, I keep
feeling like it's not even worth mentioning. ~Every time I think that,
though I remember that it's one of the things that helps to set us apart
from other subcultures out there. ~The fact that we can and will find
out more about the people we're interested in based on a few short
commands or a quick search online sounds pretty sinister - it's just not
something people in general do, at least not to the same extent. ~If you
apply for a new job, you can expect to be Googled, Facebooked, and
LinkedIn by your potential new employer, but that's about as close as
you'll get to someone looking up personal information about you. ~It's
so totally normal for us that we haven't realized that it's changed the
way we make our first impressions of each other. ~In an AR sense, this
is roughly equivalent to walking down the street and seeing someone
rather attractive, only to find out via a little thought-bubble above
their head that they secretly really enjoy being spanked, bitten, and
tied up when they have sex.

If you meet someone within the fandom now, it's easy to find out more
information on them than you would ever find out otherwise. ~Friendships
are formed more quickly than outside the subculture and are based on
much more in-depth knowledge of each other. ~Add in the benefit of sex
without physical consequences through playing around online and you've
got a strange basis for a culture that relies almost entirely on a
multi-layered channel of communication. ~The more I think about how
different these first and lasting impressions are within the fandom, the
more I think it stems from the previously mentioned difference between
character and self that is inherent within furry: we are so eager to use
any tools available to us to more completely represent our characters
online that we're willing to change the basics of personal interaction
in order to accomplish it. ~Add in the anonymity provided by the
internet and you have a whole subculture that is far more willing to
share personal details with those that they haven't even met yet than
most any group out there, online or offline.

Interacting in person with other furries, particularly at conventions,
is a strange mix of ``normal interactions'' as well as some amount of
this multi-layered communication. I'm sure that much of this has to do
with how generally tech-literate furries, or at least the con-going
crowd are. ~If you meet someone at a convention, you'll likely to do it
by scanning their con-badges for images of their character or a
recognizable name, rather than, say, looking at a face (the
``con-greeting''). ~With the information contained on a standard
con-badge, one still has as much to go on as on IRC - namely, the
ability to look someone up on FA and figure out more about them. ~Maybe
I'll try an experiment with FC 2012 and make a QR code badge and see
just who all interacts with it.

Beyond that, however, I wonder just how much of our in-depth first
impressions translate outside of the fandom, but into other, tightly
knit groups. ~If, say, an academic winds up at SIGGRAPH or a designer
winds up at TED, meets someone in the halls, and notices a convention
badge with a name on it, chances are good that they'll be able to go
check on their work somewhere on the internet. ~However, these examples
are academic and professional, not social, and I haven't had the
opportunity to go to, say, an anime or comic convention to see if
lasting personal or even sexual relationships are formed in quite the
same way as they are within our own subculture. ~Would I be able to go
to Nan Desu Kan, a local anime convention, and expect to meet two or
three people there whom I would be able to instantly look up on my
intelligent telephone, know intimate details about, form lasting
friendships with?

With this confused blur from total immersion in our characters to the
unobstructed view of self that we provide glimpses of, our mixed-up
concept of first impressions within the furry fandom is understandable.
~These first impressions are based not only on the actions of a persona
as we perceive them, but also the more static metadata left behind on
the other layers of communication within the fandom, whether it's
information left on FA, attributes on f-list or within a command such as
\texttt{wi}, or art, visual or otherwise, of a character doing whatever
that character does, providing a glimpse of how that avatar moves within
the larger arena of the whole subculture, or even reacts to the world at
large. ~Perhaps it really is no big surprise that the furry community is
both incredibly tight knit and also renowned for the drama that it puts
itself through.
