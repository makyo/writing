Every now and then, it's important to take a step back and gain a little
bit of perspective. It sounds cliché, of course, and there are a lot of
people in my life I can imagine scoffing at the type of post I'm about
to write, if not that very phrase itself. In fact, there are plenty of
other posts that I have in the docket, but they can wait for another
time, and I hope you'll begrudge me a fluff post while I gain my
perspective. ~Also, a trigger warning for some brief but frank
discussion of suicide, and excessively sentimental foxes.

There's a lot that can be said about emotion. Hell, there's a lot that
has been said about emotion; so much so that there is only the most
minuscule of portions that bear repeating. If there is one thing worth
noting, though, it's the intensely dire sensation each of our own
emotions carry to us. They press against us and burden us with
incredible weight, and even though there's a lot of really flowery prose
one could write about just how much our emotions impress on us, it
really just boils down to the fact that an entire portion of our brain
is focused on \emph{feeling} things at all times, almost without rest.
This dire aspect makes it quite difficult to accept commiseration, to
comprehend that many of us try to understand those around us be way of
relating their experiences to our own. To hear someone say that ``what
you're feeling is just like when I felt something exactly like it!'' Or
``that's something that everyone goes through.'' To hear that this
burden isn't yours and is hardly unique is not a comfortable thing to
hear, no matter how true.

I go through bouts of depression about once every six or seven months
that last for about a month. I freely admit that this is hardly
uncommon. Freely because I'm actually feeling really good right now, and
have been for a bit. I can remember the urgency and importance of the
way I felt, even when it's not something that's pressing on me
\emph{right now}, as it was then. This difference is sometimes a vague
feeling: like, ``yeah, feeling good is different than feeling bad''.
Sometimes it's a very concrete sensation, such as now being able to
tolerate heights as something that's merely scary, and not ``oh God am I
going to jump!?''.

Being able to take a step back, no matter the cliché, is the sort of
helpful thing that lets me see and understand what exactly is going on,
and, understanding, helps provide me with a path forward. Not a
solution, of course, just a path. I don't do meds; I have a deep-seated
paranoia of that attempt at a solution despite seeing them work wonders
for someone very close to me. Their reason for taking them is very
situational by their own admission: given a very nearly unsolvable
problem and no time to work on it, one takes what space one can in order
to move forward.

That's what the step back grants me. Even though the source of my own
overwhelming emotions is something decidedly innate, something more
biological, the space gives me the room to take that into account. If,
for example, I give myself the room to understand that those feelings of
hopelessness and dread that seem to be stemming from work are more just
the handicapped sense of self involved in depression, then I can more
easily make the choices I need to stay healthy.

This is really new to me, honestly, and thus my fascination. I started
to understand it last year in October and November when I was going
though a similar period, but it occurred alongside a work trip to
Copenhagen that left me no room for myself. Heathrow's terminal 5, with
it's glass-walled balconies and walkways, and the hotel's
\href{http://www.flickr.com/photos/ranna/8128316641/}{looming}
\href{http://www.flickr.com/photos/ranna/8128342888/}{15 degree}
\href{http://www.flickr.com/photos/ranna/8164105438/}{tilt} made me
frankly fear for my life. The previous March saw an attempt at suicide,
and the very limited amount of space I (figuratively) had to step back
into was hardly enough in November for me to work with this problem
constructively, and it took getting kicked back by the motor tic in my
neck coming back~after an absence halfway through the trip and forcing
me to slow down to understand just what this space meant to me.

April and May were much different. Things started to go pear shaped in
mid-April, and, though the tic had once again left, I knew right away
what I had to do. I slowed my velocity at work (with my boss's
blessing), held off on writing any articles, and~took the space I needed
to stay healthy. There was another work trip on the middle of this, but
it was out in California, where, even though I was still working my tail
off during the day, I had more of a support network than Copenhagen had
to offer outside of work hours. While things got their worst after that
trip, I still had my space, and so everything was different. The aching
pressure in my chest was far less than before, along with the sense of
dread and suicidal ideation. Things were off, but as long as I could
take that step back, they hovered a notch or two above `bad'.

That's a lot of words, and not one of them was `furry', `subculture', or
`fox-person'. For those of you still reading, I appreciate your
tenacity, because honestly, it's this furry subculture, this ability to
be a fox-person among friends that provides the framework I need to
remain grounded while taking these countless steps back, lest I just
withdraw completely into myself.

Toward the end of the summer of last year, it was JM who IMed me to ask
how I was doing. My emotions were coming through in my articles, he
said: I was on point when I was happy and maudlin when I wasn't (I know
this is basically the most maudlin thing I've ever written, but stick
with me here). I took time off then to gain some space and work on
improving things, but having this framework kept me from zooming off to
far into the distance. Most poignantly, it was the death of my friend,
Margaras, that helped prove the worth of maintaining the ties I had with
those in my social circle, furries all.

The fandom as a subculture plays a very unique role in or lives, I've
noticed, in that it provides a sort of skeleton that we can use to help
give our lives their structure. I found myself discussing this with two
LDS (that is, Mormon) missionaries who stopped by the other day, when I
asked them how their faith fit into their lives in terms of identity; I
was raised by two staunch atheists, I didn't experience religion as a
community until a brief stint attending a Unitarian Universalist church
in my early twenties. Their conversation lead to the topic of chosen
family, that closest of social circles. They said that their growth out
into the world had structure, pacing, and direction that they felt would
have been missing without the framework of their church.

I said at the time that I agreed with them: having that missing from my
life led to the described lack of direction in my own growth. ~My time
in the dorms was a stark example of that. However, in light of these
last two months, and all that I've learned over the last year and a
half, I'm not sure that I had told the truth. Furry is lacking a lot of
things that make a church, and so yes, my growth within the fandom was
hardly predictable; no mission for me. But that said, it was still just
that: growth within the fandom. I have this framework in my life to add
meaning and direction. That's what kept me and so many others going
after Margaras' death, what got me through last march and the end of the
year, even what helped me during this last sprint. I still had
structure, even if I didn't feel well. Something to hold me up and keep
me from deflating completely.

A few weeks ago, I tried to explain some of these thoughts in the form
of a small experiential game, a little bit of interaction intended to
convey a point, called \href{http://a-full-life.drab-makyo.com/}{A Full
Life}. In it, your goal is to make the fullest life you can, even when
there are things standing in your way preventing you from feeling
fulfilled, your sense of `full' handicapped. I think that these
frameworks - the church for those missionaries I'd talked to, furry for
me, and countless others - help us out. They don't necessarily solve
problems (and may often cause them), but they help keep that handicapped
sense of self from constricting too small and squeezing out
\emph{everything} that's good in life.

So. Apologies for the wash of an article, and thank you if you've made
it this far, but do me a real big favor: sometimes, when you've got a
bit of time, think about the ways this fandom is meaningful to you.
Think about the ways you must be meaningful to those around you. Maybe
take a moment to talk about it with someone, or if not, at least just
appreciate it. I know I do.
