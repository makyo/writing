\chapter{A Letter to my Daughter}

\begin{itshape}
Kayla, my dear Kayla.

My lovely, beautiful daughter, Kayla.  My lovely blossom.

I hope that living with your aunt has proven to be good move for you to make.  Sometimes, I still have a hard time accepting the fact that that is what is best for you, but in the end, I think that it was healthier for you to move away from here, just as it was healthier for me to stay.  I think we each have our own way of grieving.

For me, I needed to stay here and go through the motions of putting life back together.  The house is paid off, and the market isn't right for me to try and sell it.  I know this isn't the type of thing you want to here, but this is the type of thing that occupies my day-to-day thoughts -- ``What is the house worth?''  ``Could I get a decent deal on it in the market?''  ``Is it worth moving somewhere else within the same town to keep my job, or should I look for something else outside of this little neighborhood?''

I'm sure you're thinking about school, about the friends you've made, and about what you'll make, what you'll do.  Everything moves so much faster when you're younger -- you have less tying you down to one place, nothing except the grown-ups who seem to make all of the decisions for you.

I've been reading (I know, surprise surprise, right?), and I found your mother's old Bible.  The first few pages inform me that it is the ``Today's New International Version'', which seems like rather a mouthful to me.  I've never read the bible, so I don't know about these things.

Did you know that your mother was a spiritual person?  It's something that we had talked about early on during our courtship, but something which never seemed to come up during the time that we were raising you before her death.  She wasn't practicing, if I'm getting that term right.  She never went to church, and never mentioned her spirituality to either you or Justin as far as I knew.  I never stopped her, but it seemed that my belief that there was no God was stronger than her belief in God, because we wound up raising you to be agnostic, and decidedly not Christian.

When she died, when Karen died, my atheism was strengthened.  `How could any just God put any just (for so I thought myself at the time) man through such torture?'

I didn't know the story of Job at the time.

\ldots{}Well, okay, that's a bit of a stretch.  I didn't know the story of Job, true, but I also think that I greatly misunderstood a lot of what went on behind the scenes of Christianity.

I'd been raised by an agnostic and an atheist, and, similar to what I mentioned about you two, the atheism won out.  My father's staunch belief that there was no God was stronger than my mother's apathy, and that left me feeling as though there simply mustn't be a God.  This is the power that fathers have over their sons.

When Karen died, it was confirmation, in some small way, that my father was right.  No benevolent figure, however distant, could allow its creations to feel such pain.

I packed up so much of her stuff after that, because it was causing me so much pain just having it in my life.  You never knew her that well -- she passed when you were so young -- but you did seem to treasure the pictures that I left out of her, the one on my desk, the one on the mantle, and the one on the wall by the stairs.

You were so curious about her.  I would tell you stories about how we met, about what it was like raising Justin, about the life that we had together before we had you two.  But I never talked to you about her death, except to say that she died, and I never talked to you about her spirituality.

I didn't understand it.  I never had understood it, mostly because I had never tried.  I accompanied her to church once or twice, before either you or Justin were born, but neither of us were much on the act of attending a religious ceremony like that.  For her, it was a private thing, something that informed the way that she interacted with the world, but only on an internal level.

It was only after Justin's death and after you moved in with your aunt that I started to dig through the boxes that I'd hidden away from myself.

I found her bible, old and worn.  She had obviously read it quite a bit through the years, and I'd either never noticed or she had read it on her own when I wasn't around.

Finding that little book and seeing the dog-eared pages and the frayed bookmark ribbon sewn into the pages, I realized how little I knew about her and how her faith worked, and that in turn made me realize just how little I had tried to understand her.

Determined, I sat down and began to read.  I skimmed over a lot of the geneology bits -- I'd always found those so dry and uninspiring, and never understood why they were in there -- and I powered through a lot of the rules and dictates from the early parts of the old testament.  It wasn't until I started to get into the stories of the kings and leaders of the tribes of Israel that I learned more about what these stories actually teach.

I had to stop reading for the night once I came across this bit in 2 Samuel:

\begin{quote}
  The king was shaken.  He went up to the room over the gateway and wept.  As he went, he said, ``O my son Absalom! My son, my son Absalom! If only I had died instead of you -- O Absalom, my son, my son!''
\end{quote}

There is pain in the Bible.  Real, earnest pain.  The stories may just be stories, but the pain is there, and people hurt just as much as they do now.  I don't think I'm much at risk of becoming a Christian just because I leafed through your mother's bible, but I think I understand it all a little better now.

Stories like this -- all stories, not just biblical stories -- teach us how to feel more of the human experience than we feel on our own.  They teach us to know one another.

When I realized how little I actually knew Karen, it also made me realize how little I knew you, and how little I knew Justin.  The thing that drove him to take his own life isn't something I think I'll ever understand or know.  It's so hard to think about, Kayla.  So hard.  But I think that if there's one thing that I've learned from all that happened, it's that I need to try.  I need to try and understand and really know you, because I didn't, with Justin.  I didn't try hard enough, and I didn't really know him.  I really didn't.

I know that's hard for you, too.  I know it'll always be hard for us both, seeing what we saw and having lived through something no one but us can truly know, no one but us will believe.  I'm sorry for that, Kay-bear, I really am.

I will be honest and say that I have heard from your aunt.  I'm proud of you, Kayla, and all that you've been able to accomplish this last year.  I want to see you, I really do, but I understand how difficult it will be for you to come back and visit me here.  I'll make my way out there some day, out where it's clean and cool, out where I can walk with you down the block from the school to your house.

The house next door has been demolished, and some corporation has taken ownership of the land in some complex agreement with the homeowners' associataion that I don't understand.  I haven't read to deeply into it, truthfully, for reasons that I'm sure you can appreciate.  After the investigation, the police had the area cordoned off, and after the yellow tape disappeared, I couldn't bring myself over to the skeleton of a house, or the bare plot of land that it has become.

I know that this is hard for you, Kayla, that it must still be hard.  Your aunt Alice has mentioned that you have only just begun talking to her, talking to anyone outside of school.  I know that you're excelling in school, but I know that it's proved difficult for you to move on outside of school.

I know that it hurts, Kayla love.  I still hurt; I hurt every day.  I know that you and I share the problem of the doubt we get when we tell our story.  Even so, I want you to keep trying.  I want you to keep excelling at school, and I want you to keep trying to open up and make more friends, to open up to Alice.  No one means you harm, and everyone is rooting for you to feel better.

I'll see you soon, Kay-bear, I promise.  Keep drawing, little blossom.  I'll get things sorted out and I'll see you soon.

Love,

Daddy
\end{itshape}
