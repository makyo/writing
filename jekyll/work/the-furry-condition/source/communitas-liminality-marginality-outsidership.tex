The idea that furry is a slice of ordinary society is one well worth
keeping in mind. I wrote about it as my very first article on this site,
even. It's important to consider the ways in which we, as furries, are
not somehow separate from the rest of the world; furry does not take
place in a vacuum, as I believe I've said before. We are all members of
our own social structures both within and without this subculture, and
it's that mixture of individualities and social ideals that belong to
its members that help to make us who we are as a fandom

The very phrase `social structures', however, is telling, in that that
is precisely what some of us seek to escape by means of our membership
to this social group: structure. For many, furry is seen as something
apart from the social structures that surround them in their day-to-day
lives. That has come up several times before here, of course. I wrote
about
\href{http://adjectivespecies.com/2013/03/20/leadership-in-a-decentralized-subculture/}{leadership
in a decentralized subculture}, and JM and I have both written about the
intersection of furry and the wider cultures to which we belong, both in
terms of
\href{http://adjectivespecies.com/2013/06/17/an-argument-for-conformity/}{conformity}
and
\href{http://adjectivespecies.com/2013/06/19/an-argument-for-non-conformity/}{non-conformity}.
This puts us in something of an interesting - and ever-changing - space,
as furries. We exist somewhat apart from the wider cultural contexts of
which we are a part, though at the same time we cannot escape the
connections entirely, for they inform a large portion of the way our own
social group works.

This tension between conformity and non-conformity, belonging and not
belonging, being a part of society or rejecting it, is a type of
\emph{liminality}, exiting between states, on the threshold, and
certainly worth taking a moment to explore.

Let's take a step back and figure out what liminality is, along with the
closely related concept of marginality. Liminality (from the Latin word
\emph{limen}, meaning threshold) began as an anthropological term to
describe the process of ritual, wherein those involved enter as part of
the social structure, become something separate outside of but on the
threshold of that structure, before returning to society. This can
easily be seen in a simple ritual which has continued until today such
as marriage: those who are to be married enter as separate people, and
through the process of ritual, are socially, even legally, set aside
from the social structure during the ceremony, before they are
re-inducted back into society, this time as a single social (and often
legal) unit.

I noticed this myself recently with my own civil union ceremony: JD and
I entered as two separate people, and then, even though we were simply
signing papers for five or ten minutes, we entered a ritual sort of
liminality where we were not separate, but not together - one step
removed from society - before we were welcomed back by the county clerk
as a single, legally recognized couple, complete with an announcement
that got a small round of applause from the few others in the room.

At that point, following Victor Turner's definition, we were
\emph{liminars}: liminal entities wrapped up in the process of ritual.
However, the concept of liminality has been extended beyond the idea of
ritual in several ways since then.
\href{http://www.liminality.org/about/whatisliminality}{This delightful
essay} describes the ways in which the concept can be and has been
applied outside the context of ritual. Liminal states are all around us,
and a regular part of life. The author of the essay takes liminality far
beyond the ritual, as have others, and elevates it to state valid in
life, or even within aspects of life. There are ways in which we are
betwixt and between that tie into our lives quite a bit, setting us
somewhat apart from society into a sort of anti-structure.

This anti-structure, as a lack of the wider social structure, is
described as \emph{communitas}, which is a social anti-structure that
places emphasis on humanity, equality, and togetherness rather than the
hierarchies and strictures of society's more standardized forms. This is
evident in many social movements, such as feminism and the gay rights
movement, where, by virtue of this status of being set apart, elements
of - if not all of - social structure are set aside in favor of
communitas: a sense of ``we're all in this together'' and yet ``we're
still all human.''

In some sense, then, liminality is very similar to marginality, and
there are certainly discussions worth having on both subjects, but I
think it's important to first differentiate between marginality and
liminality as outsidership. I mentioned in the previous paragraph that
this often happens with social movements, and I think that this shows a
good example of marginality, in a way. Those at the edges of society
who, by their very existence, are set apart from society in some way
experience outsidership just as those in a liminal, between state do.
However, there is an important distinction to be made, and that's one of
choice. While liminality is often a something that one can choose to
take part in - the author of the aforementioned article chose to accept
his job in a foreign country, setting himself up in a state of
not-quite-beloning to both his native, western culture as well as the
Korean culture in which he was embedded - whereas marginality, as a
social sciences term, generally refers to those statuses which place one
outside of social structures through no choice of their own, such as
race, class, sexual orientation, and so on.

Of course, I'm sure you can see where I'm going with all of this. In a
way, furry itself, like many subcultures, is a form of outsidership, and
thus something of a liminal space. We experience our own communitas
within the fandom, and I think this is evident in a few notable ways.

The characters that we create for ourselves are, in a way, liminars -
items betwixt and between the two worlds of the imaginary and the real.
Yes, they are fictional constructs to many of us. There is no Makyo, per
se, only Matthew Scott and this idea of Makyo. And yet they are
expressed in the real world in several different ways. Art, fursuits,
role-play, and even just plain talking about characters (as in the
species selection and character creation panel at RMFC) is a way in
which we bring them closer to what we consider real. They are on the
threshold of both purely imaginary and totally real.

On a similar note, conventions are another good example, and a more
complex one at that. Cons are liminal spaces, wherein we, as a
subculture, experience our communitas more completely than perhaps we
might outside of them. We try to build the world that we want Furry to
be for a few short days, and we often do a pretty good job of it. One of
the aspects of communitas that I find interesting is that, by virtue of
this anti-structure, even leaders are still members, and so it is in
most cases with con staff and board: they are furries there to enjoy the
convention as well. And yet all of this takes place in the middle of San
Jose, or Pittsburgh, or Magdeburg. All around the convention, keeping us
from transitioning entirely to some other, more furry state, is the rest
of a bustling city that is not partaking in this communitas (and indeed,
often rejects it outright).

This applies to time as well as the space around conventions. While
conventions get closer to Turner's ritual definition of liminality, a
ritual setting aside of social structures in favor of communitas, so to
does the ritual of traveling to and from conventions. This year, on the
way to Further Confusion, I just happened to run into a few furries by
pure chance in the San Francisco airport. We even wound up on the same
train down to San Jose together. This, and so many experiences like it,
help to show the ritual nature of travel, the setting aside into a space
not quite society, where hierarchies are blurred and you're all just
Passengers, Travelers, or Pilgrims.

As I mentioned before, however, subcultures are their own kind of
outsidership. All of these things are not strictly furry, not even the
conventions. Any other group that gathers around a central idea such as
this has the chance to set themselves apart and yet still on the
threshold, in that between space. The anime culture has their own
conventions, interests, and communitas, as do so many other social
groups out there.

So how has furry changed over time?

A curious question that came up in the process of researching this post
is that, while it's understandable that the difference between
marginality and liminality is one of choice, how exactly that choice
works. That is, are there aspects of marginality to our fandom? Is it
marginal to be into something by virtue of personality, or not
understand the outsidership role interest plays in our lives? This is a
question that JM has touched on
\href{http://adjectivespecies.com/2012/04/09/geeks/}{before}, and I
think it's worth at least a look.

In some ways, geek culture as a whole, but also our furry subculture,
has been making a slow shift from marginality to liminality. No small
amount of words have been spilled over the topic of how nerds are in,
it's chic to be geek, \emph{et cetera ad nauseum}. However, that it is
so obvious is, I think, a sign of the roles that interest play in
choice. Is it a choice to participate in a subculture such as this? Of
course. One need not partake in the social aspects of interest to simply
be interested. Is interest a choice though? That is a tougher question,
I think, and I would hesitate to say so. It shows, then, that as
participation increases, the liminal aspects of interest - those based
around choosing outsidership - grow in their perceived importance, even
as the marginal aspects - those based around having outsidership forced
upon one - shrink.

This is simple membership draw, of course, and nothing mystical, but
interesting all the same, notably in the ways in which one reacts to
having one's outsidership acknowledged, or even challenged. There is a
great lead into
\href{http://warpcorecritical.wordpress.com/2013/07/17/science-fictions-queer-problem/}{this
article} about what it means to have sexual orientation (a marginal
state for some) acknowledged, and I think that similar reactions can be
seen in furry. The ways in which we reacted to MTV's Sex2k episode, or
the Salon article are different than the ways in which we react to
Maxim's recent nod to furries, and I think that, too, is a sign of us
feeling less marginal and more liminal: it's easier to feel proud of
outsidership that is freely chosen, because, to us, that outsidership is
eminently enjoyable, or even a core part of our lives.

This brings me to my standard conclusion (since I've already tackled
``is it furry?''): what does this get us? Liminality is a part of life,
whether we notice it or not. Often we do not, but it does form a core of
who we are: the ability to step outside, to gather in this communitas
with our equals, and to set ourselves outside social structure on the
threshold of real and imaginary, even if only for a time. Intentional
liminality such as membership in a subculture can help or harm depending
on the individual and how it's used, of course. We all know of the trope
of the furry so entrenched in the fandom that they cannot hold down a
job, pay bills, or interact well in social situations outside of furry
by virtue of their membership. However, furry is certainly of incredible
importance to a great many of us, and the form of escapism involved in
it is hardly unhealthy. We've created ourselves a space neither here in
society at large, nor, by necessity, there, in this fictional world of
our zoomorphized selves. It's a safe space, a space of communitas, that
draws us in.
