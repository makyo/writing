As mentioned before, I've been totally slammed by offline things over
the last few weeks. ~It's been crazy, it's been fun, and it's certainly
left almost no time for the writing process besides thinking in bed
before sleep. There certainly is a place for that in writing, however,
and so I hope you'll all forgive me for a post consisting mostly~of
introspection. ~Now that things have mostly cleared up, I hope that I'll
be able to get back into the swing of writing about the fandom in a less
navel-gazey way. ~Until then, here are three ideas that I've not been
able to get out of my head, recently.

\subsubsection{Process in Furry}\label{process-in-furry}

When I was working my way through my music composition degree, I wound
up fixating on one particular style of composition that has stuck with
me to this day. ~There are as many ways of writing music as there are
composers (many more, really), but one can discern general trends in the
process of creation. ~I've mentioned this before, actually, in the
\href{http://adjectivespecies.com/2012/04/11/meaning-within-a-subculture-part-1/}{introduction
to the article} on meaning within the fandom. ~There is the watercolor
method of writing, which I'm going through now: starting at the top and
writing until you're finished. ~In contrast to that, there is the
carefully sculpted architectural method of writing, where one creates a
blueprint then writes an article to match.

It's similar within music composition, and the style that I latched onto
was process music, which is something of a synthesis of the strictness
of form so important only a few hundred years ago and the freedom
implicit in the postmodernist ideal. ~Within process music or process
composition, one doesn't necessarily work with a form, but with a
defined transformation. ~One of the most common ways of enacting the
process is to come up with a set musical idea, a motive, and applies the
transformation to it over time in order to help construct the piece of
music. ~The use of the word `help' is key there; the idea of a
transformation in music is not a new one.

In the early-mid twentieth century, the idea of a transformation was
extended to the twelve-tone row (where one sets the twelve tones of the
western scale in a certain order and makes that a primary motive to be
used in the piece). ~One transformation is to simply shift the piece
over some number \emph{n}~where \emph{n}~is less than twelve (as a
twelve-tone row is a mode of limited transposition - more on this
later), but one may also take every instance where one tone in the row
goes up to the next tone and make it go down the same number of steps
instead, and vice versa; or to play the row backwards. ~Of course, these
are just transformations working on the same set of material; a very
strict process, as it were.

The process music that I found myself working with in my career follows
a much looser standard, playing with the motive much more freely, while
still applying a process to it over time. ~This was explained to me in
terms of music that I had already written, however, and as with most all
retrospection, it was something which I found applicable in many aspects
of my life, such as when one learns about archetypes and, on looking
back over their life, finds such scattered throughout, almost exactly
where one would expect them.

The way in which I found myself thinking about processes in furry was
within the context of conventions. ~When I interact with my furry
friends online, when I interact with my furry friends in person, and
when I interact with furries at a convention, I'm often struck by the
how we continue the wayss in which we socialize within the constraints
of the medium. ~Put differently, I feel that I interact with furries in
much the same way, no matter the medium, and all that happens is that I
tend to put the interactions through a transformative process in order
to fit them to the setting. ~The ways that I talk and move within the
fandom, and the shifting settings and participants aren't mere pixels in
some rasterized picture of my life, but more like vectors, something
purer that traces tracks through time (and I freely admit that that is
an enormously nerdy analogy).

I suppose a lot of this is fairly obvious stuff, but I find it all very
interesting, because of the correlation to music, another very important
aspect of my life. ~Indeed, the parallel can be drawn through most
aspects of my life, or even through trends in history. ~Mostly, though,
I've been thinking a lot about the idea of processes recently, though,
due to the recent familial conflagration that took place at our house
during the marriage, and all that lead up to it. ~It was easy to see it
as a single event, a goal. ~Then I started remembering the similar
gathering that took place at graduation, at various birthdays, and so
on, and it became a little clearer that life is more of a process that
we experience over time, rather than simple events taken out of time.

In the long run, I suppose we all deal with transformations of a motive
throughout our lives. We're bound, whether consciously or not, to
certain themes present in the world, and it's only the passage of time
that helps us to change or be changed by them. ~It's a little bit of the
old ``there is nothing new under the sun,'' to be sure, but it's also
heartening to think of the paths we make through each other's lives as
we live out the processes of our lives in proximity with each other.

\subsubsection{Evolution Within a
Subculture}\label{evolution-within-a-subculture}

If one were to take a step back from the individual paths that we make
in life and look at humanity as a species, it becomes clearly that we've
really got a good thing going on with tool use. ~We've been at this
whole ``living on Earth'' thing for quite a while now, and we seem to
have grown accustomed to our surroundings, or, failing that, grown
accustomed to making our surroundings fit our needs and wants. ~Sure, we
started small with simple knives of stone and bone, then moved up
through hammers and thongs to hammers and tongs, through stone to wood
and bronze, iron, steel, titanium. ~We've surpassed many other species
in a great many ways, arguably right up to species primacy. ~This is the
process taken to the utmost extreme.

Similar things happen within societies, when one takes a step back
inwards: civilizations rise and fall, and change with the times. ~The
Romans, they did great! ~Certainly a gold star for the republic, and
then the empire gets special marks for effort, to be sure. ~But they
aren't alone, of course. ~The Greeks, the Tsardom of Russia, the various
monarchies of Europe, and so on, have all striven forward and achieved
primacy in their own times. ~America did likewise, and even believed
strongly in its own exceptionalism for quite a while, and we shall see
where that leads. ~Needless to say, the same sort of evolution and
process holds true on a cultural level, as well as a species level.
~Neal Stephenson discusses this in many of his books - whether it's the
Chinese \emph{ti}~in his book \emph{The Diamond Age}~or the struggle of
societies in \emph{The Cryptonomicon}.

All of these struggles also surround tool use, in a way. ~The members of
cultures are tools of the culture, as are the things they create. ~Not
only did the individuals of the Revolutionary War help cement American
exceptionalism in the cultural mindset of the times, but the use of
inventions such as the atomic bomb helped to solidify them during times
of stress. ~I'm being a little glib, of course, but the point stands:
the use of what we're given in order to build with what we've got better
than the others describes much of human civilization, in the macro or
micro sense.

There was, however, one invention that, at least to some extent, changed
up the order a little bit, simply by virtue of ignoring the previous
geopolitical boundaries already in place. ~The Internet's a great and
grand thing - where would {[}a{]}{[}s{]} be without it? - but it's
shifted the race to primacy, at least in terms of social stability, one
step closer to the individual from species and culture. ~The subculture
is something that surely existed before the Internet, of course, as one
had such things within occupations and hobbies, but without necessarily
the same ease of communication. ~With the advent of the communications
age, the subculture gained a greater deal of prominence within the lives
of those so enabled. ~A hobby moved beyond something one might do with
close local friends and by oneself in the basement, and into something
one shared with like-minded individuals with a fervency that was
magnified by a technology that mostly just aided in communication using
written language for a good deal of its existence.

Taken that way, the contiguous furry fandom these days has a lot going
for it. ~We know our tools very, very well.

Furry fits in nicely on the web: by virtue of having much of the primary
purpose of its existence based around socialization, role playing, and
communication, a medium that lends itself particularly well to such
things was quite the opportunity for the growing fandom. ~It's not
simply that we're all tech-savvy individuals, as that's demonstrably not
the case, there are weekly journals in my own watch-list on FA and daily
statuses on Twitter made by furries requesting tech-help. ~Simply being
savvy with the underlying technology isn't what makes all of this so
useful to us, no, it's tied into something deeper, something which will
help to ensure the stability of our subculture in many ways. ~Furries
are savvy, instead, in the concept of social currency within the context
of their fandom.

The whole idea of social currency suddenly became much more important
with the invention of the 'net. ~One could have all the money in the
world, or only enough to afford the means with which to communicate on
the 'net effectively, and one could become rich in social currency: the
sharing of ideas and words with those seeking them out. ~It's a little
bit cynical, perhaps, and not very flattering for us, but
{[}adjective{]}{[}species{]} acts in its own way within that structure,
bolstering its own social currency by providing the ideas contained by
the authors, both of articles and comments, to a wider audience - not
simply forcing it on them by way of intrusive advertisements, but by
making it a genuine resource available to those in search of it. ~We do
our best to earn our social revenue, but we are, when it comes down to
it, actively seeking it out.

Furries seem to be all about this, too: there are paid sites with
limited-distribution furry images and stories, comics available only in
hard-bound format, and countless individuals seeking profit in the more
standard sense. ~However, for every image that's available only in a
paid format, there are tens, hundreds, perhaps even thousands of images
freely available to a wide audience through venues provided free of
charge. ~And just as some form of man grew and rose to some form of
species primacy, just how some forms of government grow and rise to some
form of primacy in their respective times, the fandom is growing and
rising into a space that sometimes seems made for it (avatars in
SecondLife, anyone?). ~We're evolving to fit the environs and growing in
stability as we do so.

\subsubsection{The Self-Aware Fandom}\label{the-self-aware-fandom}

I know that I have written about the idea of the contiguous fandom
before, as that which is made up of those who identify as members of the
fandom such that a semi-coherent group is formed. ~It's worth mentioning
that in many cases, the idea of `identity' is used to describe something
that is pathological, or differing from the norm. ~For instance, I
brought up the idea of basing a portion of my identity on my successes
with my psychologist, and we wound up spending the next several weeks
talking about what exactly could be causing such a problem. ~It's not so
much that we have identities, of course, we all do, but that when we are
conscious of our identities, it's indicative of some pathology,
something differing from the norm, or some dis-ease; we may always
identify as male, but when the idea of gender identity rises to the
surface and occupies one's thoughts unbidden, then we start thinking of
gender identity disorder.

Doubly interesting, then, that furry has become a matter of identity.
~It's been brought up on twitter, at least, that many within the fandom
may feel some sort of species dysphoria, or dissatisfaction or
depression associated with the feeling of being the wrong species.
~While I went through a period wherein I would have agreed to that, I
don't think that's the case for myself anymore, and I'm not sure that
describes a majority of the fandom, either. ~I think we have something
subtler and more interesting going on with the fandom. ~It seems a
simple thing for us to say that we are furries, and yet {[}a{]}{[}s{]}
is only the most blatant instance of furries exploring or attempting to
explain furries, even to other furries, never mind the world at large.
~Perhaps it's a symptom of the participation mystique I've brought up
before, and perhaps not, but it's worth exploring either way.

The idea of a furry identity is consistent with even a cursory
observation of the contiguous fandom. ~The two examples that seem to
show themselves most clearly is the combination of apologetic and
defensive attitudes in regards to adult content, and the self
deprecation that takes place in so many of the social outlets as favored
by members of the fandom.

The first of these, I believe, is due in part to a sense of just how
loose-weaved the fandom is ~perceived to be by its adherents. ~What
appears to be a split between those who are avid consumers and producers
of adult content and those reject that it is a large part of the
~experience of being furry may in fact be so visible because of the
simple perception that there is great diversity in the membership of the
subculture, and the whole gamut between porn-obsessed freaks and those
who are either most innocent of or staunchly opposed to the adult
content that exists within the fandom. ~This site is not the only outlet
of meta-furry content out there - I see fairly regular journals and
mention of many of the topics we've covered and will cover here. ~Furry
is something we obviously spend a lot of time thinking about, it's an
identity that doesn't necessarily always sit naturally within our
concepts of self and how we interact with the rest of the (non-furry)
world, and perhaps that's due to the social nature of what much of furry
has become.

As for the second example of self deprecation, I've been watching waves
of the hashtag \#furriesruineverything wash over twitter over the last
year or so. ~It began as simple snark, implying that furries really did
take everything, turn it terrible, and set it loose on the Internet, but
it's since gained additional layers of meaning. ~It's been inverted to
add some sarcasm to the mix - furries ``ruin'' everything, by making it
better - and it's been reverted back to the idea that furries can ruin
even things that aren't necessarily furry in the first place, such as
Twitter, kids shows, and so on. ~This is only the simplest and one of
the most blatant examples of the self-deprecation that seems to move
through the fandom, and it's occasionally found itself tied to the first
example through off-color remarks about how most furries are sexaholics,
but we love them anyway.

What does it mean that we are all occasionally a little uncomfortable in
our membership with this subculture? ~It's one of those questions that,
yes, is another sort of process, the type of question we're continually
finding new and better answers to, the type of process that continues to
define who we are, hopefully toward the more healthy end of the
`identity' spectrum. ~It seems that, for a majority of those involved,
the fandom has at least provided a positive influence on life, whether
or not it makes us a little too conscious of the portion of our
identities we've based on it. ~I know I wouldn't trade myself now for
who I might be without the fandom, ever.
