I pulled together a few additional ideas on the concept of character
versus self visited in a
\href{http://adjectivespecies.com/2011/11/23/character-versus-self/}{previous
post}. ~A lot of these rely on little ideas dropped here and there by
comments either on the blog itself or on Twitter. ~They're all kind of
neat, but none of them really warrant a full post. ~I pulled together
these three smaller ramblings here into one larger post in the hopes
that I can still get my thoughts out there on the subjects. ~Enjoy!

\subsubsection{Acting}\label{acting}

I was turned on to Erving Goffman by
a~\href{http://adjectivespecies.com/2011/11/16/boys-girls-and-the-in-betweens/}{commenter}
recently and found out a little about his ideas on the presentation of
one's self (mostly through secondary sources, full disclosure). ~Goffman
describes our social interactions as ``front stage'' and ``backstage''.
~Each person in a social group is an actor utilizing their props and
their role to present a positive image of themselves to their audience,
who are, of course, actors in turn doing much the same. ~This is the
front stage aspect, whereas the backstage aspect is more the idea of who
we really are outside of the social play, where we can ``deconstruct our
personas''. ~This impression management is a sort of ``artificial,
willed credulity'' (or, more glibly, consensual hallucination, a phrase
used by William Gibson, who used it when he coined the term
``cyberspace'').

In a lot of ways this concept fits in well with furries. ~Of course,
there is the surface aspect that our frontstage aspects are much
stronger in that they differ greatly from our backstage personas - I
don't have any specifics on the numbers, but I'm pretty sure that very
few of us are anthropomorphic canids sitting in front of a computer.
~Beyond that, however, the idea still holds: speaking from personal
experience, we interact with other furries very differently than how we
interact with others, and that persona that we present to our cohorts is
a strong one, often considered freer and more true to ourselves than our
other roles, but still something different from our true selves due to
the whole thing being only a portion of our personalities. ~It's not the
whole of our self that we present to our furry companions.

Whether or not the concept is strictly applicable is up for debate in my
mind, however. ~In terms of the first impressions mentioned earlier,
there seems to be this additional layer of role-playing, as if our front
stage personas were acting about being actors in a play, and the line is
blurred further when bits of information about ourselves, as well as
aspects of our other personas, are injected into our avatars via these
other layers of our channel of communication. ~I suppose that it's for
this reason that the Internet would be considered largely a backstage
environment, or at least has the potential to be such. ~The reality,
though, is that we construct our characters just as thoroughly online as
we do in real life, if not more so, with it being a conscious effort.
~They are our avatars, yes, but they are also constructed personas used
for interacting with our environment in the context of a social
structure. ~Goffman's idea of stage and backstage is more useful in
considering that, as we interact with each other within our subculture,
we're presenting ourselves in a certain way, acting a part for our
audience, yet also giving them a glimpse into our backstage lives due
not only to our interactions spanning the online and offline arenas, but
also due to the fact that our constructed personas are blatantly not
ourselves.

\subsubsection{Attention}\label{attention}

The Internet hasn't been all roses and sunshine. ~Since its inception
and rapid growth, several problems or perceived problems have been
associated, fairly or unfairly, with the liberal interconnecting of
people by technology. ~As the web increased in importance, so these
problems increased in visibility. ~One of the more interesting of these
problems is the interestingly named Münchausen by Internet. ~This is
when someone will feign a severe illness or disability for themselves on
the Internet in order to garner attention for themselves. ~It's not
quite hypochondria, which is a separate disorder, and it differs from
regular~Münchausen syndrome in my mind in that, while there's some
discussion as to whether the latter is a conscious drawing of attention
to one's self, the former requires much more forethought in order to
keep up - it is either a gross exaggeration of reality or an outright
lie.

Along the same time as the Internet was coming into its own as a serious
technological innovation, my own generation was reaching middle- and
high-school age, and the age of the over-diagnosed psychological
disorder was gaining steam. ~Friend after friend of mine was diagnosed
with Attention Deficit Disorder or Attention Deficit Hyperactivity
Disorder, manic depression, bipolar syndrome, or some other item from
the mild side of the DSM-IV (I'm speaking glibly, of course). ~I knew of
a score of classmates on Prozac, as many on Ritalin, and a few on more
extreme drugs such as Lithium.

I went through my own period of depression and restlessness, but my
solution was to hide it from my parents and escape the best way I knew
how: get online. ~I know I wasn't the only one, too. ~When I was first
getting into the Internet, I associated with many of the same age as
myself on a previously mentioned forum, and when I got into furry, I
wound up on FluffMUCK, again in the company of several other high-school
kids around the same age as myself. ~I battled my depression with
electronic affection and fought my restlessness with\ldots{} or, well, I
enabled any ADD tendencies I had with either wasting time online or
thinking about wasting time online. ~A rough childhood, I assure you,
growing up in upper-middle class Colorado with two engineers for
parents.

By the time I got to college, however, these three things - the ability
to garner attention through lying or embellishing illness online, the
over-diagnosis of youth, and the dipping mean age of the furry fandom -
had coalesced into the strange amalgam that is The Furry Disorder. ~The
Furry Disorder seems to shift as time goes on - originally, it was
bipolar syndrome or manic depression, then it shifted to ADD, and now it
seems to be Aspergers syndrome - but the distinguishing aspects seem to
be that it is often a loosely diagnosed disorder among furs in their
teens and twenties and is easily used to gain attention online. ~It's as
if a segment of the fandom agreed that the best way to gain reinforcing
attention from others was to latch onto this one disorder and capitalize
on it as much as possible. Synchronized~Münchausen by Internet.

It should be noted, however, that all this pales in comparison to how
amusing the term `cybermunch' is, in referring to someone partaking in
or suffering from Münchausen by Internet.

\subsubsection{Self-Importance}\label{self-importance}

I've mentioned the Dunning-Kruger Effect before. ~Briefly, it's the idea
that those who are less competent are more likely to overrate their
competence, while the opposite is true for those who are more competent
- they are more likely to underrate themselves. ~And boy howdy, am I
prime example of this.

It seems as though every fur goes through some creative phases, due in
part to how much the fandom itself is centered around creativity. ~Which
phase is most popular seems to change with time. ~When I first really
got into it way back in high school, everyone was drawing - to be a
furry, you had~to draw (while that's still popular now, it seems that
the thing to do now is make your own fursuit, and a few years ago, it
was making your own website). ~I was\ldots{}not good. ~I was very bad,
actually. ~It wasn't so much that I lacked a sense of proportion -
though I definitely did - or that I had very little sense of light and
shadow - though I had none at all - rather that I thought I was pretty
awesome. ~This was back when Yerf! reigned supreme in the furry art
world, and it was a struggle to even get on VCL. ~I was most definitely
convinced that I could get onto Yerf! with ease. ~I mean, look! ~I could
draw foxes! ~Foxes and foxes and foxes!

Of course, I was rejected.

When I say I was bad, I'm sure some of that is a bit of the old
Dunning-Kruger effect, because, while I was really actually bad, I was
getting better. ~Here's a bit of a progression from what I could find
(having destroyed most of what I could find years ago):
\href{http://us-p.vclart.net/vcl/Artists/Matt-Scott/YT_c.JPG}{mid-2000}
-
\href{http://us-p.vclart.net/vcl/Artists/Matt-Scott/Matt011.JPG}{late-2000}
-
\href{http://us-p.vclart.net/vcl/Artists/Matt-J-Scott/dancingfox_c.jpg}{2002}.
~I did draw quite a bit, and with experience, I was learning more and
improving. ~As my skill at creating improved, so did my skill at
appraising my own work, and I started to see more and more problems with
what I was doing. ~This is a theme that's been repeated a few times in
my life; nothing was more detrimental to my compositional output than my
composition degree: the more skill I gained as a composer, the less
competent I felt.

An interesting side effect of this is how protective I felt of my work
early on as I was working on it, and this is something I've noted in
others, not just myself. ~Every one of my drawings on those early VCL
accounts was marked ``(c) me'', which sounds pretty silly to me now.
~Silliness aside, though, I know that in the early stages of creative
growth, whether in music or art, I was so confident in it that I was
eager to copyright everything, whereas once I started to gain more
skill, I was more and more willing, even eager to use less restrictive
licenses such as Creative Commons licenses. ~I know I'm not the only one
to work this way, too; I've watched several artists within the fandom
change similarly over time - the better they got, the more professional
their attitudes, the harsher their critiques of their own work, and the
more varied (though, of course, not necessarily more liberal) their
attitudes toward the licensing of their creations became. ~In fact,
there seems to be a point in most artist's career in the fandom - lets
call it the Pre-Popufur Point - some loosely defined point in time that
is penultimate to their going in one of two general directions. ~As
their skill progresses and they get better and better, the chances that
they'll approach this point increase, and here they will either become a
popular (to whatever extent) furry artist or head the direction I did,
feeling less and less comfortable with their work until they stop, or at
least slow drastically.

Of course, everyone's different, and not everyone reaches this point at
the same time or perhaps even at all. ~There are plenty who never start
drawing because they're preemptively~hard on themselves, and there are
those who draw and increase and keep a positive outlook on things.
~There are those who invert their views on intellectual property, or
those that maintain a firm grip on their art throughout their career.
~And, lest we forget there are those who are so relentlessly polarized
in their opinions as to warrant the creation of the LiveJournal
community Artists Beware. ~Even so, the general trend of the
Dunning-Kruger effect is deeply ingrained in the fandom's art culture,
and, with our unique focus on the visual representations of our
characters, seemingly more visible than in society at large.
