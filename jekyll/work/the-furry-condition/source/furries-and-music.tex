Furries and music definitely have a thing going on. I've wanted to write
about it for quite a while now, but I've never quite found the right
entry point, the right way to piece together a story about how the two
might connect. I actually started thinking about the current topic when
Klisoura of Furry Survey lore was in town over the week between
Christmas and New Years, and the topic came up of how furries have a
tendency to consider themselves ``ahead of the curve'' when it comes to
music, television, and video games, or even trying new things, yet do
not necessarily consider themselves to be hip or paragons of pop culture
(\href{http://vis.adjectivespecies.com/microsurvey/2012/}{ref}). While
I'm really as much of a fan of new music as anybody out there in this
subculture, I wasn't quite sure how well this held up. What the data
seem to be saying is that furries showed a tendency to eschew popular
culture in favor of the type of things that would become popular
culture. While some of our number may fit within that category, it's
oddly specific for a subculture that doesn't, at its roots, have as
necessary an intersection with popular culture as might, say, the fans
of an actual genre of music, television series, or video game.

A survey is a survey, though, and can only really tell so much about
those who really should be telling the story. I turned, instead, to
Twitter, and invited an email barrage on myself to see what those who
had the stories to tell had to say about the matter, asking ``Do you
think furries are more or less musical than non-furs?'' and ``Do you
think furries are ahead of the curve in terms of music?''.

Let me take a step back and say that I've always been kind of fascinated
with the relationship that furries have with music. I spent the time and
money (lots of the latter) needed to get a bachelor's degree in music
composition, so I've always been, as I glibly put it on Twitter, super
into music, and so I'd always wondered if maybe it was the crowd that I
hung out with that was influencing my perceptions of furries as rather
musically oriented folks, or if maybe it was just everyone. Another
thing that piqued my interest, however, was the visible importance of
music at conventions and in every-day chatter. The latter could be
explained away by the fact that a lot of folks within furry aren't going
to spend every second role-playing, of course, they're going to have
conversations about the things that interest them, and music is a
natural topic even outside the fandom. The former, however, intrigued
me, even after I started regularly attending conventions. There were
dances every night. There were dance competitions, dance competition
try-outs, dance competition out-takes, dancing in the fursuit parade,
dancing for no reason. Music seemed to be everywhere, from panels to the
dealer's den, and it all made me pretty happy, if curious.

Furries, like everyone can be broken down into two, very rough,
categories when it comes to things like music: creators and consumers.
The act of creation plays a big role within the fandom, of course. Given
that we are, as was famously put, ``fans of each other'', we rely
primarily on our own membership to create the art and stories
appreciated within our subculture. Within music, however, things are a
little more gray. The question of whether or not there is such a thing
as furry music and what might define it is one for someone else to
answer, but needless to say, there are still plenty of furry musicians.
There are several out there that create music within the context of
furry, post their music to FA, or perform at conventions (such as the
jazz combo SuperPack at FC a week and change ago). ``{[}T{]}he
environment seems more conducive to the sharing of content in general,
music included. Furry musicians have a built-in audience they can reach
that many other aspiring artists might not.'' Vincent writes, and I
think this is an apt description of at least part of the reason there is
a music scene within our fandom, or indeed within many subcultures.

There's one more smaller subset we should probably take into account
given the popularity of dances and the like at cons, not to mention the
relative popularity of electronic music within the fandom, and that's
the wide variety of furry DJs out there. The reasons for the popularity
of this pursuit are varied, and hinted at by several of those who wrote
back. Technological aptitude, diversity, a focus on sharing, and
interest in EDM (electronic dance music) as trends within our subculture
may help guide many toward DJing as a mode of expression, and notably as
a way of sharing things important to themselves.

Beyond simply creating or creatively mixing music, though, we are avid
consumers of music, at least commensurate with our strongest
demographics. \href{http://twitter.com/sotopnthr}{Soto} writes, ``From a
consumption standpoint, I haven't found furries to deviate much from
their non-furry counterpoints in the same demographics. For example, age
group. Furries as a whole may be more passionate about music and stay
more current with trends, but furries as a whole have that lovely
age-skew toward the late teens and twenties, and that age group is
generally pretty up on their music as it stands.'' That is to say, we're
helped along by some of the categories that many of our members belong
to in listening to and exploring music with the sort of enthusiasm that
goes along with connoisseurship.

So, what about my two questions? As hinted about in the previous quote,
opinions are mixed on the question of whether or not furries are more
musical than their non-furry counterparts. In fact, after reading many
of the responses, I don't think the question should be whether or not
furries are more musical than their counterparts, but whether or not
they have the conception that they are.
\href{http://www.furaffinity.net/user/zenuel/}{Zenuel} offers, on the
positive side, ``I like to think that the fandom simply offers more open
and honest states of being{[}\ldots{}{]}; a furry posts to a more
receptive community like FurAffinity they generally receive more
encouraging feedback, as well as having the backing of freedom that the
fandom presents to the artist in question.'' Vincent acknowledges this,
but warns, ``This is a pro and a con, I've always seen furry as
something of a `hugbox' where criticism isn't forbidden, but it
certainly isn't forthcoming. I've found that (at least in the realm of
DJing) it's very, very hard to get good technical feedback on how to
improve, and in many instances subpar mixing is lauded as exceptional.''

One advantage that we do have that we gain from being a decently
coherent subculture is the fact that we are rather diverse in ways
unrelated to some of our stronger demographics. That is, age and gender
aside, our diversity in terms of backgrounds, social status, education,
and so on does help us with the ways in which we deal with music. As
Wolfdawn put it, ``just being part of a diverse and unusual subculture
would have to be a big {[}plus{]}, since that alone makes people more
likely to have been exposed to wider range of musical interests as
they're shared among friends.'' I noticed a similar effect outside of
furry when I moved away from my rather homogeneous upbringing and high
school to college, where much more diversity was to be found. College
was where I expanded beyond my own choral background into genres,
classical and not, far beyond what I was used to. Furry was much the
same, and in fact, much of this article was written listening to a
playlist composed almost entirely of music suggested to me by cats,
dogs, and all sorts of fuzzy creatures. In other words, are we more
musical than the non-furries that surround us? Probably not. However, do
we consider ourselves more musical than those around us at least in part
because of furry? Often times, I think so, and a lot of these responses
echo that sentiment.

As to the second question, you'll note that I put ``ahead of the curve''
above in quotes. These weren't meant to be scare-quotes, necessarily,
but I would like to highlight something before I get too far. It's
always very important to pay attention to the ways in which language is
used. I know, I write about words a lot (using, of course, as many words
as I can), but when I responded to the onslaught of emails with the two
questions, I tried to do so using language that would invite people to
provide longer, rather than shorter answers, because I think that the
thoughts of those being asked are much more interesting than simple
yes-or-no answers on the subject. It's the way that people interpret the
questions they're asked, sometimes, that provides a lot of the answer. I
understand that ``ahead of the curve'' can be a little misleading in
terms of being able to provide a concise answer, and I'm sure I could've
worded it better besides, but the answers I received in reply more than
made up for it in their thoughtful and well-put responses.

Are we ahead of the curve? A lot of folks who replied indicated that no,
we're not really all that ahead of the curve, at least not moreso than
we might necessarily be given some of our demographic skews. There are a
couple of reasons behind this, and one of the big ones is that the
Internet and mass media in general hasn't benefited only furries. ``The
increased visibility of various scenes took away the relative advantage
having a community that encourages sharing,'' writes Vincent, and this
is echoed by a lot of my own perceptions: my composition professor went
on a `where is the drop?' joke spree with almost all of his students
once dubstep became a more visible part of the music scene around us
(the idea of being separate, here, due mostly to the fact that we were
being classically trained in composition). That aside, however,
\href{http://www.lionhearted.ca/music.htm}{Branwyn} suggests that many
``are in the same arena as non-furs - they consume music in the same
way, influenced by the same sources, regardless of quality.'' That is,
being furry does not necessarily influence the ways in which we
appreciate music, so much as some of the content that we listen to. We
listen to the things our circle of friends listen to, in all
probabilities, and I believe that much the same happens when it comes to
visual art, for that matter; we don't enjoy visual art that much
differently (though we do sometimes place quite a bit of importance on a
visual representation of a character -
\href{http://adjectivespecies.com/2011/11/23/character-versus-self/}{ref}),
so much as enjoy the things that our chosen family and circle of friends
also enjoy.

A possible explanation for all of this is offered by Forneus: ``Furries
are, I would argue, more musical than the mean, but not moreso than
other geek subcultures.'' We are, of course, not the only subculture
based almost solely around a shared, intense interest. The My Little
Pony fandom has created a wealth of their own music, not to mention
filking, which as a long and well-established tradition. Several of
those who responded to the questions touched on the points of geekdom
and technology, along with their ties to the fandom. One respondent
talked at length about the fact that there are readily available tools
on the market now, and, despite the fact that many, given such tools,
will create music that might not be the best in terms of musicality and
technical ability, they are still creating quite a bit (my own
experiences with Reason are a testament to this, of course). ``I think
that if you put the tools in front of furries, they are more willing to
try creating music than regular people,'' echoes Nathaniel Hahn; this
does well at pointing out the fact that, rather than being more innately
musical or musically hip, we may simply be focused on putting something
out there given the tools we have for our subculture to enjoy.

Satori sums it up well, ``We have geeks of all kinds, and some geek on
their music. Others are too into geeking on other things that they don't
make the time for it much.'' We're just us, in the end. We're a good mix
of musical and non-musical fuzzies, no more or less of a mix than the
world at large. We have things working to our advantage, such as our
broad social circles, diversity, geekdom, the Internet, and so on, but
no matter how large a part music plays within the fandom, we're still
just us, and some of us will create, and others will consume. We're no
less interesting for being a good mix, of course, and music does still
appear to be quite important to us, but in the end, we're plenty good at
focusing on being and appreciating animal folk.
