This weekend, I had the privilege of helping facilitate a panel at Rocky
Mountain Fur Con 2013 surrounding the topic of species selection and
character creation. The panel was a delightful discussion about the ways
in which we build up the avatars we use to interact within our
subculture, and why exactly it is that we choose the animal (or animals)
that we become with our character (or characters).

That's not all, though. I also had the privilege of sitting down with
Klisoura, {[}a{]}{[}s{]} contributor of Furry Survey fame, and having
not only several delightful discussions on topics as diverse as tennis
balls and coyotes, but also a little impromptu hack-a-thon in the hotel
lobby on the subject of species selection. This tied in well enough with
the panel that some of the results of that were shown during the Q\&A
after the discussion, and even led to several other conversations with
various different furries over dinner and the next day. The whole
weekend was a blast, but I'd like to tie up some of these conversation
threads and ideas into something worth showing here on {[}a{]}{[}s{]}.

The title of that particular panel was the same as this post, ``Species
Selection and Character Creation'', and was intended to be something new
for me, and, I felt, relatively new for the convention as well. Rather
than sit behind the table at the head of the discussion room and dictate
a set of ideas to an audience, my goal was to re-arrange the chairs in
the room into a circle and have everyone participate evenly in a sort of
Socratic-style exploration of species and avatar. However, given the
hour of sleep I'd had the night before, it worked out somewhere in
between. While the Socratic ``asking questions to receive answers
everyone already knows about themselves'' part worked out pretty well, I
wasn't able to make real the truly participatory experience of everyone
being able to see each other. I offer this as an explanation for not
simply posting the audio from the panel itself, though it was recorded.
If I get around to mastering the audio well enough to make it
presentable, I'll post it here and make note of it. I think it's worth a
listen!

I began by asking the room full of furries why they chose the animal
they did for their species, and I received a lot of answers that fit in
well with my experience of the fandom. Notable among the explanations
were the oft-used words `identity', `connection', `personality', and
`characteristics'. And this, of course makes sense. Many introductions
to furry, whether they're websites (the first introductory website I
found was Captain Packrat's explanation of FurCodes) or friends, explain
that although furry is about being a fan of anthropomorphism \emph{in
general}, it often (but not always) specifically involves a personal
connection with an animal that leads to the creation of a personal
character: an avatar often used in interaction with other furries.

We all know this, of course, but it's always interesting to see the data
bear it out. A discussion with Klisoura prior to the panel led to an
experiment: is such a thing visible in the answers provided by
respondents to the furry survey? It turns out that it is, in its own
way. On the survey, users are asked the species of their character or
characters, and then given room to provide an explanation of just why
they chose the species they did. Free-text answers are hard to parse
down into simple one-way conclusions, and are not necessarily available
to be shared as they stand. However, we can draw conclusions about the
use of language itself within these answers, and in this instance, we
did so by means of one of the simpler means of textual analysis:
frequency counts.

We've analyzed the responses for many of the most popular species
represented in the responses to the 2012 Furry Survey. Breaking this
down by species not only helps us spot keywords such as mentioned above,
but also helps us see where additional words, especially emotionally or
spiritually charged words, are used when identifying with particular
species. Let's start out with one of the easier ones, for huskies, where
I can point to a few of these words in particular to explain what I
mean:

\begin{figure}[htbp]
\centering
\includegraphics{/wp-content/uploads/2013/08/strip_husky.pdf.png}
\caption{Husky word-cloud}
\end{figure}

We see our previously tagged set of words such as `traits',
`personality', and `always' (left in* because it often shows up in
constructs such as ``I have always felt like I was a husky''). However,
we can also see several emotionally charged words such as
`love'/`loved', `loyal', `cute', `playful', and `beautiful'. These
figure strongly as compared to other marked words such as `cool',
`hard', `submissive', and `spiritual'. Contrasting this with the cloud
for wolves shows the difference in species selection:

\begin{figure}[htbp]
\centering
\includegraphics{/wp-content/uploads/2013/08/strip_wolf.pdf.png}
\caption{Wolf word-cloud}
\end{figure}

Here we see a shift in the tagged words to `connection', `identify', as
well as `personality', which I think shows a different attitude used to
approach the problem of species selection when creating a character.
Indeed, we see that `spirituality' figures more strongly, along with
`pack', `strong', `spirit', and `one'/`alone', while `loyal' and
`social' are deemphasized.

Another interesting thing to note is that, among the several species**
we pulled from the database, some are more strongly marked, such as the
previous two, and some are not. Those who chose dragon as their species,
do so for many, many different reasons than wolves or huskies.

\begin{figure}[htbp]
\centering
\includegraphics{http://adjectivespecies.com/wp-content/uploads/2013/08/strip_dragon.pdf.png}
\caption{Dragon word-cloud}
\end{figure}

As you can see, there is less polarization around certain terms, both
emotionally marked and the previously tagged words; that is, the cloud
is more homogeneous. There are a few potential reasons for this. One is
the possibility that dragons have cultural ties to more than just
western culture. Wolves have both a strong mythology surrounding them in
the west, as well as the advantage of being important in current events,
given the re-homing and conservation efforts surrounding the species in
North America.

While dragons do have a mythology attached to them in the west, it's
very different than their Eastern interpretations, which will lead to
less strongly-marked words and phrases showing up in analysis due to a
wider spread. Additionally, while dragons are certainly prominent now in
fiction words, they are not nearly as prevalent in current events
outside of that setting.

These are just some examples, but I think it goes to show that there are
indeed some trends, both general and specific, that go into species
selection among furries. That's only part of what goes into the creation
of a personal character, though, as I think we might achieve some
similar results by asking ordinary people to justify their choice of
their favorite animal. Thus, during the panel, we also discussed the
processes of character creation, growth, and change.

One exercise that I think works well is imposing artificial
restrictions. This was, after all, one of the foundations for the
literary group Oulipo, of \emph{A Void} fame (\emph{A Void} being a book
written originally in French entirely without the letter `e', and then,
perhaps even more impressively, translated into English with the same
restriction in place). By imposing on ourselves restrictions, we reduce
the problem of unfettered, and thus directionless, creativity. In that
vein, I asked participants to describe their personal characters -
fursonae, if you will - in one sentence or less. The results are
telling:

\begin{quote}
My persona is a reflection of myself ahead in life which I can use as a
goal.
\end{quote}

and

\begin{quote}
My fursona is an extension of myself as I move forward in life.
\end{quote}

Some were more verbose and specific along these lines:

\begin{quote}
It's a coping mechanism, a way to become someone else and not deal with
tough times, or even provide an outside perspective on them.
\end{quote}

and

\begin{quote}
Who I strive to become, always a step ahead of me; as I gain attributes,
my character stays one step ahead of me. It is my role-model.
\end{quote}

Some people got even more creative:

\begin{quote}
The person with whom I speak.
\end{quote}

or

\begin{quote}
Convenient, exaggerated wish fulfillment.
\end{quote}

or simply,

\begin{quote}
Me.
\end{quote}

The theme of ``a better version of me'' was repeated quite often when
discussing both the ways in which characters are created, and the ways
in which they change. I really think that this reflects well on us as a
subculture. A lot of my focus, when interacting with other furries, is
centered around being what I see as an ideal version of myself, as well
as just a fox-person. Some of that's simple and mechanical: ``I wish I
were able to more clearly express my ideas'' and ``I wish I were more
glib, quippier'' are both aided by social interaction through a
text-based interface such as one might find online. Beyond that,
however, by being able to have this version of myself that is better
than me, I, as others mentioned, have something to strive for, something
to grow into.

Discussion along these lines continued after the panel itself, as a few
of the attendees convinced me to head out to dinner rather than straight
up to bed (thanks for that, it was the first real meal of the day).
While we ate, we talked about what people took away most from the panel,
and also came up with a few additional ideas to help tie together the
two ideas of species and character.

One thing that came up was the idea that some gentle joking about
species, a sort of
\href{http://tvtropes.org/pmwiki/pmwiki.php/Main/LampshadeHanging}{lampshading}
of stereotypes, helps to reinforce species identity with regards to
character. Much, if not most of this, as pointed out by Klisoura later
on, is self-deprecatory. This helps to forge familiarity between people,
especially among members of the same subculture, or even sub-groups
within that subculture. Making fun of the chase-instinct in dogs by, as
my roommate (a husky) puts it, ``huffing the scent of a new can of
tennis balls'', or the face-first pouncing of foxes lending to the
overall silliness of the species helps not only to strengthen one's
identity with that species but also to provide a conversational starter
among friends, or friends-to-be. This can, of course, be mis-applied or
simply go too far. The idea that wolves are a dime-a-dozen, or that
foxes are all ``sluts'' are complex and sometimes self-reinforcing
stereotypes that, by virtue of their being stereotypes, can rub many the
wrong way and cause no small amount of offense.

We also noted another interesting conclusion from the panel. Every time
I run the ``Exploring the Fandom Through Data'' panel, I bring up the
idea of doxa - that which we accept as truth without requiring proof -
and how sometimes it needs to be challenged when that which is accepted
is not necessarily true. For me and several others, one aspect of doxa
in particular was challenged during the convention, and it was
particularly surprising that this was the case.

One of the attendees at the panel brought up the fact that, during a
time of crisis, epiphany, or great change in life, sometimes one's
character also goes through change (in this case, a change in species
from fox to rat), in a sense reflecting external events in an extreme
way. Even though several of us were surprised that such things as a
turning point in life would be shown in something so fundamental as
one's species, it's one of those things that makes sense upon
consideration. Even looking back, for myself, the one time I truly
changed species surrounded a profound change in my life. Moving to
college - and all that is entailed in that, such as moving away from
parents, getting a job, and so on - affected me deeply. That signified a
total restructuring of my life, even to the point where the old
character I had inhabited, a red fox with two tails, the tips of which
were dyed green, no longer applied. It was high-school-me. It was
me-growing-up. It is not me now.

The reactions from around the room echoed my sentiment. While most were
surprised and intrigued at the concept of an external factor such as a
move or an epiphany having so large an effect on someone as to cause a
sudden, major restructuring of their furry identity, many, myself
included, confirmed that this is not infrequent. Those who were most
surprised felt that a sudden crisis such as this would not lead to a
major change, but rather influence the direction in which their
character grew. That is, their goals would change both for them as well
as their character, though aspects such as species would remain.
Unfortunately, we ran low on time before we had the chance to
investigate the differences in how these two rough groups dealt with
their character's identity, though it is worth investigating! That there
is even the trope of the species-change-journal on FA is proof of this.

As a meta-furry resource, {[}adjective{]}{[}species{]} explores a lot of
topics surrounding furry, though it seems of late that the focus has
been on topics that happen to be ancillary to the fandom itself. These
are all dreadfully interesting, I think, but so is much of the stuff at
the core of our subculture, this base layer that helps make us who we
are. These are the reasons we seek to meet up together at cons such as
RMFC, not simply these supplementary reasons such as being ahead of or
behind the rest of the world, any skews in sexual orientation or gender,
or even movies about cheetahs, though they may all help. These core
facets are worth exploring, as they help to form coherence among all
these different animal-folk.

If you are interested in more from the panel, the notes are available
\href{https://github.com/adjspecies/roundtables-and-discussions/blob/master/species-identity-and-character-creation/notes-RMFC2013-20130804.markdown}{here}.

\begin{center}\rule{0.5\linewidth}{\linethickness}\end{center}

* The responses were cleaned of some very common words that tended to
skew the word-clouds, such as articles (the, a, an), conjunctions (but,
and), and the species' name and plural form of the name which, of
course, show up quite often.

** \href{http://adjectivespecies.com/?attachment_id=1444}{Cats},
\href{http://adjectivespecies.com/?attachment_id=1445}{cheetahs},
\href{http://adjectivespecies.com/?attachment_id=1452}{coyotes},
\href{http://adjectivespecies.com/?attachment_id=1446}{dragons},
\href{http://adjectivespecies.com/?attachment_id=1447}{red foxes},
\href{http://adjectivespecies.com/?attachment_id=1466}{horses},
\href{http://adjectivespecies.com/?attachment_id=1448}{huskies},
\href{http://adjectivespecies.com/?attachment_id=1449}{jackals},
\href{http://adjectivespecies.com/?attachment_id=1453}{rabbits},
\href{http://adjectivespecies.com/?attachment_id=1450}{tigers}, and
\href{http://adjectivespecies.com/?attachment_id=1451}{wolves}.
