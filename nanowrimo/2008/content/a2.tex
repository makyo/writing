Full of bagel, cream cheese, and lax, Our route turned east along I-80 for the few miles it took to get to I-25.  Even though I'd gotten a coffee to go with our lunch, I was still tired.  When my mom asked, I told her it was how boring the bland landscape was after the relative excitement of the mountains and greenery.  I had barely slept the night before.  Despite trying to act cool about the whole moving to college thing, excitement really had taken its toll on me, and I had alternated between worrying in bed and worrying at my desk.  I must've mowed through half the container of olives we had in the fridge that night, sneaking out so as not to wake my mom and Jared as I made my food raids.  Those spicy olives straight from the container were one of my comfort foods.  One of those things that has to be eaten with the fingers.

I suppose I'm a little weird.

The conversation wandered around a little more between my mom and my self as she shared anecdotes from her own college life and I talked about recent stories about my friends as news from the perennial diaspora of high school graduates to colleges across the country trickled back to me.  ``Other Cory'' had wound up down in Denver at the University of Colorado's campus there, and he had sent me a few pictures from his most recent visit down there.  A few more friends from band had made their way to CU's Boulder campus, where my mom had wanted me to go, and the Inseparable Trio of Karen, Jessie and Nate had made their way to the University of Northern Colorado in Greeley, just a half hour away from where I was headed.  Only one other of my friends from band --- more an acquaintance than anything --- had picked Colorado State University as I had, trekking over to Fort Collins along with me for visits and auditions, though he wasn't moving in until later today.

Most of my other friends, though, had spread out much further than Colorado.  Many of the other band kids had filtered down to various schools in Texas for their music and education programs down there, and one or two made their way to each coast.  Their parents were loaded, though, and could afford to pay for all the plane tickets and out of state tuition that was involved in such a move.

I was comfortable heading three or four hours away, though.  I felt that it was close enough to home that I could visit if I wanted once I got my car fixed over Thanksgiving break.  Still, it was far enough away so that I wouldn't have to worry about my mom `emtpy nesting' on me and coming over to visit, except for the concerts.  Dad was down in Colorado Springs, which was a good distance away, though I didn't expect to see him quite as much.  Since he was helping with tuition as well, he made in-state tuition a must.  Not that I minded, I loved Colorado, it just meant that if I wanted to get away, I would have to choose my schools carefully.  I knew CSU from the two times I had done summer band camp, so that was my logical choice.

Sipping my way through my coffee, I let the flattening landscape and my mom's music lull me into an empty mind.  Excitement and caffeine kept me from dozing, but it felt like the first real relaxation I'd had in a while.

Dad had sent me on my way with his goofy stories from college: shooting out a street light with a .22 rifle and having to repaint twenty light posts for the city as his community service; drinking with friends; smoking enough pot at a party that he wandered into the wrong apartment when he'd tried to go home.  ``Just promise me you'll get a DD if you drink, be a DD when you don't, and call me once in a while,'' was his goodbye when I'd left for Steamboat again on Wednesday.  He'd given me a check for fifty dollars and walked me out to Jared's Honda, the car I was borrowing for this last visit.

Jared had little to say to when it came to college other than to agree with most of what my mom said and offer up common sense advice on doing my homework.  I had watched his own kid graduate and move off to college, which was a much bigger deal to him - he and Jennifer had gone out to dinner on their own four or five times in as many weeks before she made her way across town to the Colorado Mountain College, and they had planned everything meticulously.  It was understandable, I guess.  He and my mom hadn't gotten married until my Sophomore year of high school - he was just that guy that lived with us, and I was just that kid his girlfriend had from before.  Didn't matter much to us that I was leaving.  And I don't suppose it helped that I liked guys.  I was always just a little unnatural to him.

Mom and dad had taken that whole thing pretty well, at least.  There were a few long talks I had to sit through about whether or not the whole thing was a phase or not, what this meant for their hopes of grandchildren, who I would go out with, and so on, but after a while, it was all normal to them.  They both liked Chris, they both had their concerns about the whole internet dating thing, and they both treated me as they always had, which I suppose was the most important thing.  Say what you want about Colorado in general, but I guess when it's your kid, it's hard to freak out too much.  Besides, they were both hippies once.

The bigger concern amongs them, my parents and Jared, was that I wanted to go into music.  That alone had caused more strife than coming out had.

``You'll never make any money,'' was what their arguments had come down to.  Usually, it was couched in some lecture-speak, like, ``There's a fine line between doing what you love and doing what you have to do in order to live comfortably.''  I had been a good kid and rarely rolled my eyes, but after watching both parents suffer through work, after watching Jared's relief at his lay-off, and most importantly, watching Mr. Paulsen talk about how much he loved his job in music teaching our band, I had to roll my eyes at this.

The arguments went back and forth, and my only concession had been to major in music education instead of just plain music.  Teaching wouldn't be so bad, so long as I could teach music.
